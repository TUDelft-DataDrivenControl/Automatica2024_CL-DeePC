\section{Preliminaries and notation}
This section presents the employed system model description and notation that are used throughout this article.

\subsection{System Model}
A typical generic, non-deterministic discrete \ac{LTI} system considers both process and measurement noise
\begin{subequations}\label{eqn:SS_wv}
\begin{align}
	\bar{x}_{k+1} &= A\bar{x}_k + Bu_k + w_k,\label{eqn:SSwv_x}\\
	y_k &= C\bar{x}_k + Du_k + v_k \label{eqn:SSwv_y},
\end{align}
\end{subequations}
in which ${\bar{x}_k\in\mathbb{R}^n}$, ${u_k\in\mathbb{R}^r}$, ${y_k\in\mathbb{R}^l}$, ${w_k\in\mathbb{R}^n}$, ${v_k\in\mathbb{R}^l}$, respectively are states, inputs, outputs, process noise, and measurement noise at the time instant denoted by the subscript $k$. In addition, $A$, $B$, $C$, $D$ respectively are the state transition, input, output, and direct-feedthrough matrices of corresponding dimensions.

Let the process and measurement noise be zero-mean random sequences with covariance matrix
\begin{equation*}
    \mathbb{E}\left\{\begin{bmatrix}w_k\\v_k\end{bmatrix}\begin{bmatrix}w_j^\top & v_j^\top\end{bmatrix}\right\} =
\begin{bmatrix}Q & S\\S^\top & R\end{bmatrix}\delta_{kj},
\end{equation*}
in which $R>0$, $\delta_{kj}$ is the Kronecker delta function (1 if $k=j$, 0 otherwise). If the covariance matrix above, before $\delta_{kj}$, is positive semi-definite, the pair $(A,\,C)$ is observable, and $(A,\,Q^{1/2})$ is reachable, then there exists an innovation form representation that is equivalent to~\eqnref{eqn:SS_wv} given by~\cite[p.~\todo{Find pages}]{Anderson1979}, \cite[p.~162]{Verhaegen2007a}%
\begin{subequations}\label{eqn:SS_innovation}
\begin{align}
	x_{k+1} &= Ax_k + Bu_k + Ke_k,\label{eqn:SSi_x}\\
	y_k &= Cx_k + Du_k + e_k \label{eqn:SSi_y},
\end{align}
\end{subequations}
in which $x_k\in\mathbb{R}^{n}$ represents states (different from $\bar{x}_k$), $e_k$ represents so called innovation noise, and $K\in\mathbb{R}^{n\times l}$ is a constant Kalman gain matrix that renders ${\tilde{A}=A-KC}$ asymptotically stable. By substituting \eqnref{eqn:SSi_y} into \eqnref{eqn:SSi_x} one alternatively obtains the equivalent predictor form
\begin{subequations}\label{eqn:SS_innovation}
\begin{align}
	x_{k+1} &= \tilde{A}x_k + \tilde{B}u_k + Ky_k,\label{eqn:SSp_x}\\
	y_k &= Cx_k + Du_k + e_k \label{eqn:SSp_y},
\end{align}
\end{subequations}
in which $\tilde{B}=B-KD$.

\subsection{Notation}
Having described the different representations of the considered discrete, non-deterministic \ac{LTI} systems, this section introduces the notation used in this work.

To start, a block-Toeplitz matrices are defined by
\begin{align}\label{eqn:blockToeplitz} 
\mathcal{T}_s(\mathcal{A},\mathcal{B},\mathcal{C},\mathcal{D}) =\scriptsize{
	\begin{bmatrix}
		\mathcal{D}         & 0         & 0      & \cdots  & 0\\
		\mathcal{C}\mathcal{B}        & \mathcal{D}         & 0      & \cdots  & 0\\
		\mathcal{C}\mathcal{A}\mathcal{B}       & \mathcal{C}\mathcal{B}        & \mathcal{D}      & \cdots & 0\\
		\vdots    &  \vdots & \ddots & \ddots & \vdots\\
		\mathcal{C}\mathcal{A}^{s-2}\mathcal{B} & \mathcal{C}\mathcal{A}^{s-3}\mathcal{B} & \cdots  & \mathcal{C}\mathcal{B}     & \mathcal{D}
	\end{bmatrix}},
\end{align}
in which the subscript $s$ indicates the number of block-rows, and the matrices $\mathcal{A}$, $\mathcal{B}$, $\mathcal{C}$, and $\mathcal{D}$ are all of compatible dimensions. Let $I_s\in\mathbb{R}^{s\times s}$ represent an identity matrix and $f\in\mathbb{Z}^+$ a strictly-positive integer prediction window length. Equation~\eqnref{eqn:blockToeplitz} thereby defines the block-Toeplitz matrices
\begin{alignat*}{2}
\Tf{u}&=\mathcal{T}_f(A,B,C,D),\quad  &\Hf&=\mathcal{T}_f(A,K,C,I_l),\\
\tTf{u}&=\mathcal{T}_f(\tilde{A},\tilde{B},C,D),\quad  &\tHf&=\mathcal{T}_f(\tilde{A},K,-C,I_l).
\end{alignat*}

In addition, two extended observability matrices are defined by
\begin{align*}
\Gf &= \begin{bmatrix}C^\top & (CA)^\top & \cdots & (CA^{f-1})^\top\end{bmatrix}^\top\quad\text{and}\\
\tGf &= \begin{bmatrix}C^\top & (C\tilde{A})^\top & \cdots & (C\tilde{A}^{f-1})^\top\end{bmatrix}^\top,
\end{align*}
and two extended reversed controllability matrices are defined as 
\begin{align*}
\tKp{u} &= \begin{bmatrix} \tilde{A}^{p-1}\tilde{B}\:\, & \tilde{A}^{p-2}\tilde{B} & \cdots & \tilde{A}\tilde{B} & \tilde{B}\:\, \end{bmatrix},\text{ and}\\
\tKp{y} &= \begin{bmatrix} \tilde{A}^{p-1}K & \tilde{A}^{p-2}K & \cdots & \tilde{A}K & K \end{bmatrix}.
\end{align*}

Furthermore, data vectors are denoted as examplified by
\begin{align*}
    \datavec{y}{i,p} = \begin{bmatrix} y_i^\top & y_{i+1}^\top & \cdots & y_{i+p-1}^\top\end{bmatrix}^\top,
\end{align*}
which represents a vector of ordered output data starting at the time index given by the first subscript $i$, and containing a number of data samples indicated by the second subscript $p$. Akin to $f$, the subscript $p\in\mathbb{Z}^+$ will be used throughout this work to refer to a strictly-positive integer that represents a certain window length of data.

Using such data vectors it is possible to concisely define block-Hankel data matrices. Such a block-Hankel data matrix is examplified by
\begin{align*}
    Y_{i,p,N} = \begin{bmatrix}
        \datavec{y}{i,p} & \datavec{y}{i+1,p} & \cdots & \datavec{y}{i+N-1,p}
    \end{bmatrix},
\end{align*}
which contains $N\in\mathbb{Z}^+$ successive output data vectors with $p$ data samples each, starting from time index $i$.