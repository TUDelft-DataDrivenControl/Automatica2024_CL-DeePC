\section{Preliminaries}
This section presents the employed system model, notation, and the considered control problem.

\subsection{System model}
Consider a non-deterministic discrete \ac{LTI} system $\mathcal{S}$ whose dynamics is described in the so-called \textit{innovation} form by
\begin{subequations}\label{eqn:SS_innovation}
\begin{empheq}[left=\mathcal{S}_\mathcal{I}\empheqlbrace]{align}
    x_{k+1} &= Ax_k + Bu_k + Ke_k,\label{eqn:SSi_x}\\
	y_k &= Cx_k + Du_k + e_k \label{eqn:SSi_y},
  \end{empheq}
\end{subequations}
in which the subscript $k\in\mathbb{Z}$ denotes the discrete time index, ${x_k\in\mathbb{R}^n}$, ${u_k\in\mathbb{R}^r}$, ${y_k\in\mathbb{R}^l}$, ${e_k\in\mathbb{R}^l}$ respectively represent states, inputs, outputs, and zero-mean white innovation noise with variance $R_\mathrm{e} > 0$, and $\{A,B,C,D,K\}$ are system matrices of compatible dimensions. Without loss of generality we henceforth assume the data to be generated by a minimal system realization. %also done in Breschi2022
In accordance with Kalman filtering theory from which this representation derives, $K$ represents a unique (and optimal) Kalman filter gain matrix that renders ${\tilde{A}=A-KC}$ asymptotically stable (see, e.g., \citet[Sec.~5.7]{Verhaegen2007a}). By substituting \eqnref{eqn:SSi_y} into \eqnref{eqn:SSi_x} one alternatively obtains the equivalent predictor form
\begin{subequations}\label{eqn:SS_predictor}
\begin{empheq}[left=\mathcal{S}_\mathcal{P}\empheqlbrace]{align}
	x_{k+1} &= \tilde{A}x_k + \tilde{B}u_k + Ky_k,\label{eqn:SSp_x}\\
	y_k &= Cx_k + Du_k + e_k \label{eqn:SSp_y},
  \end{empheq}
% \begin{align}
	% x_{k+1} &= \tilde{A}x_k + \tilde{B}u_k + Ky_k,\label{eqn:SSp_x}\\
	% y_k &= Cx_k + Du_k + e_k \label{eqn:SSp_y},
% \end{align}
\end{subequations}
in which $\tilde{B}=B-KD$.
%
% ${x_k\in\mathbb{R}^n}$, ${e_k\in\mathbb{R}^l}$, and $K\in\mathbb{R}^{n\times l}$ respectively represent states, zero-mean white innovation noise, and a unique Kalman gain matrix that renders ${\tilde{A}=A-KC}$ asymptotically stable.
%
% in which ${\bar{x}_k\in\mathbb{R}^n}$, ${u_k\in\mathbb{R}^r}$, ${y_k\in\mathbb{R}^l}$, ${w_k\in\mathbb{R}^n}$, ${v_k\in\mathbb{R}^l}$ respectively represent states, inputs, outputs, process noise, and measurement noise, and $\{A,B,C,D\}$ are system matrices of compatible dimensions. If the process and measurement noise are either zero-mean white or colored Gaussian sequences such that the system $\mathcal{S}$ satisfies notions of detectability and reachability there exists an equivalent innovation form $\mathcal{F}_\mathcal{I}$ that is based on Kalman filtering (see~\citet[p.~112-113\todo{Check pages}]{Anderson1979}, or \citet[p.~162]{Verhaegen2007a} for details) and is given by
%
\subsection{Notation and definitions}\label{sec:notation}
Having described the different system representations of the considered system, this section introduces some useful preliminary notation and definitions.
%
% To start, several strictly-positive integers are denoted by $s,q,p,f,N,\bar{N}\in\mathbb{Z}_{>0}$ and are frequently used to indicate window lengths. Throughout this article, time indices $i$, $\hat{i}$, $j$, and $k$, will be used together with the shorthand exemplified by $k_p=k+p$.

To start, block-Toeplitz matrices are defined by
\begin{align}\label{eqn:blockToeplitz} 
\mathcal{T}_s(\mathcal{A},\mathcal{B},\mathcal{C},\mathcal{D}) =\scriptsize{
	\begin{bmatrix}
		\mathcal{D}         & 0         & 0      & \cdots  & 0\\
		\mathcal{C}\mathcal{B}        & \mathcal{D}         & 0      & \cdots  & 0\\
		\mathcal{C}\mathcal{A}\mathcal{B}       & \mathcal{C}\mathcal{B}        & \mathcal{D}      & \cdots & 0\\
		\vdots    &  \vdots & \ddots & \ddots & \vdots\\
		\mathcal{C}\mathcal{A}^{s-2}\mathcal{B} & \mathcal{C}\mathcal{A}^{s-3}\mathcal{B} & \cdots  & \mathcal{C}\mathcal{B}     & \mathcal{D}
	\end{bmatrix}},
\end{align}
in which the subscript $s\in\mathbb{Z}_{>0}$ is a generic strictly-positive integer that is used here to indicate the number of block-rows, the matrices $\mathcal{A}$, $\mathcal{B}$, $\mathcal{C}$, and $\mathcal{D}$ are all of compatible dimensions. Let ${I_s\in\mathbb{R}^{s\times s}}$ represent an identity matrix. Equation~\eqnref{eqn:blockToeplitz} thereby defines the block-Toeplitz matrices
\begin{alignat*}{2}
\mathcal{T}_s^\mathrm{u}&=\mathcal{T}_s(A,B,C,D),\quad  &\mathcal{H}_s&=\mathcal{T}_s(A,K,C,I_l),\\
\widetilde{\mathcal{T}}_s^\mathrm{u}&=\mathcal{T}_s(\tilde{A},\tilde{B},C,D),\quad  &\widetilde{\mathcal{H}}_s&=\mathcal{T}_s(\tilde{A},K,-C,I_l).
\end{alignat*}

In addition, two extended observability matrices are defined by
\begin{align*}
\Gamma_s &= \begin{bmatrix}C^\top & (CA)^\top & \cdots & (CA^{s-1})^\top\end{bmatrix}^\top,\\%\quad\text{and}\\
\widetilde{\Gamma}_s &= \begin{bmatrix}C^\top & (C\tilde{A})^\top & \cdots & (C\tilde{A}^{s-1})^\top\end{bmatrix}^\top.
\end{align*}
The extended observability matrix $\Gamma_s$ is used to define a system property commonly referred to as its lag.
\begin{defn}\label{def:lag}
    A system's lag is the smallest integer $\ell\in\mathbb{Z}_{>0}$ such that the extended observability matrix $\Gamma_\ell$ is of rank $n$.
\end{defn}
For any observable dynamic \ac{LTI} system, $1\leq \ell \leq n$.

Moreover, two extended reversed controllability matrices are defined as 
\begin{align*}
\tKp{u} &= \begin{bmatrix} \tilde{A}^{p-1}\tilde{B}\:\, & \tilde{A}^{p-2}\tilde{B} & \cdots & \tilde{A}\tilde{B} & \tilde{B}\:\, \end{bmatrix},\\%\text{ and}\\
\tKp{y} &= \begin{bmatrix} \tilde{A}^{p-1}K & \tilde{A}^{p-2}K & \cdots & \tilde{A}K & K \end{bmatrix},
\end{align*}
in which $p\in\mathbb{Z}_{>0}$ is a past data window length.

Furthermore, data vectors are denoted as examplified by
\begin{align*}
    \datavec{u}{k,s} = \begin{bmatrix} u_k^\top & u_{k+1}^\top & \cdots & u_{k+s-1}^\top\end{bmatrix}^\top,
\end{align*}
which represents a vector of ordered input data starting at time index $k$, and containing a number of samples $s$.

Using such data vectors it is possible to concisely define block-Hankel data matrices. Such a block-Hankel data matrix is examplified by
\begin{align*}
    U_{k,s,q} = \frac{1}{\sqrt{q}}\begin{bmatrix}
        \datavec{u}{k,s} & \datavec{u}{k+1,s} & \cdots & \datavec{u}{k+q-1,s}
    \end{bmatrix},
\end{align*}
in which $q\in\mathbb{Z}_{>0}$ is another generic positive-definite integer that here represents the number of successive input data vectors with $s$ data samples each, starting from time index $k$. Note the block-anti diagonal structure of block-Hankel matrices. This notion of block-Hankel data matrices is employed to define the notion of persistency of excition.
\begin{defn}\label{def:PE}
    A signal consisting of samples ${w_j\in\mathbb{R}^q},$ $j\in[k,\,k+s+N-2]$ is persistently exciting of order $s$ if the associated block-Hankel matrix ${W_{k,s,N}\in\mathbb{R}^{sq \times N}}$ is full row rank.
\end{defn}
Predicted samples, vectors, or matrices of data are indicated by $\hat{(\cdot)}$.% whilst those that are comprised in part of predictions are indicated by $\tilde{(\cdot)}$.

For convenience, the notation $\Phi$ is reserved to denote an often encountered concatenation of input and output block-Hankel matrices that is given by
\begin{align*}
    \Phi_{k,s,q} = \begin{bmatrix}
        U_{k,p,q}^\top & U_{k_p,s,q}^\top & Y_{k,p,q}^\top
    \end{bmatrix}^\top,
\end{align*}
in which $k$, and $q$ respectively indicate the starting index and parameterize the dimensions of the concatenated matrix together with $p$.
% 
% ==============================================================================================================================================================
% ==============================================================================================================================================================
\subsection{The data equations}\label{sec:DerivingDataEquations}
This section derives several fundamental relations called data-equations, which reformulate \eqref{eqn:SS_innovation} and \eqref{eqn:SS_predictor} in terms of block-Hankel matrices.

To this end, it can be shown by iterative application of respectively \eqref{eqn:SS_innovation} and \eqref{eqn:SS_predictor} that%
\begin{align}
    Y_{k_p,s,q} &= \Gamma_s X_{k_p,1,q} + \mathcal{T}_s^\mathrm{u} U_{k_p,s,q} + \mathcal{H}_s E_{k_p,s,q}\label{eq:Yf1},\\
    \begin{split}%
    Y_{k_p,s,q} &= \widetilde{\Gamma}_s X_{k_p,1,q} + \widetilde{\mathcal{T}}_s^\mathrm{u} U_{k_p,s,q} + E_{k_p,s,q}\\
    &\phantom{=}+(I_{sl}-\widetilde{\mathcal{H}}_s)Y_{k_p,s,q},
    \end{split}\label{eq:Yf2}
\end{align}
in which we employ a recurring shorthand for time indices examplified by $k_p=k+p$. Furthermore, the initial states can be rewritten in terms of preceding states and input-output data using \eqref{eqn:SS_predictor} as%
\begin{align}\label{eq:Xip}
    X_{k_p,1,q} = \tilde{A}^p X_{k,1,q} + \tKp{u} U_{k,p,q} + \tKp{y} Y_{k,p,q}.
    % \begin{bmatrix}
    %     Y_{i,p,q}\\
    %     U_{i,p,q}
    % \end{bmatrix}.
\end{align}
% in which $\tKp{}=\big[\tKp{y}\;\;\tKp{u}\big]$.
Substitute \eqref{eq:Xip} into \eqref{eq:Yf1} and \eqref{eq:Yf2} %and apply Assumption~\ref{assum:initial_contribution}
to obtain two so called data equations:
\begin{align}
    Y_{k_p,s,q} &= L_s \Phi_{k,s,q} + \mathcal{H}_s E_{k_p,s,q} + \Gamma_s \tilde{A}^p X_{k,1,q},\label{eq:DataEq1}\\
    \begin{split}
    Y_{k_p,s,q} &= \widetilde{L}_s \Phi_{k,s,q} + E_{k_p,s,q} + (I_{sl}-\widetilde{\mathcal{H}}_s) Y_{k_p,s,q} \\
    &\phantom{=}+ \widetilde{\Gamma}_s \tilde{A}^p X_{k,1,q},
    \end{split}\label{eq:DataEq2}
\end{align}
in which, for convenience, we define, two reoccurring so called `dynamic matrices' as
\begin{align*}
    L_s &= \begin{bmatrix} \Gamma_s\tKp{u} & \mathcal{T}_s^\mathrm{u} & \Gamma_s\tKp{y} \end{bmatrix},\\%\text{ and}\\
    \widetilde{L}_s &= \begin{bmatrix} \widetilde{\Gamma}_s\tKp{u} & \widetilde{\mathcal{T}}_s^\mathrm{u} & \widetilde{\Gamma}_s\tKp{y} \end{bmatrix}.
\end{align*}

% in which $L_s$, $\widetilde{L}_s$, $\Phi_{k,s,q}$ are defined in Section \ref{sec:notation}. %Similarly to \eqref{eq:DataEq1} and \eqref{eq:DataEq2}, the future outputs are defined by
% Although a more generic representation was kept above for later analysis, for \ac{CL-DeePC}, $s=1$. This reduces the complexity of the above equations since ${\widetilde{L}_1=L_1=\big[ C\tKp{u} \; D \; C\tKp{y} \big]}$ and $\widetilde{\mathcal{H}}_1=\mathcal{H}_1=I_l$.
%
\subsection{Receding horizon control problem formulation}
To clearly demarcate the scope of this work this section presents the receding horizon control problem that is considered here, which is formulated using \eqref{eq:Yf1} for a single column ($q=1$) from time index $k=\hat{i}$ and with $s=f\in\mathbb{Z}_{>0}$ as future prediction window length, as
\begin{subequations}
\begin{alignat}{2}
    &\min_{\datavec{u}{\hat{i}_p,f}} ||\datavec{\hat{y}}{\hat{i}_p,f}-\datavec{r}{\hat{i}_p,f}||_Q^2 + ||\datavec{u}{\hat{i}_p,f}||_R^2 \span\span\\
    \text{s.t.}\quad& &\datavec{\hat{y}}{\hat{i}_p,f}&=\Gamma_f x_{\hat{i}_p}+\mathcal{T}_f^\mathrm{u}\datavec{u}{\hat{i}_p,f}%+\mathcal{H}_f\datavec{e}{\hat{i}_p,f}
    ,\label{eq:SS_iter}\\
   && x_{\hat{i}_p}&=x_\mathrm{ini},\label{eq:x_ini}\\
   && u_k&\in\mathcal{U},\quad \hat{y}_k\in\mathcal{Y},\quad \forall k\in\big[\hat{i}_p,\,\hat{i}_p+f\big),
\end{alignat}
\end{subequations}
% \begin{subequations}
%     \begin{alignat}{3}
%       &\min_{\substack{u_k,\\ \mathclap{\forall k\in\left[\hat{i}_p,\,\hat{i}_p+f\right)}}}&&\quad\sum_{k=\hat{i}_p}^{\mathclap{\hat{i}_p+f-1}} \mathbb{E} \left[||y_k-r_k||_Q^2\right] + ||u_k||_R^2\span\span \\
%      \text{s.t.}& &&\text{\eqref{eqn:SS_innovation},} &&\forall k\in\left[\hat{i}_p,\,\hat{i}_p+f\right),\label{eq:SS_iter}\\
%      &&x_{\hat{i}_p}&=x_\mathrm{ini},&&\label{eq:x_ini}\\
%      && u_k&\in \mathcal{U}, \; \mathbb{E}[y_k]\in\mathcal{Y},\quad &&\forall k\in\left[\hat{i}_p,\,\hat{i}_p+f\right),
% \end{alignat}
% \end{subequations}
% in which $\hat{i}_p$ is the first future time index, at which time there is an initial state $x_\mathrm{ini}$, $f\in\mathbb{Z}_{>0}$ is a future prediction window length, and $r_k\in\mathbb{R}^l$ is a sample of an output reference trajectory. In addition, $||\cdot||_\cdot$ denotes a weighted 2-norm, with $Q\in\mathbb{R}^{l\times l}$ and $R\in\mathbb{R}^{r\times r}$ as respectively positive semi-definite and positive definite user-defined weighting matrices. Furthermore, $\mathbb{E}[\cdot]$ represents an expectation with respect to the future innovation signal that is conditioned on the future input sequence and the initial state.
in which $\datavec{r}{\hat{i}_p,f}\in\mathbb{R}^{fl}$ is a data vector of a future reference trajectory, $x_\mathrm{ini}$ is an initial state at the first future time $k=\hat{i}_p$, and $||\cdot||_{(\cdot)}$ denotes a weighted Euclidian norm, $Q\in\mathbb{R}^{fl\times fl}$ and $R\in\mathbb{R}^{fr\times fr}$ are respectively positive semi-definite and positive definite user-defined weighting matrices, and $\mathcal{U}$ and $\mathcal{Y}$ respectively represent sets of allowable inputs and outputs. %Furthermore, $\mathbb{E}[\cdot]$ represents an expectation w.r.t. the future innovation signal $\datavec{e}{\hat{i}_p,f}$ that is conditioned on the future input sequence $\datavec{u}{\hat{i}_p,f}$ and the initial state $x_{\hat{i}_p}$, and $\mathcal{U}$ and $\mathcal{Y}$ respectively represent sets of allowable inputs and outputs.

Without knowledge of the system matrices $\{A,B,C,D,K\}$ and the initial state $x_\mathrm{ini}$, but given sufficiently informative past input-output data from  intervals $k\in[i,\,i+\bar{N})$ and $k\in[\hat{i},\,\hat{i}_p)$ that may overlap\footnote{Depending on the difference $\hat{i}-i>0$ and the number of samples $\bar{N}$.} and have been collected in closed-loop, the principal goal is to find an unbiased behavioural output predictor to replace the unknown relations \eqref{eq:SS_iter} and \eqref{eq:x_ini}.