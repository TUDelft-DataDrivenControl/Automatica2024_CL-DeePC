\section{Closed-loop Data-enabled Predictive Control}
This section presents the main result of this article, providing contribution~(\ref{contribution:solves_CL_issue}) whereby we develop \ac{CL-DeePC}. An intuitive explanation is first offered before a proof of the underlying main result is provided.

As a solution to the identification bias that arises in closed-loop due to correlation between inputs and noise (a demonstration thereof is deferred to Section~\ref{sec:CL_ID_issue}) it is possible to estimate a step-ahead predictor~\citep{Ljung1996}. A prediction horizon length $f>1$ is of more practical use in receding horizon optimal control settings, to which end step-ahead predictors can be applied sequentially. 

Fig.~\ref{fig:CL-DeePC} and \eqref{eq:CL_DeePC_no_IVs} illustrate how this idea is employed in \ac{CL-DeePC}. A step-ahead predictor can be obtained from \ac{DeePC} (see Fig.~\ref{fig:regular-DeePC} and \eqref{eq:regular_DeePC_no_IVs}) with $f=1$. In \ac{CL-DeePC} the successive columns of $G$ (from left to right) and their corresponding columns on the right-hand side correspond to sequential applications of \ac{DeePC} with $f=1$ to the same matrix of sufficiently persistently exciting past input-output data on the left-hand side as well as time-shifted windows of input-output data on the right-hand side that encode information on successive initial states.\\+
%
\begin{figure}[b!]
\centering
\begin{tikzpicture}
    % defining constants
    \def\stepSize{0.25}
    \def\Nnum{9}
    \def\fnum{8}
    \def\pnum{4}
    
    % Defining lengths
    \newlength{\onelen}
    \setlength{\onelen}{\stepSize cm}
    \newlength{\BrCl}
    \setlength{\BrCl}{0.075cm}
    \newlength{\BrIn}
    \setlength{\BrIn}{0.15cm}
    \newlength{\plen}
    \setlength{\plen}{1cm}%{\pnum\stepSize cm}
    \newlength{\flen}
    \setlength{\flen}{2cm}%{\fnum*\stepSize cm}
    \newlength{\Nlen}
    \setlength{\Nlen}{2.25cm}%{\Nnum\stepSize cm}%should be 2*p+one
    \newlength{\MatClearance}
    \setlength{\MatClearance}{0.3cm}
    
    % grid lines for guidance
    % \draw[gray,step=0.5] (-0,-3) grid (8,3);

    % ======================= drawing data matrix =======================
    \path (0,-\onelen) coordinate (M1A);
    \path ([xshift=\Nlen]M1A) coordinate (M1B);
    \path ([yshift=2*\plen+2*\onelen]M1B) coordinate (M1C);
    \path ([xshift=-\Nlen]M1C) coordinate (M1D);
    \draw[line width=1.5pt] ([xshift=\BrIn,yshift=-\BrCl]M1A) -- ([xshift=-\BrCl,yshift=-\BrCl]M1A) -- ([xshift=-\BrCl,yshift=\BrCl]M1D) -- ([xshift=\BrIn,yshift=\BrCl]M1D); %left bracket
    \draw[line width=1.5pt] ([xshift=-\BrIn,yshift=-\BrCl]M1B) -- ([xshift=\BrCl,yshift=-\BrCl]M1B) -- ([xshift=\BrCl,yshift=\BrCl]M1C) -- ([xshift=-\BrIn,yshift=\BrCl]M1C); %right bracket
    \draw[line width=1pt] ([yshift=\onelen]M1A) -- ([yshift=\onelen]M1B); % dividing matrix into blocks
    \fill[black, opacity=0.5] (M1A) rectangle (M1C);
    \foreach \x in {0,...,8} { % drawing black dots
    \foreach \y in {-1,...,8} {
      \fill ( {(\x+0.5)*\onelen}, {(\y+0.5)*\onelen} ) circle (1pt);
    }}
    \draw[line width=0.1pt] ([yshift=-\plen-\onelen]M1D) -- ([yshift=-\plen-\onelen]M1C);
    \draw[line width=0.1pt] ([yshift=-\plen]M1D) -- ([yshift=-\plen]M1C);

    % coordinates for diagonals
    \path ([yshift=-\plen]M1D) coordinate (M1stair1A);
    \path ([xshift=\plen]M1D) coordinate (M1stair1I);
    % black diagonals
    % \foreach \i in {0,-1}{
    % \foreach \dy in {1,...,11}{
    % \ifthenelse{\dy<5.5}{
    % % This code will be executed if \dy<9.5
    %     \draw[dash pattern=on 1pt off 2pt, line width=0.1pt] ([xshift= 0.5*\onelen,yshift=\i*(\plen+\onelen)-(\dy+0.5)*\onelen]M1D) -- ([xshift= (\dy+0.5)*\onelen,yshift=\i*(\plen+\onelen)-0.5*\onelen]M1D);
    % }{
    % % This code will be executed if \dy>=10
    %     \ifthenelse{\dy<8.5}{
    %         % This code will be executed if \dy<=12
    %         \draw[dash pattern=on 1pt off 2pt, line width=0.1pt] ([xshift= (\dy-3.5)*\onelen,yshift=\i*(\plen+\onelen)-\plen-\onelen+0.5*\onelen]M1D) -- ([xshift=(\dy-8.5)*\onelen,yshift=\i*(\plen+\onelen)-0.5\onelen]M1C);
    %         }{
    %         % This code will be executed if \dy>=13
    %         \draw[dash pattern=on 1pt off 2pt, line width=0.1pt] ([xshift= (\dy-3.5)*\onelen,yshift=(\i-1)*(\plen+\onelen)+0.5\onelen]M1D) -- ([xshift= -0.5*\onelen,yshift=\i*(\plen+\onelen)-(\dy-7.5)*\onelen]M1C);
    %     }
    % }
    % }}
    
    % ---------------- drawing left brace and matrix ----------------
    \path ([xshift=-0.5\BrIn,yshift=-2\BrCl]M1A) coordinate (Brace1L);
    \path ([xshift=0.5\BrIn,yshift=-2\BrCl]M1B) coordinate (Brace1R);
    \draw[decorate, decoration={calligraphic brace, amplitude=3pt, mirror, aspect=0.75},line width=1pt] (Brace1L) -- (Brace1R);
    \node (mat1) at ($(Brace1L)!0.75!(Brace1R) + (0,-3pt)$) {};
    \node[below] at (mat1.center) {$\begin{bmatrix}
        U_{i,p,N}\\U_{i_p,1,N}\\Y_{i,p,N}\\ \hline Y_{i_p,1,N}
    \end{bmatrix}$};
    % brace to the left
    % \path ([xshift=-0.75cm-0.05cm,yshift=-4pt]mat1.center) coordinate (Brace1T);
    % \path ([yshift=-1.73cm]Brace1T) coordinate (Brace1B);
    % \draw[pen colour=gray!60,decorate, decoration={calligraphic brace, amplitude=3pt, mirror, aspect=0.2},line width=0.75pt] (Brace1T) -- (Brace1B);
    % % Zp
    % \node (Zp) at ($(Brace1T)!0.2!(Brace1B) + (-3pt,-0.8pt)$) {};
    % \node[left,text=gray!100,align=center] at ([xshift=0.5mm,yshift=-0.18cm]Zp.center) {\scriptsize{\shortstack{$=$\\$\Phi_{i,1,N}$}}};
  
    % ======================= drawing G =======================
    % useful coordinates
    \path ([xshift=\MatClearance,yshift=\onelen]M1B) coordinate (M2A);
    \path ([xshift=\flen]M2A) coordinate (M2B);
    \path ([yshift=\Nlen]M2B) coordinate (M2C);
    \path ([xshift=-\flen]M2C) coordinate (M2D);

    % brackets
    \draw[line width=1.5pt] ([xshift=\BrIn,yshift=-\BrCl]M2A) -- ([xshift=-\BrCl,yshift=-\BrCl]M2A) -- ([xshift=-\BrCl,yshift=\BrCl]M2D) -- ([xshift=\BrIn,yshift=\BrCl]M2D); %left bracket
    \draw[line width=1.5pt] ([xshift=-\BrIn,yshift=-\BrCl]M2B) -- ([xshift=\BrCl,yshift=-\BrCl]M2B) -- ([xshift=\BrCl,yshift=\BrCl]M2C) -- ([xshift=-\BrIn,yshift=\BrCl]M2C); %right bracket
    
    % red fill
    \fill[red!50,opacity=0.5] (M2A) rectangle (M2C);

    % drawing red dots
    \foreach \x in {0,...,7} {
    \foreach \y in {0,...,8} {
      \fill[red] ([xshift=(\x+0.5)*\onelen,yshift=(\y+0.5)*\onelen]M2A) circle (1pt);%{(\x+0.5)*\onelen+\Nlen+0.5cm}, {(\y+0.5)*\onelen}
    }}

    % dividers
    \foreach \x in {1,...,7}{\draw[line width=0.1pt] ([xshift=\x*\onelen]M2A) -- ([xshift=\x*\onelen]M2D);}

    % drawing middle brace and matrix
    \coordinate (Brace2L) at ([xshift=-0.5\BrIn]M2A |- Brace1L);
    \coordinate (Brace2R) at ([xshift=0.5\BrIn]M2B |- Brace1L);
    \draw[decorate, decoration={calligraphic brace, amplitude=3pt, mirror, aspect=0.5},line width=1pt] (Brace2L) -- (Brace2R);
    \node (mat1) at ($(Brace2L)!0.5!(Brace2R) + (0,-3pt)$) {};
    \node[below] at ([yshift=-0.75cm]mat1.center) {$\underbrace{
    \begin{bmatrix}
        g_1 & g_2 & \cdots & g_f
    \end{bmatrix}}_{= G}$};
    
    % ======================= drawing equal sign ======================= 
    \path ([xshift=\MatClearance*3/4,yshift=\Nlen/2-0.1cm]M2B) coordinate (EqA);
    \path ([xshift=0.4cm]EqA) coordinate (EqB);
    \path ([yshift=0.2cm]EqB) coordinate (EqC);
    \path ([xshift=-0.4cm]EqC) coordinate (EqD);
    \draw[line width = 1.5 pt] (EqA) -- (EqB);
    \draw[line width = 1.5 pt] (EqD) -- (EqC);

    % ---------------- equation sign below ----------------
    \node[below] at ([xshift=5.65\onelen,yshift=-1.05cm]mat1.center) {$=$};

    % ======================= drawing RHS =======================
    % inside of matrix
    \path ([xshift=\MatClearance*3/4,yshift=-\Nlen/2+0.1cm-\onelen]EqB) coordinate (M3A);
    \path ([xshift=\flen]M3A) coordinate (M3B);
    \path ([yshift=2*\plen+2*\onelen]M3B) coordinate (M3C);
    \path ([xshift=-\flen]M3C) coordinate (M3D);
    % top left black triangle
    \path ([yshift=-\plen-\onelen/2]M3D) coordinate (t1A);
    \path ([xshift=\plen+\onelen/2]M3D) coordinate (t1B);
    % top red trapezoid
    \path ([yshift=-\onelen/2]t1A) coordinate (t2A);
    \path ([xshift=\flen]t2A) coordinate (t2B);
    % bottom black triangle
    \path ([yshift=-\plen-\onelen/2]t2A) coordinate (t3A);
    \path ([xshift=\plen+\onelen/2]t2A) coordinate (t3B);
    % bottom red trapezoid
    \path ([yshift=-\onelen/2]t3A) coordinate (t4A);

    % top 'staircase' coordinates
    \path ([yshift=-\plen]M3D) coordinate (stair1A);
    \path ([xshift=\onelen]stair1A) coordinate (stair1B);
    \path ([yshift=\onelen]stair1B) coordinate (stair1C);
    \path ([xshift=\onelen]stair1C) coordinate (stair1D);
    \path ([yshift=\onelen]stair1D) coordinate (stair1E);
    \path ([xshift=\onelen]stair1E) coordinate (stair1F);
    \path ([yshift=\onelen]stair1F) coordinate (stair1G);
    \path ([xshift=\onelen]stair1G) coordinate (stair1H);
    \path ([yshift=\onelen]stair1H) coordinate (stair1I);

    % bottom 'staircase' coordinates
    \path ([yshift=-\onelen]stair1A) coordinate (stair2J);
    \path ([yshift=-\plen]stair2J) coordinate (stair2A);
    \path ([xshift=\onelen]stair2A) coordinate (stair2B);
    \path ([yshift=\onelen]stair2B) coordinate (stair2C);
    \path ([xshift=\onelen]stair2C) coordinate (stair2D);
    \path ([yshift=\onelen]stair2D) coordinate (stair2E);
    \path ([xshift=\onelen]stair2E) coordinate (stair2F);
    \path ([yshift=\onelen]stair2F) coordinate (stair2G);
    \path ([xshift=\onelen]stair2G) coordinate (stair2H);
    \path ([yshift=\onelen]stair2H) coordinate (stair2I);
    
    % fill figures
    % \fill[black, opacity=0.5]  (t1A) -- (t1B) -- (M3D) -- cycle;% top black
    % \fill[red!50, opacity=0.5] (t2A) -- (t2B) -- (M3C) -- (t1B) -- (t1A) -- cycle;% top red
    % \fill[black, opacity=0.5]  (t3A) -- (t3B) -- (t2A) -- cycle;% bottom black
    % \fill[red!50, opacity=0.5] (t4A) -- (M3B) -- (t2B) -- (t3B) -- (t3A) -- cycle; % bottom red
    \fill[black,opacity=0.5]  (M3D) -- (stair1A) -- (stair1B) -- (stair1C) -- (stair1D) -- (stair1E) -- (stair1F) -- (stair1G) -- (stair1H) -- (stair1I) -- cycle;
    \fill[red!50,opacity=0.5] ([xshift=\flen-\onelen]stair1B) -- (stair1B) -- (stair1C) -- (stair1D) -- (stair1E) -- (stair1F) -- (stair1G) -- (stair1H) -- (stair1I) -- (M3C) -- cycle;
    \fill[black,opacity=0.5]  (stair2J) -- (stair2A) -- (stair2B) -- (stair2C) -- (stair2D) -- (stair2E) -- (stair2F) -- (stair2G) -- (stair2H) -- (stair2I) -- cycle;
    \fill[red!50,opacity=0.5] ([xshift=\flen-\onelen]stair2B) -- (stair2B) -- (stair2C) -- (stair2D) -- (stair2E) -- (stair2F) -- (stair2G) -- (stair2H) -- (stair2I) -- ([yshift=-\plen-\onelen]M3C) -- cycle;
    
    % draw brackets
    \draw[line width=1.5pt] ([xshift=\BrIn,yshift=-\BrCl]M3A) -- ([xshift=-\BrCl,yshift=-\BrCl]M3A) -- ([xshift=-\BrCl,yshift=\BrCl]M3D) -- ([xshift=\BrIn,yshift=\BrCl]M3D); %left bracket
    \draw[line width=1.5pt] ([xshift=-\BrIn,yshift=-\BrCl]M3B) -- ([xshift=\BrCl,yshift=-\BrCl]M3B) -- ([xshift=\BrCl,yshift=\BrCl]M3C) -- ([xshift=-\BrIn,yshift=\BrCl]M3C); %right bracket
    
    % U_{i,p,N}
    \path ([xshift=0.5*\onelen,yshift=-\plen+0.5*\onelen]M3D) coordinate (tlbA);
    \foreach \x in {0,...,7}{
    \foreach \y in {0,...,3}{
        \ifthenelse{{\y>\x}\OR{\y=\x}}{%
        % \fill[black, opacity=0.5] ([xshift={(\x-0.5)*\onelen},yshift={(\y-0.5)*\onelen}]tlbA) rectangle ([xshift={(\x+0.5)*\onelen},yshift={(\y+0.5)*\onelen}]tlbA);
        \fill[black] ([xshift={\x*\onelen},yshift={\y*\onelen}]tlbA) circle (1pt);
        }{%
        % \fill[red!50,opacity=0.5] ([xshift={(\x-0.5)*\onelen},yshift={(\y-0.5)*\onelen}]tlbA) rectangle ([xshift={(\x+0.5)*\onelen},yshift={(\y+0.5)*\onelen}]tlbA);%<do this if false>
        \fill[red] ([xshift={\x*\onelen},yshift={(\y*\onelen}]tlbA) circle (1pt);
        }%
    }}
    
    % U_{i_p,1,N}
    \path ([yshift=-\onelen]tlbA) coordinate (tlbB);
    \fill[red!50,opacity=0.5] (stair1A) rectangle ([xshift=\flen,yshift=-\onelen]stair1A);
    \foreach \x in {0,...,7}{\fill[red] ([xshift=\x*\onelen]tlbB) circle (1pt);}
    
    % Y_{i,p,N}
    \path ([yshift=-\plen]tlbB) coordinate (tlbC);
    \foreach \x in {0,...,7}{
    \foreach \y in {0,...,3}{
        \ifthenelse{{\y>\x}\OR{\y=\x}}{%
        \fill[black] ([xshift={\x*\onelen},yshift={\y*\onelen}]tlbC) circle (1pt);%<do this if true>
        }{%
        \fill[red] ([xshift={\x*\onelen},yshift={(\y*\onelen}]tlbC) circle (1pt);%<do this if false>
        }%
    }}

    % Y_{i_p,1,N}
    \path ([yshift=-\onelen]tlbC) coordinate (tlbD);
    \fill[red!50,opacity=0.5] (stair2A) rectangle ([xshift=\flen,yshift=-\onelen]stair2A);
    \foreach \x in {0,...,7}{\fill[red] ([xshift=\x*\onelen]tlbD) circle (1pt);}

    \foreach \dy in {0,-\plen-\onelen}{%
    % black diagonals
    \draw[dash pattern=on 1pt off 2pt, line width=0.1pt] ([xshift= 1/2*\onelen,yshift=\dy+5/2*\onelen]stair1A) -- ([xshift=-5/2*\onelen,yshift=\dy-1/2*\onelen]stair1I);
    \draw[dash pattern=on 1pt off 2pt, line width=0.1pt] ([xshift= 1/2*\onelen,yshift=\dy+3/2*\onelen]stair1A) -- ([xshift=-3/2*\onelen,yshift=\dy-1/2*\onelen]stair1I);
    \draw[dash pattern=on 1pt off 2pt, line width=0.1pt] ([xshift= 1/2*\onelen,yshift=\dy+1/2*\onelen]stair1A) -- ([xshift=-1/2*\onelen,yshift=\dy-1/2*\onelen]stair1I);
    % red diagonals
    \draw[red,dash pattern=on 1pt off 2pt, line width=0.1pt] ([xshift= 1/2*\onelen,yshift=\dy-1/2*\onelen]stair1A) -- ([xshift=1/2*\onelen,yshift=\dy-1/2*\onelen]stair1I);
    \draw[red,dash pattern=on 1pt off 2pt, line width=0.1pt] ([xshift= 3/2*\onelen,yshift=\dy-1/2*\onelen]stair1A) -- ([xshift=3/2*\onelen,yshift=\dy-1/2*\onelen]stair1I);
    \draw[red,dash pattern=on 1pt off 2pt, line width=0.1pt] ([xshift= 5/2*\onelen,yshift=\dy-1/2*\onelen]stair1A) -- ([xshift=5/2*\onelen,yshift=\dy-1/2*\onelen]stair1I);
    \draw[red,dash pattern=on 1pt off 2pt, line width=0.1pt] ([xshift= 7/2*\onelen,yshift=\dy-1/2*\onelen]stair1A) -- ([xshift=7/2*\onelen,yshift=\dy-1/2*\onelen]stair1I);
    \draw[red,dash pattern=on 1pt off 2pt, line width=0.1pt] ([xshift= 9/2*\onelen,yshift=\dy-1/2*\onelen]stair1A) -- ([xshift=7/2*\onelen,yshift=\dy-3/2*\onelen]stair1I);
    \draw[red,dash pattern=on 1pt off 2pt, line width=0.1pt] ([xshift=11/2*\onelen,yshift=\dy-1/2*\onelen]stair1A) -- ([xshift=7/2*\onelen,yshift=\dy-5/2*\onelen]stair1I);
    \draw[red,dash pattern=on 1pt off 2pt, line width=0.1pt] ([xshift=13/2*\onelen,yshift=\dy-1/2*\onelen]stair1A) -- ([xshift=7/2*\onelen,yshift=\dy-7/2*\onelen]stair1I);
    }
    
    % draw dividers
    \draw[line width=0.1pt] ([yshift=-\plen-\onelen]M3D) -- ([yshift=-\plen-\onelen]M3C);
    \draw[line width=0.1pt] ([yshift=-\plen]M3D) -- ([yshift=-\plen]M3C);
    \draw[line width=1pt] ([yshift=\onelen]M3A) -- ([yshift=\onelen]M3B);

    % ---------------- drawing right brace and matrix ----------------
    % brace below matrix
    \path ([xshift=-0.5\BrIn,yshift=-2\BrCl]M3A) coordinate (Brace3L);
    \path ([xshift=0.5\BrIn,yshift=-2\BrCl]M3B) coordinate (Brace3R);
    \draw[decorate, decoration={calligraphic brace, amplitude=3pt, mirror, aspect=0.25},line width=1pt] (Brace3L) -- (Brace3R);
    % matrix below brace
    \node (mat3) at ($(Brace3L)!0.25!(Brace3R) + (0,-3pt)$) {};
    \node[below] at (mat3.center) {$\begin{bmatrix}
        U_{\hat{i},p,f}\\U_{\hat{i}_p,1,f}\\Y_{\hat{i},p,f}\\ \hline \widehat{Y}_{\hat{i}_p,1,f}
    \end{bmatrix}$};
    % tag & label (on RHS of column)
    \node[below] at ([xshift=2.13cm,yshift=-0.9cm]mat3.center) {$\refstepcounter{equation}(\theequation)\label{eq:CL_DeePC_no_IVs}$};
    % brace to the right
    % \path ([xshift=0.7cm+0.05cm]mat3.center |- Brace1T) coordinate (Brace3T);%([xshift=0.7cm,yshift=-4pt]mat3.center) coordinate (Brace3T);
    % \path (Brace3T |- Brace1B) coordinate (Brace3B);
    % \draw[pen colour=gray!60,decorate, decoration={calligraphic brace, amplitude=3pt, aspect=0.2},line width=0.75pt] (Brace3T) -- (Brace3B);
    % % Zf
    % \node (Zf) at ($(Brace3T)!0.2!(Brace3B) + (3pt,-1.5pt)$) {};
    % \node[right,text=gray!100] at ([yshift=-0.18cm]Zf.center) {\scriptsize{\shortstack{$=$\\$\Phi_{\hat{i},1,f}$}}};
    
    % ======================= length indicators ======================= 
    \draw[|-|] ([yshift=\onelen]M1D) -- node[above] {\scriptsize$N$} ([yshift=\onelen]M1C);
    \draw[|-|] ([yshift=\onelen]M2D) -- node[above] {\scriptsize$f$} ([yshift=\onelen]M2C);
    \draw[|-|] ([yshift=\onelen]M3D) -- node[above] {\scriptsize$f$} ([yshift=\onelen]M3C);
    \draw[|-|] ([xshift=\onelen]M3C) -- node[right] {\scriptsize$pr$} ([xshift=\onelen,yshift=\onelen]t2B);
    \draw[|-|] ([xshift=\onelen,yshift=\onelen]t2B) -- node[right] {\scriptsize$r$} ([xshift=\onelen]t2B);
    \draw[|-|] ([xshift=\onelen]t2B) -- node[right] {\scriptsize$pl$} ([xshift=\onelen,yshift=\onelen]M3B);
    \draw[|-|] ([xshift=\onelen,yshift=\onelen]M3B) -- node[right] {\scriptsize$l$} ([xshift=\onelen]M3B);
\end{tikzpicture}
\caption{Visualization of known (black) and unknown (red) variables in \ac{CL-DeePC} without \ac{IVs}. Each dot represents an input $u_k\in\mathbb{R}^r$, output $y_k\in\mathbb{R}^l$, or element of the matrix $G$. \ac{CL-DeePC} involves $f$ sequential applications of a step-ahead predictor obtained from regular \ac{DeePC} with $f=1$ (see also Fig.~\ref{fig:regular-DeePC}), resulting in the dashed block-anti diagonals with the same $u_k$ or $y_k$ on the right hand side.}
\label{fig:CL-DeePC}
\end{figure}
\begin{figure}[b!]
\centering
\begin{tikzpicture}
    % defining constants
    \def\stepSize{0.25}
    \def\Nnum{9}
    \def\fnum{8}
    \def\pnum{4}
    
    % Defining lengths
    % \newlength{\onelen}
    \setlength{\onelen}{\stepSize cm}
    % \newlength{\BrCl}
    \setlength{\BrCl}{0.075cm}
    % \newlength{\BrIn}
    \setlength{\BrIn}{0.15cm}
    % \newlength{\plen}
    \setlength{\plen}{1cm}%{\pnum\stepSize cm}
    % \newlength{\flen}
    \setlength{\flen}{2cm}%{\fnum*\stepSize cm}
    % \newlength{\Nlen}
    \setlength{\Nlen}{2.25cm}%{\Nnum\stepSize cm}%should be 2*p+one
    % \newlength{\MatClearance}
    \setlength{\MatClearance}{0.3cm}
    
    % grid lines for guidance
    % \draw[gray,step=0.5] (-0,-7) grid (8,3);

    % ======================= drawing data matrix =======================
    \path (0,2\plen+\onelen) coordinate (M1D);
    \path ([yshift=-2\plen-2\flen]M1D) coordinate (M1A);
    \path ([xshift=\Nlen]M1A) coordinate (M1B);
    \path ([yshift=2*\plen+2*\flen]M1B) coordinate (M1C);
    \draw[line width=1.5pt] ([xshift=\BrIn,yshift=-\BrCl]M1A) -- ([xshift=-\BrCl,yshift=-\BrCl]M1A) -- ([xshift=-\BrCl,yshift=\BrCl]M1D) -- ([xshift=\BrIn,yshift=\BrCl]M1D); %left bracket
    \draw[line width=1.5pt] ([xshift=-\BrIn,yshift=-\BrCl]M1B) -- ([xshift=\BrCl,yshift=-\BrCl]M1B) -- ([xshift=\BrCl,yshift=\BrCl]M1C) -- ([xshift=-\BrIn,yshift=\BrCl]M1C); %right bracket
    \draw[line width=1pt] ([yshift=\flen]M1A) -- ([yshift=\flen]M1B); % dividing matrix into blocks
    \fill[black, opacity=0.5] (M1A) rectangle (M1C);
    \foreach \x in {0,...,8} { % drawing black dots
    \foreach \y in {-15,...,8} {
      \fill ( {(\x+0.5)*\onelen}, {(\y+0.5)*\onelen} ) circle (1pt);
    }}
    \draw[line width=0.1pt] ([yshift=-\plen-\flen]M1D) -- ([yshift=-\plen-\flen]M1C);
    \draw[line width=0.1pt] ([yshift=-\plen]M1D) -- ([yshift=-\plen]M1C);

    % coordinates for diagonals
    \path ([yshift=-\plen]M1D) coordinate (M1stair1A);
    \path ([xshift=\Nlen]M1D) coordinate (M1stair1I);
    % black diagonals
    % \foreach \i in {0,-1}{
    % \foreach \dy in {1,...,18}{
    % \ifthenelse{\dy<9.5}{
    % % This code will be executed if \dy<9.5
    %     \draw[dash pattern=on 1pt off 2pt, line width=0.1pt] ([xshift= 0.5*\onelen,yshift=\i*(\plen+\flen)-(\dy+0.5)*\onelen]M1D) -- ([xshift= (\dy+0.5)*\onelen,yshift=\i*(\plen+\flen)-0.5*\onelen]M1D);
    % }{
    % % This code will be executed if \dy>=10
    %     \ifthenelse{\dy<12.5}{
    %         % This code will be executed if \dy<=12
    %         \draw[dash pattern=on 1pt off 2pt, line width=0.1pt] ([xshift= 0.5*\onelen,yshift=\i*(\plen+\flen)-(\dy+0.5)*\onelen]M1D) -- ([xshift= -0.5*\onelen,yshift=\i*(\plen+\flen)-(\dy-7.5)*\onelen]M1C);
    %         }{
    %         % This code will be executed if \dy>=13
    %         \draw[dash pattern=on 1pt off 2pt, line width=0.1pt] ([xshift= (\dy-10.5)*\onelen,yshift=\i*(\plen+\flen)-\plen-\flen+0.5*\onelen]M1D) -- ([xshift= -0.5*\onelen,yshift=\i*(\plen+\flen)-(\dy-7.5)*\onelen]M1C);
    %     }
    % }
    % }}
    
    % ---------------- drawing left brace and matrix ----------------
    \path ([xshift=-0.5\BrIn,yshift=-2\BrCl]M1A) coordinate (Brace1L);
    \path ([xshift=0.5\BrIn,yshift=-2\BrCl]M1B) coordinate (Brace1R);
    \draw[decorate, decoration={calligraphic brace, amplitude=3pt, mirror, aspect=0.75},line width=1pt] (Brace1L) -- (Brace1R);
    \node (mat1) at ($(Brace1L)!0.75!(Brace1R) + (0,-3pt)$) {};
    \node[below] at (mat1.center) {$\begin{bmatrix}
        U_{i,p,N}\\U_{i_p,f,N}\\Y_{i,p,N}\\ \hline Y_{i_p,f,N}
    \end{bmatrix}$};
    % brace to the left
    % \path ([xshift=-0.75cm-0.05cm,yshift=-4pt]mat1.center) coordinate (Brace1T);
    % \path ([yshift=-1.73cm]Brace1T) coordinate (Brace1B);
    % \draw[pen colour=gray!60,decorate, decoration={calligraphic brace, amplitude=3pt, mirror, aspect=0.2},line width=0.75pt] (Brace1T) -- (Brace1B);
    % % Zp
    % \node (Zp) at ($(Brace1T)!0.2!(Brace1B) + (-3pt,-0.8pt)$) {};
    % \node[left,text=gray!100,align=center] at ([xshift=0.5mm,yshift=-0.18cm]Zp.center) {\scriptsize{\shortstack{$=$\\$\Phi_{i,f,N}$}}};
  
    % ======================= drawing G =======================
    % useful coordinates
    \path ([xshift=\MatClearance,yshift=-0.5\Nlen-\plen-\flen]M1C) coordinate (M2A);
    \path ([xshift=\onelen]M2A) coordinate (M2B);
    \path ([yshift=\Nlen]M2B) coordinate (M2C);
    \path ([xshift=-\onelen]M2C) coordinate (M2D);

    % brackets
    \draw[line width=1.5pt] ([xshift=0.5\BrIn,yshift=-\BrCl]M2A) -- ([xshift=-\BrCl,yshift=-\BrCl]M2A) -- ([xshift=-\BrCl,yshift=\BrCl]M2D) -- ([xshift=0.5\BrIn,yshift=\BrCl]M2D); %left bracket
    \draw[line width=1.5pt] ([xshift=-0.5\BrIn,yshift=-\BrCl]M2B) -- ([xshift=\BrCl,yshift=-\BrCl]M2B) -- ([xshift=\BrCl,yshift=\BrCl]M2C) -- ([xshift=-0.5\BrIn,yshift=\BrCl]M2C); %right bracket
    
    % red fill
    \fill[red!50,opacity=0.5] (M2A) rectangle (M2C);

    % drawing red dots
    \foreach \x in {0} {
    \foreach \y in {0,...,8} {
      \fill[red] ([xshift=(\x+0.5)*\onelen,yshift=(\y+0.5)*\onelen]M2A) circle (1pt);
    }}

    % drawing middle brace and matrix
    \coordinate (Brace2L) at ([xshift=-0.5\BrIn]M2A |- Brace1L);
    \coordinate (Brace2R) at ([xshift=0.5\BrIn]M2B |- Brace1L);
    \draw[decorate, decoration={calligraphic brace, amplitude=3pt, mirror, aspect=0.5},line width=1pt] (Brace2L) -- (Brace2R);
    \node (mat1) at ($(Brace2L)!0.5!(Brace2R) + (0,-3pt)$) {};
    \node[below] at ([yshift=-1.05cm]mat1.center) {$g$};
    
    % ======================= drawing equal sign ======================= 
    \path ([xshift=\MatClearance*3/4,yshift=\Nlen/2-0.1cm]M2B) coordinate (EqA);
    \path ([xshift=0.4cm]EqA) coordinate (EqB);
    \path ([yshift=0.2cm]EqB) coordinate (EqC);
    \path ([xshift=-0.4cm]EqC) coordinate (EqD);
    \draw[line width = 1.5 pt] (EqA) -- (EqB);
    \draw[line width = 1.5 pt] (EqD) -- (EqC);

    % ---------------- equation sign below ----------------
    \node[below] at ([xshift=2\onelen,yshift=-1.05cm]mat1.center) {$=$};

    % ======================= drawing RHS =======================
    % inside of matrix
    \path ([xshift=0.75\MatClearance]EqB |- M1A) coordinate (M3A);
    \path ([xshift=\onelen]M3A) coordinate (M3B);
    \path ([yshift=2\plen+2\flen]M3B) coordinate (M3C);
    \path ([xshift=-\onelen]M3C) coordinate (M3D);
    
    % fill figures
    \fill[black,opacity=0.5]  (M3D) -- ([yshift=-\plen]M3D) -- ([yshift=-\plen]M3C) -- (M3C) -- cycle;
    \fill[red!50,opacity=0.5] ([yshift=-\plen]M3D) -- ([yshift=-\plen-\flen]M3D) -- ([yshift=-\plen-\flen]M3C) -- ([yshift=-\plen]M3C) -- cycle;
    \fill[black,opacity=0.5] ([yshift=\flen]M3A) -- ([yshift=\flen]M3B) -- ([yshift=\flen+\plen]M3B) -- ([yshift=\flen+\plen]M3A) -- cycle;
    \fill[red!50,opacity=0.5]  (M3A) -- (M3B) -- ([yshift=\flen]M3B) -- ([yshift=\flen]M3A) -- cycle;
    
    % draw brackets
    \draw[line width=1.5pt] ([xshift=0.5\BrIn,yshift=-\BrCl]M3A) -- ([xshift=-\BrCl,yshift=-\BrCl]M3A) -- ([xshift=-\BrCl,yshift=\BrCl]M3D) -- ([xshift=0.5\BrIn,yshift=\BrCl]M3D); %left bracket
    \draw[line width=1.5pt] ([xshift=-0.5\BrIn,yshift=-\BrCl]M3B) -- ([xshift=\BrCl,yshift=-\BrCl]M3B) -- ([xshift=\BrCl,yshift=\BrCl]M3C) -- ([xshift=-0.5\BrIn,yshift=\BrCl]M3C); %right bracket
    
    % U_{i,p,1}
    \foreach \dy in {0,...,-3}{
        \fill[black] ([xshift=0.5\onelen,yshift=(\dy-0.5)*\onelen]M3D) circle (1pt);
    }

    % U_{i_p,f,1}
    \foreach \dy in {-4,...,-11}{
        \fill[red] ([xshift=0.5\onelen,yshift=(\dy-0.5)*\onelen]M3D) circle (1pt);
    }
    
    % Y_{i,p,1}
    \foreach \dy in {0,...,-3}{
        \fill[black] ([xshift=0.5\onelen,yshift=(\dy-0.5)*\onelen-\plen-\flen]M3D) circle (1pt);
    }

    % Y_{i_p,f,1}
    \foreach \dy in {-4,...,-11}{
        \fill[red] ([xshift=0.5\onelen,yshift=(\dy-0.5)*\onelen-\plen-\flen]M3D) circle (1pt);
    }

    % draw dividers
    \draw[line width=0.1pt] ([yshift=-\plen]M3D) -- ([yshift=-\plen]M3C);
    \draw[line width=0.1pt] ([yshift=-\plen-\flen]M3D) -- ([yshift=-\plen-\flen]M3C);
    \draw[line width=1pt] ([yshift=\flen]M3A) -- ([yshift=\flen]M3B);

    % ---------------- drawing right brace and matrix ----------------
    % brace below matrix
    \path ([xshift=-0.5\BrIn,yshift=-2\BrCl]M3A) coordinate (Brace3L);
    \path ([xshift=0.5\BrIn,yshift=-2\BrCl]M3B) coordinate (Brace3R);
    \draw[decorate, decoration={calligraphic brace, amplitude=3pt, mirror, aspect=0.5},line width=1pt] (Brace3L) -- (Brace3R);
    % matrix below brace
    \node (mat3) at ($(Brace3L)!0.5!(Brace3R) + (0,-3pt)$) {};
    \node[below] at ([xshift=0.35cm]mat3.center) {$\begin{bmatrix}
        U_{\hat{i},p,1}\\U_{\hat{i}_p,f,1}\\Y_{\hat{i},p,1}\\ \hline \widehat{Y}_{\hat{i}_p,f,1}
    \end{bmatrix}$};
    % tag & label (on RHS of column)
    \node[below left] at (8.35cm,-5cm) {$\refstepcounter{equation}(\theequation)\label{eq:regular_DeePC_no_IVs}$};
    % brace to the right
    % \path ([xshift=1.1cm]mat3.center |- Brace1T) coordinate (Brace3T);%([xshift=0.7cm,yshift=-4pt]mat3.center) coordinate (Brace3T);
    % \path (Brace3T |- Brace1B) coordinate (Brace3B);
    % \draw[pen colour=gray!60,decorate, decoration={calligraphic brace, amplitude=3pt, aspect=0.2},line width=0.75pt] (Brace3T) -- (Brace3B);
    % % Zf
    % \node (Zf) at ($(Brace3T)!0.2!(Brace3B) + (3pt,-1.5pt)$) {};
    % \node[right,text=gray!100] at ([yshift=-0.18cm]Zf.center) {\scriptsize{\shortstack{$=$\\$\Phi_{\hat{i},f,1}$}}};
    
    % ======================= length indicators ======================= 
    \draw[|-|] ([yshift=\onelen]M1D) -- node[above] {\scriptsize$N$} ([yshift=\onelen]M1C);
    \draw[|-|] ([xshift=\onelen]M3C) -- node[right] {\scriptsize$pr$} ([xshift=\onelen,yshift=-\plen]M3C);
    \draw[|-|] ([xshift=\onelen,yshift=-\plen]M3C) -- node[right] {\scriptsize$fr$} ([xshift=\onelen,yshift=-\plen-\flen]M3C);
    \draw[|-|] ([xshift=\onelen,yshift=-\plen-\flen]M3C) -- node[right] {\scriptsize$pl$} ([xshift=\onelen,yshift=\flen]M3B);
    \draw[|-|] ([xshift=\onelen,yshift=\flen]M3B) -- node[right] {\scriptsize$fl$} ([xshift=\onelen]M3B);
\end{tikzpicture}
\caption{Visualization of known (black) and unknown (red) variables in \ac{DeePC} without \ac{IVs}. Each dot represents an input $u_k\in\mathbb{R}^r$, output $y_k\in\mathbb{R}^l$, or element of the matrix $G$. A multi-step ahead predictor of prediction length $f$ is formed directly by taking a linear combination of past input and output data.\\\vspace{0.75mm}}
\label{fig:regular-DeePC}
\end{figure}
%
\setcounter{thm}{0}
\begin{thm}\label{theorem:main_result}
    Consider the minimal discrete non-deterministic \ac{LTI} system given by~\eqref{eqn:SS_innovation} to generate input-output data in closed-loop with a strictly causal controller. Define data matrices $\Phi_{i,1,N}$ and $\overline{\Phi}_{\hat{i},1,f}$ as in \eqref{eq:Phi_def}. %
    %
    If the joint input and noise sequences are sufficiently persistently exciting such that $\left[X_{i,1,N}^\top \; U_{i,p,N}^\top \; U_{i_p,1,N}^\top \; E_{i,p,N}^\top\right]^\top$ is full row rank %
    %If the input sequence $\{u_k\}_{k=i}^{i+\bar{N}-1}$ of length $\bar{N}=p+s+N-1$ %, with $N\geq(p+s+n)(r+l)+n$ and $p\geq\ell$\todo{don't forget},
    % is persistently exciting of order $p+s+n$, and has sample correlations such that%
    % \begin{alignat}{2}%see also https://www.cis.upenn.edu/~jean/schur-comp.pdf
    % % \widehat{\Sigma}_{u,u} &> 0,\label{eq:PE_corU}\\
    % &\widehat{\Sigma}_{\mathrm{ee}} - \widehat{\Sigma}_{\mathrm{ue}^\top} \widehat{\Sigma}_{\mathrm{uu}}\inv \widehat{\Sigma}_{\mathrm{ue}}\succ0,\span\span\label{eq:PE_corUE2}\\
    % &&\text{with}\quad\widehat{\Sigma}_{\mathrm{ee}}&=E_{i,p+s+n,N-n}E_{i,p+s+n,N-n}^\top,\notag\\
    % &&\widehat{\Sigma}_{\mathrm{ue}}&=U_{i,p+s+n,N-n}E_{i,p+s+n,N-n}^\top,\notag\\
    % &&\widehat{\Sigma}_{\mathrm{uu}}&=U_{i,p+s+n,N-n}U_{i,p+s+n,N-n}^\top,\notag
    % \end{alignat}
    then \\
    $\mathrm{(i)}$ $\exists G\in\mathbb{R}^{N\times f}$ such that
    \begin{align}\tag{\ref{eq:CL_DeePC_no_IVs}}%\label{eq:Theorem1}
        \begin{bmatrix}
            \Phi_{i,1,N}\\Y_{i_p,1,N}
        \end{bmatrix}G =
        \begin{bmatrix}
            \overline{\Phi}_{\hat{i},1,f}\\\widehat{Y}_{\hat{i}_p,1,f}
        \end{bmatrix},
    \end{align}
    $\mathrm{(ii)}$ and with $\widehat{Y}_{\hat{i}_p,1,f}$ asymptotically unbiased as $p\rightarrow\infty$.
\end{thm}

% ---------------- Proof of (i) ----------------
\noindent\textbf{Proof of $(\mathrm{i})$:} Note from \eqref{eq:CL_DeePC_no_IVs} that $G$ is determined by $\Phi_{i,1,N}G=\overline{\Phi}_{\hat{i},1,f}$ since $\widehat{Y}_{\hat{i}_p,1,f}$ is unknown a priori. There exists at least one solution to this equation if $\Phi_{i,1,N}$ is full row rank. Decomposing $\Phi_{i,1,N}$ into its dependencies begets
\begin{align}\label{eq:Phi_isN}
    \mkern-8mu\Phi_{i,1,N}=\mkern-1mu\begin{bmatrix}
        U_{i,p,N}\\
        U_{i_p,1,N}\\
        Y_{i,p,N}
    \end{bmatrix}\mkern-1mu=\mkern-1mu
    \underbrace{\begin{bmatrix}
        0        & I_{pr} & 0      & 0\\
        0        & 0      & I_{r} & 0\\
        \Gamma_p & \mathcal{T}_p^\mathrm{u} & 0 & \mathcal{H}_p
    \end{bmatrix}}\mkern-3mu\begin{bmatrix}
        X_{i,1,N}\\
        U_{i,p,N}\\
        U_{i_p,1,N}\\
        E_{i,p,N}
    \end{bmatrix}\mkern-1mu,
\end{align}
in which the matrix on the right hand side is assumed to be full row rank, and the underbraced matrix is also full row rank. Since both of these matrices are full row rank, so is its product $\Phi_{i,1,N}$, meaning that there exists at least one $G$ that satisfies \eqref{eq:CL_DeePC_no_IVs}. $\hfill\qed$

\noindent\textbf{Remark 1:} For the matrix on the right hand side of \eqref{eq:Phi_isN} to be full row rank a necessary condition is that $N\geq n+(p+1)r+pl$.

\noindent\textbf{Remark 2:} The full row rank matrix $\Phi_{i,1,N}\in\mathbb{R}^{((p+1)r+pl)\times N}$ has more columns than rows, meaning that in fact there are an infinite number of possibilities for $G$ that satisfy \eqref{eq:CL_DeePC_no_IVs}. These possibilities are given by
\begin{align}\label{eq:G_sols}
    G = \Phi_{i,1,N}^\dagger\overline{\Phi}_{\hat{i},1,f} + \Pi_{\Phi_{i,1,N}}^\bot W,
\end{align}
in which the dagger $\dagger$ denotes the right inverse ($\mathcal{Q}^\dagger=\mathcal{Q}^\top(\mathcal{Q}\mathcal{Q}^\top)\inv$ with $\mathcal{Q}$ as a real, full row rank matrix), $\Pi_{\Phi_{i,1,N}}^\bot=I_N-\Phi_{i,1,N}^\dagger\Phi_{i,1,N}$ is a projection matrix onto the orthogonal complement of the row space of $\Phi_{i,1,N}$, %see Overschee1996, pg. 19
and $W\in\mathbb{R}^{N\times f}$ is a matrix of decision variables. If no form of regularization is applied, this $W$ does not affect the cost function and is free\todo{uniform random?}.

% ---------------- Proof of (ii) ----------------
\noindent\textbf{Proof of $(\mathrm{ii})$:} Equation \eqref{eq:CL_DeePC_no_IVs} stipulates an output predictor as $\widehat{Y}_{\hat{i}_p,1,f}=Y_{i_p,1,N}G$. Consider the error of this prediction, which using \eqref{eq:DataEq1} is written as
\begin{align}\label{eq:Yf_error1}
    \begin{split}
        &\!\!\!\widehat{Y}_{\hat{i}_p,1,f}-Y_{\hat{i}_p,1,f} = \Gamma_1 \tilde{A}^p (\underbrace{X_{i,1,N}G}_{=\widehat{X}_{\hat{i},1,f}}-X_{\hat{i},1,f}) \\
        &+L_1(\underbrace{\Phi_{i,1,N}G}_{=\overline{\Phi}_{\hat{i},1,f}}-\Phi_{\hat{i},1,f}) +\mathcal{H}_1 (\underbrace{E_{i_p,1,N}G}_{=\widehat{E}_{\hat{i}_p,1,f}}-E_{\hat{i}_p,1,f}),
    \end{split}
\end{align}
in which interpretations of underbraced terms are indicated. Applying the limit $p\rightarrow\infty$ asymptotically attenuates the top contribution by the error of the initial state estimates since, by the definition of $K$ in \secref{sec:sys_model}, $\tilde{A}$ has all of its eigenvalues strictly inside the unit circle. In addition, substituting $G$ from \eqref{eq:G_sols} and cancelling equal terms in $\overline{\Phi}_{\hat{i},1,f}$ and $\Phi_{\hat{i},1,f}$ results in
\begin{alignat}{2}
        &\!\!\!\!\lim_{p\rightarrow\infty} \widehat{Y}_{\hat{i}_p,1,f}-Y_{\hat{i}_p,1,f} = \lim_{p\rightarrow\infty}\Gamma_1\tKp{y}\left(\overline{Y}_{\hat{i},p,f}-Y_{\hat{i},p,f}\right)\notag\\
        &+\mathcal{H}_1\Big(E_{i_p,1,N} W-E_{\hat{i}_p,1,f}\label{eq:Yf_error2}\\
        &+\underbrace{E_{i_p,1,N}\Phi_{i,1,N}^\top}(\Phi_{i,1,N}\Phi_{i,1,N}^\top)\inv(\overline{\Phi}_{\hat{i},1,f}-\Phi_{i,1,N}W)\Big)\notag.
\end{alignat}%
In the above formulation, $W$ is a free matrix that parameterizes $G$. As such, for the predictor to be unbiased, the expectation ($\mathbb{E}[\cdot]$) of this error w.r.t. the noise and conditioned on $W$ must be zero. \todo{$\mathbb{E}[\cdot]$,\\$EW$,\\$\hat{\Sigma}\hat{\Sigma}\inv$?} This expectation renders the second row zero, but not the bottom row because of the underbraced term. Due to feedback, there is a correlation between past inputs and preceding noise such that the expectation of the sample correlation
\begin{align*}
    \begin{split}
        &E_{i_p,1,N}\Phi_{i,1,N}^\top = \\
        &\;\frac{1}{N}\;\sum\limits_{k=i_p}^{\mathclap{i_p+N-1}} e_k \begin{bmatrix}u_{k-p}^\top & \cdots & u_{k-1}^\top & u_k^\top & y_{k-p}^\top & \cdots & y_{k-1}^\top \end{bmatrix}
    \end{split}
\end{align*}
which is contained in the underbraced term in \eqref{eq:Yf_error2}, is nonzero. Due to feedback $\mathbb{E}[e_j u_k^\top]\neq 0,\;\forall k>j$. However, if, as assumed by Theorem~\ref{theorem:main_result}, the controller has no direct feedthrough, then $\mathbb{E}[e_k u_k^\top]= 0,\;\forall k$. \todo{in lim N?}%or is this only true in the limit N->infty for above. Need for quasi-stationarity here
Hence, if ${s=1}$, the expected value of the correlation $E_{i_p,s,N}U_{i_p,s,N}^\top$ reduces to zero, rendering the expectation of the bottom row of \eqref{eq:Yf_error2} zero.

Given the structure of $\overline{\Phi}_{\hat{i},1,f}$ and $\overline{Y}_{\hat{i},p,f}$ shown in Fig.~\ref{fig:CL-DeePC}, consider \eqref{eq:Yf_error2} column by column for $s=1$. Taking the expectation leaves
\begin{align}\label{eq:Yf_error3}
\begin{split}
    &\mkern-3mu\lim_{p\rightarrow\infty} \mathbb{E}\left[\hat{y}_{\hat{i}_p+k}-y_{\hat{i}_p+k}\right] = \\ &\;\;\;\lim_{p\rightarrow\infty}\Gamma_1\tKp{y}\mathbb{E}\left[\datavec{\overline{y}}{\hat{i}+k,p}-\datavec{y}{\hat{i}+k,p}\right],\forall k\in[0,f-1].
\end{split}
\end{align}
By \eqref{eq:Yf_error3} in the limit $p\rightarrow\infty$ the prediction $\hat{y}_{\hat{i}_p+k}$ is unbiased if the $p$ preceding output estimates are unbiased. For $k=0$, this is the case because none of the relevant preceding output data is estimated ($\datavec{\overline{y}}{\hat{i},p}=\datavec{y}{\hat{i},p}$). No bias is thereby introduced on the right hand side for $k=1$ such that $\hat{y}_{\hat{i}_p+1}$ is also unbiased. Repetition of this process until $k=f-1$ demonstrates that $\widehat{Y}_{\hat{i}_p,1,f}$ is indeed an asymptotically unbiased predictor in the limit $p\rightarrow\infty$. This concludes the proof of $\mathrm{(ii)}$. $\hfill\qed$

\subsection{Systematic noise mitigation using \ac{IVs}}
The previous section demonstrated that the predictor that is obtained from \eqref{eq:CL_DeePC_no_IVs} is asymptotically unbiased. This section demonstrates the use of \ac{IVs} as a means to decrease the variance of this predictor.

As shown by \eqref{eq:Yf_error1}, the columns of $G$ take linear combinations of the past noise $E_{i_p,s,N}$, increasing the variance of the implicitly estimated noise in $\widehat{E}_{\hat{i}_p,s,f}$. The idea is to redefine $G$ such that its columns are orthogonal to the noise. The following theorem demonstrates the use of an \ac{IV} $\mathcal{Z}$ for this purpose based on the same assumptions as Theorem~\ref{theorem:main_result}.

\begin{thm}
    Consider the minimal discrete non-deterministic \ac{LTI} system given by~\eqref{eqn:SS_innovation} to generate input-output data in closed-loop with a strictly causal controller. Define data matrices $\Phi_{i,s,N}$ and $\overline{\Phi}_{\hat{i},s,f}$ as in \eqref{eq:Phi_def}. %
    %
    If the joint input and noise sequences are sufficiently persistently exciting such that $\left[X_{i,1,N}^\top \; U_{i,p,N}^\top \; U_{i_p,s,N}^\top \; E_{i,p,N}^\top\right]^\top$ is full row rank, and with the choice of \ac{IV} given by $\mathcal{Z}=\Phi_{i,s,N}$ then\\
    $\mathrm{(i)}$ $\exists G\in\mathbb{R}^{((p+s)r+pl)\times f}$ such that
    \begin{align}\label{eq:Theorem2}
        \begin{bmatrix}
            \Phi_{i,s,N}\\Y_{i_p,s,N}
        \end{bmatrix}\mathcal{Z}^\top G =
        \begin{bmatrix}
            \overline{\Phi}_{\hat{i},s,f}\\\widehat{Y}_{\hat{i}_p,s,f}^\mathrm{IV}
        \end{bmatrix},
    \end{align}
    $\mathrm{(ii)}$ and which if $s=1$ provides an asymptotically unbiased predictor $\widehat{Y}_{\hat{i}_p,s,f}^\mathrm{IV}$ with $p\rightarrow\infty$\\
    $\mathrm{(iii)}$ that has a variance that is smaller than or equal to the variance of $\widehat{Y}_{\hat{i}_p,s,f}$.
\end{thm}
%%%%%%%%%%%%%%%%%%%%%%%%%%%%%%%%%%%%%%%%%%%%%%%%%%%%%%%%%%%%%%%%%%%%%%%%%%%%%%%%%%%%%%%%%%%%%%%%%%%%%%%%%%%%%%%%%%%%%%%%%%%%%
%%%%%%%%%%%%%%%%%%%%%%%%%%%%%%%%%%%%%%%%%%%%%%%%%%%%%%%%%%%%%%%%%%%%%%%%%%%%%%%%%%%%%%%%%%%%%%%%%%%%%%%%%%%%%%%%%%%%%%%%%%%%%
\begin{align*}
    \hline
\end{align*}
%
The proof of Theorem~\ref{theorem:main_result} is deferred till after the treatment of several auxiliary results.
\subsection{Auxiliary results}\label{sec:aux_results}
For the development of sufficient conditions for persistency of excitation, first consider the following result for deterministic systems.
\begin{lem}\citep[Cor.~2(iii)]{Willems2005}\label{lem:D_det_full_row_rank}
    If the input sequence $\{u_k\}_{k=i}^{i+\epsilon+q-2}$ of a controllable discrete \ac{LTI} system without noise is persistently exciting of order $\epsilon+n$ then the matrix $\left[X_{i,1,q}^\top\;U_{i,\epsilon,q}^\top\right]^\top$ is full row rank.
\end{lem}
Lemma~\ref{lem:D_det_full_row_rank} can be extended to non-deterministic systems as shown by the following lemma.
% \setcounter{thm}{0}
\begin{lem}\label{lem:D_full_row_rank}
    If for a controllable non-deterministic \ac{LTI} system of the form given by \eqref{eqn:SS_innovation} the sequence of inputs and noise $\{[u_k^\top\;e_k^\top]^\top\}_{k=i}^{i+\epsilon+q-2}$ is persistently exciting of order $\epsilon+n$ then the matrix $\left[X_{i,1,q}^\top\;U_{i,\epsilon,q}^\top\;E_{i,\epsilon,q}^\top\right]^\top$ is full row rank.
\end{lem}
\textbf{Proof:} Lemma~\ref{lem:D_full_row_rank} follows from Lemma~\ref{lem:D_det_full_row_rank} by extending the exogenous inputs to include the innovation noise, thus also requiring controllable $(A,[B\,K])$. This latter condition is satisfied if $(A,B)$ is controllable.$\hfill\qed$

Since the closed-loop identification problem arises due to correlation between inputs and noise the persistency of excitation condition in Lemma~\ref{lem:D_full_row_rank} is rewritten in terms of such correlations. For this the following lemma concerning Schur complements is used.
\begin{lem}\citep[Lem.~2.7(i)]{Verhaegen2007a}\label{lem:Schur_comp}
    Let $S\in\mathbb{R}^{(\delta+\kappa)\times(\delta+\kappa)}$ be the symmetric matrix
    \begin{align*}
        S=\begin{bmatrix}
            \mathcal{A} & \mathcal{B}\\
            \mathcal{B}^\top & \mathcal{C}
        \end{bmatrix},
    \end{align*}
    with $\mathcal{A}\in\mathbb{R}^{\delta \times \delta}$, $\mathcal{B}\in\mathbb{R}^{\delta \times \kappa}$, $\mathcal{C}\in\mathbb{R}^{\kappa \times \kappa}$. If $\mathcal{A}\succ0$ then $S\succ0$ if and only if $\mathcal{C}-\mathcal{B}^\top\mathcal{A}\inv\mathcal{B}\succ0$.
\end{lem}

This allows the following lemma to express persistency of excitation conditions using correlation matrices.
\begin{lem}\label{lem:D_full_row_rank2}
    Consider a controllable non-deterministic \ac{LTI} system of the form given by \eqref{eqn:SS_innovation} with an input sequence $\{u_k\}_{k=i}^{i+\epsilon+q-2}$ that is persistently exciting of order $\epsilon+n$ and innovation sequence $\{e_k\}_{k=i}^{i+\epsilon+q-2}$. If the sample correlation matrices between inputs and noise are such that
    \begin{align}
        & \span\span \hat{\Sigma}_{\mathrm{ee},2} - \hat{\Sigma}_{\mathrm{ue},2}^\top \hat{\Sigma}_{\mathrm{uu},2}\inv \hat{\Sigma}_{\mathrm{ue},2} \succ 0,\label{eq:PE_corUE3}\\
        & \text{with}\;\;\;&&\hat{\Sigma}_{\mathrm{ee},2}=E_{i,\epsilon+n,q-n}E_{i,\epsilon+n,q-n}^\top\notag\\
        & &&\hat{\Sigma}_{\mathrm{ue},2}=U_{i,\epsilon+n,q-n}E_{i,\epsilon+n,q-n}^\top\notag\\
        & &&\hat{\Sigma}_{\mathrm{uu},2}=U_{i,\epsilon+n,q-n}U_{i,\epsilon+n,q-n}^\top\notag
    \end{align}
    then the matrix $\left[X_{i,1,q}^\top\;U_{i,\epsilon,q}^\top\;E_{i,\epsilon,q}^\top\right]^\top$ is full row rank.
\end{lem}
\textbf{Proof:} By Lemma~\ref{lem:D_full_row_rank} the matrix $\left[X_{i,1,q}^\top\;U_{i,\epsilon,q}^\top\;E_{i,\epsilon,q}^\top\right]^\top$ is full row rank if the combined input and noise sequence is such that, by Definition~\ref{def:PE}, $q\geq(\epsilon+n)(r+l)+n$, and
\begin{align}\label{eq:PE_corUE}
    \begin{bmatrix}
        U_{i,\epsilon+n,q-n}\\
        E_{i,\epsilon+n,q-n}
    \end{bmatrix}\!\!
    \begin{bmatrix}
        U_{i,\epsilon+n,q-n}\\
        E_{i,\epsilon+n,q-n}
    \end{bmatrix}^\top\!=
    \begin{bmatrix}
        \hat{\Sigma}_{\mathrm{uu},2} & \hat{\Sigma}_{\mathrm{ue},2}\\
        \hat{\Sigma}_{\mathrm{ue},2}^\top & \hat{\Sigma}_{\mathrm{ee},2}
    \end{bmatrix}\succ 0,
\end{align}
The persistency of excitation condition of the input sequence of order $\epsilon+n$ entails by Definition~\ref{def:PE} that ${\hat{\Sigma}_{\mathrm{uu},2}\succ0}$. Then by Lemma~\ref{lem:Schur_comp}, condition \eqref{eq:PE_corUE} is met such that $\left[X_{i,1,q}^\top\;U_{i,\epsilon,q}^\top\;E_{i,\epsilon,q}^\top\right]^\top$ is full row rank if and only if \eqref{eq:PE_corUE3} is satisfied.$\hfill\qed$
\subsection{Proof of Theorem~\ref{theorem:main_result}}%\textbf{Proof:}
This section builds on the auxiliary results presented in \secref{sec:aux_results} to provide a proof of Theorem~\ref{theorem:main_result}, which follows next.

%%%%%%%%%%%%%%%%%%%%%%%% Proof of (i) %%%%%%%%%%%%%%%%%%%%%%%%%%%%%%%%%%%%%%%%
\noindent\textbf{Proof of $\mathrm{\mathbf{(i)}}$:} 
Equation \eqref{eq:DataEq1} can be rewritten with $k=i,\;q=N$ in terms of actual states, inputs, outputs, and noise or with $k=\hat{i},\;q=f$ for predictions along the lines of \eqref{eq:CL_DeePC_no_IVs} as respectively
\begin{align}
    &\begin{bmatrix}-\Gamma_s \tilde{A}^p & -L_s & I_{sl}&-\mathcal{H}_s\end{bmatrix}
    \underbrace{\begin{bmatrix}
        X_{i,1,N}\\
        \Phi_{i,s,N}\\
        Y_{i_p,s,N}\\
        E_{i_p,s,N}
    \end{bmatrix}}_{=\mathfrak{BD}_N}=\mathcal{O},\label{eq:RBD1}\\
    &\underbrace{\begin{bmatrix}-\Gamma_s \tilde{A}^p & -L_s & I_{sl}&-\mathcal{H}_s\end{bmatrix}}_{= \mathfrak{R}}
    \underbrace{\begin{bmatrix}
        \widehat{X}_{\hat{i},1,f}\\
        \overline{\Phi}_{\hat{i},s,f}\\
        \widehat{Y}_{\hat{i}_p,s,f}\\
        \widehat{E}_{\hat{i}_p,s,f}
    \end{bmatrix}}_{=\mathfrak{BD}_f}=\mathcal{O},\label{eq:RBD2}\\
    &\mathfrak{R}\in\mathbb{R}^{sl\times (n+(p+s)(r+l)+sl)},\;\mathfrak{BD}_q\in\mathbb{R}^{(n+(p+s)(r+l)+sl) \times q},\notag
\end{align}
with $\mathfrak{R}$, and $\mathfrak{BD}_q$ for $q=N$ and $q=f$ defined as indicated for brevity. The estimated states $\widehat{X}_{\hat{i},1,f}$ and future innovation noise $\widehat{E}_{\hat{i}_p,s,f}$ are needed to explain the estimates $\overline{\Phi}_{\hat{i},s,f}$ and $\widehat{Y}_{\hat{i}_p,s,f}$ in \eqref{eq:RBD2} along the lines of \eqref{eq:RBD1}. Proof that the columns of $\mathfrak{BD}_f$ indeed all lie in the nullspace of $\mathfrak{R}$ follows soon. To this end, let $\mathcal{R}(\cdot)$ and $\mathcal{N}(\cdot)$ respectively denote the range and nullspace of a matrix.

The central idea of this proof is that \eqref{eq:CL_DeePC_no_IVs} holds true if there exists a $G$ such that $\mathfrak{BD}_N G=\mathfrak{BD}_f$, for which sufficient conditions are that
\begin{enumerate}
    \item[C1.] $\mathcal{R}(\mathfrak{BD}_N)=\mathcal{N}(\mathfrak{R})$, and
    \item[C2.] $\mathcal{R}(\mathfrak{BD}_f)\;\subseteq\mathcal{N}(\mathfrak{R})$.
\end{enumerate}
To prove that these conditions are satisfied use \eqref{eq:DataEq1} to rewrite $\mathfrak{BD}_N$ and $\mathfrak{BD}_f$ as contributions of initial states and exogenous inputs (respectively denoted by $\mathfrak{D}_N$ and $\mathfrak{D}_f$) as well as a matrix that describes their effects $\mathfrak{B}$. This factorization is given by
\begin{alignat}{2}
    \begin{bmatrix}
        X_{i,1,N}\\
        \hline
        \Phi_{i,s,N}\\
        \hline
        Y_{i_p,s,N}\\
        E_{i_p,s,N}
    \end{bmatrix}&=
    \underbrace{\begin{bmatrix}
        I_n      & 0      & 0       & 0 & 0\\
        \hline
        0        & I_{pr} & 0       & 0 & 0\\
        0        & 0      & I_{sr}  & 0 & 0\\
        \Gamma_p & \mathcal{T}_p^\mathrm{u} & 0 & \mathcal{H}_p & 0\\
        \hline
        \varepsilon_1 & \varepsilon_2 & \mathcal{T}_s^\mathrm{u} & \varepsilon_3 & \mathcal{H}_s\\
        0 & 0 & 0 & 0 & I_{sl}
    \end{bmatrix}}_{=\mathfrak{B}}
    \underbrace{\begin{bmatrix}
        X_{i,1,N}\\
        U_{i,p,N}\\
        U_{i_p,s,N}\\
        E_{i,p,N}\\
        E_{i_p,s,N}
    \end{bmatrix}}_{=\mathfrak{D}_N},\notag\\%\label{eq:BD_N}\\
    \begin{bmatrix}
        \widehat{X}_{\hat{i},1,f}\\
        \overline{\Phi}_{\hat{i},s,f}\\
        \widehat{Y}_{\hat{i}_p,s,f}\\
        \widehat{E}_{\hat{i}_p,s,f}
    \end{bmatrix}&=
    \mathfrak{B}
    \underbrace{\begin{bmatrix}
        \widehat{X}_{\hat{i},1,f}\\
        U_{\hat{i},p,f}\\
        U_{\hat{i}_p,s,f}\\
        \overline{E}_{\hat{i},p,f}\\
        \widehat{E}_{\hat{i}_p,s,f}
    \end{bmatrix}}_{=\mathfrak{D}_f},\label{eq:BD_f}\\%\label{eq:BD_f}\\
    \span\mathfrak{B}\in\mathbb{R}^{(n+(p+s)(r+l)+sl)\times (n+(p+s)(r+l))},\notag\\
    \span\mathfrak{D}_N\in\mathbb{R}^{(n+(p+s)(r+l))\times N},\quad \mathfrak{D}_f\in\mathbb{R}^{(n+(p+s)(r+l))\times f}\notag
\end{alignat}
with $\mathfrak{B}$, $\mathfrak{D}_N$, and $\mathfrak{D}_f$ defined as shown and $\varepsilon_1=\Gamma_s(\tilde{A}^p+\tKp{y}\Gamma_p)$, $\varepsilon_2=\Gamma_s(\tKp{u}+\tKp{y}\mathcal{T}_p^\mathrm{u})$, and $\varepsilon_3=\Gamma_s\tKp{y}\mathcal{H}_p$. Note that $\mathfrak{RB}=\mathcal{O}$, which means that $\mathcal{R}(\mathfrak{B})\subseteq\mathcal{N}(\mathfrak{R})$. In fact, by inspection one may verify that $\mathfrak{B}$ is full column rank and that its number of columns corresponds to the nullity of $\mathfrak{R}$. Hence, $\mathcal{R}(\mathfrak{B})=\mathcal{N}(\mathfrak{R})$. This has two important implications.

Firstly, since ${\mathcal{R}(\mathfrak{BD}_f)\subseteq\mathcal{R}(\mathfrak{B})=\mathcal{N}(\mathfrak{R})}$, condition C2 is satisfied.

Secondly, by similar reasoning, it must hold that ${\mathcal{R}(\mathfrak{BD}_N)\subseteq\mathcal{R}(\mathfrak{B})=\mathcal{N}(\mathfrak{R})}$. To prove condition C1 it must therefore be shown that $\mathcal{R}(\mathfrak{BD}_N)=\mathcal{R}(\mathfrak{B})$, for which $\mathfrak{D}_N$ must be full row rank. Under the stipulated persistency of excitation of the input sequence of order $p+s+n$, condition \eqref{eq:PE_corUE2}, and presumed system controllability, Theorem~\ref{theorem:main_result} satisfies the conditions of Lemma~\ref{lem:D_full_row_rank2} for $\epsilon=p+s$ and $q=N$ such that the matrix $\mathfrak{D}_N$ is full row rank. This proves condition C1. Having already proven condition C2 this concludes the proof of $\mathrm{(i)}$. $\hfill\qed$

\noindent\textbf{Remark 1:} A necessary condition for \eqref{eq:PE_corUE2} is that ${N\geq(p+s+n)(r+l)+n}$. This follows from the above proof, in which $\mathfrak{D}_N$ is required to be full row rank. The top matrix equation of \eqref{eq:CL_DeePC_no_IVs} defines $G$, for which there are multiple solutions given the aforementioned necessary lower bound on $N$ and the dimensions of $\Phi_{i,s,N}\in\mathbb{R}^{((p+s)r+pl)\times N}$. These different possible solutions of $G$ are given by
\begin{align}%\label{eq:G_sols}
    G = \Phi_{i,s,N}^\dagger\overline{\Phi}_{\hat{i},s,f} + \Pi_{\Phi_{i,s,N}}^\bot W,
\end{align}
in which the dagger $\dagger$ denotes the right inverse ($\mathcal{Q}^\dagger=\mathcal{Q}^\top(\mathcal{Q}\mathcal{Q}^\top)\inv$ with $\mathcal{Q}$ as a real, full row rank matrix), $\Pi_{\Phi_{i,s,N}}^\bot=I_N-\Phi_{i,s,N}^\dagger\Phi_{i,s,N}$ is a projection matrix onto the orthogonal complement of the row space of $\Phi_{i,s,N}$, %see Overschee1996, pg. 19
and $W\in\mathbb{R}^{N\times f}$ is a free matrix.

\noindent\textbf{Proof of $\mathrm{(ii)}:$} To prove that \eqref{eq:CL_DeePC_no_IVs} defines $\widehat{Y}_{\hat{i}_p,s,f}$ as an asymptotically unbiased predictor as $p\rightarrow\infty$ if $s=1$, first consider the error of this prediction. By the bottom matrix equation of~\eqref{eq:CL_DeePC_no_IVs} and subsequent application of~\eqref{eq:DataEq1} to rewrite $Y_{i_p,s,N}$ and $Y_{\hat{i}_p,s,f}$ we find
\begin{align}%\label{eq:Yf_error1}
    \begin{split}
        &\!\!\!\widehat{Y}_{\hat{i}_p,s,f}-Y_{\hat{i}_p,s,f} = \Gamma_s \tilde{A}^p (\underbrace{X_{i,1,N}G}_{=\widehat{X}_{\hat{i},1,f}}-X_{\hat{i},1,f}) \\
        &+L_s(\underbrace{\Phi_{i,s,N}G}_{=\overline{\Phi}_{\hat{i},s,f}}-\Phi_{\hat{i},s,f}) +\mathcal{H}_s (\underbrace{E_{i_p,s,N}G}_{=\widehat{E}_{\hat{i}_p,s,f}}-E_{\hat{i}_p,s,f})
    \end{split}
\end{align}
The interpretations of the underbraced terms are obtained from $\mathfrak{BD}_N G=\mathfrak{BD}_f$, which was central to the preceding proof of $\mathrm{(i)}$. Applying the limit $p\rightarrow\infty$ asymptotically attenuates the top contribution by the error of the state estimates since, by the definition of $K$ in \secref{sec:sys_model}, $\tilde{A}$ has all of its eigenvalues strictly inside the unit circle. In addition, substituting $G$ from \eqref{eq:G_sols} and cancelling equal terms in $\overline{\Phi}_{\hat{i},s,f}$ and $\Phi_{\hat{i},s,f}$ results in
\begin{alignat}{2}
        &\!\!\!\!\lim_{p\rightarrow\infty} \widehat{Y}_{\hat{i}_p,s,f}-Y_{\hat{i}_p,s,f} = \Gamma_s\tKp{y}\left(\overline{Y}_{\hat{i},p,f}-Y_{\hat{i},p,f}\right)\notag\\
        &+\mathcal{H}_s\Big(E_{i_p,s,N} W-E_{\hat{i}_p,s,f}\\%\label{eq:Yf_error2}\\
        &+\underbrace{E_{i_p,s,N}\Phi_{i,s,N}^\top}(\Phi_{i,s,N}\Phi_{i,s,N}^\top)\inv(\overline{\Phi}_{\hat{i},s,f}-\Phi_{i,s,N}W)\Big)\notag.
\end{alignat}%
In the above formulation, $W$ is a free matrix that parameterizes $G$. As such, for the predictor to be unbiased, the expectation ($\mathbb{E}[\cdot]$) of this error w.r.t. the noise and conditioned on $W$ must be zero. \todo{$\mathbb{E}[\cdot]$,\\$EW$,\\$\hat{\Sigma}\hat{\Sigma}\inv$?} This expectation renders the second row zero, but not the bottom row because of the underbraced term. Due to feedback, there is a correlation between past inputs and preceding noise such that the expectation of the sample correlation
\begin{align}
E_{i_p,s,N}U_{i_p,s,N}^\top =
    \frac{1}{N}\sum\limits_{k=i_p-1}^{i_p+N-2}
    \begin{bmatrix}
        e_{k+1} u_{k+1}^\top & \cdots & e_{k+1} u_{k+s}^\top\\
        e_{k+2} u_{k+1}^\top & \cdots & e_{k+2} u_{k+s}^\top\\
        \vdots &  & \vdots\\
        e_{k+s}u_{k+1}^\top & \cdots & e_{k+s}u_{k+s}^\top
    \end{bmatrix},\notag%
\end{align}
which is contained in the underbraced term in \eqref{eq:Yf_error2}, is nonzero. Due to feedback $\mathbb{E}[e_j u_k^\top]\neq 0,\;\forall k>j$. However, if, as assumed by Theorem~\ref{theorem:main_result}, the controller has no direct feedthrough, then $\mathbb{E}[e_j u_k^\top]= 0,\;\forall k=j$. Hence, if $s=1$, the expected value of the correlation $E_{i_p,s,N}U_{i_p,s,N}^\top$ reduces to zero, rendering the expectation of the bottom row of \eqref{eq:Yf_error2} zero.

Given the structure of $\overline{\Phi}_{\hat{i},1,f}$ and $\overline{Y}_{\hat{i},p,f}$ shown in Fig.~\ref{fig:CL-DeePC}, consider \eqref{eq:Yf_error2} column by column for $s=1$. Taking the expectation leaves
\begin{align}%\label{eq:Yf_error3}
\begin{split}
    \mkern-3mu\lim_{p\rightarrow\infty} &\mathbb{E}\left[\widehat{y}_{\hat{i}_p+k}-y_{\hat{i}_p+k}\right] = \\ &\Gamma_1\tKp{y}\left(\mathbb{E}\left[\datavec{\overline{y}}{\hat{i}+k,p}-\datavec{y}{\hat{i}+k,p}\right]\right),\;\forall k\in[0,f-1].
\end{split}
\end{align}
For $k=0$, \eqref{eq:Yf_error3} is zero because none of the relevant ouput data is estimated ($\datavec{\overline{y}}{\hat{i},p}=\datavec{y}{\hat{i},p}$). For $k=1$ no bias is thereby introduced on the right hand side, rendering the expectation zero again. Repetition of this process until $k=f-1$ demonstrates that $\widehat{Y}_{\hat{i}_p,1,f}$ is indeed an asymptotically unbiased predictor in the limit $p\rightarrow\infty$. This concludes the proof of $\mathrm{(ii)}$. $\hfill\qed$

%%%%%%%%%%%%%%%%%%%%%%%%%%%%%%%%%%%%%%%%%%%%%%%%%%%%%%%%%%%%%%%%%%%%%%%%%%%%%%%%%%%%%%%%%%%%%%%%%%%%%%%%%%%%%%%%%%%%%%%%%%%%%%%%%%%%%%%
\begin{align*}
    \hline
\end{align*}
% As a result, by the Rouch\'{e}-Capelli theorem there then exists a vector $g_{k-\hat{i}+1}$ such that
% \begin{align}\label{eq:Dg}
%     \mathfrak{D} g_{k-\hat{i}+1} =
%     \begin{bmatrix}
%         x_k^\top & \datavec{u}{k,p}^\top & \datavec{u}{k_p,s}^\top & \datavec{e}{k,p}^\top & \datavec{e}{k_p,s}^\top
%     \end{bmatrix}^\top\in\mathbb{R}^{n+(p+s)(r+l)}.
% \end{align}
% With reference to~\eqref{eq:Yf1} and~\eqref{eq:DataEq1}, pre-multiplying both sides of~\eqref{eq:Dg} by $\mathfrak{B}$ from~\eqref{eq:RBD2} yields
% \begin{align}\label{eq:BDg}
%     \begin{bmatrix}
%         X_{i,1,N}\\
%         \Phi_{i,s,N}\\
%         Y_{i_p,s,N}\\
%         E_{i_p,s,N}
%     \end{bmatrix}g_{k-\hat{i}+1}=
%     \begin{bmatrix}
%         x_k\\
%         \Phi_{k,s,1}\\
%         \datavec{y}{k_p,s}\\
%         \datavec{e}{k_p,s}
%     \end{bmatrix}\in\mathcal{N}(\mathfrak{R}).
% \end{align}
% Proof of~\eqref{eq:CL_DeePC_no_IVs} is obtained by sequential application of \eqref{eq:BDg} for $k={\hat{i},\dots,\hat{i}+f-1}$ such that $G=\left[g_1\;g_2\;\cdots\;g_f\right]$, and without the unknown top and bottom matrix equations. As a result the future outputs in $\bar{\Phi}_{\hat{i},s,f}$ and $\widehat{Y}_{\hat{i}_p,s,f}$ of \eqref{eq:CL_DeePC_no_IVs} are predictions.

% To prove statement 1) consider that $G$ is determined only by the top matrix equation in~\eqref{eq:CL_DeePC_no_IVs}, which contains known past input-output data as well as future inputs that can be chosen. Under the assumed persistency of excitation conditions $\mathfrak{D}$ is full row rank such that, by inspection of the product $\mathfrak{BD}$ in~\eqref{eq:RBD2}, $\Phi_{i,s,N}$ must also be full row rank. %This matrix is invertible if $N=p(r+l)+sl$ <- not possible
% Since the condition ${N\geq(p+s+n)(r+l)+n}$ ensures that $\Phi_{i,s,N}\in\mathbb{R}^{p(r+l)+sl\times N}$ has more columns then rows, there are multiple solutions to $G$ in \eqref{eq:CL_DeePC_no_IVs}. This proves statement~1). The different solutions of $G$ are given by
% \begin{align}%\label{eq:G_sols}
%     G = \Phi_{i,s,N}^\dagger\overline{\Phi}_{\hat{i},s,f} + (I_N-\Phi_{i,s,N}^\dagger\Phi_{i,s,N})W,
% \end{align}
% in which the dagger $\dagger$ denotes the right inverse ($\mathcal{Q}^\dagger=\mathcal{Q}^\top(\mathcal{Q}\mathcal{Q}^\top)\inv$ with $\mathcal{Q}$ as a real, full row rank matrix), and $W\in\mathbb{R}^{N\times f}$ is a free matrix.

% To prove statement~2) consider the error of the output prediction, which using \eqref{eq:DataEq1} to rewrite $Y_{i_p,s,N}$ and $Y_{\hat{i},s,f}$ and by subsequent application of~\eqref{eq:CL_DeePC_no_IVs} is found as
% \begin{align}%\label{eq:Yf_error1}
%     \begin{split}
%         \widehat{Y}_{\hat{i}_p,s,f}-Y_{\hat{i}_p,s,f} = &\; L_s(\overline{\Phi}_{\hat{i},s,f}-\Phi_{\hat{i},s,f})\\
%         &+\mathcal{H}_s (E_{i_p,s,N}G-E_{\hat{i}_p,s,f})\\
%         &+ \Gamma_s \tilde{A}^p (X_{i,1,N}G-X_{\hat{i},1,f}).
%     \end{split}
% \end{align}\todo{incl. $\overline{\Phi}$}
% Note that the terms in parentheses correspond to the errors that are induced by removal of the unknown top and bottom matrix equations in~\eqref{eq:BDg}. Moreover, the bottom line is asymptotically attenuated as $p\rightarrow\infty$ because $\tilde{A}$, by its definition in \secref{sec:sys_model}, has all of its eigenvalues strictly inside the unit circle. Taking the limit $p\rightarrow\infty$ and substituting $G$ from \eqref{eq:G_sols} in \eqref{eq:Yf_error1} yields

% \begin{align}%\label{eq:Yf_error2}
%     \begin{split}
%         &\lim_{p\rightarrow\infty} \Big( \widehat{Y}_{\hat{i}_p,s,f}-Y_{\hat{i}_p,s,f} \Big)=\mathcal{H}_s\Big(E_{i_p,s,N}W-E_{\hat{i}_p,s,f} \\
%         % &-E_{i_p,s,N}\Phi_{i,s,N}^\top(\Phi_{i,s,N}\Phi_{i,s,N}^\top)\inv\\
%         &+\underbrace{E_{i_p,s,N}\Phi_{i,s,N}^\top}(\Phi_{i,s,N}\Phi_{i,s,N}^\top)\inv(\Phi_{\hat{i},s,f}-\Phi_{i,s,N}W)\Big),
%     \end{split}
% \end{align}
% in which the underbraced product is the sample correlation $[E_{i_p,s,N}U_{i,p,N}^\top\;\;E_{i_p,s,N}U_{i_p,s,N}^\top\;\;E_{i_p,s,N}Y_{i,p,N}^\top]$. By causality the expectation of the leftmost and rightmost sample correlations herein are zero. The expectation of the middle correlation is however not zero in general due to feedback in closed-loop operation. The middle sample correlation $E_{i_p,s,N}U_{i_p,s,N}^\top$ is
% \begin{align}
%     \frac{1}{N}\sum\limits_{k=i_p}^{i_p+N-1}\begin{bmatrix}
%         e_k u_k^\top & \cdots & e_k u_{k+s-1}^\top\\
%         e_{k+1} u_k^\top & \cdots & e_{k+1} u_{k+s-1}^\top\\
%         \vdots &  & \vdots\\
%         e_{k+s-1}u_k^\top & \cdots & e_{k+s-1}u_{k+s-1}^\top
%     \end{bmatrix}.\notag%
% \end{align}
% Since the employed controller is assumed to have no direct feedthrough, the correlation between noise and inputs is strictly causal: $\mathbb{E}[e_ku_j^\top]=0,\forall{j\leq k}$. With $s=1$ only such correlations are employed such that the expectation of \eqref{eq:Yf_error2} is the null matrix. This proves statement~2).

\section{Willems' Fundamental Lemma \& Noise}
\todo{Oud: reuse?}Equation~\eqref{eq:DataEq1} can be reformulated with $k=i$, $q=N$ or for an ideal noiseless output prediction with $k=\hat{i}$ as respectively
\begin{alignat}{2}
    \begin{bmatrix}
        -L_s & I_{sl}
    \end{bmatrix}&
    \begin{bmatrix}
        \Phi_{i,s,N}\\
        Y_{i_p,s,N}-\mathcal{H}_s E_{i_p,s,N}
    \end{bmatrix} = \mathcal{O},\label{eq:NoisyWFL1}\\%\mathcal{H}_s E_{i_p,s,N}, 
    \begin{bmatrix}
        -L_s & I_{sl}
    \end{bmatrix}&
    \begin{bmatrix}
        \Phi_{\hat{i},s,q}\\
        \widehat{Y}^*_{\hat{i}_p,s,q}
    \end{bmatrix} = \mathcal{O}, \label{eq:NoisyWFL2}
\end{alignat}
in which the asterisk indicates that the output prediction is ideal in the sense of being asymptotically unbiased. Multiplying \eqref{eq:NoisyWFL1} by $\mathcal{Z}^\top G\in\mathbb{R}^{N\times q}$, and subtracting \eqref{eq:NoisyWFL2} obtains
\begin{align}\label{eq:NoisyWFL3}
    \mkern-14mu\begin{bmatrix}
        \shortminus L_s & I_{sl}
    \end{bmatrix}
    \mkern-9mu\left(\mkern-3mu%
    \begin{bmatrix}
        \Phi_{i,s,N}\\
        Y_{i_p,s,N}\shortminus\mathcal{H}_s E_{i_p,s,N}
    \end{bmatrix}%
    \mkern-4mu\mathcal{Z}^\top G%\mkern-2mu
    -%-%
    \mkern-5mu\begin{bmatrix}
        \Phi_{\hat{i},s,q}\\
        \widehat{Y}^*_{\hat{i}_p,s,q}
    \end{bmatrix}\mkern-3mu\right)\mkern-6mu=\mkern-3mu\mathcal{O}\mkern-5mu%\mathcal{H}_s E_{i_p,s,N}\mathcal{Z}G
\end{align}
in which $\mathcal{Z}$ represents a yet unspecified matrix and $G$ represents a matrix that is akin to the likewise defined matrix from \eqref{eq:CL_DeePC_no_IVs} that contains all of the vectors $g_k$.

If the columns of the matrix with data on the left hand side of \eqref{eq:NoisyWFL1} span the entire nullspace of $\left[\shortminus L_s\;I_{sl}\right]$ and $\mathcal{Z}$ is full rank then all solutions to \eqref{eq:NoisyWFL3} are described by equating the term inside the parenthesis to zero. For now, consider the case that $\mathcal{Z}=I_N$, $s=f$, and $q=1$ in the absence of noise to recover the regular deterministic \ac{DeePC} equation~\citep{Coulson2019}. %Then one possible solution (since the matrix $\left[\shortminus L_s\;I_{sl}\right]$ is not full column rank) to \eqref{eq:NoisyWFL3} with $s=f$ and $q=1$ is given by the regular deterministic \ac{DeePC} equation~\cite{Coulson2019}:
\begin{align}\label{eq:regular_DeePC}
    \begin{bmatrix}
        \Phi_{i,f,N}\\
        Y_{i_p,f,N}
    \end{bmatrix}g=%
    \begin{bmatrix}
        \Phi_{\hat{i},f,1}\\
        \widehat{Y}_{\hat{i}_p,f,1}
    \end{bmatrix}.
\end{align}
Willems' Fundamental Lemma makes use of Assumptions~\ref{assum:PE} and~\ref{assum:controllability} to ensure that the entire nullspace of $\left[\shortminus L_f\;I_{fl}\right]$ is spanned by the data matrix on the left hand side of \eqref{eq:regular_DeePC}~\citep{Willems2005}. Assumption~\ref{assum:unique_initial} is furthermore necessary to guarantee the existence of a unique initial state and therefore output predictor. %This clearly reflects Willems' Fundamental Lemma, which states that for a deterministic \ac{LTI} system, any sufficiently persistently exciting past input-output trajectory parameterizes all possible future input-output trajectories~\cite{Willems2005}.\todo{WFL: what about nullspace in (12)}

In the presence of (unknown) noise, the term $Y_{i_p,s,N}-\mathcal{H}_s E_{i_p,s,N}$ from \eqref{eq:NoisyWFL3} cannot be determined to obtain an ideal output predictor. Instead, linear combinations of a noise-corrupted output $Y_{i_p,s,N}$ as in \eqref{eq:regular_DeePC} are taken, resulting in an error of the obtained output predictor due to implicit sampling of $\mathcal{H}_s E_{i_p,s,N}$. Moreover, the regular \ac{DeePC} formulation provided by \eqref{eq:regular_DeePC} may become inconsistent in the presence of noise, prompting the use of, e.g., slack variables and regularization~\citep{Coulson2019}.
%
% ==============================================================================================================================================================
% ==============================================================================================================================================================
\subsection{Noise mitigation using \acl{IVs}}
Notwithstanding potential benefits of beforementioned mechanisms to cope with noise, such methods do not provide a systematic way to mitigate noise at the source. To that end this section introduces the use of an \ac{IV}: $\mathcal{Z}\neq I_N$. In this context, the \ac{IV} is defined such that it is uncorrelated with the noise $E_{i_p,s,N}$ and preserves the (full row) rank of the data matrix $\Phi_{i,s,N}$ obtained from a sufficiently persistently exciting input. These two conditions are respectively formulated as
%
\begin{align}
    &\lim_{N\rightarrow\infty} \frac{1}{N}E_{i_p,s,N}\mathcal{Z}^\top = \mathcal{O},\label{eq:uncorrelated}\\
    \text{rank}\biggl(&\lim_{N\rightarrow\infty} \frac{1}{N}\Phi_{i,s,N}\mathcal{Z}^\top\biggl) =  \text{rank}(\Phi_{i,s,N}),\label{eq:rankconservation}
\end{align}
%
which motivates choosing $\mathcal{Z}=\Phi_{i,s,N}$~\cite[Chapt. 9.6]{Verhaegen2007a}. An important assumption that is hereby introduced to satisfy \eqref{eq:uncorrelated} is that inputs are uncorrelated with noise. To fulfill this assumption Section~\ref{sec:CL_ID_issue} will motivate the choice $s=1$. Furthermore, to then still obtain a multi-step-ahead predictor, $q=f$ is chosen.

Since the noise contribution in \eqref{eq:NoisyWFL3} is then asymptotically attenuated with increasing $N$ this motivates the use of
\begin{align}\label{eq:CL_DeePC_with_IV}
    \begin{bmatrix}
   \Phi_{i,1,N}\Phi_{i,1,N}^\top\\
   \hline
   Y_{i_p,1,N}\Phi_{i,1,N}^\top
    \end{bmatrix}
G =
\begin{bmatrix}
    \Phi_{\hat{i},1,f}\\
    \hline
    \widehat{Y}_{\hat{i}_p,1,f}
\end{bmatrix},
\end{align}
for sufficiently large $N$. Note that the structure of this equation is very similar to \eqref{eq:CL_DeePC_no_IVs} as shown by Fig.~\ref{fig:CL-DeePC}. The main difference is that the matrix with past data on the left hand side loses its indicated block-anti-diagonal structure and has $(p+1)r+pl$ instead of $N$ columns.

Solving \eqref{eq:CL_DeePC_with_IV} for the output predictor using the data equation examplified by \eqref{eq:DataEq1} yields
\begin{align}\label{eq:OutputPredictor}
    \widehat{Y}_{\hat{i}_p,1,f} = L_1 \Phi_{\hat{i},1,f} + \mathcal{H}_1 E_{i_p,1,N}\Phi_{i,1,N}^\dagger\Phi_{\hat{i},1,f},
\end{align}
in which the dagger $\dagger$ denotes the right inverse: ${\Phi_{i,1,N}^\dagger=\Phi_{i,1,N}^\top\left(\Phi_{i,1,N}\Phi_{i,1,N}^\top\right)\inv}$. Similar scrutiny of \eqref{eq:NoisyWFL3} demonstrates that according to \eqref{eq:uncorrelated} the ideal output predictor is recovered from \eqref{eq:OutputPredictor} in the limit $N\rightarrow\infty$.