\section{Closed-loop Data-enabled Predictive Control}
This section presents the main result of this article, providing contribution~(\ref{contribution:solves_CL_issue}) whereby we develop \ac{CL-DeePC}. An intuitive explanation is first offered before a more rigorous proof of the method is provided.

As a solution to the identification bias that arises in closed-loop due to correlation between inputs and noise (a demonstration thereof is deferred to Section~\ref{sec:CL_ID_issue}) it is possible to estimate a step-ahead predictor~\citep{Ljung1996}. A prediction horizon length $f>1$ is of more practical use in receding horizon optimal control settings, to which end step-ahead predictors can be applied sequentially. 

Fig.~\ref{fig:CL-DeePC} and \eqref{eq:CL_DeePC_no_IVs} illustrate how this idea is employed in \ac{CL-DeePC}. A step-ahead predictor can be obtained from regular \ac{DeePC} (see Fig.~\ref{fig:regular-DeePC} and \eqref{eq:regular_DeePC_no_IVs}) with $f=1$. In \ac{CL-DeePC} the successive columns of $G$ (from left to right) and their corresponding columns on the right-hand side correspond to sequential applications of regular \ac{DeePC} with $f=1$ to the same matrix of sufficiently persistently exciting past input-output data on the left-hand side as well as time-shifted windows of input-output data on the right-hand side that parameterize successive initial states.
% In this section \ac{CL-DeePC} is first introduced and compared to regular \ac{DeePC}, thereby offering an intuitive explanation for contribution \ref{contribution:solves_CL_issue}. More rigorous explanations follow in subsequent sections. What follows directly below is the principal result of this work.
%
% \begin{figure*}[h!]
%      \centering
%      \begin{subfigure}[b]{\columnwidth}
%          \centering
%          \begin{tikzpicture}
    % defining constants
    \def\stepSize{0.25}
    \def\Nnum{9}
    \def\fnum{8}
    \def\pnum{4}
    
    % Defining lengths
    \newlength{\onelen}
    \setlength{\onelen}{\stepSize cm}
    \newlength{\BrCl}
    \setlength{\BrCl}{0.075cm}
    \newlength{\BrIn}
    \setlength{\BrIn}{0.15cm}
    \newlength{\plen}
    \setlength{\plen}{1cm}%{\pnum\stepSize cm}
    \newlength{\flen}
    \setlength{\flen}{2cm}%{\fnum*\stepSize cm}
    \newlength{\Nlen}
    \setlength{\Nlen}{2.25cm}%{\Nnum\stepSize cm}%should be 2*p+one
    \newlength{\MatClearance}
    \setlength{\MatClearance}{0.3cm}
    
    % grid lines for guidance
    % \draw[gray,step=0.5] (-0,-3) grid (8,3);

    % ======================= drawing data matrix =======================
    \path (0,-\onelen) coordinate (M1A);
    \path ([xshift=\Nlen]M1A) coordinate (M1B);
    \path ([yshift=2*\plen+2*\onelen]M1B) coordinate (M1C);
    \path ([xshift=-\Nlen]M1C) coordinate (M1D);
    \draw[line width=1.5pt] ([xshift=\BrIn,yshift=-\BrCl]M1A) -- ([xshift=-\BrCl,yshift=-\BrCl]M1A) -- ([xshift=-\BrCl,yshift=\BrCl]M1D) -- ([xshift=\BrIn,yshift=\BrCl]M1D); %left bracket
    \draw[line width=1.5pt] ([xshift=-\BrIn,yshift=-\BrCl]M1B) -- ([xshift=\BrCl,yshift=-\BrCl]M1B) -- ([xshift=\BrCl,yshift=\BrCl]M1C) -- ([xshift=-\BrIn,yshift=\BrCl]M1C); %right bracket
    \draw[line width=1pt] ([yshift=\onelen]M1A) -- ([yshift=\onelen]M1B); % dividing matrix into blocks
    \fill[black, opacity=0.5] (M1A) rectangle (M1C);
    \foreach \x in {0,...,8} { % drawing black dots
    \foreach \y in {-1,...,8} {
      \fill ( {(\x+0.5)*\onelen}, {(\y+0.5)*\onelen} ) circle (1pt);
    }}
    \draw[line width=0.1pt] ([yshift=-\plen-\onelen]M1D) -- ([yshift=-\plen-\onelen]M1C);
    \draw[line width=0.1pt] ([yshift=-\plen]M1D) -- ([yshift=-\plen]M1C);

    % coordinates for diagonals
    \path ([yshift=-\plen]M1D) coordinate (M1stair1A);
    \path ([xshift=\plen]M1D) coordinate (M1stair1I);
    % black diagonals
    % \foreach \i in {0,-1}{
    % \foreach \dy in {1,...,11}{
    % \ifthenelse{\dy<5.5}{
    % % This code will be executed if \dy<9.5
    %     \draw[dash pattern=on 1pt off 2pt, line width=0.1pt] ([xshift= 0.5*\onelen,yshift=\i*(\plen+\onelen)-(\dy+0.5)*\onelen]M1D) -- ([xshift= (\dy+0.5)*\onelen,yshift=\i*(\plen+\onelen)-0.5*\onelen]M1D);
    % }{
    % % This code will be executed if \dy>=10
    %     \ifthenelse{\dy<8.5}{
    %         % This code will be executed if \dy<=12
    %         \draw[dash pattern=on 1pt off 2pt, line width=0.1pt] ([xshift= (\dy-3.5)*\onelen,yshift=\i*(\plen+\onelen)-\plen-\onelen+0.5*\onelen]M1D) -- ([xshift=(\dy-8.5)*\onelen,yshift=\i*(\plen+\onelen)-0.5\onelen]M1C);
    %         }{
    %         % This code will be executed if \dy>=13
    %         \draw[dash pattern=on 1pt off 2pt, line width=0.1pt] ([xshift= (\dy-3.5)*\onelen,yshift=(\i-1)*(\plen+\onelen)+0.5\onelen]M1D) -- ([xshift= -0.5*\onelen,yshift=\i*(\plen+\onelen)-(\dy-7.5)*\onelen]M1C);
    %     }
    % }
    % }}
    
    % ---------------- drawing left brace and matrix ----------------
    \path ([xshift=-0.5\BrIn,yshift=-2\BrCl]M1A) coordinate (Brace1L);
    \path ([xshift=0.5\BrIn,yshift=-2\BrCl]M1B) coordinate (Brace1R);
    \draw[decorate, decoration={calligraphic brace, amplitude=3pt, mirror, aspect=0.75},line width=1pt] (Brace1L) -- (Brace1R);
    \node (mat1) at ($(Brace1L)!0.75!(Brace1R) + (0,-3pt)$) {};
    \node[below] at (mat1.center) {$\begin{bmatrix}
        U_{i,p,N}\\U_{i_p,1,N}\\Y_{i,p,N}\\ \hline Y_{i_p,1,N}
    \end{bmatrix}$};
    % brace to the left
    % \path ([xshift=-0.75cm-0.05cm,yshift=-4pt]mat1.center) coordinate (Brace1T);
    % \path ([yshift=-1.73cm]Brace1T) coordinate (Brace1B);
    % \draw[pen colour=gray!60,decorate, decoration={calligraphic brace, amplitude=3pt, mirror, aspect=0.2},line width=0.75pt] (Brace1T) -- (Brace1B);
    % % Zp
    % \node (Zp) at ($(Brace1T)!0.2!(Brace1B) + (-3pt,-0.8pt)$) {};
    % \node[left,text=gray!100,align=center] at ([xshift=0.5mm,yshift=-0.18cm]Zp.center) {\scriptsize{\shortstack{$=$\\$\Phi_{i,1,N}$}}};
  
    % ======================= drawing G =======================
    % useful coordinates
    \path ([xshift=\MatClearance,yshift=\onelen]M1B) coordinate (M2A);
    \path ([xshift=\flen]M2A) coordinate (M2B);
    \path ([yshift=\Nlen]M2B) coordinate (M2C);
    \path ([xshift=-\flen]M2C) coordinate (M2D);

    % brackets
    \draw[line width=1.5pt] ([xshift=\BrIn,yshift=-\BrCl]M2A) -- ([xshift=-\BrCl,yshift=-\BrCl]M2A) -- ([xshift=-\BrCl,yshift=\BrCl]M2D) -- ([xshift=\BrIn,yshift=\BrCl]M2D); %left bracket
    \draw[line width=1.5pt] ([xshift=-\BrIn,yshift=-\BrCl]M2B) -- ([xshift=\BrCl,yshift=-\BrCl]M2B) -- ([xshift=\BrCl,yshift=\BrCl]M2C) -- ([xshift=-\BrIn,yshift=\BrCl]M2C); %right bracket
    
    % red fill
    \fill[red!50,opacity=0.5] (M2A) rectangle (M2C);

    % drawing red dots
    \foreach \x in {0,...,7} {
    \foreach \y in {0,...,8} {
      \fill[red] ([xshift=(\x+0.5)*\onelen,yshift=(\y+0.5)*\onelen]M2A) circle (1pt);%{(\x+0.5)*\onelen+\Nlen+0.5cm}, {(\y+0.5)*\onelen}
    }}

    % dividers
    \foreach \x in {1,...,7}{\draw[line width=0.1pt] ([xshift=\x*\onelen]M2A) -- ([xshift=\x*\onelen]M2D);}

    % drawing middle brace and matrix
    \coordinate (Brace2L) at ([xshift=-0.5\BrIn]M2A |- Brace1L);
    \coordinate (Brace2R) at ([xshift=0.5\BrIn]M2B |- Brace1L);
    \draw[decorate, decoration={calligraphic brace, amplitude=3pt, mirror, aspect=0.5},line width=1pt] (Brace2L) -- (Brace2R);
    \node (mat1) at ($(Brace2L)!0.5!(Brace2R) + (0,-3pt)$) {};
    \node[below] at ([yshift=-0.75cm]mat1.center) {$\underbrace{
    \begin{bmatrix}
        g_1 & g_2 & \cdots & g_f
    \end{bmatrix}}_{= G}$};
    
    % ======================= drawing equal sign ======================= 
    \path ([xshift=\MatClearance*3/4,yshift=\Nlen/2-0.1cm]M2B) coordinate (EqA);
    \path ([xshift=0.4cm]EqA) coordinate (EqB);
    \path ([yshift=0.2cm]EqB) coordinate (EqC);
    \path ([xshift=-0.4cm]EqC) coordinate (EqD);
    \draw[line width = 1.5 pt] (EqA) -- (EqB);
    \draw[line width = 1.5 pt] (EqD) -- (EqC);

    % ---------------- equation sign below ----------------
    \node[below] at ([xshift=5.65\onelen,yshift=-1.05cm]mat1.center) {$=$};

    % ======================= drawing RHS =======================
    % inside of matrix
    \path ([xshift=\MatClearance*3/4,yshift=-\Nlen/2+0.1cm-\onelen]EqB) coordinate (M3A);
    \path ([xshift=\flen]M3A) coordinate (M3B);
    \path ([yshift=2*\plen+2*\onelen]M3B) coordinate (M3C);
    \path ([xshift=-\flen]M3C) coordinate (M3D);
    % top left black triangle
    \path ([yshift=-\plen-\onelen/2]M3D) coordinate (t1A);
    \path ([xshift=\plen+\onelen/2]M3D) coordinate (t1B);
    % top red trapezoid
    \path ([yshift=-\onelen/2]t1A) coordinate (t2A);
    \path ([xshift=\flen]t2A) coordinate (t2B);
    % bottom black triangle
    \path ([yshift=-\plen-\onelen/2]t2A) coordinate (t3A);
    \path ([xshift=\plen+\onelen/2]t2A) coordinate (t3B);
    % bottom red trapezoid
    \path ([yshift=-\onelen/2]t3A) coordinate (t4A);

    % top 'staircase' coordinates
    \path ([yshift=-\plen]M3D) coordinate (stair1A);
    \path ([xshift=\onelen]stair1A) coordinate (stair1B);
    \path ([yshift=\onelen]stair1B) coordinate (stair1C);
    \path ([xshift=\onelen]stair1C) coordinate (stair1D);
    \path ([yshift=\onelen]stair1D) coordinate (stair1E);
    \path ([xshift=\onelen]stair1E) coordinate (stair1F);
    \path ([yshift=\onelen]stair1F) coordinate (stair1G);
    \path ([xshift=\onelen]stair1G) coordinate (stair1H);
    \path ([yshift=\onelen]stair1H) coordinate (stair1I);

    % bottom 'staircase' coordinates
    \path ([yshift=-\onelen]stair1A) coordinate (stair2J);
    \path ([yshift=-\plen]stair2J) coordinate (stair2A);
    \path ([xshift=\onelen]stair2A) coordinate (stair2B);
    \path ([yshift=\onelen]stair2B) coordinate (stair2C);
    \path ([xshift=\onelen]stair2C) coordinate (stair2D);
    \path ([yshift=\onelen]stair2D) coordinate (stair2E);
    \path ([xshift=\onelen]stair2E) coordinate (stair2F);
    \path ([yshift=\onelen]stair2F) coordinate (stair2G);
    \path ([xshift=\onelen]stair2G) coordinate (stair2H);
    \path ([yshift=\onelen]stair2H) coordinate (stair2I);
    
    % fill figures
    % \fill[black, opacity=0.5]  (t1A) -- (t1B) -- (M3D) -- cycle;% top black
    % \fill[red!50, opacity=0.5] (t2A) -- (t2B) -- (M3C) -- (t1B) -- (t1A) -- cycle;% top red
    % \fill[black, opacity=0.5]  (t3A) -- (t3B) -- (t2A) -- cycle;% bottom black
    % \fill[red!50, opacity=0.5] (t4A) -- (M3B) -- (t2B) -- (t3B) -- (t3A) -- cycle; % bottom red
    \fill[black,opacity=0.5]  (M3D) -- (stair1A) -- (stair1B) -- (stair1C) -- (stair1D) -- (stair1E) -- (stair1F) -- (stair1G) -- (stair1H) -- (stair1I) -- cycle;
    \fill[red!50,opacity=0.5] ([xshift=\flen-\onelen]stair1B) -- (stair1B) -- (stair1C) -- (stair1D) -- (stair1E) -- (stair1F) -- (stair1G) -- (stair1H) -- (stair1I) -- (M3C) -- cycle;
    \fill[black,opacity=0.5]  (stair2J) -- (stair2A) -- (stair2B) -- (stair2C) -- (stair2D) -- (stair2E) -- (stair2F) -- (stair2G) -- (stair2H) -- (stair2I) -- cycle;
    \fill[red!50,opacity=0.5] ([xshift=\flen-\onelen]stair2B) -- (stair2B) -- (stair2C) -- (stair2D) -- (stair2E) -- (stair2F) -- (stair2G) -- (stair2H) -- (stair2I) -- ([yshift=-\plen-\onelen]M3C) -- cycle;
    
    % draw brackets
    \draw[line width=1.5pt] ([xshift=\BrIn,yshift=-\BrCl]M3A) -- ([xshift=-\BrCl,yshift=-\BrCl]M3A) -- ([xshift=-\BrCl,yshift=\BrCl]M3D) -- ([xshift=\BrIn,yshift=\BrCl]M3D); %left bracket
    \draw[line width=1.5pt] ([xshift=-\BrIn,yshift=-\BrCl]M3B) -- ([xshift=\BrCl,yshift=-\BrCl]M3B) -- ([xshift=\BrCl,yshift=\BrCl]M3C) -- ([xshift=-\BrIn,yshift=\BrCl]M3C); %right bracket
    
    % U_{i,p,N}
    \path ([xshift=0.5*\onelen,yshift=-\plen+0.5*\onelen]M3D) coordinate (tlbA);
    \foreach \x in {0,...,7}{
    \foreach \y in {0,...,3}{
        \ifthenelse{{\y>\x}\OR{\y=\x}}{%
        % \fill[black, opacity=0.5] ([xshift={(\x-0.5)*\onelen},yshift={(\y-0.5)*\onelen}]tlbA) rectangle ([xshift={(\x+0.5)*\onelen},yshift={(\y+0.5)*\onelen}]tlbA);
        \fill[black] ([xshift={\x*\onelen},yshift={\y*\onelen}]tlbA) circle (1pt);
        }{%
        % \fill[red!50,opacity=0.5] ([xshift={(\x-0.5)*\onelen},yshift={(\y-0.5)*\onelen}]tlbA) rectangle ([xshift={(\x+0.5)*\onelen},yshift={(\y+0.5)*\onelen}]tlbA);%<do this if false>
        \fill[red] ([xshift={\x*\onelen},yshift={(\y*\onelen}]tlbA) circle (1pt);
        }%
    }}
    
    % U_{i_p,1,N}
    \path ([yshift=-\onelen]tlbA) coordinate (tlbB);
    \fill[red!50,opacity=0.5] (stair1A) rectangle ([xshift=\flen,yshift=-\onelen]stair1A);
    \foreach \x in {0,...,7}{\fill[red] ([xshift=\x*\onelen]tlbB) circle (1pt);}
    
    % Y_{i,p,N}
    \path ([yshift=-\plen]tlbB) coordinate (tlbC);
    \foreach \x in {0,...,7}{
    \foreach \y in {0,...,3}{
        \ifthenelse{{\y>\x}\OR{\y=\x}}{%
        \fill[black] ([xshift={\x*\onelen},yshift={\y*\onelen}]tlbC) circle (1pt);%<do this if true>
        }{%
        \fill[red] ([xshift={\x*\onelen},yshift={(\y*\onelen}]tlbC) circle (1pt);%<do this if false>
        }%
    }}

    % Y_{i_p,1,N}
    \path ([yshift=-\onelen]tlbC) coordinate (tlbD);
    \fill[red!50,opacity=0.5] (stair2A) rectangle ([xshift=\flen,yshift=-\onelen]stair2A);
    \foreach \x in {0,...,7}{\fill[red] ([xshift=\x*\onelen]tlbD) circle (1pt);}

    \foreach \dy in {0,-\plen-\onelen}{%
    % black diagonals
    \draw[dash pattern=on 1pt off 2pt, line width=0.1pt] ([xshift= 1/2*\onelen,yshift=\dy+5/2*\onelen]stair1A) -- ([xshift=-5/2*\onelen,yshift=\dy-1/2*\onelen]stair1I);
    \draw[dash pattern=on 1pt off 2pt, line width=0.1pt] ([xshift= 1/2*\onelen,yshift=\dy+3/2*\onelen]stair1A) -- ([xshift=-3/2*\onelen,yshift=\dy-1/2*\onelen]stair1I);
    \draw[dash pattern=on 1pt off 2pt, line width=0.1pt] ([xshift= 1/2*\onelen,yshift=\dy+1/2*\onelen]stair1A) -- ([xshift=-1/2*\onelen,yshift=\dy-1/2*\onelen]stair1I);
    % red diagonals
    \draw[red,dash pattern=on 1pt off 2pt, line width=0.1pt] ([xshift= 1/2*\onelen,yshift=\dy-1/2*\onelen]stair1A) -- ([xshift=1/2*\onelen,yshift=\dy-1/2*\onelen]stair1I);
    \draw[red,dash pattern=on 1pt off 2pt, line width=0.1pt] ([xshift= 3/2*\onelen,yshift=\dy-1/2*\onelen]stair1A) -- ([xshift=3/2*\onelen,yshift=\dy-1/2*\onelen]stair1I);
    \draw[red,dash pattern=on 1pt off 2pt, line width=0.1pt] ([xshift= 5/2*\onelen,yshift=\dy-1/2*\onelen]stair1A) -- ([xshift=5/2*\onelen,yshift=\dy-1/2*\onelen]stair1I);
    \draw[red,dash pattern=on 1pt off 2pt, line width=0.1pt] ([xshift= 7/2*\onelen,yshift=\dy-1/2*\onelen]stair1A) -- ([xshift=7/2*\onelen,yshift=\dy-1/2*\onelen]stair1I);
    \draw[red,dash pattern=on 1pt off 2pt, line width=0.1pt] ([xshift= 9/2*\onelen,yshift=\dy-1/2*\onelen]stair1A) -- ([xshift=7/2*\onelen,yshift=\dy-3/2*\onelen]stair1I);
    \draw[red,dash pattern=on 1pt off 2pt, line width=0.1pt] ([xshift=11/2*\onelen,yshift=\dy-1/2*\onelen]stair1A) -- ([xshift=7/2*\onelen,yshift=\dy-5/2*\onelen]stair1I);
    \draw[red,dash pattern=on 1pt off 2pt, line width=0.1pt] ([xshift=13/2*\onelen,yshift=\dy-1/2*\onelen]stair1A) -- ([xshift=7/2*\onelen,yshift=\dy-7/2*\onelen]stair1I);
    }
    
    % draw dividers
    \draw[line width=0.1pt] ([yshift=-\plen-\onelen]M3D) -- ([yshift=-\plen-\onelen]M3C);
    \draw[line width=0.1pt] ([yshift=-\plen]M3D) -- ([yshift=-\plen]M3C);
    \draw[line width=1pt] ([yshift=\onelen]M3A) -- ([yshift=\onelen]M3B);

    % ---------------- drawing right brace and matrix ----------------
    % brace below matrix
    \path ([xshift=-0.5\BrIn,yshift=-2\BrCl]M3A) coordinate (Brace3L);
    \path ([xshift=0.5\BrIn,yshift=-2\BrCl]M3B) coordinate (Brace3R);
    \draw[decorate, decoration={calligraphic brace, amplitude=3pt, mirror, aspect=0.25},line width=1pt] (Brace3L) -- (Brace3R);
    % matrix below brace
    \node (mat3) at ($(Brace3L)!0.25!(Brace3R) + (0,-3pt)$) {};
    \node[below] at (mat3.center) {$\begin{bmatrix}
        U_{\hat{i},p,f}\\U_{\hat{i}_p,1,f}\\Y_{\hat{i},p,f}\\ \hline \widehat{Y}_{\hat{i}_p,1,f}
    \end{bmatrix}$};
    % tag & label (on RHS of column)
    \node[below] at ([xshift=2.13cm,yshift=-0.9cm]mat3.center) {$\refstepcounter{equation}(\theequation)\label{eq:CL_DeePC_no_IVs}$};
    % brace to the right
    % \path ([xshift=0.7cm+0.05cm]mat3.center |- Brace1T) coordinate (Brace3T);%([xshift=0.7cm,yshift=-4pt]mat3.center) coordinate (Brace3T);
    % \path (Brace3T |- Brace1B) coordinate (Brace3B);
    % \draw[pen colour=gray!60,decorate, decoration={calligraphic brace, amplitude=3pt, aspect=0.2},line width=0.75pt] (Brace3T) -- (Brace3B);
    % % Zf
    % \node (Zf) at ($(Brace3T)!0.2!(Brace3B) + (3pt,-1.5pt)$) {};
    % \node[right,text=gray!100] at ([yshift=-0.18cm]Zf.center) {\scriptsize{\shortstack{$=$\\$\Phi_{\hat{i},1,f}$}}};
    
    % ======================= length indicators ======================= 
    \draw[|-|] ([yshift=\onelen]M1D) -- node[above] {\scriptsize$N$} ([yshift=\onelen]M1C);
    \draw[|-|] ([yshift=\onelen]M2D) -- node[above] {\scriptsize$f$} ([yshift=\onelen]M2C);
    \draw[|-|] ([yshift=\onelen]M3D) -- node[above] {\scriptsize$f$} ([yshift=\onelen]M3C);
    \draw[|-|] ([xshift=\onelen]M3C) -- node[right] {\scriptsize$pr$} ([xshift=\onelen,yshift=\onelen]t2B);
    \draw[|-|] ([xshift=\onelen,yshift=\onelen]t2B) -- node[right] {\scriptsize$r$} ([xshift=\onelen]t2B);
    \draw[|-|] ([xshift=\onelen]t2B) -- node[right] {\scriptsize$pl$} ([xshift=\onelen,yshift=\onelen]M3B);
    \draw[|-|] ([xshift=\onelen,yshift=\onelen]M3B) -- node[right] {\scriptsize$l$} ([xshift=\onelen]M3B);
\end{tikzpicture}
%          \caption{\ac{CL-DeePC} involves $f$ sequential applications of a step-ahead predictor obtained from regular \ac{DeePC} with $f=1$ (see also Fig.~(b)), resulting in the dashed block-anti diagonals with the same $u_k$ or $y_k$ on the right hand side.}
%          \label{fig:CL-DeePC}
%      \end{subfigure}
%      \hfill
%      \begin{subfigure}[b]{\columnwidth}
%          \centering
%          \begin{tikzpicture}
    % defining constants
    \def\stepSize{0.25}
    \def\Nnum{9}
    \def\fnum{8}
    \def\pnum{4}
    
    % Defining lengths
    % \newlength{\onelen}
    \setlength{\onelen}{\stepSize cm}
    % \newlength{\BrCl}
    \setlength{\BrCl}{0.075cm}
    % \newlength{\BrIn}
    \setlength{\BrIn}{0.15cm}
    % \newlength{\plen}
    \setlength{\plen}{1cm}%{\pnum\stepSize cm}
    % \newlength{\flen}
    \setlength{\flen}{2cm}%{\fnum*\stepSize cm}
    % \newlength{\Nlen}
    \setlength{\Nlen}{2.25cm}%{\Nnum\stepSize cm}%should be 2*p+one
    % \newlength{\MatClearance}
    \setlength{\MatClearance}{0.3cm}
    
    % grid lines for guidance
    % \draw[gray,step=0.5] (-0,-7) grid (8,3);

    % ======================= drawing data matrix =======================
    \path (0,2\plen+\onelen) coordinate (M1D);
    \path ([yshift=-2\plen-2\flen]M1D) coordinate (M1A);
    \path ([xshift=\Nlen]M1A) coordinate (M1B);
    \path ([yshift=2*\plen+2*\flen]M1B) coordinate (M1C);
    \draw[line width=1.5pt] ([xshift=\BrIn,yshift=-\BrCl]M1A) -- ([xshift=-\BrCl,yshift=-\BrCl]M1A) -- ([xshift=-\BrCl,yshift=\BrCl]M1D) -- ([xshift=\BrIn,yshift=\BrCl]M1D); %left bracket
    \draw[line width=1.5pt] ([xshift=-\BrIn,yshift=-\BrCl]M1B) -- ([xshift=\BrCl,yshift=-\BrCl]M1B) -- ([xshift=\BrCl,yshift=\BrCl]M1C) -- ([xshift=-\BrIn,yshift=\BrCl]M1C); %right bracket
    \draw[line width=1pt] ([yshift=\flen]M1A) -- ([yshift=\flen]M1B); % dividing matrix into blocks
    \fill[black, opacity=0.5] (M1A) rectangle (M1C);
    \foreach \x in {0,...,8} { % drawing black dots
    \foreach \y in {-15,...,8} {
      \fill ( {(\x+0.5)*\onelen}, {(\y+0.5)*\onelen} ) circle (1pt);
    }}
    \draw[line width=0.1pt] ([yshift=-\plen-\flen]M1D) -- ([yshift=-\plen-\flen]M1C);
    \draw[line width=0.1pt] ([yshift=-\plen]M1D) -- ([yshift=-\plen]M1C);

    % coordinates for diagonals
    \path ([yshift=-\plen]M1D) coordinate (M1stair1A);
    \path ([xshift=\Nlen]M1D) coordinate (M1stair1I);
    % black diagonals
    % \foreach \i in {0,-1}{
    % \foreach \dy in {1,...,18}{
    % \ifthenelse{\dy<9.5}{
    % % This code will be executed if \dy<9.5
    %     \draw[dash pattern=on 1pt off 2pt, line width=0.1pt] ([xshift= 0.5*\onelen,yshift=\i*(\plen+\flen)-(\dy+0.5)*\onelen]M1D) -- ([xshift= (\dy+0.5)*\onelen,yshift=\i*(\plen+\flen)-0.5*\onelen]M1D);
    % }{
    % % This code will be executed if \dy>=10
    %     \ifthenelse{\dy<12.5}{
    %         % This code will be executed if \dy<=12
    %         \draw[dash pattern=on 1pt off 2pt, line width=0.1pt] ([xshift= 0.5*\onelen,yshift=\i*(\plen+\flen)-(\dy+0.5)*\onelen]M1D) -- ([xshift= -0.5*\onelen,yshift=\i*(\plen+\flen)-(\dy-7.5)*\onelen]M1C);
    %         }{
    %         % This code will be executed if \dy>=13
    %         \draw[dash pattern=on 1pt off 2pt, line width=0.1pt] ([xshift= (\dy-10.5)*\onelen,yshift=\i*(\plen+\flen)-\plen-\flen+0.5*\onelen]M1D) -- ([xshift= -0.5*\onelen,yshift=\i*(\plen+\flen)-(\dy-7.5)*\onelen]M1C);
    %     }
    % }
    % }}
    
    % ---------------- drawing left brace and matrix ----------------
    \path ([xshift=-0.5\BrIn,yshift=-2\BrCl]M1A) coordinate (Brace1L);
    \path ([xshift=0.5\BrIn,yshift=-2\BrCl]M1B) coordinate (Brace1R);
    \draw[decorate, decoration={calligraphic brace, amplitude=3pt, mirror, aspect=0.75},line width=1pt] (Brace1L) -- (Brace1R);
    \node (mat1) at ($(Brace1L)!0.75!(Brace1R) + (0,-3pt)$) {};
    \node[below] at (mat1.center) {$\begin{bmatrix}
        U_{i,p,N}\\U_{i_p,f,N}\\Y_{i,p,N}\\ \hline Y_{i_p,f,N}
    \end{bmatrix}$};
    % brace to the left
    % \path ([xshift=-0.75cm-0.05cm,yshift=-4pt]mat1.center) coordinate (Brace1T);
    % \path ([yshift=-1.73cm]Brace1T) coordinate (Brace1B);
    % \draw[pen colour=gray!60,decorate, decoration={calligraphic brace, amplitude=3pt, mirror, aspect=0.2},line width=0.75pt] (Brace1T) -- (Brace1B);
    % % Zp
    % \node (Zp) at ($(Brace1T)!0.2!(Brace1B) + (-3pt,-0.8pt)$) {};
    % \node[left,text=gray!100,align=center] at ([xshift=0.5mm,yshift=-0.18cm]Zp.center) {\scriptsize{\shortstack{$=$\\$\Phi_{i,f,N}$}}};
  
    % ======================= drawing G =======================
    % useful coordinates
    \path ([xshift=\MatClearance,yshift=-0.5\Nlen-\plen-\flen]M1C) coordinate (M2A);
    \path ([xshift=\onelen]M2A) coordinate (M2B);
    \path ([yshift=\Nlen]M2B) coordinate (M2C);
    \path ([xshift=-\onelen]M2C) coordinate (M2D);

    % brackets
    \draw[line width=1.5pt] ([xshift=0.5\BrIn,yshift=-\BrCl]M2A) -- ([xshift=-\BrCl,yshift=-\BrCl]M2A) -- ([xshift=-\BrCl,yshift=\BrCl]M2D) -- ([xshift=0.5\BrIn,yshift=\BrCl]M2D); %left bracket
    \draw[line width=1.5pt] ([xshift=-0.5\BrIn,yshift=-\BrCl]M2B) -- ([xshift=\BrCl,yshift=-\BrCl]M2B) -- ([xshift=\BrCl,yshift=\BrCl]M2C) -- ([xshift=-0.5\BrIn,yshift=\BrCl]M2C); %right bracket
    
    % red fill
    \fill[red!50,opacity=0.5] (M2A) rectangle (M2C);

    % drawing red dots
    \foreach \x in {0} {
    \foreach \y in {0,...,8} {
      \fill[red] ([xshift=(\x+0.5)*\onelen,yshift=(\y+0.5)*\onelen]M2A) circle (1pt);
    }}

    % drawing middle brace and matrix
    \coordinate (Brace2L) at ([xshift=-0.5\BrIn]M2A |- Brace1L);
    \coordinate (Brace2R) at ([xshift=0.5\BrIn]M2B |- Brace1L);
    \draw[decorate, decoration={calligraphic brace, amplitude=3pt, mirror, aspect=0.5},line width=1pt] (Brace2L) -- (Brace2R);
    \node (mat1) at ($(Brace2L)!0.5!(Brace2R) + (0,-3pt)$) {};
    \node[below] at ([yshift=-1.05cm]mat1.center) {$g$};
    
    % ======================= drawing equal sign ======================= 
    \path ([xshift=\MatClearance*3/4,yshift=\Nlen/2-0.1cm]M2B) coordinate (EqA);
    \path ([xshift=0.4cm]EqA) coordinate (EqB);
    \path ([yshift=0.2cm]EqB) coordinate (EqC);
    \path ([xshift=-0.4cm]EqC) coordinate (EqD);
    \draw[line width = 1.5 pt] (EqA) -- (EqB);
    \draw[line width = 1.5 pt] (EqD) -- (EqC);

    % ---------------- equation sign below ----------------
    \node[below] at ([xshift=2\onelen,yshift=-1.05cm]mat1.center) {$=$};

    % ======================= drawing RHS =======================
    % inside of matrix
    \path ([xshift=0.75\MatClearance]EqB |- M1A) coordinate (M3A);
    \path ([xshift=\onelen]M3A) coordinate (M3B);
    \path ([yshift=2\plen+2\flen]M3B) coordinate (M3C);
    \path ([xshift=-\onelen]M3C) coordinate (M3D);
    
    % fill figures
    \fill[black,opacity=0.5]  (M3D) -- ([yshift=-\plen]M3D) -- ([yshift=-\plen]M3C) -- (M3C) -- cycle;
    \fill[red!50,opacity=0.5] ([yshift=-\plen]M3D) -- ([yshift=-\plen-\flen]M3D) -- ([yshift=-\plen-\flen]M3C) -- ([yshift=-\plen]M3C) -- cycle;
    \fill[black,opacity=0.5] ([yshift=\flen]M3A) -- ([yshift=\flen]M3B) -- ([yshift=\flen+\plen]M3B) -- ([yshift=\flen+\plen]M3A) -- cycle;
    \fill[red!50,opacity=0.5]  (M3A) -- (M3B) -- ([yshift=\flen]M3B) -- ([yshift=\flen]M3A) -- cycle;
    
    % draw brackets
    \draw[line width=1.5pt] ([xshift=0.5\BrIn,yshift=-\BrCl]M3A) -- ([xshift=-\BrCl,yshift=-\BrCl]M3A) -- ([xshift=-\BrCl,yshift=\BrCl]M3D) -- ([xshift=0.5\BrIn,yshift=\BrCl]M3D); %left bracket
    \draw[line width=1.5pt] ([xshift=-0.5\BrIn,yshift=-\BrCl]M3B) -- ([xshift=\BrCl,yshift=-\BrCl]M3B) -- ([xshift=\BrCl,yshift=\BrCl]M3C) -- ([xshift=-0.5\BrIn,yshift=\BrCl]M3C); %right bracket
    
    % U_{i,p,1}
    \foreach \dy in {0,...,-3}{
        \fill[black] ([xshift=0.5\onelen,yshift=(\dy-0.5)*\onelen]M3D) circle (1pt);
    }

    % U_{i_p,f,1}
    \foreach \dy in {-4,...,-11}{
        \fill[red] ([xshift=0.5\onelen,yshift=(\dy-0.5)*\onelen]M3D) circle (1pt);
    }
    
    % Y_{i,p,1}
    \foreach \dy in {0,...,-3}{
        \fill[black] ([xshift=0.5\onelen,yshift=(\dy-0.5)*\onelen-\plen-\flen]M3D) circle (1pt);
    }

    % Y_{i_p,f,1}
    \foreach \dy in {-4,...,-11}{
        \fill[red] ([xshift=0.5\onelen,yshift=(\dy-0.5)*\onelen-\plen-\flen]M3D) circle (1pt);
    }

    % draw dividers
    \draw[line width=0.1pt] ([yshift=-\plen]M3D) -- ([yshift=-\plen]M3C);
    \draw[line width=0.1pt] ([yshift=-\plen-\flen]M3D) -- ([yshift=-\plen-\flen]M3C);
    \draw[line width=1pt] ([yshift=\flen]M3A) -- ([yshift=\flen]M3B);

    % ---------------- drawing right brace and matrix ----------------
    % brace below matrix
    \path ([xshift=-0.5\BrIn,yshift=-2\BrCl]M3A) coordinate (Brace3L);
    \path ([xshift=0.5\BrIn,yshift=-2\BrCl]M3B) coordinate (Brace3R);
    \draw[decorate, decoration={calligraphic brace, amplitude=3pt, mirror, aspect=0.5},line width=1pt] (Brace3L) -- (Brace3R);
    % matrix below brace
    \node (mat3) at ($(Brace3L)!0.5!(Brace3R) + (0,-3pt)$) {};
    \node[below] at ([xshift=0.35cm]mat3.center) {$\begin{bmatrix}
        U_{\hat{i},p,1}\\U_{\hat{i}_p,f,1}\\Y_{\hat{i},p,1}\\ \hline \widehat{Y}_{\hat{i}_p,f,1}
    \end{bmatrix}$};
    % tag & label (on RHS of column)
    \node[below left] at (8.35cm,-5cm) {$\refstepcounter{equation}(\theequation)\label{eq:regular_DeePC_no_IVs}$};
    % brace to the right
    % \path ([xshift=1.1cm]mat3.center |- Brace1T) coordinate (Brace3T);%([xshift=0.7cm,yshift=-4pt]mat3.center) coordinate (Brace3T);
    % \path (Brace3T |- Brace1B) coordinate (Brace3B);
    % \draw[pen colour=gray!60,decorate, decoration={calligraphic brace, amplitude=3pt, aspect=0.2},line width=0.75pt] (Brace3T) -- (Brace3B);
    % % Zf
    % \node (Zf) at ($(Brace3T)!0.2!(Brace3B) + (3pt,-1.5pt)$) {};
    % \node[right,text=gray!100] at ([yshift=-0.18cm]Zf.center) {\scriptsize{\shortstack{$=$\\$\Phi_{\hat{i},f,1}$}}};
    
    % ======================= length indicators ======================= 
    \draw[|-|] ([yshift=\onelen]M1D) -- node[above] {\scriptsize$N$} ([yshift=\onelen]M1C);
    \draw[|-|] ([xshift=\onelen]M3C) -- node[right] {\scriptsize$pr$} ([xshift=\onelen,yshift=-\plen]M3C);
    \draw[|-|] ([xshift=\onelen,yshift=-\plen]M3C) -- node[right] {\scriptsize$fr$} ([xshift=\onelen,yshift=-\plen-\flen]M3C);
    \draw[|-|] ([xshift=\onelen,yshift=-\plen-\flen]M3C) -- node[right] {\scriptsize$pl$} ([xshift=\onelen,yshift=\flen]M3B);
    \draw[|-|] ([xshift=\onelen,yshift=\flen]M3B) -- node[right] {\scriptsize$fl$} ([xshift=\onelen]M3B);
\end{tikzpicture}
%          \caption{In regular \ac{DeePC} a multi-step ahead predictor of prediction length $f$ is formed directly by taking a single linear combination $g$ of past input-output data. Past data on the right-hand side encodes information on an initial state.}
%         \label{fig:regular-DeePC}
%      \end{subfigure}
%      \caption{Visualization of known (black) and unknown (red) variables in \ac{CL-DeePC} (a) and regular \ac{DeePC} from~\citep{Coulson2019} (b) without \ac{IVs}. Each dot represents an input $u_k\in\mathbb{R}^r$, output $y_k\in\mathbb{R}^l$, or element of a vector $g$.}
% \end{figure*}
\begin{figure}[b!]
\centering
\begin{tikzpicture}
    % defining constants
    \def\stepSize{0.25}
    \def\Nnum{9}
    \def\fnum{8}
    \def\pnum{4}
    
    % Defining lengths
    \newlength{\onelen}
    \setlength{\onelen}{\stepSize cm}
    \newlength{\BrCl}
    \setlength{\BrCl}{0.075cm}
    \newlength{\BrIn}
    \setlength{\BrIn}{0.15cm}
    \newlength{\plen}
    \setlength{\plen}{1cm}%{\pnum\stepSize cm}
    \newlength{\flen}
    \setlength{\flen}{2cm}%{\fnum*\stepSize cm}
    \newlength{\Nlen}
    \setlength{\Nlen}{2.25cm}%{\Nnum\stepSize cm}%should be 2*p+one
    \newlength{\MatClearance}
    \setlength{\MatClearance}{0.3cm}
    
    % grid lines for guidance
    % \draw[gray,step=0.5] (-0,-3) grid (8,3);

    % ======================= drawing data matrix =======================
    \path (0,-\onelen) coordinate (M1A);
    \path ([xshift=\Nlen]M1A) coordinate (M1B);
    \path ([yshift=2*\plen+2*\onelen]M1B) coordinate (M1C);
    \path ([xshift=-\Nlen]M1C) coordinate (M1D);
    \draw[line width=1.5pt] ([xshift=\BrIn,yshift=-\BrCl]M1A) -- ([xshift=-\BrCl,yshift=-\BrCl]M1A) -- ([xshift=-\BrCl,yshift=\BrCl]M1D) -- ([xshift=\BrIn,yshift=\BrCl]M1D); %left bracket
    \draw[line width=1.5pt] ([xshift=-\BrIn,yshift=-\BrCl]M1B) -- ([xshift=\BrCl,yshift=-\BrCl]M1B) -- ([xshift=\BrCl,yshift=\BrCl]M1C) -- ([xshift=-\BrIn,yshift=\BrCl]M1C); %right bracket
    \draw[line width=1pt] ([yshift=\onelen]M1A) -- ([yshift=\onelen]M1B); % dividing matrix into blocks
    \fill[black, opacity=0.5] (M1A) rectangle (M1C);
    \foreach \x in {0,...,8} { % drawing black dots
    \foreach \y in {-1,...,8} {
      \fill ( {(\x+0.5)*\onelen}, {(\y+0.5)*\onelen} ) circle (1pt);
    }}
    \draw[line width=0.1pt] ([yshift=-\plen-\onelen]M1D) -- ([yshift=-\plen-\onelen]M1C);
    \draw[line width=0.1pt] ([yshift=-\plen]M1D) -- ([yshift=-\plen]M1C);

    % coordinates for diagonals
    \path ([yshift=-\plen]M1D) coordinate (M1stair1A);
    \path ([xshift=\plen]M1D) coordinate (M1stair1I);
    % black diagonals
    % \foreach \i in {0,-1}{
    % \foreach \dy in {1,...,11}{
    % \ifthenelse{\dy<5.5}{
    % % This code will be executed if \dy<9.5
    %     \draw[dash pattern=on 1pt off 2pt, line width=0.1pt] ([xshift= 0.5*\onelen,yshift=\i*(\plen+\onelen)-(\dy+0.5)*\onelen]M1D) -- ([xshift= (\dy+0.5)*\onelen,yshift=\i*(\plen+\onelen)-0.5*\onelen]M1D);
    % }{
    % % This code will be executed if \dy>=10
    %     \ifthenelse{\dy<8.5}{
    %         % This code will be executed if \dy<=12
    %         \draw[dash pattern=on 1pt off 2pt, line width=0.1pt] ([xshift= (\dy-3.5)*\onelen,yshift=\i*(\plen+\onelen)-\plen-\onelen+0.5*\onelen]M1D) -- ([xshift=(\dy-8.5)*\onelen,yshift=\i*(\plen+\onelen)-0.5\onelen]M1C);
    %         }{
    %         % This code will be executed if \dy>=13
    %         \draw[dash pattern=on 1pt off 2pt, line width=0.1pt] ([xshift= (\dy-3.5)*\onelen,yshift=(\i-1)*(\plen+\onelen)+0.5\onelen]M1D) -- ([xshift= -0.5*\onelen,yshift=\i*(\plen+\onelen)-(\dy-7.5)*\onelen]M1C);
    %     }
    % }
    % }}
    
    % ---------------- drawing left brace and matrix ----------------
    \path ([xshift=-0.5\BrIn,yshift=-2\BrCl]M1A) coordinate (Brace1L);
    \path ([xshift=0.5\BrIn,yshift=-2\BrCl]M1B) coordinate (Brace1R);
    \draw[decorate, decoration={calligraphic brace, amplitude=3pt, mirror, aspect=0.75},line width=1pt] (Brace1L) -- (Brace1R);
    \node (mat1) at ($(Brace1L)!0.75!(Brace1R) + (0,-3pt)$) {};
    \node[below] at (mat1.center) {$\begin{bmatrix}
        U_{i,p,N}\\U_{i_p,1,N}\\Y_{i,p,N}\\ \hline Y_{i_p,1,N}
    \end{bmatrix}$};
    % brace to the left
    % \path ([xshift=-0.75cm-0.05cm,yshift=-4pt]mat1.center) coordinate (Brace1T);
    % \path ([yshift=-1.73cm]Brace1T) coordinate (Brace1B);
    % \draw[pen colour=gray!60,decorate, decoration={calligraphic brace, amplitude=3pt, mirror, aspect=0.2},line width=0.75pt] (Brace1T) -- (Brace1B);
    % % Zp
    % \node (Zp) at ($(Brace1T)!0.2!(Brace1B) + (-3pt,-0.8pt)$) {};
    % \node[left,text=gray!100,align=center] at ([xshift=0.5mm,yshift=-0.18cm]Zp.center) {\scriptsize{\shortstack{$=$\\$\Phi_{i,1,N}$}}};
  
    % ======================= drawing G =======================
    % useful coordinates
    \path ([xshift=\MatClearance,yshift=\onelen]M1B) coordinate (M2A);
    \path ([xshift=\flen]M2A) coordinate (M2B);
    \path ([yshift=\Nlen]M2B) coordinate (M2C);
    \path ([xshift=-\flen]M2C) coordinate (M2D);

    % brackets
    \draw[line width=1.5pt] ([xshift=\BrIn,yshift=-\BrCl]M2A) -- ([xshift=-\BrCl,yshift=-\BrCl]M2A) -- ([xshift=-\BrCl,yshift=\BrCl]M2D) -- ([xshift=\BrIn,yshift=\BrCl]M2D); %left bracket
    \draw[line width=1.5pt] ([xshift=-\BrIn,yshift=-\BrCl]M2B) -- ([xshift=\BrCl,yshift=-\BrCl]M2B) -- ([xshift=\BrCl,yshift=\BrCl]M2C) -- ([xshift=-\BrIn,yshift=\BrCl]M2C); %right bracket
    
    % red fill
    \fill[red!50,opacity=0.5] (M2A) rectangle (M2C);

    % drawing red dots
    \foreach \x in {0,...,7} {
    \foreach \y in {0,...,8} {
      \fill[red] ([xshift=(\x+0.5)*\onelen,yshift=(\y+0.5)*\onelen]M2A) circle (1pt);%{(\x+0.5)*\onelen+\Nlen+0.5cm}, {(\y+0.5)*\onelen}
    }}

    % dividers
    \foreach \x in {1,...,7}{\draw[line width=0.1pt] ([xshift=\x*\onelen]M2A) -- ([xshift=\x*\onelen]M2D);}

    % drawing middle brace and matrix
    \coordinate (Brace2L) at ([xshift=-0.5\BrIn]M2A |- Brace1L);
    \coordinate (Brace2R) at ([xshift=0.5\BrIn]M2B |- Brace1L);
    \draw[decorate, decoration={calligraphic brace, amplitude=3pt, mirror, aspect=0.5},line width=1pt] (Brace2L) -- (Brace2R);
    \node (mat1) at ($(Brace2L)!0.5!(Brace2R) + (0,-3pt)$) {};
    \node[below] at ([yshift=-0.75cm]mat1.center) {$\underbrace{
    \begin{bmatrix}
        g_1 & g_2 & \cdots & g_f
    \end{bmatrix}}_{= G}$};
    
    % ======================= drawing equal sign ======================= 
    \path ([xshift=\MatClearance*3/4,yshift=\Nlen/2-0.1cm]M2B) coordinate (EqA);
    \path ([xshift=0.4cm]EqA) coordinate (EqB);
    \path ([yshift=0.2cm]EqB) coordinate (EqC);
    \path ([xshift=-0.4cm]EqC) coordinate (EqD);
    \draw[line width = 1.5 pt] (EqA) -- (EqB);
    \draw[line width = 1.5 pt] (EqD) -- (EqC);

    % ---------------- equation sign below ----------------
    \node[below] at ([xshift=5.65\onelen,yshift=-1.05cm]mat1.center) {$=$};

    % ======================= drawing RHS =======================
    % inside of matrix
    \path ([xshift=\MatClearance*3/4,yshift=-\Nlen/2+0.1cm-\onelen]EqB) coordinate (M3A);
    \path ([xshift=\flen]M3A) coordinate (M3B);
    \path ([yshift=2*\plen+2*\onelen]M3B) coordinate (M3C);
    \path ([xshift=-\flen]M3C) coordinate (M3D);
    % top left black triangle
    \path ([yshift=-\plen-\onelen/2]M3D) coordinate (t1A);
    \path ([xshift=\plen+\onelen/2]M3D) coordinate (t1B);
    % top red trapezoid
    \path ([yshift=-\onelen/2]t1A) coordinate (t2A);
    \path ([xshift=\flen]t2A) coordinate (t2B);
    % bottom black triangle
    \path ([yshift=-\plen-\onelen/2]t2A) coordinate (t3A);
    \path ([xshift=\plen+\onelen/2]t2A) coordinate (t3B);
    % bottom red trapezoid
    \path ([yshift=-\onelen/2]t3A) coordinate (t4A);

    % top 'staircase' coordinates
    \path ([yshift=-\plen]M3D) coordinate (stair1A);
    \path ([xshift=\onelen]stair1A) coordinate (stair1B);
    \path ([yshift=\onelen]stair1B) coordinate (stair1C);
    \path ([xshift=\onelen]stair1C) coordinate (stair1D);
    \path ([yshift=\onelen]stair1D) coordinate (stair1E);
    \path ([xshift=\onelen]stair1E) coordinate (stair1F);
    \path ([yshift=\onelen]stair1F) coordinate (stair1G);
    \path ([xshift=\onelen]stair1G) coordinate (stair1H);
    \path ([yshift=\onelen]stair1H) coordinate (stair1I);

    % bottom 'staircase' coordinates
    \path ([yshift=-\onelen]stair1A) coordinate (stair2J);
    \path ([yshift=-\plen]stair2J) coordinate (stair2A);
    \path ([xshift=\onelen]stair2A) coordinate (stair2B);
    \path ([yshift=\onelen]stair2B) coordinate (stair2C);
    \path ([xshift=\onelen]stair2C) coordinate (stair2D);
    \path ([yshift=\onelen]stair2D) coordinate (stair2E);
    \path ([xshift=\onelen]stair2E) coordinate (stair2F);
    \path ([yshift=\onelen]stair2F) coordinate (stair2G);
    \path ([xshift=\onelen]stair2G) coordinate (stair2H);
    \path ([yshift=\onelen]stair2H) coordinate (stair2I);
    
    % fill figures
    % \fill[black, opacity=0.5]  (t1A) -- (t1B) -- (M3D) -- cycle;% top black
    % \fill[red!50, opacity=0.5] (t2A) -- (t2B) -- (M3C) -- (t1B) -- (t1A) -- cycle;% top red
    % \fill[black, opacity=0.5]  (t3A) -- (t3B) -- (t2A) -- cycle;% bottom black
    % \fill[red!50, opacity=0.5] (t4A) -- (M3B) -- (t2B) -- (t3B) -- (t3A) -- cycle; % bottom red
    \fill[black,opacity=0.5]  (M3D) -- (stair1A) -- (stair1B) -- (stair1C) -- (stair1D) -- (stair1E) -- (stair1F) -- (stair1G) -- (stair1H) -- (stair1I) -- cycle;
    \fill[red!50,opacity=0.5] ([xshift=\flen-\onelen]stair1B) -- (stair1B) -- (stair1C) -- (stair1D) -- (stair1E) -- (stair1F) -- (stair1G) -- (stair1H) -- (stair1I) -- (M3C) -- cycle;
    \fill[black,opacity=0.5]  (stair2J) -- (stair2A) -- (stair2B) -- (stair2C) -- (stair2D) -- (stair2E) -- (stair2F) -- (stair2G) -- (stair2H) -- (stair2I) -- cycle;
    \fill[red!50,opacity=0.5] ([xshift=\flen-\onelen]stair2B) -- (stair2B) -- (stair2C) -- (stair2D) -- (stair2E) -- (stair2F) -- (stair2G) -- (stair2H) -- (stair2I) -- ([yshift=-\plen-\onelen]M3C) -- cycle;
    
    % draw brackets
    \draw[line width=1.5pt] ([xshift=\BrIn,yshift=-\BrCl]M3A) -- ([xshift=-\BrCl,yshift=-\BrCl]M3A) -- ([xshift=-\BrCl,yshift=\BrCl]M3D) -- ([xshift=\BrIn,yshift=\BrCl]M3D); %left bracket
    \draw[line width=1.5pt] ([xshift=-\BrIn,yshift=-\BrCl]M3B) -- ([xshift=\BrCl,yshift=-\BrCl]M3B) -- ([xshift=\BrCl,yshift=\BrCl]M3C) -- ([xshift=-\BrIn,yshift=\BrCl]M3C); %right bracket
    
    % U_{i,p,N}
    \path ([xshift=0.5*\onelen,yshift=-\plen+0.5*\onelen]M3D) coordinate (tlbA);
    \foreach \x in {0,...,7}{
    \foreach \y in {0,...,3}{
        \ifthenelse{{\y>\x}\OR{\y=\x}}{%
        % \fill[black, opacity=0.5] ([xshift={(\x-0.5)*\onelen},yshift={(\y-0.5)*\onelen}]tlbA) rectangle ([xshift={(\x+0.5)*\onelen},yshift={(\y+0.5)*\onelen}]tlbA);
        \fill[black] ([xshift={\x*\onelen},yshift={\y*\onelen}]tlbA) circle (1pt);
        }{%
        % \fill[red!50,opacity=0.5] ([xshift={(\x-0.5)*\onelen},yshift={(\y-0.5)*\onelen}]tlbA) rectangle ([xshift={(\x+0.5)*\onelen},yshift={(\y+0.5)*\onelen}]tlbA);%<do this if false>
        \fill[red] ([xshift={\x*\onelen},yshift={(\y*\onelen}]tlbA) circle (1pt);
        }%
    }}
    
    % U_{i_p,1,N}
    \path ([yshift=-\onelen]tlbA) coordinate (tlbB);
    \fill[red!50,opacity=0.5] (stair1A) rectangle ([xshift=\flen,yshift=-\onelen]stair1A);
    \foreach \x in {0,...,7}{\fill[red] ([xshift=\x*\onelen]tlbB) circle (1pt);}
    
    % Y_{i,p,N}
    \path ([yshift=-\plen]tlbB) coordinate (tlbC);
    \foreach \x in {0,...,7}{
    \foreach \y in {0,...,3}{
        \ifthenelse{{\y>\x}\OR{\y=\x}}{%
        \fill[black] ([xshift={\x*\onelen},yshift={\y*\onelen}]tlbC) circle (1pt);%<do this if true>
        }{%
        \fill[red] ([xshift={\x*\onelen},yshift={(\y*\onelen}]tlbC) circle (1pt);%<do this if false>
        }%
    }}

    % Y_{i_p,1,N}
    \path ([yshift=-\onelen]tlbC) coordinate (tlbD);
    \fill[red!50,opacity=0.5] (stair2A) rectangle ([xshift=\flen,yshift=-\onelen]stair2A);
    \foreach \x in {0,...,7}{\fill[red] ([xshift=\x*\onelen]tlbD) circle (1pt);}

    \foreach \dy in {0,-\plen-\onelen}{%
    % black diagonals
    \draw[dash pattern=on 1pt off 2pt, line width=0.1pt] ([xshift= 1/2*\onelen,yshift=\dy+5/2*\onelen]stair1A) -- ([xshift=-5/2*\onelen,yshift=\dy-1/2*\onelen]stair1I);
    \draw[dash pattern=on 1pt off 2pt, line width=0.1pt] ([xshift= 1/2*\onelen,yshift=\dy+3/2*\onelen]stair1A) -- ([xshift=-3/2*\onelen,yshift=\dy-1/2*\onelen]stair1I);
    \draw[dash pattern=on 1pt off 2pt, line width=0.1pt] ([xshift= 1/2*\onelen,yshift=\dy+1/2*\onelen]stair1A) -- ([xshift=-1/2*\onelen,yshift=\dy-1/2*\onelen]stair1I);
    % red diagonals
    \draw[red,dash pattern=on 1pt off 2pt, line width=0.1pt] ([xshift= 1/2*\onelen,yshift=\dy-1/2*\onelen]stair1A) -- ([xshift=1/2*\onelen,yshift=\dy-1/2*\onelen]stair1I);
    \draw[red,dash pattern=on 1pt off 2pt, line width=0.1pt] ([xshift= 3/2*\onelen,yshift=\dy-1/2*\onelen]stair1A) -- ([xshift=3/2*\onelen,yshift=\dy-1/2*\onelen]stair1I);
    \draw[red,dash pattern=on 1pt off 2pt, line width=0.1pt] ([xshift= 5/2*\onelen,yshift=\dy-1/2*\onelen]stair1A) -- ([xshift=5/2*\onelen,yshift=\dy-1/2*\onelen]stair1I);
    \draw[red,dash pattern=on 1pt off 2pt, line width=0.1pt] ([xshift= 7/2*\onelen,yshift=\dy-1/2*\onelen]stair1A) -- ([xshift=7/2*\onelen,yshift=\dy-1/2*\onelen]stair1I);
    \draw[red,dash pattern=on 1pt off 2pt, line width=0.1pt] ([xshift= 9/2*\onelen,yshift=\dy-1/2*\onelen]stair1A) -- ([xshift=7/2*\onelen,yshift=\dy-3/2*\onelen]stair1I);
    \draw[red,dash pattern=on 1pt off 2pt, line width=0.1pt] ([xshift=11/2*\onelen,yshift=\dy-1/2*\onelen]stair1A) -- ([xshift=7/2*\onelen,yshift=\dy-5/2*\onelen]stair1I);
    \draw[red,dash pattern=on 1pt off 2pt, line width=0.1pt] ([xshift=13/2*\onelen,yshift=\dy-1/2*\onelen]stair1A) -- ([xshift=7/2*\onelen,yshift=\dy-7/2*\onelen]stair1I);
    }
    
    % draw dividers
    \draw[line width=0.1pt] ([yshift=-\plen-\onelen]M3D) -- ([yshift=-\plen-\onelen]M3C);
    \draw[line width=0.1pt] ([yshift=-\plen]M3D) -- ([yshift=-\plen]M3C);
    \draw[line width=1pt] ([yshift=\onelen]M3A) -- ([yshift=\onelen]M3B);

    % ---------------- drawing right brace and matrix ----------------
    % brace below matrix
    \path ([xshift=-0.5\BrIn,yshift=-2\BrCl]M3A) coordinate (Brace3L);
    \path ([xshift=0.5\BrIn,yshift=-2\BrCl]M3B) coordinate (Brace3R);
    \draw[decorate, decoration={calligraphic brace, amplitude=3pt, mirror, aspect=0.25},line width=1pt] (Brace3L) -- (Brace3R);
    % matrix below brace
    \node (mat3) at ($(Brace3L)!0.25!(Brace3R) + (0,-3pt)$) {};
    \node[below] at (mat3.center) {$\begin{bmatrix}
        U_{\hat{i},p,f}\\U_{\hat{i}_p,1,f}\\Y_{\hat{i},p,f}\\ \hline \widehat{Y}_{\hat{i}_p,1,f}
    \end{bmatrix}$};
    % tag & label (on RHS of column)
    \node[below] at ([xshift=2.13cm,yshift=-0.9cm]mat3.center) {$\refstepcounter{equation}(\theequation)\label{eq:CL_DeePC_no_IVs}$};
    % brace to the right
    % \path ([xshift=0.7cm+0.05cm]mat3.center |- Brace1T) coordinate (Brace3T);%([xshift=0.7cm,yshift=-4pt]mat3.center) coordinate (Brace3T);
    % \path (Brace3T |- Brace1B) coordinate (Brace3B);
    % \draw[pen colour=gray!60,decorate, decoration={calligraphic brace, amplitude=3pt, aspect=0.2},line width=0.75pt] (Brace3T) -- (Brace3B);
    % % Zf
    % \node (Zf) at ($(Brace3T)!0.2!(Brace3B) + (3pt,-1.5pt)$) {};
    % \node[right,text=gray!100] at ([yshift=-0.18cm]Zf.center) {\scriptsize{\shortstack{$=$\\$\Phi_{\hat{i},1,f}$}}};
    
    % ======================= length indicators ======================= 
    \draw[|-|] ([yshift=\onelen]M1D) -- node[above] {\scriptsize$N$} ([yshift=\onelen]M1C);
    \draw[|-|] ([yshift=\onelen]M2D) -- node[above] {\scriptsize$f$} ([yshift=\onelen]M2C);
    \draw[|-|] ([yshift=\onelen]M3D) -- node[above] {\scriptsize$f$} ([yshift=\onelen]M3C);
    \draw[|-|] ([xshift=\onelen]M3C) -- node[right] {\scriptsize$pr$} ([xshift=\onelen,yshift=\onelen]t2B);
    \draw[|-|] ([xshift=\onelen,yshift=\onelen]t2B) -- node[right] {\scriptsize$r$} ([xshift=\onelen]t2B);
    \draw[|-|] ([xshift=\onelen]t2B) -- node[right] {\scriptsize$pl$} ([xshift=\onelen,yshift=\onelen]M3B);
    \draw[|-|] ([xshift=\onelen,yshift=\onelen]M3B) -- node[right] {\scriptsize$l$} ([xshift=\onelen]M3B);
\end{tikzpicture}
\caption{Visualization of known (black) and unknown (red) variables in \ac{CL-DeePC} without \ac{IVs}. Each dot represents an input $u_k\in\mathbb{R}^r$, output $y_k\in\mathbb{R}^l$, or element of the matrix $G$. \ac{CL-DeePC} involves $f$ sequential applications of a step-ahead predictor obtained from regular \ac{DeePC} with $f=1$ (see also Fig.~\ref{fig:regular-DeePC}), resulting in the dashed block-anti diagonals with the same $u_k$ or $y_k$ on the right hand side.}
\label{fig:CL-DeePC}
\end{figure}
\begin{figure}[b!]
\centering
\begin{tikzpicture}
    % defining constants
    \def\stepSize{0.25}
    \def\Nnum{9}
    \def\fnum{8}
    \def\pnum{4}
    
    % Defining lengths
    % \newlength{\onelen}
    \setlength{\onelen}{\stepSize cm}
    % \newlength{\BrCl}
    \setlength{\BrCl}{0.075cm}
    % \newlength{\BrIn}
    \setlength{\BrIn}{0.15cm}
    % \newlength{\plen}
    \setlength{\plen}{1cm}%{\pnum\stepSize cm}
    % \newlength{\flen}
    \setlength{\flen}{2cm}%{\fnum*\stepSize cm}
    % \newlength{\Nlen}
    \setlength{\Nlen}{2.25cm}%{\Nnum\stepSize cm}%should be 2*p+one
    % \newlength{\MatClearance}
    \setlength{\MatClearance}{0.3cm}
    
    % grid lines for guidance
    % \draw[gray,step=0.5] (-0,-7) grid (8,3);

    % ======================= drawing data matrix =======================
    \path (0,2\plen+\onelen) coordinate (M1D);
    \path ([yshift=-2\plen-2\flen]M1D) coordinate (M1A);
    \path ([xshift=\Nlen]M1A) coordinate (M1B);
    \path ([yshift=2*\plen+2*\flen]M1B) coordinate (M1C);
    \draw[line width=1.5pt] ([xshift=\BrIn,yshift=-\BrCl]M1A) -- ([xshift=-\BrCl,yshift=-\BrCl]M1A) -- ([xshift=-\BrCl,yshift=\BrCl]M1D) -- ([xshift=\BrIn,yshift=\BrCl]M1D); %left bracket
    \draw[line width=1.5pt] ([xshift=-\BrIn,yshift=-\BrCl]M1B) -- ([xshift=\BrCl,yshift=-\BrCl]M1B) -- ([xshift=\BrCl,yshift=\BrCl]M1C) -- ([xshift=-\BrIn,yshift=\BrCl]M1C); %right bracket
    \draw[line width=1pt] ([yshift=\flen]M1A) -- ([yshift=\flen]M1B); % dividing matrix into blocks
    \fill[black, opacity=0.5] (M1A) rectangle (M1C);
    \foreach \x in {0,...,8} { % drawing black dots
    \foreach \y in {-15,...,8} {
      \fill ( {(\x+0.5)*\onelen}, {(\y+0.5)*\onelen} ) circle (1pt);
    }}
    \draw[line width=0.1pt] ([yshift=-\plen-\flen]M1D) -- ([yshift=-\plen-\flen]M1C);
    \draw[line width=0.1pt] ([yshift=-\plen]M1D) -- ([yshift=-\plen]M1C);

    % coordinates for diagonals
    \path ([yshift=-\plen]M1D) coordinate (M1stair1A);
    \path ([xshift=\Nlen]M1D) coordinate (M1stair1I);
    % black diagonals
    % \foreach \i in {0,-1}{
    % \foreach \dy in {1,...,18}{
    % \ifthenelse{\dy<9.5}{
    % % This code will be executed if \dy<9.5
    %     \draw[dash pattern=on 1pt off 2pt, line width=0.1pt] ([xshift= 0.5*\onelen,yshift=\i*(\plen+\flen)-(\dy+0.5)*\onelen]M1D) -- ([xshift= (\dy+0.5)*\onelen,yshift=\i*(\plen+\flen)-0.5*\onelen]M1D);
    % }{
    % % This code will be executed if \dy>=10
    %     \ifthenelse{\dy<12.5}{
    %         % This code will be executed if \dy<=12
    %         \draw[dash pattern=on 1pt off 2pt, line width=0.1pt] ([xshift= 0.5*\onelen,yshift=\i*(\plen+\flen)-(\dy+0.5)*\onelen]M1D) -- ([xshift= -0.5*\onelen,yshift=\i*(\plen+\flen)-(\dy-7.5)*\onelen]M1C);
    %         }{
    %         % This code will be executed if \dy>=13
    %         \draw[dash pattern=on 1pt off 2pt, line width=0.1pt] ([xshift= (\dy-10.5)*\onelen,yshift=\i*(\plen+\flen)-\plen-\flen+0.5*\onelen]M1D) -- ([xshift= -0.5*\onelen,yshift=\i*(\plen+\flen)-(\dy-7.5)*\onelen]M1C);
    %     }
    % }
    % }}
    
    % ---------------- drawing left brace and matrix ----------------
    \path ([xshift=-0.5\BrIn,yshift=-2\BrCl]M1A) coordinate (Brace1L);
    \path ([xshift=0.5\BrIn,yshift=-2\BrCl]M1B) coordinate (Brace1R);
    \draw[decorate, decoration={calligraphic brace, amplitude=3pt, mirror, aspect=0.75},line width=1pt] (Brace1L) -- (Brace1R);
    \node (mat1) at ($(Brace1L)!0.75!(Brace1R) + (0,-3pt)$) {};
    \node[below] at (mat1.center) {$\begin{bmatrix}
        U_{i,p,N}\\U_{i_p,f,N}\\Y_{i,p,N}\\ \hline Y_{i_p,f,N}
    \end{bmatrix}$};
    % brace to the left
    % \path ([xshift=-0.75cm-0.05cm,yshift=-4pt]mat1.center) coordinate (Brace1T);
    % \path ([yshift=-1.73cm]Brace1T) coordinate (Brace1B);
    % \draw[pen colour=gray!60,decorate, decoration={calligraphic brace, amplitude=3pt, mirror, aspect=0.2},line width=0.75pt] (Brace1T) -- (Brace1B);
    % % Zp
    % \node (Zp) at ($(Brace1T)!0.2!(Brace1B) + (-3pt,-0.8pt)$) {};
    % \node[left,text=gray!100,align=center] at ([xshift=0.5mm,yshift=-0.18cm]Zp.center) {\scriptsize{\shortstack{$=$\\$\Phi_{i,f,N}$}}};
  
    % ======================= drawing G =======================
    % useful coordinates
    \path ([xshift=\MatClearance,yshift=-0.5\Nlen-\plen-\flen]M1C) coordinate (M2A);
    \path ([xshift=\onelen]M2A) coordinate (M2B);
    \path ([yshift=\Nlen]M2B) coordinate (M2C);
    \path ([xshift=-\onelen]M2C) coordinate (M2D);

    % brackets
    \draw[line width=1.5pt] ([xshift=0.5\BrIn,yshift=-\BrCl]M2A) -- ([xshift=-\BrCl,yshift=-\BrCl]M2A) -- ([xshift=-\BrCl,yshift=\BrCl]M2D) -- ([xshift=0.5\BrIn,yshift=\BrCl]M2D); %left bracket
    \draw[line width=1.5pt] ([xshift=-0.5\BrIn,yshift=-\BrCl]M2B) -- ([xshift=\BrCl,yshift=-\BrCl]M2B) -- ([xshift=\BrCl,yshift=\BrCl]M2C) -- ([xshift=-0.5\BrIn,yshift=\BrCl]M2C); %right bracket
    
    % red fill
    \fill[red!50,opacity=0.5] (M2A) rectangle (M2C);

    % drawing red dots
    \foreach \x in {0} {
    \foreach \y in {0,...,8} {
      \fill[red] ([xshift=(\x+0.5)*\onelen,yshift=(\y+0.5)*\onelen]M2A) circle (1pt);
    }}

    % drawing middle brace and matrix
    \coordinate (Brace2L) at ([xshift=-0.5\BrIn]M2A |- Brace1L);
    \coordinate (Brace2R) at ([xshift=0.5\BrIn]M2B |- Brace1L);
    \draw[decorate, decoration={calligraphic brace, amplitude=3pt, mirror, aspect=0.5},line width=1pt] (Brace2L) -- (Brace2R);
    \node (mat1) at ($(Brace2L)!0.5!(Brace2R) + (0,-3pt)$) {};
    \node[below] at ([yshift=-1.05cm]mat1.center) {$g$};
    
    % ======================= drawing equal sign ======================= 
    \path ([xshift=\MatClearance*3/4,yshift=\Nlen/2-0.1cm]M2B) coordinate (EqA);
    \path ([xshift=0.4cm]EqA) coordinate (EqB);
    \path ([yshift=0.2cm]EqB) coordinate (EqC);
    \path ([xshift=-0.4cm]EqC) coordinate (EqD);
    \draw[line width = 1.5 pt] (EqA) -- (EqB);
    \draw[line width = 1.5 pt] (EqD) -- (EqC);

    % ---------------- equation sign below ----------------
    \node[below] at ([xshift=2\onelen,yshift=-1.05cm]mat1.center) {$=$};

    % ======================= drawing RHS =======================
    % inside of matrix
    \path ([xshift=0.75\MatClearance]EqB |- M1A) coordinate (M3A);
    \path ([xshift=\onelen]M3A) coordinate (M3B);
    \path ([yshift=2\plen+2\flen]M3B) coordinate (M3C);
    \path ([xshift=-\onelen]M3C) coordinate (M3D);
    
    % fill figures
    \fill[black,opacity=0.5]  (M3D) -- ([yshift=-\plen]M3D) -- ([yshift=-\plen]M3C) -- (M3C) -- cycle;
    \fill[red!50,opacity=0.5] ([yshift=-\plen]M3D) -- ([yshift=-\plen-\flen]M3D) -- ([yshift=-\plen-\flen]M3C) -- ([yshift=-\plen]M3C) -- cycle;
    \fill[black,opacity=0.5] ([yshift=\flen]M3A) -- ([yshift=\flen]M3B) -- ([yshift=\flen+\plen]M3B) -- ([yshift=\flen+\plen]M3A) -- cycle;
    \fill[red!50,opacity=0.5]  (M3A) -- (M3B) -- ([yshift=\flen]M3B) -- ([yshift=\flen]M3A) -- cycle;
    
    % draw brackets
    \draw[line width=1.5pt] ([xshift=0.5\BrIn,yshift=-\BrCl]M3A) -- ([xshift=-\BrCl,yshift=-\BrCl]M3A) -- ([xshift=-\BrCl,yshift=\BrCl]M3D) -- ([xshift=0.5\BrIn,yshift=\BrCl]M3D); %left bracket
    \draw[line width=1.5pt] ([xshift=-0.5\BrIn,yshift=-\BrCl]M3B) -- ([xshift=\BrCl,yshift=-\BrCl]M3B) -- ([xshift=\BrCl,yshift=\BrCl]M3C) -- ([xshift=-0.5\BrIn,yshift=\BrCl]M3C); %right bracket
    
    % U_{i,p,1}
    \foreach \dy in {0,...,-3}{
        \fill[black] ([xshift=0.5\onelen,yshift=(\dy-0.5)*\onelen]M3D) circle (1pt);
    }

    % U_{i_p,f,1}
    \foreach \dy in {-4,...,-11}{
        \fill[red] ([xshift=0.5\onelen,yshift=(\dy-0.5)*\onelen]M3D) circle (1pt);
    }
    
    % Y_{i,p,1}
    \foreach \dy in {0,...,-3}{
        \fill[black] ([xshift=0.5\onelen,yshift=(\dy-0.5)*\onelen-\plen-\flen]M3D) circle (1pt);
    }

    % Y_{i_p,f,1}
    \foreach \dy in {-4,...,-11}{
        \fill[red] ([xshift=0.5\onelen,yshift=(\dy-0.5)*\onelen-\plen-\flen]M3D) circle (1pt);
    }

    % draw dividers
    \draw[line width=0.1pt] ([yshift=-\plen]M3D) -- ([yshift=-\plen]M3C);
    \draw[line width=0.1pt] ([yshift=-\plen-\flen]M3D) -- ([yshift=-\plen-\flen]M3C);
    \draw[line width=1pt] ([yshift=\flen]M3A) -- ([yshift=\flen]M3B);

    % ---------------- drawing right brace and matrix ----------------
    % brace below matrix
    \path ([xshift=-0.5\BrIn,yshift=-2\BrCl]M3A) coordinate (Brace3L);
    \path ([xshift=0.5\BrIn,yshift=-2\BrCl]M3B) coordinate (Brace3R);
    \draw[decorate, decoration={calligraphic brace, amplitude=3pt, mirror, aspect=0.5},line width=1pt] (Brace3L) -- (Brace3R);
    % matrix below brace
    \node (mat3) at ($(Brace3L)!0.5!(Brace3R) + (0,-3pt)$) {};
    \node[below] at ([xshift=0.35cm]mat3.center) {$\begin{bmatrix}
        U_{\hat{i},p,1}\\U_{\hat{i}_p,f,1}\\Y_{\hat{i},p,1}\\ \hline \widehat{Y}_{\hat{i}_p,f,1}
    \end{bmatrix}$};
    % tag & label (on RHS of column)
    \node[below left] at (8.35cm,-5cm) {$\refstepcounter{equation}(\theequation)\label{eq:regular_DeePC_no_IVs}$};
    % brace to the right
    % \path ([xshift=1.1cm]mat3.center |- Brace1T) coordinate (Brace3T);%([xshift=0.7cm,yshift=-4pt]mat3.center) coordinate (Brace3T);
    % \path (Brace3T |- Brace1B) coordinate (Brace3B);
    % \draw[pen colour=gray!60,decorate, decoration={calligraphic brace, amplitude=3pt, aspect=0.2},line width=0.75pt] (Brace3T) -- (Brace3B);
    % % Zf
    % \node (Zf) at ($(Brace3T)!0.2!(Brace3B) + (3pt,-1.5pt)$) {};
    % \node[right,text=gray!100] at ([yshift=-0.18cm]Zf.center) {\scriptsize{\shortstack{$=$\\$\Phi_{\hat{i},f,1}$}}};
    
    % ======================= length indicators ======================= 
    \draw[|-|] ([yshift=\onelen]M1D) -- node[above] {\scriptsize$N$} ([yshift=\onelen]M1C);
    \draw[|-|] ([xshift=\onelen]M3C) -- node[right] {\scriptsize$pr$} ([xshift=\onelen,yshift=-\plen]M3C);
    \draw[|-|] ([xshift=\onelen,yshift=-\plen]M3C) -- node[right] {\scriptsize$fr$} ([xshift=\onelen,yshift=-\plen-\flen]M3C);
    \draw[|-|] ([xshift=\onelen,yshift=-\plen-\flen]M3C) -- node[right] {\scriptsize$pl$} ([xshift=\onelen,yshift=\flen]M3B);
    \draw[|-|] ([xshift=\onelen,yshift=\flen]M3B) -- node[right] {\scriptsize$fl$} ([xshift=\onelen]M3B);
\end{tikzpicture}
\caption{Visualization of known (black) and unknown (red) variables in regular \ac{DeePC} without \ac{IVs}. Each dot represents an input $u_k\in\mathbb{R}^r$, output $y_k\in\mathbb{R}^l$, or element of the matrix $G$. A multi-step ahead predictor of prediction length $f$ is formed directly by taking a linear combination of past input and output data.\\\vspace{0.75mm}}
\label{fig:regular-DeePC}
\end{figure}
%
\setcounter{thm}{0}
\begin{thm}\label{theorem:main_result}
    Consider the minimal discrete non-deterministic \ac{LTI} system given by~\eqref{eqn:SS_innovation}. If $p\geq\ell$, the sequence of inputs $\{u_k\}_{k=i}^{i+\bar{N}-1}$ of length $\bar{N}=p+s+N-1$ is persistently exciting of order $p+s+n$, and with sample correlations such that%
    \begin{alignat}{2}%see also https://www.cis.upenn.edu/~jean/schur-comp.pdf
    % \widehat{\Sigma}_{u,u} &> 0,\label{eq:PE_corU}\\
    &\widehat{\Sigma}_{e,e} - \widehat{\Sigma}_{u,e} \widehat{\Sigma}_{u,u}\inv \widehat{\Sigma}_{u,e}^\top>0,\span\span\label{eq:PE_corUE2}\\
    &&\text{with}\quad\widehat{\Sigma}_{e,e}&=E_{i,p+s+n,N-n}E_{i,p+s+n,N-n}^\top,\notag\\
    &&\widehat{\Sigma}_{u,e}&=U_{i,p+s+n,N-n}E_{i,p+s+n,N-n}^\top,\notag\\
    &&\widehat{\Sigma}_{u,u}&=U_{i,p+s+n,N-n}U_{i,p+s+n,N-n}^\top,\notag
    \end{alignat}
    then for
    \begin{align}\label{eq:Theorem1}
        \begin{bmatrix}
            \Phi_{i,s,N}\\Y_{i_p,s,N}
        \end{bmatrix}G =
        \begin{bmatrix}
            \Phi_{\hat{i},s,f}\\\widehat{Y}_{\hat{i}_p,s,f}
        \end{bmatrix}
    \end{align}
    \begin{enumerate}
        \item[1)] there exists a solution $G$\label{claim:G_exists}
        \item[2)] that is unique if $\Phi_{i,s,N}$ is square, such that %the corresponding output predictor 
        $\widehat{Y}_{\hat{i}_p,s,f}$
        \begin{enumerate}
            \item is uniquely determined by past noise and input-output data, and%that uniquely specifies an output predictor
            \item is asymptotically unbiased with $p\rightarrow\infty$ if $s=1$.
        \end{enumerate} %$\widehat{Y}_{\hat{i}_p,s,f}$
    \end{enumerate}
    % Let $\bar{u}_k=\left[u_k^\top \; e_k^\top\right]^\top$ represent an `extended' input. Given a sequence of input $u_k$ and output $y_k$ data of length $\bar{N}=N+p$, if the extended input sequence is persistently exciting of order $p+1+n$ and $p\geq\ell$, then the \ac{CL-DeePC} formulation given by~\eqref{eq:CL_DeePC_no_IVs} provides an unbiased output predictor.%asymptotically unbiased output predictor in the limit $N\rightarrow \infty$.
\end{thm}
\textbf{Proof:} 
Rewriting \eqref{eq:DataEq1} with $k=i$ and $q=N$ yields
\begin{align*}
    &\underbrace{\begin{bmatrix}-\Gamma_s \tilde{A}^p & -L_s & I_{sl}&-\mathcal{H}_s\end{bmatrix}}_{= \mathfrak{R}}
    \underbrace{\begin{bmatrix}
        X_{i,1,N}\\
        \Phi_{i,s,N}\\
        Y_{i_p,s,N}\\
        E_{i_p,s,N}
    \end{bmatrix}}_{\mathfrak{B}\mathfrak{D}}=\mathcal{O},\\
    &\mathfrak{R}\in\mathbb{R}^{sl\times n+(p+s)(r+l)+sl},\;\mathfrak{BD}\in\mathbb{R}^{n+(p+s)(r+l)+sl \times N},
\end{align*}
with $\mathfrak{R}$, and $\mathfrak{BD}$ as indicated for brevity. Similarly, using \eqref{eq:DataEq1} to decompose $\mathfrak{BD}$ into a matrix $\mathfrak{D}$ of exogenous inputs and initial states and a matrix $\mathfrak{B}$ to describe their effects obtains
\begin{align}
    &\mathfrak{R}
    \underbrace{\begin{bmatrix}
        I_n      & 0      & 0       & 0 & 0\\
        0        & I_{pr} & 0       & 0 & 0\\
        0        & 0      & I_{sr}  & 0 & 0\\
        \Gamma_p & \mathcal{T}_p^\mathrm{u} & 0 & \mathcal{H}_p & 0\\
        \varepsilon_1 & \varepsilon_2 & \mathcal{T}_s^\mathrm{u} & \varepsilon_3 & \mathcal{H}_s\\
        0 & 0 & 0 & 0 & I_{sl}
    \end{bmatrix}}_{=\mathfrak{B}}
    \underbrace{\begin{bmatrix}
        X_{i,1,N}\\
        U_{i,p,N}\\
        U_{i_p,s,N}\\
        E_{i,p,N}\\
        E_{i_p,s,N}
    \end{bmatrix}}_{=\mathfrak{D}}=\mathcal{O},\label{eq:RBD2}\\
    &\mathfrak{B}\in\mathbb{R}^{n+(p+s)(r+l)+sl\times n+(p+s)(r+l)},\notag\\
    &\mathfrak{D}\in\mathbb{R}^{n+(p+s)(r+l)\times N},\notag
\end{align}
with $\varepsilon_1=\Gamma_s(\tilde{A}^p+\tKp{y}\Gamma_p)$, $\varepsilon_2=\Gamma_s(\tKp{u}+\tKp{y}\mathcal{T}_p^\mathrm{u})$, and $\varepsilon_3=\Gamma_s\tKp{y}\mathcal{H}_p$ defined as shown. By inspection, with $\mathcal{R}(\cdot)$ and $\mathcal{N}(\cdot)$ respectively denoting the range and nullspace of a matrix, $\mathcal{R}(\mathfrak{B}\mathfrak{D})\subseteq\mathcal{R}(\mathfrak{B})=\mathcal{N}(\mathfrak{R})$. Only if $\mathfrak{D}$ is full row rank $\mathcal{R}(\mathfrak{B}\mathfrak{D})=\mathcal{N}(\mathfrak{R})$. To see when this latter condition is the case, consider the following.
\setcounter{thm}{0}
\begin{lem}[\cite{Willems2005}, Cor.~2(iii)]
    If the input sequence $\{u_k\}_{k=i}^{i+\epsilon+q-2}$ of a controllable discrete \ac{LTI} system without noise is persistently exciting of order $\epsilon+n$ then the matrix $\left[X_{i,1,q}^\top\;U_{i,\epsilon,q}^\top\right]^\top$ is full row rank.
\end{lem}
This yields the following corollary for extensions to non-deterministic systems, for which we note that controllable $(A,B)$ implies controllable $(A,[B\,K])$.
\setcounter{thm}{0}
\begin{cor}
    If for a controllable non-deterministic \ac{LTI} system of the form given by \eqref{eqn:SS_innovation} the sequence of inputs and noise $\{[u_k^\top\;e_k^\top]^\top\}_{k=i}^{i+\epsilon+q-2}$ is persistently exciting of order $\epsilon+n$ then the matrix $\left[X_{i,1,q}^\top\;U_{i,\epsilon,q}^\top\;E_{i,\epsilon,q}^\top\right]^\top$ is full row rank.
\end{cor}
Hence, since \eqref{eqn:SS_innovation} is assumed to be minimal and therefore also controllable, $\mathfrak{D}$ in \eqref{eq:RBD2} is full row rank if $\{[u_k^\top\;e_k^\top]^\top\}_{k=i}^{i+p+s+N-2}$ is persistently exciting of order $p+s+n$. Definition~\ref{def:PE} thereby implies that $N\geq (p+s+n)(r+l)+n$ and that, by extension
\begin{align}\label{eq:PE_corUE}
    \begin{bmatrix}
        U_{i,p+s+n,N-n}\\
        E_{i,p+s+n,N-n}
    \end{bmatrix}
    \begin{bmatrix}
        U_{i,p+s+n,N-n}\\
        E_{i,p+s+n,N-n}
    \end{bmatrix}^\top > 0,
\end{align}
which is equivalent to the positive definiteness of an experimental correlation matrix that can be decomposed into its constituent components. By considering Schur complements it follows that \eqref{eq:PE_corUE} is satisfied if both \eqref{eq:PE_corUE2} and $\widehat{\Sigma}_{u,u} > 0$. Note that both of these conditions are satisfied by assumption, with the latter condition being implied by a persistently exciting input of order $p+s+n$. %see also https://www.cis.upenn.edu/~jean/schur-comp.pdf

Hence, if the aforementioned persistency of excitation conditions are met and the \ac{LTI} system is controllable then $\mathfrak{D}$ from \eqref{eq:RBD2} is full row rank. As a result, by the Rouch\'{e}-Capelli theorem there then exists a vector $g_{k-\hat{i}+1}$ such that
\begin{align}\label{eq:Dg}
    \mathfrak{D} g_{k-\hat{i}+1} =
    \begin{bmatrix}
        x_k^\top & \datavec{u}{k,p}^\top & \datavec{u}{k_p,s}^\top & \datavec{e}{k,p}^\top & \datavec{e}{k_p,s}^\top
    \end{bmatrix}^\top\in\mathcal{N}(\mathfrak{RB}).
\end{align}
With reference to~\eqref{eq:Yf1} and~\eqref{eq:DataEq1}, premultiplying both sides of~\eqref{eq:Dg} by $\mathfrak{B}$ from~\eqref{eq:RBD2} yields
\begin{align}\label{eq:Dg}
    \begin{bmatrix}
        X_{i,1,N}\\
        \Phi_{i,s,N}\\
        Y_{i_p,s,N}\\
        E_{i_p,s,N}
    \end{bmatrix}g_{k-\hat{i}+1}=
    \begin{bmatrix}
        x_k\\
        \Phi_{k,s,1}\\
        \datavec{y}{k_p,s}\\
        \datavec{e}{k_p,s}
    \end{bmatrix}.
\end{align}
Equation~\eqref{eq:Theorem1} is obtained by sequential application of \eqref{eq:Dg} for $k={\hat{i},\dots,\hat{i}+f-1}$ without the top and bottom matrix equations. As a result the future outputs resemble predictions and $G=\left[g_1\;g_2\;\cdots\;g_f\right]$. This proves statement~\ref{claim:G_exists}).
%
% Without the top and bottom matrix equations this is written more compactly for consecutive time indices $k={\hat{i},\dots,\hat{i}+f-1}$ as given by \eqref{eq:Theorem1}, in which $G$ contains the vectors $\{g_{k-\hat{i}+1}\}_{k=\hat{i}}^{\hat{i}+f-1}$. Moreover, the future outputs under consideration become estimates since the top and bottom matrix equations with relevant but unknown data are not considered by~\eqref{eq:Theorem1}. 

Under the assumed persistency of excitation conditions $\mathfrak{D}$ is full row rank such that, by inspection of the product $\mathfrak{BD}$ in~\eqref{eq:RBD2}, $\Phi_{i,s,N}$ must also be full row rank.
initial state specification by data with asymptotically decreasing error


% ==============================================================================================================================================================
% ==============================================================================================================================================================
% \subsection{The basic idea}
% To derive a variant of \ac{DeePC} that does not suffer from the aforementioned closed-loop identification issue lets start by considering regular \ac{DeePC}. Closed-loop identification bias can be avoided by using a step-ahead predictor~\citep{Ljung1996}. However, such a short prediction horizon length is typically not conducive to good performance in a receding horizon control setting. Hence, to obtain another output prediction, the previous regular \ac{DeePC} problem is repeated with the same past data (but a different vector $g$ to span it) to obtain trajectories of input-output data that are shifted forwards one time step. This procedure can be repeated to obtain a desired prediction horizon length $f$. The entire procedure is succinctly described by Fig.~\ref{fig:CL-DeePC} and \eqref{eq:CL_DeePC_no_IVs}, \todo{check length conditions DeePC}

% in which $i$, $i_p$, $\hat{i}$, and $\hat{i}_p$ are discrete time indices (the first three indices lie in the past, and the last index $\hat{i}_p$ resembles the first future time index), and $G$ defines a matrix with columns given by the vectors $\{g_k\}^f_{k=1}$. Furthermore, note that $\Phi_{i,1,N}$ and $\Phi_{\hat{i},1,f}$ are present on respectively the top left and top right hand side.

% Treatment of different \ac{CL-DeePC} solution strategies is deferred to Section \ref{sec:SolutionMethods}. Suffice it for now to take note of the structure of \eqref{eq:CL_DeePC_no_IVs} illustrated by Fig.~\ref{fig:CL-DeePC} and to say that if the input is sufficiently persistently exciting such that $\Phi_{i,1,N}$ is full row rank~\cite[Chapt.~9.6.1]{Verhaegen2007a} then making $\Phi_{i,1,N}$ square and invertible by selecting $N=(p+1)r+pl$ minimizes the number of optimization variables.
% 
% As with regular \ac{DeePC} the idea is to find an optimal combination of allowable future inputs and outputs that minimizes a cost function that is possibly subject to constraints. To see how \eqref{eq:CL_DeePC_no_IVs} can be used in a receding horizon optimal control setting, first consider the top three blocks of the past data matrix. If the input is sufficiently persistently exciting then this matrix, $\Phi_{i,1,N}$, is full row rank~\cite[Chapt.~9.6.1]{Verhaegen2007a}. If, furthermore, ${N=(p+1)r+pl}$, then $\Phi_{i,1,N}$ becomes square and invertible. Hence, a unique solution for $G$ can then be obtained from the top three block equations of \eqref{eq:CL_DeePC_no_IVs} (in terms of future inputs and outputs), which can then be used to obtain output predictions by using the bottom block equation. The structure of \eqref{eq:CL_DeePC_no_IVs} is visualized by Fig.~\ref{fig:CL-DeePC}. This figure demonstrates that successive future output predictions are dependent on preceding input-output data as well as their concurrent input, opening the door to the sequential construction of an output predictor. This is described in \todo{section}, and can be used in a receding horizon optimal control framework.
% 
% 
% ==============================================================================================================================================================
% ==============================================================================================================================================================
% \subsection{The data equations}\label{sec:DerivingDataEquations}
% To motivate a noise mitigation strategy based that is based on \ac{IVs} that is explained hereafter, the data equations that justify this approach are first derived here using a state-space approach.

% To this end, it is straightforward to show by iterative application of respectively \eqref{eqn:SS_innovation} and \eqref{eqn:SS_predictor} that%
% \begin{align}
%     Y_{k_p,s,q} &= \Gamma_s X_{k_p,1,q} + \mathcal{T}_s^\mathrm{u} U_{k_p,s,q} + \mathcal{H}_s E_{k_p,s,q}\label{eq:Yf1},\\
%     \begin{split}%
%     Y_{k_p,s,q} &= \widetilde{\Gamma}_s X_{k_p,1,q} + \widetilde{\mathcal{T}}_s^\mathrm{u} U_{k_p,s,q} + E_{k_p,s,q}\\
%     &\phantom{=}+(I_{sl}-\widetilde{\mathcal{H}}_s)Y_{k_p,s,q}.
%     \end{split}\label{eq:Yf2}
% \end{align}
% It is possible to rewrite the initial states in terms of preceding states and input-output data using \eqref{eqn:SS_predictor} as%
% \begin{align}\label{eq:Xip}
%     X_{k_p,1,q} = \tilde{A}^p X_{k,1,q} + \tKp{u} U_{k,p,q} + \tKp{y} Y_{k,p,q}.
%     % \begin{bmatrix}
%     %     Y_{i,p,q}\\
%     %     U_{i,p,q}
%     % \end{bmatrix}.
% \end{align}
% % in which $\tKp{}=\big[\tKp{y}\;\;\tKp{u}\big]$.
% Substitute \eqref{eq:Xip} into \eqref{eq:Yf1} and \eqref{eq:Yf2} and apply Assumption~\ref{assum:initial_contribution} to obtain two so called data equations:
% \begin{align}
%     Y_{k_p,s,q} &= L_s \Phi_{k,s,q} + \mathcal{H}_s E_{k_p,s,q}\label{eq:DataEq1}\\
%     Y_{k_p,s,q} &= \widetilde{L}_s \Phi_{k,s,q} + E_{k_p,s,q} + (I_{sl}-\widetilde{\mathcal{H}}_s) Y_{k_p,s,q},\label{eq:DataEq2}
% \end{align}
% in which $L_s$, $\widetilde{L}_s$, $\Phi_{k,s,q}$ are defined in Section \ref{sec:notation}. %Similarly to \eqref{eq:DataEq1} and \eqref{eq:DataEq2}, the future outputs are defined by
% % \todo{use noiseless?}%always refer to noiseless version or beter to refer to ideal predictor here?
% % \begin{align}
% %     Y_{\hat{i}_p,s,f} &= L_s \Phi_{\hat{i},s,f} + \mathcal{H}_s E_{\hat{i}_p,s,f},\label{eq:DataEq1.2}\\
% %     Y_{\hat{i}_p,s,f} &= \widetilde{L}_s \Phi_{\hat{i},s,f} + (I_{sl}-\widetilde{\mathcal{H}}_s) Y_{\hat{i}_p,s,f} + E_{\hat{i}_p,s,f}\label{eq:DataEq2.2}.
% % \end{align}
% Although a more generic representation was kept above for later analysis, for \ac{CL-DeePC}, $s=1$. This reduces the complexity of the above equations since ${\widetilde{L}_1=L_1=\big[ C\tKp{u} \; D \; C\tKp{y} \big]}$ and $\widetilde{\mathcal{H}}_1=\mathcal{H}_1=I_l$.
%
% ==============================================================================================================================================================
% ==============================================================================================================================================================
\subsection{Willems' Fundamental Lemma \& Noise}
Equation~\eqref{eq:DataEq1} can be reformulated with $k=i$, $q=N$ or for an ideal noiseless output prediction with $k=\hat{i}$ as respectively
\begin{alignat}{2}
    \begin{bmatrix}
        -L_s & I_{sl}
    \end{bmatrix}&
    \begin{bmatrix}
        \Phi_{i,s,N}\\
        Y_{i_p,s,N}-\mathcal{H}_s E_{i_p,s,N}
    \end{bmatrix} = \mathcal{O},\label{eq:NoisyWFL1}\\%\mathcal{H}_s E_{i_p,s,N}, 
    \begin{bmatrix}
        -L_s & I_{sl}
    \end{bmatrix}&
    \begin{bmatrix}
        \Phi_{\hat{i},s,q}\\
        \widehat{Y}^*_{\hat{i}_p,s,q}
    \end{bmatrix} = \mathcal{O}, \label{eq:NoisyWFL2}
\end{alignat}
in which the asterisk indicates that the output prediction is ideal in the sense of being asymptotically unbiased. Multiplying \eqref{eq:NoisyWFL1} by $\mathcal{Z}^\top G\in\mathbb{R}^{N\times q}$, and subtracting \eqref{eq:NoisyWFL2} obtains
\begin{align}\label{eq:NoisyWFL3}
    \mkern-14mu\begin{bmatrix}
        \shortminus L_s & I_{sl}
    \end{bmatrix}
    \mkern-9mu\left(\mkern-3mu%
    \begin{bmatrix}
        \Phi_{i,s,N}\\
        Y_{i_p,s,N}\shortminus\mathcal{H}_s E_{i_p,s,N}
    \end{bmatrix}%
    \mkern-4mu\mathcal{Z}^\top G%\mkern-2mu
    -%-%
    \mkern-5mu\begin{bmatrix}
        \Phi_{\hat{i},s,q}\\
        \widehat{Y}^*_{\hat{i}_p,s,q}
    \end{bmatrix}\mkern-3mu\right)\mkern-6mu=\mkern-3mu\mathcal{O}\mkern-5mu%\mathcal{H}_s E_{i_p,s,N}\mathcal{Z}G
\end{align}
in which $\mathcal{Z}$ represents a yet unspecified matrix and $G$ represents a matrix that is akin to the likewise defined matrix from \eqref{eq:CL_DeePC_no_IVs} that contains all of the vectors $g_k$.

If the columns of the matrix with data on the left hand side of \eqref{eq:NoisyWFL1} span the entire nullspace of $\left[\shortminus L_s\;I_{sl}\right]$ and $\mathcal{Z}$ is full rank then all solutions to \eqref{eq:NoisyWFL3} are described by equating the term inside the parenthesis to zero. For now, consider the case that $\mathcal{Z}=I_N$, $s=f$, and $q=1$ in the absence of noise to recover the regular deterministic \ac{DeePC} equation~\citep{Coulson2019}. %Then one possible solution (since the matrix $\left[\shortminus L_s\;I_{sl}\right]$ is not full column rank) to \eqref{eq:NoisyWFL3} with $s=f$ and $q=1$ is given by the regular deterministic \ac{DeePC} equation~\cite{Coulson2019}:
\begin{align}\label{eq:regular_DeePC}
    \begin{bmatrix}
        \Phi_{i,f,N}\\
        Y_{i_p,f,N}
    \end{bmatrix}g=%
    \begin{bmatrix}
        \Phi_{\hat{i},f,1}\\
        \widehat{Y}_{\hat{i}_p,f,1}
    \end{bmatrix}.
\end{align}
Willems' Fundamental Lemma makes use of Assumptions~\ref{assum:PE} and~\ref{assum:controllability} to ensure that the entire nullspace of $\left[\shortminus L_f\;I_{fl}\right]$ is spanned by the data matrix on the left hand side of \eqref{eq:regular_DeePC}~\citep{Willems2005}. Assumption~\ref{assum:unique_initial} is furthermore necessary to guarantee the existence of a unique initial state and therefore output predictor. %This clearly reflects Willems' Fundamental Lemma, which states that for a deterministic \ac{LTI} system, any sufficiently persistently exciting past input-output trajectory parameterizes all possible future input-output trajectories~\cite{Willems2005}.\todo{WFL: what about nullspace in (12)}

In the presence of (unknown) noise, the term $Y_{i_p,s,N}-\mathcal{H}_s E_{i_p,s,N}$ from \eqref{eq:NoisyWFL3} cannot be determined to obtain an ideal output predictor. Instead, linear combinations of a noise-corrupted output $Y_{i_p,s,N}$ as in \eqref{eq:regular_DeePC} are taken, resulting in an error of the obtained output predictor due to implicit sampling of $\mathcal{H}_s E_{i_p,s,N}$. Moreover, the regular \ac{DeePC} formulation provided by \eqref{eq:regular_DeePC} may become inconsistent in the presence of noise, prompting the use of, e.g., slack variables and regularization~\citep{Coulson2019}.
%
% ==============================================================================================================================================================
% ==============================================================================================================================================================
\subsection{Noise mitigation using \acl{IVs}}
Notwithstanding potential benefits of beforementioned mechanisms to cope with noise, such methods do not provide a systematic way to mitigate noise at the source. To that end this section introduces the use of an \ac{IV}: $\mathcal{Z}\neq I_N$. In this context, the \ac{IV} is defined such that it is uncorrelated with the noise $E_{i_p,s,N}$ and preserves the (full row) rank of the data matrix $\Phi_{i,s,N}$ obtained from a sufficiently persistently exciting input. These two conditions are respectively formulated as
%
\begin{align}
    &\lim_{N\rightarrow\infty} \frac{1}{N}E_{i_p,s,N}\mathcal{Z}^\top = \mathcal{O},\label{eq:uncorrelated}\\
    \text{rank}\biggl(&\lim_{N\rightarrow\infty} \frac{1}{N}\Phi_{i,s,N}\mathcal{Z}^\top\biggl) =  \text{rank}(\Phi_{i,s,N}),\label{eq:rankconservation}
\end{align}
%
which motivates choosing $\mathcal{Z}=\Phi_{i,s,N}$~\cite[Chapt. 9.6]{Verhaegen2007a}. An important assumption that is hereby introduced to satisfy \eqref{eq:uncorrelated} is that inputs are uncorrelated with noise. To fulfill this assumption Section~\ref{sec:CL_ID_issue} will motivate the choice $s=1$. Furthermore, to then still obtain a multi-step-ahead predictor, $q=f$ is chosen.

Since the noise contribution in \eqref{eq:NoisyWFL3} is then asymptotically attenuated with increasing $N$ this motivates the use of
\begin{align}\label{eq:CL_DeePC_with_IV}
    \begin{bmatrix}
   \Phi_{i,1,N}\Phi_{i,1,N}^\top\\
   \hline
   Y_{i_p,1,N}\Phi_{i,1,N}^\top
    \end{bmatrix}
G =
\begin{bmatrix}
    \Phi_{\hat{i},1,f}\\
    \hline
    \widehat{Y}_{\hat{i}_p,1,f}
\end{bmatrix},
\end{align}
for sufficiently large $N$. Note that the structure of this equation is very similar to \eqref{eq:CL_DeePC_no_IVs} as shown by Fig.~\ref{fig:CL-DeePC}. The main difference is that the matrix with past data on the left hand side loses its indicated block-anti-diagonal structure and has $(p+1)r+pl$ instead of $N$ columns.

Solving \eqref{eq:CL_DeePC_with_IV} for the output predictor using the data equation examplified by \eqref{eq:DataEq1} yields
\begin{align}\label{eq:OutputPredictor}
    \widehat{Y}_{\hat{i}_p,1,f} = L_1 \Phi_{\hat{i},1,f} + \mathcal{H}_1 E_{i_p,1,N}\Phi_{i,1,N}^\dagger\Phi_{\hat{i},1,f},
\end{align}
in which the dagger $\dagger$ denotes the right inverse: ${\Phi_{i,1,N}^\dagger=\Phi_{i,1,N}^\top\left(\Phi_{i,1,N}\Phi_{i,1,N}^\top\right)\inv}$. Similar scrutiny of \eqref{eq:NoisyWFL3} demonstrates that according to \eqref{eq:uncorrelated} the ideal output predictor is recovered from \eqref{eq:OutputPredictor} in the limit $N\rightarrow\infty$.
% In obtaining an output predictor, no systematic noise mitigation strategy is yet applied by \eqref{eq:CL_DeePC_no_IVs} as the columns of $G$ simply take linear combinations of the noise in the output demonstrated by \eqref{eq:DataEq1}. An altered formulation of \eqref{eq:CL_DeePC_no_IVs} is therefore considered that allows the use of \ac{IVs} ($\mathcal{Z}_\mathrm{IV}$) as in~\cite{vanWingerden2022}
% \begin{align}\label{eq:CL_DeePC_with_IV}
%     \begin{bmatrix}
%    \Phi_{i,1,N}\\
%    \hline
%    Y_{i_p,1,N}
%     \end{bmatrix}
% {\mathcal{Z}_\mathrm{IV}}^\top G =
% \begin{bmatrix}
%     \Phi_{\hat{i},1,f}\\
%     \hline
%     \widehat{Y}_{\hat{i}_p,1,f}
% \end{bmatrix},
% \end{align}
% in which $G$ may be different from its previous definition depending on the definition of $\mathcal{Z}_\mathrm{IV}$, which follows shortly. Note that \eqref{eq:CL_DeePC_no_IVs} is recovered with $\mathcal{Z}_\mathrm{IV}=I_N$.

% From \eqref{eq:DataEq1} and \eqref{eq:CL_DeePC_with_IV} the output predictor becomes
% \begin{align}\label{eq:OutputPred}
%     \widehat{Y}_{\hat{i}_p,1,f} = L_1 \Phi_{\hat{i},1,f} + \mathcal{H}_1 E_{i_p,1,N}{\mathcal{Z}_\mathrm{IV}}^\top G.
% \end{align}
% To obtain an output estimate that best resembles a noiseless version of the actual future outputs given by \eqref{eq:DataEq1.2} it is desirable to reduce the noise contribution on the right hand side above. In addition, from an optimization point of view, it would be favorable to choose an \acs{IV} that uniquely determines $G$ from \eqref{eq:CL_DeePC_with_IV} given $\Phi_{\hat{i},1,f}$ and a sufficiently persistently exciting input that ensures that $\Phi_{i,1,N}$ is full row rank.

% This motivates choosing $\mathcal{Z}_\mathrm{IV}=\Phi_{i,1,N}$ as an \acs{IV} since\footnote{Since ${\mathcal{Z}_\mathrm{IV}}^\top G$ is fixed by \eqref{eq:CL_DeePC_IV} scalar multiples of the chosen \ac{IV} would be equally valid, simply resulting in a different $G$.}%
% \begin{align}
%     &\lim_{N\rightarrow\infty} \frac{1}{N}E_{i_p,1,N}{\Phi_{i,1,N}}^\top = \mathcal{O},\\%\label{eq:uncorrelated}\\
%     \text{rank}\biggl(&\lim_{N\rightarrow\infty} \frac{1}{N}\Phi_{i,1,N}{\Phi_{i,1,N}}^\top\biggl) =  \text{rank}(\Phi_{i,1,N}).%\label{eq:rankconservation}
% \end{align}
% As demonstrated by \eqref{eq:uncorrelated}, the instrumental variable and the noise are uncorrelated\footnote{Note that the choice $s=1$ is essential here to avoid correlation between inputs and noise during operation with data obtained in closed-loop.}. Hence, \eqref{eq:OutputPred} asymptotically converges to an ideal, noiseless output predictor with increasing $N$. In addition, provided that the input is sufficiently persistently exciting, $\Phi_{i,1,N}$ is full row rank such that \eqref{eq:rankconservation} permits only a single, unique solution for $G$ in \eqref{eq:CL_DeePC_IV}.