% autosam.tex
% Annotated sample file for the preparation of LaTeX files
% for the final versions of papers submitted to or accepted for 
% publication in AUTOMATICA.

% See also the Information for Authors.

% Make sure that the zip file that you send contains all the 
% files, including the files for the figures and the bib file.

% Output produced with the elsart style file does not imitate the
% AUTOMATICA style. The style file is generic for all Elsevier
% journals and the output is laid out for easy copy editing. The
% final document is produced from the source file in the
% AUTOMATICA style at Elsevier.

% You may use the style file autart.cls to obtain a two-column 
% document (see below) that more or less imitates the printed 
% Automatica style. This may helpful to improve the formatting 
% of the equations, tables and figures, and also serves to check 
% whether the paper satisfies the length requirements.

% Please note: Authors must not create their own macros.

% For further information regarding the preparation of LaTeX files 
% for Elsevier, please refer to the "Full Instructions to Authors" 
% from Elsevier's anonymous ftp server on ftp.elsevier.nl in the
% directory pub/styles, or from the internet (CTAN sites) on
% ftp.shsu.edu, ftp.dante.de and ftp.tex.ac.uk in the directory
% tex-archive/macros/latex/contrib/supported/elsevier.


%\documentclass{elsart}               % The use of LaTeX2e is preferred.

\documentclass[twocolumn]{autart}    % Enable this line and disable the 
                                     % preceding line to obtain a two-column 
                                     % document whose style resembles the
                                     % printed Automatica style.


\usepackage{graphicx}          % Include this line if your 
                               % document contains figures,
%\usepackage[dvips]{epsfig}    % or this line, depending on which
                               % you prefer.

% ===================================================================
\usepackage{amsmath}
\usepackage{amsfonts}
\usepackage[nolist]{acronym}    % required for \acro

\usepackage[textsize=tiny]{todonotes}
\usepackage{tikz,ifthen}
\usetikzlibrary{calc}

% own definitions:
\newcommand{\eqnref}[1]{(\ref{#1})}
\newcommand{\figref}[1]{Fig.~\ref{#1}}
\newcommand{\secref}[1]{Section~\ref{#1}}
\newcommand{\inv}{^{\raisebox{.2ex}{$\scriptscriptstyle-1$}}}
\newcommand{\Tf}[1]{\mathcal{T}_f^\mathrm{{#1}}}
\newcommand{\tTf}[1]{\widetilde{\mathcal{T}}_f^\mathrm{{#1}}}
\newcommand{\hTf}[1]{\widehat{\mathcal{T}}_f^\mathrm{{#1}}}
\newcommand{\Hf}{\mathcal{H}_f}
\newcommand{\tHf}{\widetilde{\mathcal{H}}_f}
\newcommand{\hHf}{\widehat{\mathcal{H}}_f}
\newcommand{\Kp}[1]{\mathcal{K}_{p}^{\mathrm{{#1}}}}
\newcommand{\tKp}[1]{\widetilde{\mathcal{K}}_{p}^{\mathrm{{#1}}}}
\newcommand{\Gf}{\Gamma_f}
\newcommand{\tGf}{\widetilde{\Gamma}_f}
\newcommand{\datavec}[2]{\underline{#1}_{#2}} % <-- add own preamble stuff
% ===================================================================

\begin{document}

\begin{frontmatter}
%\runtitle{Insert a suggested running title}  % Running title for regular 
                                              % papers but only if the title  
                                              % is over 5 words. Running title 
                                              % is not shown in output.

\title{Closed-loop Data-Enabled Predictive Control\thanksref{footnoteinfo}} % Title, preferably not more 
                                                % than 10 words.

\thanks[footnoteinfo]{This paper was not presented at any IFAC meeting.}
% \corauth[cor1]{Corresponding author.% R.~Dinkla. Tel. +XXXIX-VI-mmmxxi. Fax +XXXIX-VI-mmmxxv.
% }

\author[TUD]{R. Dinkla\corauthref{cor}}\ead{r.t.o.dinkla@tudelft.nl},    % Add the
\corauth[cor]{Corresponding author.% R.~Dinkla. Tel. +XXXIX-VI-mmmxxi. Fax +XXXIX-VI-mmmxxv.
}
\author[TUD]{S. P. Mulders}\ead{s.p.mulders@tudelft.nl},               % e-mail address 
\author[TUD,TUE]{T. Oomen}\ead{t.a.e.oomen@tudelft.nl},  % (ead) as shown
\author[TUD]{J.W. van Wingerden}\ead{j.w.vanwingerden@tudelft.nl}
\address[TUD]{Delft Center for Systems and Control, Delft University of Technology, Mekelweg 2, 2628CD Delft, The Netherlands}  % Please supply                                              
\address[TUE]{Control Systems Technology Group, Eindhoven University of Technology, 5600
MB Eindhoven, The Netherlands}        % Ful addresses here.

          
\begin{keyword}                           % Five to ten keywords,  
Cicero; Catiline; orations.               % chosen from the IFAC 
\end{keyword}                             % keyword list or with the 
                                          % help of the Automatica 
                                          % keyword wizard


\begin{abstract}                          % Abstract of not more than 200 words.
Cum M.~Cicero consul Nonis Decembribus senatum in aede Iovis 
Statoris consuleret, quid de iis coniurationis Catilinae sociis 
fieri placeret, qui in custodiam traditi essent, factum est, ut 
duae potissimum sententiae proponerentur, una D.~Silani consulis 
designati, qui morte multandos illos censebat, altera C.~Caesaris, 
qui illos publicatis bonis per municipia Italiae distribuendos 
ac vinculis sempiternis tenendos existimabat.
Cum M.~Cicero consul Nonis Decembribus senatum in aede Iovis 
Statoris consuleret, quid de iis coniurationis Catilinae sociis 
fieri placeret, qui in custodiam traditi essent, factum est, ut 
duae potissimum sententiae proponerentur, una D.~Silani consulis 
designati, qui morte multandos illos censebat, altera C.~Caesaris, 
qui illos publicatis bonis per municipia Italiae distribuendos 
ac vinculis sempiternis tenendos existimabat.
Cum M.~Cicero consul Nonis Decembribus senatum in aede Iovis 
Statoris consuleret, quid de iis coniurationis Catilinae sociis 
fieri placeret, qui in custodiam traditi essent, factum est, ut 
duae potissimum sententiae proponerentur, una D.~Silani consulis 
designati, qui morte multandos illos censebat, altera C.~Caesaris, 
qui illos publicatis bonis per municipia Italiae distribuendos 
ac vinculis sempiternis tenendos existimabat.
\end{abstract}

\end{frontmatter}
%===============================================================================
% define acronyms here
\begin{acronym}%
    \acro{SPC}{Subspace Predictive Control}
    \acro{CL-SPC}{closed-loop Subspace Predictive Control}
    \acro{DeePC}{Data-enabled Predictive Control}
    \acro{LTI}{linear time-invariant}
    \acro{IV}{instrumental variable}
    \acro{IVs}{instrumental variables}
    \acro{N4SID}{Numerical algorithm for Subspace State Space System Identification}
    \acro{MPC}{Model Predictive Control}
    \acro{CL-DeePC}{closed-loop Data-enabled Predictive Control}
\end{acronym}%
%===============================================================================
\section{Introduction}
Trends of increasing data availability as well as increasing system complexity make a compelling case for data-driven control methods as an alternative to model-based approaches~\cite{Hou2013}. Whilst data can be used with indirect data-driven approaches to synthesise a model by means of system identification, direct data-driven control approaches are promising because they derive a control law directly from data without having to realize an explicit system model as an (often costly) intermediate step.

Since its conception in~\cite{Coulson2019}, a form of direct data-driven control called \ac{DeePC} has seen considerable development. \ac{DeePC} uses Willems' Fundamental Lemma from the field of behavioural systems theory, which states that for a deterministic system any sufficiently persistently exciting input-output trajectory parameterizes all possible future input-output trajectories~\cite{Willems2005}. In effect, \ac{DeePC} exploits Willems' Fundamental Lemma in a receding horizon optimal control framework found also in \ac{MPC}. For non-deterministic systems, the parameterization described by Willems' Fundamental Lemma needs to consider noise as well. However, since the noise is typically unknown, parameterizations based on only input-output data r


Contributions
\begin{enumerate}
\item Demonstration of a closed-loop identification issue in the presence of noise that arises with regular \ac{DeePC}.
\item The development of two equivalent \ac{CL-DeePC} algorithms (using \ac{IVs}) that do not suffer from this closed-loop issue and are also more sample efficient.
\item Demonstrate an equivalence between the developed \ac{CL-DeePC} algorithm and \ac{CL-SPC}.
\item A comparison of the attained performance of the regular and closed-loop \ac{DeePC} algorithms is demonstrated by means of simulations.
\end{enumerate}
\section{Preliminaries}
This section presents the employed system model, notation, and the considered control problem.

\subsection{System model}
Consider a non-deterministic discrete \ac{LTI} system $\mathcal{S}$ whose dynamics is described in the so-called \textit{innovation} form by
\begin{subequations}\label{eqn:SS_innovation}
\begin{empheq}[left=\mathcal{S}_\mathcal{I}\empheqlbrace]{align}
    x_{k+1} &= Ax_k + Bu_k + Ke_k,\label{eqn:SSi_x}\\
	y_k &= Cx_k + Du_k + e_k \label{eqn:SSi_y},
  \end{empheq}
\end{subequations}
in which the subscript $k\in\mathbb{Z}$ denotes the discrete time index, ${x_k\in\mathbb{R}^n}$, ${u_k\in\mathbb{R}^r}$, ${y_k\in\mathbb{R}^l}$, ${e_k\in\mathbb{R}^l}$ respectively represent states, inputs, outputs, and zero-mean white innovation noise with variance $R_\mathrm{e} > 0$, and $\{A,B,C,D,K\}$ are system matrices of compatible dimensions. Without loss of generality we henceforth assume the data to be generated by a minimal system realization. %also done in Breschi2022
In accordance with Kalman filtering theory from which this representation derives, $K$ represents a unique (and optimal) Kalman filter gain matrix that renders ${\tilde{A}=A-KC}$ asymptotically stable (see, e.g., \citet[Sec.~5.7]{Verhaegen2007a}). By substituting \eqnref{eqn:SSi_y} into \eqnref{eqn:SSi_x} one alternatively obtains the equivalent predictor form
\begin{subequations}\label{eqn:SS_predictor}
\begin{empheq}[left=\mathcal{S}_\mathcal{P}\empheqlbrace]{align}
	x_{k+1} &= \tilde{A}x_k + \tilde{B}u_k + Ky_k,\label{eqn:SSp_x}\\
	y_k &= Cx_k + Du_k + e_k \label{eqn:SSp_y},
  \end{empheq}
% \begin{align}
	% x_{k+1} &= \tilde{A}x_k + \tilde{B}u_k + Ky_k,\label{eqn:SSp_x}\\
	% y_k &= Cx_k + Du_k + e_k \label{eqn:SSp_y},
% \end{align}
\end{subequations}
in which $\tilde{B}=B-KD$.
%
% ${x_k\in\mathbb{R}^n}$, ${e_k\in\mathbb{R}^l}$, and $K\in\mathbb{R}^{n\times l}$ respectively represent states, zero-mean white innovation noise, and a unique Kalman gain matrix that renders ${\tilde{A}=A-KC}$ asymptotically stable.
%
% in which ${\bar{x}_k\in\mathbb{R}^n}$, ${u_k\in\mathbb{R}^r}$, ${y_k\in\mathbb{R}^l}$, ${w_k\in\mathbb{R}^n}$, ${v_k\in\mathbb{R}^l}$ respectively represent states, inputs, outputs, process noise, and measurement noise, and $\{A,B,C,D\}$ are system matrices of compatible dimensions. If the process and measurement noise are either zero-mean white or colored Gaussian sequences such that the system $\mathcal{S}$ satisfies notions of detectability and reachability there exists an equivalent innovation form $\mathcal{F}_\mathcal{I}$ that is based on Kalman filtering (see~\citet[p.~112-113\todo{Check pages}]{Anderson1979}, or \citet[p.~162]{Verhaegen2007a} for details) and is given by
%
\subsection{Notation and definitions}\label{sec:notation}
Having described the different system representations of the considered system, this section introduces some useful preliminary notation and definitions.
%
% To start, several strictly-positive integers are denoted by $s,q,p,f,N,\bar{N}\in\mathbb{Z}_{>0}$ and are frequently used to indicate window lengths. Throughout this article, time indices $i$, $\hat{i}$, $j$, and $k$, will be used together with the shorthand exemplified by $k_p=k+p$.

To start, block-Toeplitz matrices are defined by
\begin{align}\label{eqn:blockToeplitz} 
\mathcal{T}_s(\mathcal{A},\mathcal{B},\mathcal{C},\mathcal{D}) =\scriptsize{
	\begin{bmatrix}
		\mathcal{D}         & 0         & 0      & \cdots  & 0\\
		\mathcal{C}\mathcal{B}        & \mathcal{D}         & 0      & \cdots  & 0\\
		\mathcal{C}\mathcal{A}\mathcal{B}       & \mathcal{C}\mathcal{B}        & \mathcal{D}      & \cdots & 0\\
		\vdots    &  \vdots & \ddots & \ddots & \vdots\\
		\mathcal{C}\mathcal{A}^{s-2}\mathcal{B} & \mathcal{C}\mathcal{A}^{s-3}\mathcal{B} & \cdots  & \mathcal{C}\mathcal{B}     & \mathcal{D}
	\end{bmatrix}},
\end{align}
in which the subscript $s\in\mathbb{Z}_{>0}$ is a generic strictly-positive integer that is used here to indicate the number of block-rows, the matrices $\mathcal{A}$, $\mathcal{B}$, $\mathcal{C}$, and $\mathcal{D}$ are all of compatible dimensions. Let ${I_s\in\mathbb{R}^{s\times s}}$ represent an identity matrix. Equation~\eqnref{eqn:blockToeplitz} thereby defines the block-Toeplitz matrices
\begin{alignat*}{2}
\mathcal{T}_s^\mathrm{u}&=\mathcal{T}_s(A,B,C,D),\quad  &\mathcal{H}_s&=\mathcal{T}_s(A,K,C,I_l),\\
\widetilde{\mathcal{T}}_s^\mathrm{u}&=\mathcal{T}_s(\tilde{A},\tilde{B},C,D),\quad  &\widetilde{\mathcal{H}}_s&=\mathcal{T}_s(\tilde{A},K,-C,I_l).
\end{alignat*}

In addition, two extended observability matrices are defined by
\begin{align*}
\Gamma_s &= \begin{bmatrix}C^\top & (CA)^\top & \cdots & (CA^{s-1})^\top\end{bmatrix}^\top,\\%\quad\text{and}\\
\widetilde{\Gamma}_s &= \begin{bmatrix}C^\top & (C\tilde{A})^\top & \cdots & (C\tilde{A}^{s-1})^\top\end{bmatrix}^\top.
\end{align*}
The extended observability matrix $\Gamma_s$ is used to define a system property commonly referred to as its lag.
\begin{defn}\label{def:lag}
    A system's lag is the smallest integer $\ell\in\mathbb{Z}_{>0}$ such that the extended observability matrix $\Gamma_\ell$ is of rank $n$.
\end{defn}
For any observable dynamic \ac{LTI} system, $1\leq \ell \leq n$.

Moreover, two extended reversed controllability matrices are defined as 
\begin{align*}
\tKp{u} &= \begin{bmatrix} \tilde{A}^{p-1}\tilde{B}\:\, & \tilde{A}^{p-2}\tilde{B} & \cdots & \tilde{A}\tilde{B} & \tilde{B}\:\, \end{bmatrix},\\%\text{ and}\\
\tKp{y} &= \begin{bmatrix} \tilde{A}^{p-1}K & \tilde{A}^{p-2}K & \cdots & \tilde{A}K & K \end{bmatrix},
\end{align*}
in which $p\in\mathbb{Z}_{>0}$ is a past data window length.

Furthermore, data vectors are denoted as examplified by
\begin{align*}
    \datavec{u}{k,s} = \begin{bmatrix} u_k^\top & u_{k+1}^\top & \cdots & u_{k+s-1}^\top\end{bmatrix}^\top,
\end{align*}
which represents a vector of ordered input data starting at time index $k$, and containing a number of samples $s$.

Using such data vectors it is possible to concisely define block-Hankel data matrices. Such a block-Hankel data matrix is examplified by
\begin{align*}
    U_{k,s,q} = \frac{1}{\sqrt{q}}\begin{bmatrix}
        \datavec{u}{k,s} & \datavec{u}{k+1,s} & \cdots & \datavec{u}{k+q-1,s}
    \end{bmatrix},
\end{align*}
in which $q\in\mathbb{Z}_{>0}$ is another generic positive-definite integer that here represents the number of successive input data vectors with $s$ data samples each, starting from time index $k$. Note the block-anti diagonal structure of block-Hankel matrices. This notion of block-Hankel data matrices is employed to define the notion of persistency of excition.
\begin{defn}\label{def:PE}
    A signal consisting of samples ${w_j\in\mathbb{R}^q},$ $j\in[k,\,k+s+N-2]$ is persistently exciting of order $s$ if the associated block-Hankel matrix ${W_{k,s,N}\in\mathbb{R}^{sq \times N}}$ is full row rank.
\end{defn}
Predicted samples, vectors, or matrices of data are indicated by $\hat{(\cdot)}$.% whilst those that are comprised in part of predictions are indicated by $\tilde{(\cdot)}$.

For convenience, the notation $\Phi$ is reserved to denote an often encountered concatenation of input and output block-Hankel matrices that is given by
\begin{align*}
    \Phi_{k,s,q} = \begin{bmatrix}
        U_{k,p,q}^\top & U_{k_p,s,q}^\top & Y_{k,p,q}^\top
    \end{bmatrix}^\top,
\end{align*}
in which $k$, and $q$ respectively indicate the starting index and parameterize the dimensions of the concatenated matrix together with $p$.
% 
% ==============================================================================================================================================================
% ==============================================================================================================================================================
\subsection{The data equations}\label{sec:DerivingDataEquations}
This section derives several fundamental relations called data-equations, which reformulate \eqref{eqn:SS_innovation} and \eqref{eqn:SS_predictor} in terms of block-Hankel matrices.

To this end, it can be shown by iterative application of respectively \eqref{eqn:SS_innovation} and \eqref{eqn:SS_predictor} that%
\begin{align}
    Y_{k_p,s,q} &= \Gamma_s X_{k_p,1,q} + \mathcal{T}_s^\mathrm{u} U_{k_p,s,q} + \mathcal{H}_s E_{k_p,s,q}\label{eq:Yf1},\\
    \begin{split}%
    Y_{k_p,s,q} &= \widetilde{\Gamma}_s X_{k_p,1,q} + \widetilde{\mathcal{T}}_s^\mathrm{u} U_{k_p,s,q} + E_{k_p,s,q}\\
    &\phantom{=}+(I_{sl}-\widetilde{\mathcal{H}}_s)Y_{k_p,s,q},
    \end{split}\label{eq:Yf2}
\end{align}
in which we employ a recurring shorthand for time indices examplified by $k_p=k+p$. Furthermore, the initial states can be rewritten in terms of preceding states and input-output data using \eqref{eqn:SS_predictor} as%
\begin{align}\label{eq:Xip}
    X_{k_p,1,q} = \tilde{A}^p X_{k,1,q} + \tKp{u} U_{k,p,q} + \tKp{y} Y_{k,p,q}.
    % \begin{bmatrix}
    %     Y_{i,p,q}\\
    %     U_{i,p,q}
    % \end{bmatrix}.
\end{align}
% in which $\tKp{}=\big[\tKp{y}\;\;\tKp{u}\big]$.
Substitute \eqref{eq:Xip} into \eqref{eq:Yf1} and \eqref{eq:Yf2} %and apply Assumption~\ref{assum:initial_contribution}
to obtain two so called data equations:
\begin{align}
    Y_{k_p,s,q} &= L_s \Phi_{k,s,q} + \mathcal{H}_s E_{k_p,s,q} + \Gamma_s \tilde{A}^p X_{k,1,q},\label{eq:DataEq1}\\
    \begin{split}
    Y_{k_p,s,q} &= \widetilde{L}_s \Phi_{k,s,q} + E_{k_p,s,q} + (I_{sl}-\widetilde{\mathcal{H}}_s) Y_{k_p,s,q} \\
    &\phantom{=}+ \widetilde{\Gamma}_s \tilde{A}^p X_{k,1,q},
    \end{split}\label{eq:DataEq2}
\end{align}
in which, for convenience, we define, two reoccurring so called `dynamic matrices' as
\begin{align*}
    L_s &= \begin{bmatrix} \Gamma_s\tKp{u} & \mathcal{T}_s^\mathrm{u} & \Gamma_s\tKp{y} \end{bmatrix},\\%\text{ and}\\
    \widetilde{L}_s &= \begin{bmatrix} \widetilde{\Gamma}_s\tKp{u} & \widetilde{\mathcal{T}}_s^\mathrm{u} & \widetilde{\Gamma}_s\tKp{y} \end{bmatrix}.
\end{align*}

% in which $L_s$, $\widetilde{L}_s$, $\Phi_{k,s,q}$ are defined in Section \ref{sec:notation}. %Similarly to \eqref{eq:DataEq1} and \eqref{eq:DataEq2}, the future outputs are defined by
% Although a more generic representation was kept above for later analysis, for \ac{CL-DeePC}, $s=1$. This reduces the complexity of the above equations since ${\widetilde{L}_1=L_1=\big[ C\tKp{u} \; D \; C\tKp{y} \big]}$ and $\widetilde{\mathcal{H}}_1=\mathcal{H}_1=I_l$.
%
\subsection{Receding horizon control problem formulation}
To clearly demarcate the scope of this work this section presents the receding horizon control problem that is considered here, which is formulated using \eqref{eq:Yf1} for a single column ($q=1$) from time index $k=\hat{i}$ and with $s=f\in\mathbb{Z}_{>0}$ as future prediction window length, as
\begin{subequations}
\begin{alignat}{2}
    &\min_{\datavec{u}{\hat{i}_p,f}} ||\datavec{\hat{y}}{\hat{i}_p,f}-\datavec{r}{\hat{i}_p,f}||_Q^2 + ||\datavec{u}{\hat{i}_p,f}||_R^2 \span\span\\
    \text{s.t.}\quad& &\datavec{\hat{y}}{\hat{i}_p,f}&=\Gamma_f x_{\hat{i}_p}+\mathcal{T}_f^\mathrm{u}\datavec{u}{\hat{i}_p,f}%+\mathcal{H}_f\datavec{e}{\hat{i}_p,f}
    ,\label{eq:SS_iter}\\
   && x_{\hat{i}_p}&=x_\mathrm{ini},\label{eq:x_ini}\\
   && u_k&\in\mathcal{U},\quad \hat{y}_k\in\mathcal{Y},\quad \forall k\in\big[\hat{i}_p,\,\hat{i}_p+f\big),
\end{alignat}
\end{subequations}
% \begin{subequations}
%     \begin{alignat}{3}
%       &\min_{\substack{u_k,\\ \mathclap{\forall k\in\left[\hat{i}_p,\,\hat{i}_p+f\right)}}}&&\quad\sum_{k=\hat{i}_p}^{\mathclap{\hat{i}_p+f-1}} \mathbb{E} \left[||y_k-r_k||_Q^2\right] + ||u_k||_R^2\span\span \\
%      \text{s.t.}& &&\text{\eqref{eqn:SS_innovation},} &&\forall k\in\left[\hat{i}_p,\,\hat{i}_p+f\right),\label{eq:SS_iter}\\
%      &&x_{\hat{i}_p}&=x_\mathrm{ini},&&\label{eq:x_ini}\\
%      && u_k&\in \mathcal{U}, \; \mathbb{E}[y_k]\in\mathcal{Y},\quad &&\forall k\in\left[\hat{i}_p,\,\hat{i}_p+f\right),
% \end{alignat}
% \end{subequations}
% in which $\hat{i}_p$ is the first future time index, at which time there is an initial state $x_\mathrm{ini}$, $f\in\mathbb{Z}_{>0}$ is a future prediction window length, and $r_k\in\mathbb{R}^l$ is a sample of an output reference trajectory. In addition, $||\cdot||_\cdot$ denotes a weighted 2-norm, with $Q\in\mathbb{R}^{l\times l}$ and $R\in\mathbb{R}^{r\times r}$ as respectively positive semi-definite and positive definite user-defined weighting matrices. Furthermore, $\mathbb{E}[\cdot]$ represents an expectation with respect to the future innovation signal that is conditioned on the future input sequence and the initial state.
in which $\datavec{r}{\hat{i}_p,f}\in\mathbb{R}^{fl}$ is a data vector of a future reference trajectory, $x_\mathrm{ini}$ is an initial state at the first future time $k=\hat{i}_p$, and $||\cdot||_{(\cdot)}$ denotes a weighted Euclidian norm, $Q\in\mathbb{R}^{fl\times fl}$ and $R\in\mathbb{R}^{fr\times fr}$ are respectively positive semi-definite and positive definite user-defined weighting matrices, and $\mathcal{U}$ and $\mathcal{Y}$ respectively represent sets of allowable inputs and outputs. %Furthermore, $\mathbb{E}[\cdot]$ represents an expectation w.r.t. the future innovation signal $\datavec{e}{\hat{i}_p,f}$ that is conditioned on the future input sequence $\datavec{u}{\hat{i}_p,f}$ and the initial state $x_{\hat{i}_p}$, and $\mathcal{U}$ and $\mathcal{Y}$ respectively represent sets of allowable inputs and outputs.

Without knowledge of the system matrices $\{A,B,C,D,K\}$ and the initial state $x_\mathrm{ini}$, but given sufficiently informative past input-output data from  intervals $k\in[i,\,i+\bar{N})$ and $k\in[\hat{i},\,\hat{i}_p)$ that may overlap\footnote{Depending on the difference $\hat{i}-i>0$ and the number of samples $\bar{N}$.} and have been collected in closed-loop, the principal goal is to find an unbiased behavioural output predictor to replace the unknown relations \eqref{eq:SS_iter} and \eqref{eq:x_ini}.
\section{Closed-loop Data-enabled Predictive Control}
This section presents the main result of this article, providing contribution~(\ref{contribution:solves_CL_issue}) whereby we develop \ac{CL-DeePC}. An intuitive explanation is first offered before a more rigorous proof of the method is provided.

As a solution to the identification bias that arises in closed-loop due to correlation between inputs and noise (a demonstration thereof is deferred to Section~\ref{sec:CL_ID_issue}) it is possible to estimate a step-ahead predictor~\citep{Ljung1996}. A prediction horizon length $f>1$ is of more practical use in receding horizon optimal control settings, to which end step-ahead predictors can be applied sequentially. 

Fig.~\ref{fig:CL-DeePC} and \eqref{eq:CL_DeePC_no_IVs} illustrate how this idea is employed in \ac{CL-DeePC}. A step-ahead predictor can be obtained from regular \ac{DeePC} (see Fig.~\ref{fig:regular-DeePC} and \eqref{eq:regular_DeePC_no_IVs}) with $f=1$. In \ac{CL-DeePC} the successive columns of $G$ (from left to right) and their corresponding columns on the right-hand side correspond to sequential applications of regular \ac{DeePC} with $f=1$ to the same matrix of sufficiently persistently exciting past input-output data on the left-hand side as well as time-shifted windows of input-output data on the right-hand side that parameterize successive initial states.
% In this section \ac{CL-DeePC} is first introduced and compared to regular \ac{DeePC}, thereby offering an intuitive explanation for contribution \ref{contribution:solves_CL_issue}. More rigorous explanations follow in subsequent sections. What follows directly below is the principal result of this work.
%
% \begin{figure*}[h!]
%      \centering
%      \begin{subfigure}[b]{\columnwidth}
%          \centering
%          \begin{tikzpicture}
    % defining constants
    \def\stepSize{0.25}
    \def\Nnum{9}
    \def\fnum{8}
    \def\pnum{4}
    
    % Defining lengths
    \newlength{\onelen}
    \setlength{\onelen}{\stepSize cm}
    \newlength{\BrCl}
    \setlength{\BrCl}{0.075cm}
    \newlength{\BrIn}
    \setlength{\BrIn}{0.15cm}
    \newlength{\plen}
    \setlength{\plen}{1cm}%{\pnum\stepSize cm}
    \newlength{\flen}
    \setlength{\flen}{2cm}%{\fnum*\stepSize cm}
    \newlength{\Nlen}
    \setlength{\Nlen}{2.25cm}%{\Nnum\stepSize cm}%should be 2*p+one
    \newlength{\MatClearance}
    \setlength{\MatClearance}{0.3cm}
    
    % grid lines for guidance
    % \draw[gray,step=0.5] (-0,-3) grid (8,3);

    % ======================= drawing data matrix =======================
    \path (0,-\onelen) coordinate (M1A);
    \path ([xshift=\Nlen]M1A) coordinate (M1B);
    \path ([yshift=2*\plen+2*\onelen]M1B) coordinate (M1C);
    \path ([xshift=-\Nlen]M1C) coordinate (M1D);
    \draw[line width=1.5pt] ([xshift=\BrIn,yshift=-\BrCl]M1A) -- ([xshift=-\BrCl,yshift=-\BrCl]M1A) -- ([xshift=-\BrCl,yshift=\BrCl]M1D) -- ([xshift=\BrIn,yshift=\BrCl]M1D); %left bracket
    \draw[line width=1.5pt] ([xshift=-\BrIn,yshift=-\BrCl]M1B) -- ([xshift=\BrCl,yshift=-\BrCl]M1B) -- ([xshift=\BrCl,yshift=\BrCl]M1C) -- ([xshift=-\BrIn,yshift=\BrCl]M1C); %right bracket
    \draw[line width=1pt] ([yshift=\onelen]M1A) -- ([yshift=\onelen]M1B); % dividing matrix into blocks
    \fill[black, opacity=0.5] (M1A) rectangle (M1C);
    \foreach \x in {0,...,8} { % drawing black dots
    \foreach \y in {-1,...,8} {
      \fill ( {(\x+0.5)*\onelen}, {(\y+0.5)*\onelen} ) circle (1pt);
    }}
    \draw[line width=0.1pt] ([yshift=-\plen-\onelen]M1D) -- ([yshift=-\plen-\onelen]M1C);
    \draw[line width=0.1pt] ([yshift=-\plen]M1D) -- ([yshift=-\plen]M1C);

    % coordinates for diagonals
    \path ([yshift=-\plen]M1D) coordinate (M1stair1A);
    \path ([xshift=\plen]M1D) coordinate (M1stair1I);
    % black diagonals
    % \foreach \i in {0,-1}{
    % \foreach \dy in {1,...,11}{
    % \ifthenelse{\dy<5.5}{
    % % This code will be executed if \dy<9.5
    %     \draw[dash pattern=on 1pt off 2pt, line width=0.1pt] ([xshift= 0.5*\onelen,yshift=\i*(\plen+\onelen)-(\dy+0.5)*\onelen]M1D) -- ([xshift= (\dy+0.5)*\onelen,yshift=\i*(\plen+\onelen)-0.5*\onelen]M1D);
    % }{
    % % This code will be executed if \dy>=10
    %     \ifthenelse{\dy<8.5}{
    %         % This code will be executed if \dy<=12
    %         \draw[dash pattern=on 1pt off 2pt, line width=0.1pt] ([xshift= (\dy-3.5)*\onelen,yshift=\i*(\plen+\onelen)-\plen-\onelen+0.5*\onelen]M1D) -- ([xshift=(\dy-8.5)*\onelen,yshift=\i*(\plen+\onelen)-0.5\onelen]M1C);
    %         }{
    %         % This code will be executed if \dy>=13
    %         \draw[dash pattern=on 1pt off 2pt, line width=0.1pt] ([xshift= (\dy-3.5)*\onelen,yshift=(\i-1)*(\plen+\onelen)+0.5\onelen]M1D) -- ([xshift= -0.5*\onelen,yshift=\i*(\plen+\onelen)-(\dy-7.5)*\onelen]M1C);
    %     }
    % }
    % }}
    
    % ---------------- drawing left brace and matrix ----------------
    \path ([xshift=-0.5\BrIn,yshift=-2\BrCl]M1A) coordinate (Brace1L);
    \path ([xshift=0.5\BrIn,yshift=-2\BrCl]M1B) coordinate (Brace1R);
    \draw[decorate, decoration={calligraphic brace, amplitude=3pt, mirror, aspect=0.75},line width=1pt] (Brace1L) -- (Brace1R);
    \node (mat1) at ($(Brace1L)!0.75!(Brace1R) + (0,-3pt)$) {};
    \node[below] at (mat1.center) {$\begin{bmatrix}
        U_{i,p,N}\\U_{i_p,1,N}\\Y_{i,p,N}\\ \hline Y_{i_p,1,N}
    \end{bmatrix}$};
    % brace to the left
    % \path ([xshift=-0.75cm-0.05cm,yshift=-4pt]mat1.center) coordinate (Brace1T);
    % \path ([yshift=-1.73cm]Brace1T) coordinate (Brace1B);
    % \draw[pen colour=gray!60,decorate, decoration={calligraphic brace, amplitude=3pt, mirror, aspect=0.2},line width=0.75pt] (Brace1T) -- (Brace1B);
    % % Zp
    % \node (Zp) at ($(Brace1T)!0.2!(Brace1B) + (-3pt,-0.8pt)$) {};
    % \node[left,text=gray!100,align=center] at ([xshift=0.5mm,yshift=-0.18cm]Zp.center) {\scriptsize{\shortstack{$=$\\$\Phi_{i,1,N}$}}};
  
    % ======================= drawing G =======================
    % useful coordinates
    \path ([xshift=\MatClearance,yshift=\onelen]M1B) coordinate (M2A);
    \path ([xshift=\flen]M2A) coordinate (M2B);
    \path ([yshift=\Nlen]M2B) coordinate (M2C);
    \path ([xshift=-\flen]M2C) coordinate (M2D);

    % brackets
    \draw[line width=1.5pt] ([xshift=\BrIn,yshift=-\BrCl]M2A) -- ([xshift=-\BrCl,yshift=-\BrCl]M2A) -- ([xshift=-\BrCl,yshift=\BrCl]M2D) -- ([xshift=\BrIn,yshift=\BrCl]M2D); %left bracket
    \draw[line width=1.5pt] ([xshift=-\BrIn,yshift=-\BrCl]M2B) -- ([xshift=\BrCl,yshift=-\BrCl]M2B) -- ([xshift=\BrCl,yshift=\BrCl]M2C) -- ([xshift=-\BrIn,yshift=\BrCl]M2C); %right bracket
    
    % red fill
    \fill[red!50,opacity=0.5] (M2A) rectangle (M2C);

    % drawing red dots
    \foreach \x in {0,...,7} {
    \foreach \y in {0,...,8} {
      \fill[red] ([xshift=(\x+0.5)*\onelen,yshift=(\y+0.5)*\onelen]M2A) circle (1pt);%{(\x+0.5)*\onelen+\Nlen+0.5cm}, {(\y+0.5)*\onelen}
    }}

    % dividers
    \foreach \x in {1,...,7}{\draw[line width=0.1pt] ([xshift=\x*\onelen]M2A) -- ([xshift=\x*\onelen]M2D);}

    % drawing middle brace and matrix
    \coordinate (Brace2L) at ([xshift=-0.5\BrIn]M2A |- Brace1L);
    \coordinate (Brace2R) at ([xshift=0.5\BrIn]M2B |- Brace1L);
    \draw[decorate, decoration={calligraphic brace, amplitude=3pt, mirror, aspect=0.5},line width=1pt] (Brace2L) -- (Brace2R);
    \node (mat1) at ($(Brace2L)!0.5!(Brace2R) + (0,-3pt)$) {};
    \node[below] at ([yshift=-0.75cm]mat1.center) {$\underbrace{
    \begin{bmatrix}
        g_1 & g_2 & \cdots & g_f
    \end{bmatrix}}_{= G}$};
    
    % ======================= drawing equal sign ======================= 
    \path ([xshift=\MatClearance*3/4,yshift=\Nlen/2-0.1cm]M2B) coordinate (EqA);
    \path ([xshift=0.4cm]EqA) coordinate (EqB);
    \path ([yshift=0.2cm]EqB) coordinate (EqC);
    \path ([xshift=-0.4cm]EqC) coordinate (EqD);
    \draw[line width = 1.5 pt] (EqA) -- (EqB);
    \draw[line width = 1.5 pt] (EqD) -- (EqC);

    % ---------------- equation sign below ----------------
    \node[below] at ([xshift=5.65\onelen,yshift=-1.05cm]mat1.center) {$=$};

    % ======================= drawing RHS =======================
    % inside of matrix
    \path ([xshift=\MatClearance*3/4,yshift=-\Nlen/2+0.1cm-\onelen]EqB) coordinate (M3A);
    \path ([xshift=\flen]M3A) coordinate (M3B);
    \path ([yshift=2*\plen+2*\onelen]M3B) coordinate (M3C);
    \path ([xshift=-\flen]M3C) coordinate (M3D);
    % top left black triangle
    \path ([yshift=-\plen-\onelen/2]M3D) coordinate (t1A);
    \path ([xshift=\plen+\onelen/2]M3D) coordinate (t1B);
    % top red trapezoid
    \path ([yshift=-\onelen/2]t1A) coordinate (t2A);
    \path ([xshift=\flen]t2A) coordinate (t2B);
    % bottom black triangle
    \path ([yshift=-\plen-\onelen/2]t2A) coordinate (t3A);
    \path ([xshift=\plen+\onelen/2]t2A) coordinate (t3B);
    % bottom red trapezoid
    \path ([yshift=-\onelen/2]t3A) coordinate (t4A);

    % top 'staircase' coordinates
    \path ([yshift=-\plen]M3D) coordinate (stair1A);
    \path ([xshift=\onelen]stair1A) coordinate (stair1B);
    \path ([yshift=\onelen]stair1B) coordinate (stair1C);
    \path ([xshift=\onelen]stair1C) coordinate (stair1D);
    \path ([yshift=\onelen]stair1D) coordinate (stair1E);
    \path ([xshift=\onelen]stair1E) coordinate (stair1F);
    \path ([yshift=\onelen]stair1F) coordinate (stair1G);
    \path ([xshift=\onelen]stair1G) coordinate (stair1H);
    \path ([yshift=\onelen]stair1H) coordinate (stair1I);

    % bottom 'staircase' coordinates
    \path ([yshift=-\onelen]stair1A) coordinate (stair2J);
    \path ([yshift=-\plen]stair2J) coordinate (stair2A);
    \path ([xshift=\onelen]stair2A) coordinate (stair2B);
    \path ([yshift=\onelen]stair2B) coordinate (stair2C);
    \path ([xshift=\onelen]stair2C) coordinate (stair2D);
    \path ([yshift=\onelen]stair2D) coordinate (stair2E);
    \path ([xshift=\onelen]stair2E) coordinate (stair2F);
    \path ([yshift=\onelen]stair2F) coordinate (stair2G);
    \path ([xshift=\onelen]stair2G) coordinate (stair2H);
    \path ([yshift=\onelen]stair2H) coordinate (stair2I);
    
    % fill figures
    % \fill[black, opacity=0.5]  (t1A) -- (t1B) -- (M3D) -- cycle;% top black
    % \fill[red!50, opacity=0.5] (t2A) -- (t2B) -- (M3C) -- (t1B) -- (t1A) -- cycle;% top red
    % \fill[black, opacity=0.5]  (t3A) -- (t3B) -- (t2A) -- cycle;% bottom black
    % \fill[red!50, opacity=0.5] (t4A) -- (M3B) -- (t2B) -- (t3B) -- (t3A) -- cycle; % bottom red
    \fill[black,opacity=0.5]  (M3D) -- (stair1A) -- (stair1B) -- (stair1C) -- (stair1D) -- (stair1E) -- (stair1F) -- (stair1G) -- (stair1H) -- (stair1I) -- cycle;
    \fill[red!50,opacity=0.5] ([xshift=\flen-\onelen]stair1B) -- (stair1B) -- (stair1C) -- (stair1D) -- (stair1E) -- (stair1F) -- (stair1G) -- (stair1H) -- (stair1I) -- (M3C) -- cycle;
    \fill[black,opacity=0.5]  (stair2J) -- (stair2A) -- (stair2B) -- (stair2C) -- (stair2D) -- (stair2E) -- (stair2F) -- (stair2G) -- (stair2H) -- (stair2I) -- cycle;
    \fill[red!50,opacity=0.5] ([xshift=\flen-\onelen]stair2B) -- (stair2B) -- (stair2C) -- (stair2D) -- (stair2E) -- (stair2F) -- (stair2G) -- (stair2H) -- (stair2I) -- ([yshift=-\plen-\onelen]M3C) -- cycle;
    
    % draw brackets
    \draw[line width=1.5pt] ([xshift=\BrIn,yshift=-\BrCl]M3A) -- ([xshift=-\BrCl,yshift=-\BrCl]M3A) -- ([xshift=-\BrCl,yshift=\BrCl]M3D) -- ([xshift=\BrIn,yshift=\BrCl]M3D); %left bracket
    \draw[line width=1.5pt] ([xshift=-\BrIn,yshift=-\BrCl]M3B) -- ([xshift=\BrCl,yshift=-\BrCl]M3B) -- ([xshift=\BrCl,yshift=\BrCl]M3C) -- ([xshift=-\BrIn,yshift=\BrCl]M3C); %right bracket
    
    % U_{i,p,N}
    \path ([xshift=0.5*\onelen,yshift=-\plen+0.5*\onelen]M3D) coordinate (tlbA);
    \foreach \x in {0,...,7}{
    \foreach \y in {0,...,3}{
        \ifthenelse{{\y>\x}\OR{\y=\x}}{%
        % \fill[black, opacity=0.5] ([xshift={(\x-0.5)*\onelen},yshift={(\y-0.5)*\onelen}]tlbA) rectangle ([xshift={(\x+0.5)*\onelen},yshift={(\y+0.5)*\onelen}]tlbA);
        \fill[black] ([xshift={\x*\onelen},yshift={\y*\onelen}]tlbA) circle (1pt);
        }{%
        % \fill[red!50,opacity=0.5] ([xshift={(\x-0.5)*\onelen},yshift={(\y-0.5)*\onelen}]tlbA) rectangle ([xshift={(\x+0.5)*\onelen},yshift={(\y+0.5)*\onelen}]tlbA);%<do this if false>
        \fill[red] ([xshift={\x*\onelen},yshift={(\y*\onelen}]tlbA) circle (1pt);
        }%
    }}
    
    % U_{i_p,1,N}
    \path ([yshift=-\onelen]tlbA) coordinate (tlbB);
    \fill[red!50,opacity=0.5] (stair1A) rectangle ([xshift=\flen,yshift=-\onelen]stair1A);
    \foreach \x in {0,...,7}{\fill[red] ([xshift=\x*\onelen]tlbB) circle (1pt);}
    
    % Y_{i,p,N}
    \path ([yshift=-\plen]tlbB) coordinate (tlbC);
    \foreach \x in {0,...,7}{
    \foreach \y in {0,...,3}{
        \ifthenelse{{\y>\x}\OR{\y=\x}}{%
        \fill[black] ([xshift={\x*\onelen},yshift={\y*\onelen}]tlbC) circle (1pt);%<do this if true>
        }{%
        \fill[red] ([xshift={\x*\onelen},yshift={(\y*\onelen}]tlbC) circle (1pt);%<do this if false>
        }%
    }}

    % Y_{i_p,1,N}
    \path ([yshift=-\onelen]tlbC) coordinate (tlbD);
    \fill[red!50,opacity=0.5] (stair2A) rectangle ([xshift=\flen,yshift=-\onelen]stair2A);
    \foreach \x in {0,...,7}{\fill[red] ([xshift=\x*\onelen]tlbD) circle (1pt);}

    \foreach \dy in {0,-\plen-\onelen}{%
    % black diagonals
    \draw[dash pattern=on 1pt off 2pt, line width=0.1pt] ([xshift= 1/2*\onelen,yshift=\dy+5/2*\onelen]stair1A) -- ([xshift=-5/2*\onelen,yshift=\dy-1/2*\onelen]stair1I);
    \draw[dash pattern=on 1pt off 2pt, line width=0.1pt] ([xshift= 1/2*\onelen,yshift=\dy+3/2*\onelen]stair1A) -- ([xshift=-3/2*\onelen,yshift=\dy-1/2*\onelen]stair1I);
    \draw[dash pattern=on 1pt off 2pt, line width=0.1pt] ([xshift= 1/2*\onelen,yshift=\dy+1/2*\onelen]stair1A) -- ([xshift=-1/2*\onelen,yshift=\dy-1/2*\onelen]stair1I);
    % red diagonals
    \draw[red,dash pattern=on 1pt off 2pt, line width=0.1pt] ([xshift= 1/2*\onelen,yshift=\dy-1/2*\onelen]stair1A) -- ([xshift=1/2*\onelen,yshift=\dy-1/2*\onelen]stair1I);
    \draw[red,dash pattern=on 1pt off 2pt, line width=0.1pt] ([xshift= 3/2*\onelen,yshift=\dy-1/2*\onelen]stair1A) -- ([xshift=3/2*\onelen,yshift=\dy-1/2*\onelen]stair1I);
    \draw[red,dash pattern=on 1pt off 2pt, line width=0.1pt] ([xshift= 5/2*\onelen,yshift=\dy-1/2*\onelen]stair1A) -- ([xshift=5/2*\onelen,yshift=\dy-1/2*\onelen]stair1I);
    \draw[red,dash pattern=on 1pt off 2pt, line width=0.1pt] ([xshift= 7/2*\onelen,yshift=\dy-1/2*\onelen]stair1A) -- ([xshift=7/2*\onelen,yshift=\dy-1/2*\onelen]stair1I);
    \draw[red,dash pattern=on 1pt off 2pt, line width=0.1pt] ([xshift= 9/2*\onelen,yshift=\dy-1/2*\onelen]stair1A) -- ([xshift=7/2*\onelen,yshift=\dy-3/2*\onelen]stair1I);
    \draw[red,dash pattern=on 1pt off 2pt, line width=0.1pt] ([xshift=11/2*\onelen,yshift=\dy-1/2*\onelen]stair1A) -- ([xshift=7/2*\onelen,yshift=\dy-5/2*\onelen]stair1I);
    \draw[red,dash pattern=on 1pt off 2pt, line width=0.1pt] ([xshift=13/2*\onelen,yshift=\dy-1/2*\onelen]stair1A) -- ([xshift=7/2*\onelen,yshift=\dy-7/2*\onelen]stair1I);
    }
    
    % draw dividers
    \draw[line width=0.1pt] ([yshift=-\plen-\onelen]M3D) -- ([yshift=-\plen-\onelen]M3C);
    \draw[line width=0.1pt] ([yshift=-\plen]M3D) -- ([yshift=-\plen]M3C);
    \draw[line width=1pt] ([yshift=\onelen]M3A) -- ([yshift=\onelen]M3B);

    % ---------------- drawing right brace and matrix ----------------
    % brace below matrix
    \path ([xshift=-0.5\BrIn,yshift=-2\BrCl]M3A) coordinate (Brace3L);
    \path ([xshift=0.5\BrIn,yshift=-2\BrCl]M3B) coordinate (Brace3R);
    \draw[decorate, decoration={calligraphic brace, amplitude=3pt, mirror, aspect=0.25},line width=1pt] (Brace3L) -- (Brace3R);
    % matrix below brace
    \node (mat3) at ($(Brace3L)!0.25!(Brace3R) + (0,-3pt)$) {};
    \node[below] at (mat3.center) {$\begin{bmatrix}
        U_{\hat{i},p,f}\\U_{\hat{i}_p,1,f}\\Y_{\hat{i},p,f}\\ \hline \widehat{Y}_{\hat{i}_p,1,f}
    \end{bmatrix}$};
    % tag & label (on RHS of column)
    \node[below] at ([xshift=2.13cm,yshift=-0.9cm]mat3.center) {$\refstepcounter{equation}(\theequation)\label{eq:CL_DeePC_no_IVs}$};
    % brace to the right
    % \path ([xshift=0.7cm+0.05cm]mat3.center |- Brace1T) coordinate (Brace3T);%([xshift=0.7cm,yshift=-4pt]mat3.center) coordinate (Brace3T);
    % \path (Brace3T |- Brace1B) coordinate (Brace3B);
    % \draw[pen colour=gray!60,decorate, decoration={calligraphic brace, amplitude=3pt, aspect=0.2},line width=0.75pt] (Brace3T) -- (Brace3B);
    % % Zf
    % \node (Zf) at ($(Brace3T)!0.2!(Brace3B) + (3pt,-1.5pt)$) {};
    % \node[right,text=gray!100] at ([yshift=-0.18cm]Zf.center) {\scriptsize{\shortstack{$=$\\$\Phi_{\hat{i},1,f}$}}};
    
    % ======================= length indicators ======================= 
    \draw[|-|] ([yshift=\onelen]M1D) -- node[above] {\scriptsize$N$} ([yshift=\onelen]M1C);
    \draw[|-|] ([yshift=\onelen]M2D) -- node[above] {\scriptsize$f$} ([yshift=\onelen]M2C);
    \draw[|-|] ([yshift=\onelen]M3D) -- node[above] {\scriptsize$f$} ([yshift=\onelen]M3C);
    \draw[|-|] ([xshift=\onelen]M3C) -- node[right] {\scriptsize$pr$} ([xshift=\onelen,yshift=\onelen]t2B);
    \draw[|-|] ([xshift=\onelen,yshift=\onelen]t2B) -- node[right] {\scriptsize$r$} ([xshift=\onelen]t2B);
    \draw[|-|] ([xshift=\onelen]t2B) -- node[right] {\scriptsize$pl$} ([xshift=\onelen,yshift=\onelen]M3B);
    \draw[|-|] ([xshift=\onelen,yshift=\onelen]M3B) -- node[right] {\scriptsize$l$} ([xshift=\onelen]M3B);
\end{tikzpicture}
%          \caption{\ac{CL-DeePC} involves $f$ sequential applications of a step-ahead predictor obtained from regular \ac{DeePC} with $f=1$ (see also Fig.~(b)), resulting in the dashed block-anti diagonals with the same $u_k$ or $y_k$ on the right hand side.}
%          \label{fig:CL-DeePC}
%      \end{subfigure}
%      \hfill
%      \begin{subfigure}[b]{\columnwidth}
%          \centering
%          \begin{tikzpicture}
    % defining constants
    \def\stepSize{0.25}
    \def\Nnum{9}
    \def\fnum{8}
    \def\pnum{4}
    
    % Defining lengths
    % \newlength{\onelen}
    \setlength{\onelen}{\stepSize cm}
    % \newlength{\BrCl}
    \setlength{\BrCl}{0.075cm}
    % \newlength{\BrIn}
    \setlength{\BrIn}{0.15cm}
    % \newlength{\plen}
    \setlength{\plen}{1cm}%{\pnum\stepSize cm}
    % \newlength{\flen}
    \setlength{\flen}{2cm}%{\fnum*\stepSize cm}
    % \newlength{\Nlen}
    \setlength{\Nlen}{2.25cm}%{\Nnum\stepSize cm}%should be 2*p+one
    % \newlength{\MatClearance}
    \setlength{\MatClearance}{0.3cm}
    
    % grid lines for guidance
    % \draw[gray,step=0.5] (-0,-7) grid (8,3);

    % ======================= drawing data matrix =======================
    \path (0,2\plen+\onelen) coordinate (M1D);
    \path ([yshift=-2\plen-2\flen]M1D) coordinate (M1A);
    \path ([xshift=\Nlen]M1A) coordinate (M1B);
    \path ([yshift=2*\plen+2*\flen]M1B) coordinate (M1C);
    \draw[line width=1.5pt] ([xshift=\BrIn,yshift=-\BrCl]M1A) -- ([xshift=-\BrCl,yshift=-\BrCl]M1A) -- ([xshift=-\BrCl,yshift=\BrCl]M1D) -- ([xshift=\BrIn,yshift=\BrCl]M1D); %left bracket
    \draw[line width=1.5pt] ([xshift=-\BrIn,yshift=-\BrCl]M1B) -- ([xshift=\BrCl,yshift=-\BrCl]M1B) -- ([xshift=\BrCl,yshift=\BrCl]M1C) -- ([xshift=-\BrIn,yshift=\BrCl]M1C); %right bracket
    \draw[line width=1pt] ([yshift=\flen]M1A) -- ([yshift=\flen]M1B); % dividing matrix into blocks
    \fill[black, opacity=0.5] (M1A) rectangle (M1C);
    \foreach \x in {0,...,8} { % drawing black dots
    \foreach \y in {-15,...,8} {
      \fill ( {(\x+0.5)*\onelen}, {(\y+0.5)*\onelen} ) circle (1pt);
    }}
    \draw[line width=0.1pt] ([yshift=-\plen-\flen]M1D) -- ([yshift=-\plen-\flen]M1C);
    \draw[line width=0.1pt] ([yshift=-\plen]M1D) -- ([yshift=-\plen]M1C);

    % coordinates for diagonals
    \path ([yshift=-\plen]M1D) coordinate (M1stair1A);
    \path ([xshift=\Nlen]M1D) coordinate (M1stair1I);
    % black diagonals
    % \foreach \i in {0,-1}{
    % \foreach \dy in {1,...,18}{
    % \ifthenelse{\dy<9.5}{
    % % This code will be executed if \dy<9.5
    %     \draw[dash pattern=on 1pt off 2pt, line width=0.1pt] ([xshift= 0.5*\onelen,yshift=\i*(\plen+\flen)-(\dy+0.5)*\onelen]M1D) -- ([xshift= (\dy+0.5)*\onelen,yshift=\i*(\plen+\flen)-0.5*\onelen]M1D);
    % }{
    % % This code will be executed if \dy>=10
    %     \ifthenelse{\dy<12.5}{
    %         % This code will be executed if \dy<=12
    %         \draw[dash pattern=on 1pt off 2pt, line width=0.1pt] ([xshift= 0.5*\onelen,yshift=\i*(\plen+\flen)-(\dy+0.5)*\onelen]M1D) -- ([xshift= -0.5*\onelen,yshift=\i*(\plen+\flen)-(\dy-7.5)*\onelen]M1C);
    %         }{
    %         % This code will be executed if \dy>=13
    %         \draw[dash pattern=on 1pt off 2pt, line width=0.1pt] ([xshift= (\dy-10.5)*\onelen,yshift=\i*(\plen+\flen)-\plen-\flen+0.5*\onelen]M1D) -- ([xshift= -0.5*\onelen,yshift=\i*(\plen+\flen)-(\dy-7.5)*\onelen]M1C);
    %     }
    % }
    % }}
    
    % ---------------- drawing left brace and matrix ----------------
    \path ([xshift=-0.5\BrIn,yshift=-2\BrCl]M1A) coordinate (Brace1L);
    \path ([xshift=0.5\BrIn,yshift=-2\BrCl]M1B) coordinate (Brace1R);
    \draw[decorate, decoration={calligraphic brace, amplitude=3pt, mirror, aspect=0.75},line width=1pt] (Brace1L) -- (Brace1R);
    \node (mat1) at ($(Brace1L)!0.75!(Brace1R) + (0,-3pt)$) {};
    \node[below] at (mat1.center) {$\begin{bmatrix}
        U_{i,p,N}\\U_{i_p,f,N}\\Y_{i,p,N}\\ \hline Y_{i_p,f,N}
    \end{bmatrix}$};
    % brace to the left
    % \path ([xshift=-0.75cm-0.05cm,yshift=-4pt]mat1.center) coordinate (Brace1T);
    % \path ([yshift=-1.73cm]Brace1T) coordinate (Brace1B);
    % \draw[pen colour=gray!60,decorate, decoration={calligraphic brace, amplitude=3pt, mirror, aspect=0.2},line width=0.75pt] (Brace1T) -- (Brace1B);
    % % Zp
    % \node (Zp) at ($(Brace1T)!0.2!(Brace1B) + (-3pt,-0.8pt)$) {};
    % \node[left,text=gray!100,align=center] at ([xshift=0.5mm,yshift=-0.18cm]Zp.center) {\scriptsize{\shortstack{$=$\\$\Phi_{i,f,N}$}}};
  
    % ======================= drawing G =======================
    % useful coordinates
    \path ([xshift=\MatClearance,yshift=-0.5\Nlen-\plen-\flen]M1C) coordinate (M2A);
    \path ([xshift=\onelen]M2A) coordinate (M2B);
    \path ([yshift=\Nlen]M2B) coordinate (M2C);
    \path ([xshift=-\onelen]M2C) coordinate (M2D);

    % brackets
    \draw[line width=1.5pt] ([xshift=0.5\BrIn,yshift=-\BrCl]M2A) -- ([xshift=-\BrCl,yshift=-\BrCl]M2A) -- ([xshift=-\BrCl,yshift=\BrCl]M2D) -- ([xshift=0.5\BrIn,yshift=\BrCl]M2D); %left bracket
    \draw[line width=1.5pt] ([xshift=-0.5\BrIn,yshift=-\BrCl]M2B) -- ([xshift=\BrCl,yshift=-\BrCl]M2B) -- ([xshift=\BrCl,yshift=\BrCl]M2C) -- ([xshift=-0.5\BrIn,yshift=\BrCl]M2C); %right bracket
    
    % red fill
    \fill[red!50,opacity=0.5] (M2A) rectangle (M2C);

    % drawing red dots
    \foreach \x in {0} {
    \foreach \y in {0,...,8} {
      \fill[red] ([xshift=(\x+0.5)*\onelen,yshift=(\y+0.5)*\onelen]M2A) circle (1pt);
    }}

    % drawing middle brace and matrix
    \coordinate (Brace2L) at ([xshift=-0.5\BrIn]M2A |- Brace1L);
    \coordinate (Brace2R) at ([xshift=0.5\BrIn]M2B |- Brace1L);
    \draw[decorate, decoration={calligraphic brace, amplitude=3pt, mirror, aspect=0.5},line width=1pt] (Brace2L) -- (Brace2R);
    \node (mat1) at ($(Brace2L)!0.5!(Brace2R) + (0,-3pt)$) {};
    \node[below] at ([yshift=-1.05cm]mat1.center) {$g$};
    
    % ======================= drawing equal sign ======================= 
    \path ([xshift=\MatClearance*3/4,yshift=\Nlen/2-0.1cm]M2B) coordinate (EqA);
    \path ([xshift=0.4cm]EqA) coordinate (EqB);
    \path ([yshift=0.2cm]EqB) coordinate (EqC);
    \path ([xshift=-0.4cm]EqC) coordinate (EqD);
    \draw[line width = 1.5 pt] (EqA) -- (EqB);
    \draw[line width = 1.5 pt] (EqD) -- (EqC);

    % ---------------- equation sign below ----------------
    \node[below] at ([xshift=2\onelen,yshift=-1.05cm]mat1.center) {$=$};

    % ======================= drawing RHS =======================
    % inside of matrix
    \path ([xshift=0.75\MatClearance]EqB |- M1A) coordinate (M3A);
    \path ([xshift=\onelen]M3A) coordinate (M3B);
    \path ([yshift=2\plen+2\flen]M3B) coordinate (M3C);
    \path ([xshift=-\onelen]M3C) coordinate (M3D);
    
    % fill figures
    \fill[black,opacity=0.5]  (M3D) -- ([yshift=-\plen]M3D) -- ([yshift=-\plen]M3C) -- (M3C) -- cycle;
    \fill[red!50,opacity=0.5] ([yshift=-\plen]M3D) -- ([yshift=-\plen-\flen]M3D) -- ([yshift=-\plen-\flen]M3C) -- ([yshift=-\plen]M3C) -- cycle;
    \fill[black,opacity=0.5] ([yshift=\flen]M3A) -- ([yshift=\flen]M3B) -- ([yshift=\flen+\plen]M3B) -- ([yshift=\flen+\plen]M3A) -- cycle;
    \fill[red!50,opacity=0.5]  (M3A) -- (M3B) -- ([yshift=\flen]M3B) -- ([yshift=\flen]M3A) -- cycle;
    
    % draw brackets
    \draw[line width=1.5pt] ([xshift=0.5\BrIn,yshift=-\BrCl]M3A) -- ([xshift=-\BrCl,yshift=-\BrCl]M3A) -- ([xshift=-\BrCl,yshift=\BrCl]M3D) -- ([xshift=0.5\BrIn,yshift=\BrCl]M3D); %left bracket
    \draw[line width=1.5pt] ([xshift=-0.5\BrIn,yshift=-\BrCl]M3B) -- ([xshift=\BrCl,yshift=-\BrCl]M3B) -- ([xshift=\BrCl,yshift=\BrCl]M3C) -- ([xshift=-0.5\BrIn,yshift=\BrCl]M3C); %right bracket
    
    % U_{i,p,1}
    \foreach \dy in {0,...,-3}{
        \fill[black] ([xshift=0.5\onelen,yshift=(\dy-0.5)*\onelen]M3D) circle (1pt);
    }

    % U_{i_p,f,1}
    \foreach \dy in {-4,...,-11}{
        \fill[red] ([xshift=0.5\onelen,yshift=(\dy-0.5)*\onelen]M3D) circle (1pt);
    }
    
    % Y_{i,p,1}
    \foreach \dy in {0,...,-3}{
        \fill[black] ([xshift=0.5\onelen,yshift=(\dy-0.5)*\onelen-\plen-\flen]M3D) circle (1pt);
    }

    % Y_{i_p,f,1}
    \foreach \dy in {-4,...,-11}{
        \fill[red] ([xshift=0.5\onelen,yshift=(\dy-0.5)*\onelen-\plen-\flen]M3D) circle (1pt);
    }

    % draw dividers
    \draw[line width=0.1pt] ([yshift=-\plen]M3D) -- ([yshift=-\plen]M3C);
    \draw[line width=0.1pt] ([yshift=-\plen-\flen]M3D) -- ([yshift=-\plen-\flen]M3C);
    \draw[line width=1pt] ([yshift=\flen]M3A) -- ([yshift=\flen]M3B);

    % ---------------- drawing right brace and matrix ----------------
    % brace below matrix
    \path ([xshift=-0.5\BrIn,yshift=-2\BrCl]M3A) coordinate (Brace3L);
    \path ([xshift=0.5\BrIn,yshift=-2\BrCl]M3B) coordinate (Brace3R);
    \draw[decorate, decoration={calligraphic brace, amplitude=3pt, mirror, aspect=0.5},line width=1pt] (Brace3L) -- (Brace3R);
    % matrix below brace
    \node (mat3) at ($(Brace3L)!0.5!(Brace3R) + (0,-3pt)$) {};
    \node[below] at ([xshift=0.35cm]mat3.center) {$\begin{bmatrix}
        U_{\hat{i},p,1}\\U_{\hat{i}_p,f,1}\\Y_{\hat{i},p,1}\\ \hline \widehat{Y}_{\hat{i}_p,f,1}
    \end{bmatrix}$};
    % tag & label (on RHS of column)
    \node[below left] at (8.35cm,-5cm) {$\refstepcounter{equation}(\theequation)\label{eq:regular_DeePC_no_IVs}$};
    % brace to the right
    % \path ([xshift=1.1cm]mat3.center |- Brace1T) coordinate (Brace3T);%([xshift=0.7cm,yshift=-4pt]mat3.center) coordinate (Brace3T);
    % \path (Brace3T |- Brace1B) coordinate (Brace3B);
    % \draw[pen colour=gray!60,decorate, decoration={calligraphic brace, amplitude=3pt, aspect=0.2},line width=0.75pt] (Brace3T) -- (Brace3B);
    % % Zf
    % \node (Zf) at ($(Brace3T)!0.2!(Brace3B) + (3pt,-1.5pt)$) {};
    % \node[right,text=gray!100] at ([yshift=-0.18cm]Zf.center) {\scriptsize{\shortstack{$=$\\$\Phi_{\hat{i},f,1}$}}};
    
    % ======================= length indicators ======================= 
    \draw[|-|] ([yshift=\onelen]M1D) -- node[above] {\scriptsize$N$} ([yshift=\onelen]M1C);
    \draw[|-|] ([xshift=\onelen]M3C) -- node[right] {\scriptsize$pr$} ([xshift=\onelen,yshift=-\plen]M3C);
    \draw[|-|] ([xshift=\onelen,yshift=-\plen]M3C) -- node[right] {\scriptsize$fr$} ([xshift=\onelen,yshift=-\plen-\flen]M3C);
    \draw[|-|] ([xshift=\onelen,yshift=-\plen-\flen]M3C) -- node[right] {\scriptsize$pl$} ([xshift=\onelen,yshift=\flen]M3B);
    \draw[|-|] ([xshift=\onelen,yshift=\flen]M3B) -- node[right] {\scriptsize$fl$} ([xshift=\onelen]M3B);
\end{tikzpicture}
%          \caption{In regular \ac{DeePC} a multi-step ahead predictor of prediction length $f$ is formed directly by taking a single linear combination $g$ of past input-output data. Past data on the right-hand side encodes information on an initial state.}
%         \label{fig:regular-DeePC}
%      \end{subfigure}
%      \caption{Visualization of known (black) and unknown (red) variables in \ac{CL-DeePC} (a) and regular \ac{DeePC} from~\citep{Coulson2019} (b) without \ac{IVs}. Each dot represents an input $u_k\in\mathbb{R}^r$, output $y_k\in\mathbb{R}^l$, or element of a vector $g$.}
% \end{figure*}
\begin{figure}[b!]
\centering
\begin{tikzpicture}
    % defining constants
    \def\stepSize{0.25}
    \def\Nnum{9}
    \def\fnum{8}
    \def\pnum{4}
    
    % Defining lengths
    \newlength{\onelen}
    \setlength{\onelen}{\stepSize cm}
    \newlength{\BrCl}
    \setlength{\BrCl}{0.075cm}
    \newlength{\BrIn}
    \setlength{\BrIn}{0.15cm}
    \newlength{\plen}
    \setlength{\plen}{1cm}%{\pnum\stepSize cm}
    \newlength{\flen}
    \setlength{\flen}{2cm}%{\fnum*\stepSize cm}
    \newlength{\Nlen}
    \setlength{\Nlen}{2.25cm}%{\Nnum\stepSize cm}%should be 2*p+one
    \newlength{\MatClearance}
    \setlength{\MatClearance}{0.3cm}
    
    % grid lines for guidance
    % \draw[gray,step=0.5] (-0,-3) grid (8,3);

    % ======================= drawing data matrix =======================
    \path (0,-\onelen) coordinate (M1A);
    \path ([xshift=\Nlen]M1A) coordinate (M1B);
    \path ([yshift=2*\plen+2*\onelen]M1B) coordinate (M1C);
    \path ([xshift=-\Nlen]M1C) coordinate (M1D);
    \draw[line width=1.5pt] ([xshift=\BrIn,yshift=-\BrCl]M1A) -- ([xshift=-\BrCl,yshift=-\BrCl]M1A) -- ([xshift=-\BrCl,yshift=\BrCl]M1D) -- ([xshift=\BrIn,yshift=\BrCl]M1D); %left bracket
    \draw[line width=1.5pt] ([xshift=-\BrIn,yshift=-\BrCl]M1B) -- ([xshift=\BrCl,yshift=-\BrCl]M1B) -- ([xshift=\BrCl,yshift=\BrCl]M1C) -- ([xshift=-\BrIn,yshift=\BrCl]M1C); %right bracket
    \draw[line width=1pt] ([yshift=\onelen]M1A) -- ([yshift=\onelen]M1B); % dividing matrix into blocks
    \fill[black, opacity=0.5] (M1A) rectangle (M1C);
    \foreach \x in {0,...,8} { % drawing black dots
    \foreach \y in {-1,...,8} {
      \fill ( {(\x+0.5)*\onelen}, {(\y+0.5)*\onelen} ) circle (1pt);
    }}
    \draw[line width=0.1pt] ([yshift=-\plen-\onelen]M1D) -- ([yshift=-\plen-\onelen]M1C);
    \draw[line width=0.1pt] ([yshift=-\plen]M1D) -- ([yshift=-\plen]M1C);

    % coordinates for diagonals
    \path ([yshift=-\plen]M1D) coordinate (M1stair1A);
    \path ([xshift=\plen]M1D) coordinate (M1stair1I);
    % black diagonals
    % \foreach \i in {0,-1}{
    % \foreach \dy in {1,...,11}{
    % \ifthenelse{\dy<5.5}{
    % % This code will be executed if \dy<9.5
    %     \draw[dash pattern=on 1pt off 2pt, line width=0.1pt] ([xshift= 0.5*\onelen,yshift=\i*(\plen+\onelen)-(\dy+0.5)*\onelen]M1D) -- ([xshift= (\dy+0.5)*\onelen,yshift=\i*(\plen+\onelen)-0.5*\onelen]M1D);
    % }{
    % % This code will be executed if \dy>=10
    %     \ifthenelse{\dy<8.5}{
    %         % This code will be executed if \dy<=12
    %         \draw[dash pattern=on 1pt off 2pt, line width=0.1pt] ([xshift= (\dy-3.5)*\onelen,yshift=\i*(\plen+\onelen)-\plen-\onelen+0.5*\onelen]M1D) -- ([xshift=(\dy-8.5)*\onelen,yshift=\i*(\plen+\onelen)-0.5\onelen]M1C);
    %         }{
    %         % This code will be executed if \dy>=13
    %         \draw[dash pattern=on 1pt off 2pt, line width=0.1pt] ([xshift= (\dy-3.5)*\onelen,yshift=(\i-1)*(\plen+\onelen)+0.5\onelen]M1D) -- ([xshift= -0.5*\onelen,yshift=\i*(\plen+\onelen)-(\dy-7.5)*\onelen]M1C);
    %     }
    % }
    % }}
    
    % ---------------- drawing left brace and matrix ----------------
    \path ([xshift=-0.5\BrIn,yshift=-2\BrCl]M1A) coordinate (Brace1L);
    \path ([xshift=0.5\BrIn,yshift=-2\BrCl]M1B) coordinate (Brace1R);
    \draw[decorate, decoration={calligraphic brace, amplitude=3pt, mirror, aspect=0.75},line width=1pt] (Brace1L) -- (Brace1R);
    \node (mat1) at ($(Brace1L)!0.75!(Brace1R) + (0,-3pt)$) {};
    \node[below] at (mat1.center) {$\begin{bmatrix}
        U_{i,p,N}\\U_{i_p,1,N}\\Y_{i,p,N}\\ \hline Y_{i_p,1,N}
    \end{bmatrix}$};
    % brace to the left
    % \path ([xshift=-0.75cm-0.05cm,yshift=-4pt]mat1.center) coordinate (Brace1T);
    % \path ([yshift=-1.73cm]Brace1T) coordinate (Brace1B);
    % \draw[pen colour=gray!60,decorate, decoration={calligraphic brace, amplitude=3pt, mirror, aspect=0.2},line width=0.75pt] (Brace1T) -- (Brace1B);
    % % Zp
    % \node (Zp) at ($(Brace1T)!0.2!(Brace1B) + (-3pt,-0.8pt)$) {};
    % \node[left,text=gray!100,align=center] at ([xshift=0.5mm,yshift=-0.18cm]Zp.center) {\scriptsize{\shortstack{$=$\\$\Phi_{i,1,N}$}}};
  
    % ======================= drawing G =======================
    % useful coordinates
    \path ([xshift=\MatClearance,yshift=\onelen]M1B) coordinate (M2A);
    \path ([xshift=\flen]M2A) coordinate (M2B);
    \path ([yshift=\Nlen]M2B) coordinate (M2C);
    \path ([xshift=-\flen]M2C) coordinate (M2D);

    % brackets
    \draw[line width=1.5pt] ([xshift=\BrIn,yshift=-\BrCl]M2A) -- ([xshift=-\BrCl,yshift=-\BrCl]M2A) -- ([xshift=-\BrCl,yshift=\BrCl]M2D) -- ([xshift=\BrIn,yshift=\BrCl]M2D); %left bracket
    \draw[line width=1.5pt] ([xshift=-\BrIn,yshift=-\BrCl]M2B) -- ([xshift=\BrCl,yshift=-\BrCl]M2B) -- ([xshift=\BrCl,yshift=\BrCl]M2C) -- ([xshift=-\BrIn,yshift=\BrCl]M2C); %right bracket
    
    % red fill
    \fill[red!50,opacity=0.5] (M2A) rectangle (M2C);

    % drawing red dots
    \foreach \x in {0,...,7} {
    \foreach \y in {0,...,8} {
      \fill[red] ([xshift=(\x+0.5)*\onelen,yshift=(\y+0.5)*\onelen]M2A) circle (1pt);%{(\x+0.5)*\onelen+\Nlen+0.5cm}, {(\y+0.5)*\onelen}
    }}

    % dividers
    \foreach \x in {1,...,7}{\draw[line width=0.1pt] ([xshift=\x*\onelen]M2A) -- ([xshift=\x*\onelen]M2D);}

    % drawing middle brace and matrix
    \coordinate (Brace2L) at ([xshift=-0.5\BrIn]M2A |- Brace1L);
    \coordinate (Brace2R) at ([xshift=0.5\BrIn]M2B |- Brace1L);
    \draw[decorate, decoration={calligraphic brace, amplitude=3pt, mirror, aspect=0.5},line width=1pt] (Brace2L) -- (Brace2R);
    \node (mat1) at ($(Brace2L)!0.5!(Brace2R) + (0,-3pt)$) {};
    \node[below] at ([yshift=-0.75cm]mat1.center) {$\underbrace{
    \begin{bmatrix}
        g_1 & g_2 & \cdots & g_f
    \end{bmatrix}}_{= G}$};
    
    % ======================= drawing equal sign ======================= 
    \path ([xshift=\MatClearance*3/4,yshift=\Nlen/2-0.1cm]M2B) coordinate (EqA);
    \path ([xshift=0.4cm]EqA) coordinate (EqB);
    \path ([yshift=0.2cm]EqB) coordinate (EqC);
    \path ([xshift=-0.4cm]EqC) coordinate (EqD);
    \draw[line width = 1.5 pt] (EqA) -- (EqB);
    \draw[line width = 1.5 pt] (EqD) -- (EqC);

    % ---------------- equation sign below ----------------
    \node[below] at ([xshift=5.65\onelen,yshift=-1.05cm]mat1.center) {$=$};

    % ======================= drawing RHS =======================
    % inside of matrix
    \path ([xshift=\MatClearance*3/4,yshift=-\Nlen/2+0.1cm-\onelen]EqB) coordinate (M3A);
    \path ([xshift=\flen]M3A) coordinate (M3B);
    \path ([yshift=2*\plen+2*\onelen]M3B) coordinate (M3C);
    \path ([xshift=-\flen]M3C) coordinate (M3D);
    % top left black triangle
    \path ([yshift=-\plen-\onelen/2]M3D) coordinate (t1A);
    \path ([xshift=\plen+\onelen/2]M3D) coordinate (t1B);
    % top red trapezoid
    \path ([yshift=-\onelen/2]t1A) coordinate (t2A);
    \path ([xshift=\flen]t2A) coordinate (t2B);
    % bottom black triangle
    \path ([yshift=-\plen-\onelen/2]t2A) coordinate (t3A);
    \path ([xshift=\plen+\onelen/2]t2A) coordinate (t3B);
    % bottom red trapezoid
    \path ([yshift=-\onelen/2]t3A) coordinate (t4A);

    % top 'staircase' coordinates
    \path ([yshift=-\plen]M3D) coordinate (stair1A);
    \path ([xshift=\onelen]stair1A) coordinate (stair1B);
    \path ([yshift=\onelen]stair1B) coordinate (stair1C);
    \path ([xshift=\onelen]stair1C) coordinate (stair1D);
    \path ([yshift=\onelen]stair1D) coordinate (stair1E);
    \path ([xshift=\onelen]stair1E) coordinate (stair1F);
    \path ([yshift=\onelen]stair1F) coordinate (stair1G);
    \path ([xshift=\onelen]stair1G) coordinate (stair1H);
    \path ([yshift=\onelen]stair1H) coordinate (stair1I);

    % bottom 'staircase' coordinates
    \path ([yshift=-\onelen]stair1A) coordinate (stair2J);
    \path ([yshift=-\plen]stair2J) coordinate (stair2A);
    \path ([xshift=\onelen]stair2A) coordinate (stair2B);
    \path ([yshift=\onelen]stair2B) coordinate (stair2C);
    \path ([xshift=\onelen]stair2C) coordinate (stair2D);
    \path ([yshift=\onelen]stair2D) coordinate (stair2E);
    \path ([xshift=\onelen]stair2E) coordinate (stair2F);
    \path ([yshift=\onelen]stair2F) coordinate (stair2G);
    \path ([xshift=\onelen]stair2G) coordinate (stair2H);
    \path ([yshift=\onelen]stair2H) coordinate (stair2I);
    
    % fill figures
    % \fill[black, opacity=0.5]  (t1A) -- (t1B) -- (M3D) -- cycle;% top black
    % \fill[red!50, opacity=0.5] (t2A) -- (t2B) -- (M3C) -- (t1B) -- (t1A) -- cycle;% top red
    % \fill[black, opacity=0.5]  (t3A) -- (t3B) -- (t2A) -- cycle;% bottom black
    % \fill[red!50, opacity=0.5] (t4A) -- (M3B) -- (t2B) -- (t3B) -- (t3A) -- cycle; % bottom red
    \fill[black,opacity=0.5]  (M3D) -- (stair1A) -- (stair1B) -- (stair1C) -- (stair1D) -- (stair1E) -- (stair1F) -- (stair1G) -- (stair1H) -- (stair1I) -- cycle;
    \fill[red!50,opacity=0.5] ([xshift=\flen-\onelen]stair1B) -- (stair1B) -- (stair1C) -- (stair1D) -- (stair1E) -- (stair1F) -- (stair1G) -- (stair1H) -- (stair1I) -- (M3C) -- cycle;
    \fill[black,opacity=0.5]  (stair2J) -- (stair2A) -- (stair2B) -- (stair2C) -- (stair2D) -- (stair2E) -- (stair2F) -- (stair2G) -- (stair2H) -- (stair2I) -- cycle;
    \fill[red!50,opacity=0.5] ([xshift=\flen-\onelen]stair2B) -- (stair2B) -- (stair2C) -- (stair2D) -- (stair2E) -- (stair2F) -- (stair2G) -- (stair2H) -- (stair2I) -- ([yshift=-\plen-\onelen]M3C) -- cycle;
    
    % draw brackets
    \draw[line width=1.5pt] ([xshift=\BrIn,yshift=-\BrCl]M3A) -- ([xshift=-\BrCl,yshift=-\BrCl]M3A) -- ([xshift=-\BrCl,yshift=\BrCl]M3D) -- ([xshift=\BrIn,yshift=\BrCl]M3D); %left bracket
    \draw[line width=1.5pt] ([xshift=-\BrIn,yshift=-\BrCl]M3B) -- ([xshift=\BrCl,yshift=-\BrCl]M3B) -- ([xshift=\BrCl,yshift=\BrCl]M3C) -- ([xshift=-\BrIn,yshift=\BrCl]M3C); %right bracket
    
    % U_{i,p,N}
    \path ([xshift=0.5*\onelen,yshift=-\plen+0.5*\onelen]M3D) coordinate (tlbA);
    \foreach \x in {0,...,7}{
    \foreach \y in {0,...,3}{
        \ifthenelse{{\y>\x}\OR{\y=\x}}{%
        % \fill[black, opacity=0.5] ([xshift={(\x-0.5)*\onelen},yshift={(\y-0.5)*\onelen}]tlbA) rectangle ([xshift={(\x+0.5)*\onelen},yshift={(\y+0.5)*\onelen}]tlbA);
        \fill[black] ([xshift={\x*\onelen},yshift={\y*\onelen}]tlbA) circle (1pt);
        }{%
        % \fill[red!50,opacity=0.5] ([xshift={(\x-0.5)*\onelen},yshift={(\y-0.5)*\onelen}]tlbA) rectangle ([xshift={(\x+0.5)*\onelen},yshift={(\y+0.5)*\onelen}]tlbA);%<do this if false>
        \fill[red] ([xshift={\x*\onelen},yshift={(\y*\onelen}]tlbA) circle (1pt);
        }%
    }}
    
    % U_{i_p,1,N}
    \path ([yshift=-\onelen]tlbA) coordinate (tlbB);
    \fill[red!50,opacity=0.5] (stair1A) rectangle ([xshift=\flen,yshift=-\onelen]stair1A);
    \foreach \x in {0,...,7}{\fill[red] ([xshift=\x*\onelen]tlbB) circle (1pt);}
    
    % Y_{i,p,N}
    \path ([yshift=-\plen]tlbB) coordinate (tlbC);
    \foreach \x in {0,...,7}{
    \foreach \y in {0,...,3}{
        \ifthenelse{{\y>\x}\OR{\y=\x}}{%
        \fill[black] ([xshift={\x*\onelen},yshift={\y*\onelen}]tlbC) circle (1pt);%<do this if true>
        }{%
        \fill[red] ([xshift={\x*\onelen},yshift={(\y*\onelen}]tlbC) circle (1pt);%<do this if false>
        }%
    }}

    % Y_{i_p,1,N}
    \path ([yshift=-\onelen]tlbC) coordinate (tlbD);
    \fill[red!50,opacity=0.5] (stair2A) rectangle ([xshift=\flen,yshift=-\onelen]stair2A);
    \foreach \x in {0,...,7}{\fill[red] ([xshift=\x*\onelen]tlbD) circle (1pt);}

    \foreach \dy in {0,-\plen-\onelen}{%
    % black diagonals
    \draw[dash pattern=on 1pt off 2pt, line width=0.1pt] ([xshift= 1/2*\onelen,yshift=\dy+5/2*\onelen]stair1A) -- ([xshift=-5/2*\onelen,yshift=\dy-1/2*\onelen]stair1I);
    \draw[dash pattern=on 1pt off 2pt, line width=0.1pt] ([xshift= 1/2*\onelen,yshift=\dy+3/2*\onelen]stair1A) -- ([xshift=-3/2*\onelen,yshift=\dy-1/2*\onelen]stair1I);
    \draw[dash pattern=on 1pt off 2pt, line width=0.1pt] ([xshift= 1/2*\onelen,yshift=\dy+1/2*\onelen]stair1A) -- ([xshift=-1/2*\onelen,yshift=\dy-1/2*\onelen]stair1I);
    % red diagonals
    \draw[red,dash pattern=on 1pt off 2pt, line width=0.1pt] ([xshift= 1/2*\onelen,yshift=\dy-1/2*\onelen]stair1A) -- ([xshift=1/2*\onelen,yshift=\dy-1/2*\onelen]stair1I);
    \draw[red,dash pattern=on 1pt off 2pt, line width=0.1pt] ([xshift= 3/2*\onelen,yshift=\dy-1/2*\onelen]stair1A) -- ([xshift=3/2*\onelen,yshift=\dy-1/2*\onelen]stair1I);
    \draw[red,dash pattern=on 1pt off 2pt, line width=0.1pt] ([xshift= 5/2*\onelen,yshift=\dy-1/2*\onelen]stair1A) -- ([xshift=5/2*\onelen,yshift=\dy-1/2*\onelen]stair1I);
    \draw[red,dash pattern=on 1pt off 2pt, line width=0.1pt] ([xshift= 7/2*\onelen,yshift=\dy-1/2*\onelen]stair1A) -- ([xshift=7/2*\onelen,yshift=\dy-1/2*\onelen]stair1I);
    \draw[red,dash pattern=on 1pt off 2pt, line width=0.1pt] ([xshift= 9/2*\onelen,yshift=\dy-1/2*\onelen]stair1A) -- ([xshift=7/2*\onelen,yshift=\dy-3/2*\onelen]stair1I);
    \draw[red,dash pattern=on 1pt off 2pt, line width=0.1pt] ([xshift=11/2*\onelen,yshift=\dy-1/2*\onelen]stair1A) -- ([xshift=7/2*\onelen,yshift=\dy-5/2*\onelen]stair1I);
    \draw[red,dash pattern=on 1pt off 2pt, line width=0.1pt] ([xshift=13/2*\onelen,yshift=\dy-1/2*\onelen]stair1A) -- ([xshift=7/2*\onelen,yshift=\dy-7/2*\onelen]stair1I);
    }
    
    % draw dividers
    \draw[line width=0.1pt] ([yshift=-\plen-\onelen]M3D) -- ([yshift=-\plen-\onelen]M3C);
    \draw[line width=0.1pt] ([yshift=-\plen]M3D) -- ([yshift=-\plen]M3C);
    \draw[line width=1pt] ([yshift=\onelen]M3A) -- ([yshift=\onelen]M3B);

    % ---------------- drawing right brace and matrix ----------------
    % brace below matrix
    \path ([xshift=-0.5\BrIn,yshift=-2\BrCl]M3A) coordinate (Brace3L);
    \path ([xshift=0.5\BrIn,yshift=-2\BrCl]M3B) coordinate (Brace3R);
    \draw[decorate, decoration={calligraphic brace, amplitude=3pt, mirror, aspect=0.25},line width=1pt] (Brace3L) -- (Brace3R);
    % matrix below brace
    \node (mat3) at ($(Brace3L)!0.25!(Brace3R) + (0,-3pt)$) {};
    \node[below] at (mat3.center) {$\begin{bmatrix}
        U_{\hat{i},p,f}\\U_{\hat{i}_p,1,f}\\Y_{\hat{i},p,f}\\ \hline \widehat{Y}_{\hat{i}_p,1,f}
    \end{bmatrix}$};
    % tag & label (on RHS of column)
    \node[below] at ([xshift=2.13cm,yshift=-0.9cm]mat3.center) {$\refstepcounter{equation}(\theequation)\label{eq:CL_DeePC_no_IVs}$};
    % brace to the right
    % \path ([xshift=0.7cm+0.05cm]mat3.center |- Brace1T) coordinate (Brace3T);%([xshift=0.7cm,yshift=-4pt]mat3.center) coordinate (Brace3T);
    % \path (Brace3T |- Brace1B) coordinate (Brace3B);
    % \draw[pen colour=gray!60,decorate, decoration={calligraphic brace, amplitude=3pt, aspect=0.2},line width=0.75pt] (Brace3T) -- (Brace3B);
    % % Zf
    % \node (Zf) at ($(Brace3T)!0.2!(Brace3B) + (3pt,-1.5pt)$) {};
    % \node[right,text=gray!100] at ([yshift=-0.18cm]Zf.center) {\scriptsize{\shortstack{$=$\\$\Phi_{\hat{i},1,f}$}}};
    
    % ======================= length indicators ======================= 
    \draw[|-|] ([yshift=\onelen]M1D) -- node[above] {\scriptsize$N$} ([yshift=\onelen]M1C);
    \draw[|-|] ([yshift=\onelen]M2D) -- node[above] {\scriptsize$f$} ([yshift=\onelen]M2C);
    \draw[|-|] ([yshift=\onelen]M3D) -- node[above] {\scriptsize$f$} ([yshift=\onelen]M3C);
    \draw[|-|] ([xshift=\onelen]M3C) -- node[right] {\scriptsize$pr$} ([xshift=\onelen,yshift=\onelen]t2B);
    \draw[|-|] ([xshift=\onelen,yshift=\onelen]t2B) -- node[right] {\scriptsize$r$} ([xshift=\onelen]t2B);
    \draw[|-|] ([xshift=\onelen]t2B) -- node[right] {\scriptsize$pl$} ([xshift=\onelen,yshift=\onelen]M3B);
    \draw[|-|] ([xshift=\onelen,yshift=\onelen]M3B) -- node[right] {\scriptsize$l$} ([xshift=\onelen]M3B);
\end{tikzpicture}
\caption{Visualization of known (black) and unknown (red) variables in \ac{CL-DeePC} without \ac{IVs}. Each dot represents an input $u_k\in\mathbb{R}^r$, output $y_k\in\mathbb{R}^l$, or element of the matrix $G$. \ac{CL-DeePC} involves $f$ sequential applications of a step-ahead predictor obtained from regular \ac{DeePC} with $f=1$ (see also Fig.~\ref{fig:regular-DeePC}), resulting in the dashed block-anti diagonals with the same $u_k$ or $y_k$ on the right hand side.}
\label{fig:CL-DeePC}
\end{figure}
\begin{figure}[b!]
\centering
\begin{tikzpicture}
    % defining constants
    \def\stepSize{0.25}
    \def\Nnum{9}
    \def\fnum{8}
    \def\pnum{4}
    
    % Defining lengths
    % \newlength{\onelen}
    \setlength{\onelen}{\stepSize cm}
    % \newlength{\BrCl}
    \setlength{\BrCl}{0.075cm}
    % \newlength{\BrIn}
    \setlength{\BrIn}{0.15cm}
    % \newlength{\plen}
    \setlength{\plen}{1cm}%{\pnum\stepSize cm}
    % \newlength{\flen}
    \setlength{\flen}{2cm}%{\fnum*\stepSize cm}
    % \newlength{\Nlen}
    \setlength{\Nlen}{2.25cm}%{\Nnum\stepSize cm}%should be 2*p+one
    % \newlength{\MatClearance}
    \setlength{\MatClearance}{0.3cm}
    
    % grid lines for guidance
    % \draw[gray,step=0.5] (-0,-7) grid (8,3);

    % ======================= drawing data matrix =======================
    \path (0,2\plen+\onelen) coordinate (M1D);
    \path ([yshift=-2\plen-2\flen]M1D) coordinate (M1A);
    \path ([xshift=\Nlen]M1A) coordinate (M1B);
    \path ([yshift=2*\plen+2*\flen]M1B) coordinate (M1C);
    \draw[line width=1.5pt] ([xshift=\BrIn,yshift=-\BrCl]M1A) -- ([xshift=-\BrCl,yshift=-\BrCl]M1A) -- ([xshift=-\BrCl,yshift=\BrCl]M1D) -- ([xshift=\BrIn,yshift=\BrCl]M1D); %left bracket
    \draw[line width=1.5pt] ([xshift=-\BrIn,yshift=-\BrCl]M1B) -- ([xshift=\BrCl,yshift=-\BrCl]M1B) -- ([xshift=\BrCl,yshift=\BrCl]M1C) -- ([xshift=-\BrIn,yshift=\BrCl]M1C); %right bracket
    \draw[line width=1pt] ([yshift=\flen]M1A) -- ([yshift=\flen]M1B); % dividing matrix into blocks
    \fill[black, opacity=0.5] (M1A) rectangle (M1C);
    \foreach \x in {0,...,8} { % drawing black dots
    \foreach \y in {-15,...,8} {
      \fill ( {(\x+0.5)*\onelen}, {(\y+0.5)*\onelen} ) circle (1pt);
    }}
    \draw[line width=0.1pt] ([yshift=-\plen-\flen]M1D) -- ([yshift=-\plen-\flen]M1C);
    \draw[line width=0.1pt] ([yshift=-\plen]M1D) -- ([yshift=-\plen]M1C);

    % coordinates for diagonals
    \path ([yshift=-\plen]M1D) coordinate (M1stair1A);
    \path ([xshift=\Nlen]M1D) coordinate (M1stair1I);
    % black diagonals
    % \foreach \i in {0,-1}{
    % \foreach \dy in {1,...,18}{
    % \ifthenelse{\dy<9.5}{
    % % This code will be executed if \dy<9.5
    %     \draw[dash pattern=on 1pt off 2pt, line width=0.1pt] ([xshift= 0.5*\onelen,yshift=\i*(\plen+\flen)-(\dy+0.5)*\onelen]M1D) -- ([xshift= (\dy+0.5)*\onelen,yshift=\i*(\plen+\flen)-0.5*\onelen]M1D);
    % }{
    % % This code will be executed if \dy>=10
    %     \ifthenelse{\dy<12.5}{
    %         % This code will be executed if \dy<=12
    %         \draw[dash pattern=on 1pt off 2pt, line width=0.1pt] ([xshift= 0.5*\onelen,yshift=\i*(\plen+\flen)-(\dy+0.5)*\onelen]M1D) -- ([xshift= -0.5*\onelen,yshift=\i*(\plen+\flen)-(\dy-7.5)*\onelen]M1C);
    %         }{
    %         % This code will be executed if \dy>=13
    %         \draw[dash pattern=on 1pt off 2pt, line width=0.1pt] ([xshift= (\dy-10.5)*\onelen,yshift=\i*(\plen+\flen)-\plen-\flen+0.5*\onelen]M1D) -- ([xshift= -0.5*\onelen,yshift=\i*(\plen+\flen)-(\dy-7.5)*\onelen]M1C);
    %     }
    % }
    % }}
    
    % ---------------- drawing left brace and matrix ----------------
    \path ([xshift=-0.5\BrIn,yshift=-2\BrCl]M1A) coordinate (Brace1L);
    \path ([xshift=0.5\BrIn,yshift=-2\BrCl]M1B) coordinate (Brace1R);
    \draw[decorate, decoration={calligraphic brace, amplitude=3pt, mirror, aspect=0.75},line width=1pt] (Brace1L) -- (Brace1R);
    \node (mat1) at ($(Brace1L)!0.75!(Brace1R) + (0,-3pt)$) {};
    \node[below] at (mat1.center) {$\begin{bmatrix}
        U_{i,p,N}\\U_{i_p,f,N}\\Y_{i,p,N}\\ \hline Y_{i_p,f,N}
    \end{bmatrix}$};
    % brace to the left
    % \path ([xshift=-0.75cm-0.05cm,yshift=-4pt]mat1.center) coordinate (Brace1T);
    % \path ([yshift=-1.73cm]Brace1T) coordinate (Brace1B);
    % \draw[pen colour=gray!60,decorate, decoration={calligraphic brace, amplitude=3pt, mirror, aspect=0.2},line width=0.75pt] (Brace1T) -- (Brace1B);
    % % Zp
    % \node (Zp) at ($(Brace1T)!0.2!(Brace1B) + (-3pt,-0.8pt)$) {};
    % \node[left,text=gray!100,align=center] at ([xshift=0.5mm,yshift=-0.18cm]Zp.center) {\scriptsize{\shortstack{$=$\\$\Phi_{i,f,N}$}}};
  
    % ======================= drawing G =======================
    % useful coordinates
    \path ([xshift=\MatClearance,yshift=-0.5\Nlen-\plen-\flen]M1C) coordinate (M2A);
    \path ([xshift=\onelen]M2A) coordinate (M2B);
    \path ([yshift=\Nlen]M2B) coordinate (M2C);
    \path ([xshift=-\onelen]M2C) coordinate (M2D);

    % brackets
    \draw[line width=1.5pt] ([xshift=0.5\BrIn,yshift=-\BrCl]M2A) -- ([xshift=-\BrCl,yshift=-\BrCl]M2A) -- ([xshift=-\BrCl,yshift=\BrCl]M2D) -- ([xshift=0.5\BrIn,yshift=\BrCl]M2D); %left bracket
    \draw[line width=1.5pt] ([xshift=-0.5\BrIn,yshift=-\BrCl]M2B) -- ([xshift=\BrCl,yshift=-\BrCl]M2B) -- ([xshift=\BrCl,yshift=\BrCl]M2C) -- ([xshift=-0.5\BrIn,yshift=\BrCl]M2C); %right bracket
    
    % red fill
    \fill[red!50,opacity=0.5] (M2A) rectangle (M2C);

    % drawing red dots
    \foreach \x in {0} {
    \foreach \y in {0,...,8} {
      \fill[red] ([xshift=(\x+0.5)*\onelen,yshift=(\y+0.5)*\onelen]M2A) circle (1pt);
    }}

    % drawing middle brace and matrix
    \coordinate (Brace2L) at ([xshift=-0.5\BrIn]M2A |- Brace1L);
    \coordinate (Brace2R) at ([xshift=0.5\BrIn]M2B |- Brace1L);
    \draw[decorate, decoration={calligraphic brace, amplitude=3pt, mirror, aspect=0.5},line width=1pt] (Brace2L) -- (Brace2R);
    \node (mat1) at ($(Brace2L)!0.5!(Brace2R) + (0,-3pt)$) {};
    \node[below] at ([yshift=-1.05cm]mat1.center) {$g$};
    
    % ======================= drawing equal sign ======================= 
    \path ([xshift=\MatClearance*3/4,yshift=\Nlen/2-0.1cm]M2B) coordinate (EqA);
    \path ([xshift=0.4cm]EqA) coordinate (EqB);
    \path ([yshift=0.2cm]EqB) coordinate (EqC);
    \path ([xshift=-0.4cm]EqC) coordinate (EqD);
    \draw[line width = 1.5 pt] (EqA) -- (EqB);
    \draw[line width = 1.5 pt] (EqD) -- (EqC);

    % ---------------- equation sign below ----------------
    \node[below] at ([xshift=2\onelen,yshift=-1.05cm]mat1.center) {$=$};

    % ======================= drawing RHS =======================
    % inside of matrix
    \path ([xshift=0.75\MatClearance]EqB |- M1A) coordinate (M3A);
    \path ([xshift=\onelen]M3A) coordinate (M3B);
    \path ([yshift=2\plen+2\flen]M3B) coordinate (M3C);
    \path ([xshift=-\onelen]M3C) coordinate (M3D);
    
    % fill figures
    \fill[black,opacity=0.5]  (M3D) -- ([yshift=-\plen]M3D) -- ([yshift=-\plen]M3C) -- (M3C) -- cycle;
    \fill[red!50,opacity=0.5] ([yshift=-\plen]M3D) -- ([yshift=-\plen-\flen]M3D) -- ([yshift=-\plen-\flen]M3C) -- ([yshift=-\plen]M3C) -- cycle;
    \fill[black,opacity=0.5] ([yshift=\flen]M3A) -- ([yshift=\flen]M3B) -- ([yshift=\flen+\plen]M3B) -- ([yshift=\flen+\plen]M3A) -- cycle;
    \fill[red!50,opacity=0.5]  (M3A) -- (M3B) -- ([yshift=\flen]M3B) -- ([yshift=\flen]M3A) -- cycle;
    
    % draw brackets
    \draw[line width=1.5pt] ([xshift=0.5\BrIn,yshift=-\BrCl]M3A) -- ([xshift=-\BrCl,yshift=-\BrCl]M3A) -- ([xshift=-\BrCl,yshift=\BrCl]M3D) -- ([xshift=0.5\BrIn,yshift=\BrCl]M3D); %left bracket
    \draw[line width=1.5pt] ([xshift=-0.5\BrIn,yshift=-\BrCl]M3B) -- ([xshift=\BrCl,yshift=-\BrCl]M3B) -- ([xshift=\BrCl,yshift=\BrCl]M3C) -- ([xshift=-0.5\BrIn,yshift=\BrCl]M3C); %right bracket
    
    % U_{i,p,1}
    \foreach \dy in {0,...,-3}{
        \fill[black] ([xshift=0.5\onelen,yshift=(\dy-0.5)*\onelen]M3D) circle (1pt);
    }

    % U_{i_p,f,1}
    \foreach \dy in {-4,...,-11}{
        \fill[red] ([xshift=0.5\onelen,yshift=(\dy-0.5)*\onelen]M3D) circle (1pt);
    }
    
    % Y_{i,p,1}
    \foreach \dy in {0,...,-3}{
        \fill[black] ([xshift=0.5\onelen,yshift=(\dy-0.5)*\onelen-\plen-\flen]M3D) circle (1pt);
    }

    % Y_{i_p,f,1}
    \foreach \dy in {-4,...,-11}{
        \fill[red] ([xshift=0.5\onelen,yshift=(\dy-0.5)*\onelen-\plen-\flen]M3D) circle (1pt);
    }

    % draw dividers
    \draw[line width=0.1pt] ([yshift=-\plen]M3D) -- ([yshift=-\plen]M3C);
    \draw[line width=0.1pt] ([yshift=-\plen-\flen]M3D) -- ([yshift=-\plen-\flen]M3C);
    \draw[line width=1pt] ([yshift=\flen]M3A) -- ([yshift=\flen]M3B);

    % ---------------- drawing right brace and matrix ----------------
    % brace below matrix
    \path ([xshift=-0.5\BrIn,yshift=-2\BrCl]M3A) coordinate (Brace3L);
    \path ([xshift=0.5\BrIn,yshift=-2\BrCl]M3B) coordinate (Brace3R);
    \draw[decorate, decoration={calligraphic brace, amplitude=3pt, mirror, aspect=0.5},line width=1pt] (Brace3L) -- (Brace3R);
    % matrix below brace
    \node (mat3) at ($(Brace3L)!0.5!(Brace3R) + (0,-3pt)$) {};
    \node[below] at ([xshift=0.35cm]mat3.center) {$\begin{bmatrix}
        U_{\hat{i},p,1}\\U_{\hat{i}_p,f,1}\\Y_{\hat{i},p,1}\\ \hline \widehat{Y}_{\hat{i}_p,f,1}
    \end{bmatrix}$};
    % tag & label (on RHS of column)
    \node[below left] at (8.35cm,-5cm) {$\refstepcounter{equation}(\theequation)\label{eq:regular_DeePC_no_IVs}$};
    % brace to the right
    % \path ([xshift=1.1cm]mat3.center |- Brace1T) coordinate (Brace3T);%([xshift=0.7cm,yshift=-4pt]mat3.center) coordinate (Brace3T);
    % \path (Brace3T |- Brace1B) coordinate (Brace3B);
    % \draw[pen colour=gray!60,decorate, decoration={calligraphic brace, amplitude=3pt, aspect=0.2},line width=0.75pt] (Brace3T) -- (Brace3B);
    % % Zf
    % \node (Zf) at ($(Brace3T)!0.2!(Brace3B) + (3pt,-1.5pt)$) {};
    % \node[right,text=gray!100] at ([yshift=-0.18cm]Zf.center) {\scriptsize{\shortstack{$=$\\$\Phi_{\hat{i},f,1}$}}};
    
    % ======================= length indicators ======================= 
    \draw[|-|] ([yshift=\onelen]M1D) -- node[above] {\scriptsize$N$} ([yshift=\onelen]M1C);
    \draw[|-|] ([xshift=\onelen]M3C) -- node[right] {\scriptsize$pr$} ([xshift=\onelen,yshift=-\plen]M3C);
    \draw[|-|] ([xshift=\onelen,yshift=-\plen]M3C) -- node[right] {\scriptsize$fr$} ([xshift=\onelen,yshift=-\plen-\flen]M3C);
    \draw[|-|] ([xshift=\onelen,yshift=-\plen-\flen]M3C) -- node[right] {\scriptsize$pl$} ([xshift=\onelen,yshift=\flen]M3B);
    \draw[|-|] ([xshift=\onelen,yshift=\flen]M3B) -- node[right] {\scriptsize$fl$} ([xshift=\onelen]M3B);
\end{tikzpicture}
\caption{Visualization of known (black) and unknown (red) variables in regular \ac{DeePC} without \ac{IVs}. Each dot represents an input $u_k\in\mathbb{R}^r$, output $y_k\in\mathbb{R}^l$, or element of the matrix $G$. A multi-step ahead predictor of prediction length $f$ is formed directly by taking a linear combination of past input and output data.\\\vspace{0.75mm}}
\label{fig:regular-DeePC}
\end{figure}
%
\setcounter{thm}{0}
\begin{thm}\label{theorem:main_result}
    Consider the minimal representation of a discrete non-deterministic system given by~\eqref{eqn:SS_innovation}. Let $\bar{u}_k=\left[u_k^\top \; e_k^\top\right]^\top$ represent an `extended' input. Given a sequence of input $u_k$ and output $y_k$ data of length $\bar{N}=N+p$, if the extended input sequence is persistently exciting of order $p+1+n$ and $p\geq\ell$, then the \ac{CL-DeePC} formulation given by~\eqref{eq:CL_DeePC_no_IVs} provides an unbiased output predictor.%asymptotically unbiased output predictor in the limit $N\rightarrow \infty$.
\end{thm}
\textbf{Proof:} 
Rewriting \eqref{eq:DataEq1} yields
\begin{align*}
    &\underbrace{\begin{bmatrix}-\Gamma_s \tilde{A}^p & -L_s & I_{sl}&-\mathcal{H}_s\end{bmatrix}}_{= \mathfrak{R}}
    \underbrace{\begin{bmatrix}
        X_{k,1,q}\\
        \Phi_{k,s,q}\\
        Y_{k_p,s,q}\\
        E_{k_p,s,q}
    \end{bmatrix}}_{\mathfrak{B}\mathfrak{D}}=\mathcal{O},\\
    &\mathfrak{R}\in\mathbb{R}^{sl\times n+(p+s)(r+l)+sl},\;\mathfrak{BD}\in\mathbb{R}^{n+(p+s)(r+l)+sl \times q},
\end{align*}
with $\mathfrak{R}$, and $\mathfrak{BD}$ as indicated for brevity. Similarly, using \eqref{eq:DataEq1} to decompose $\mathfrak{BD}$ into a matrix $\mathfrak{D}$ of exogenous inputs and initial states and a matrix $\mathfrak{B}$ to describe their effects obtains
\begin{align}
    &\mathfrak{R}
    \underbrace{\begin{bmatrix}
        I_n      & 0      & 0       & 0 & 0\\
        0        & I_{pr} & 0       & 0 & 0\\
        0        & 0      & I_{sr}  & 0 & 0\\
        \Gamma_p & \mathcal{T}_p^\mathrm{u} & 0 & \mathcal{H}_p & 0\\
        \varepsilon_1 & \varepsilon_2 & \mathcal{T}_s^\mathrm{u} & \varepsilon_3 & \mathcal{H}_s\\
        0 & 0 & 0 & 0 & I_{sl}
    \end{bmatrix}}_{=\mathfrak{B}}
    \underbrace{\begin{bmatrix}
        X_{k,1,q}\\
        U_{k,p,q}\\
        U_{k_p,s,q}\\
        E_{k,p,q}\\
        E_{k_p,s,q}
    \end{bmatrix}}_{=\mathfrak{D}}=\mathcal{O},\\
    &\mathfrak{B}\in\mathbb{R}^{n+(p+s)(r+l)+sl\times n+(p+s)(r+l)},\notag\\
    &\mathfrak{D}\in\mathbb{R}^{n+(p+s)(r+l)\times q},\notag
\end{align}
with $\varepsilon_1=\Gamma_s(\tilde{A}^p+\tKp{y}\Gamma_p)$, $\varepsilon_2=\Gamma_s(\tKp{u}+\tKp{y}\mathcal{T}_p^\mathrm{u})$, and $\varepsilon_3=\Gamma_s\tKp{y}\mathcal{H}_p$ defined as shown. By inspection, with $\mathcal{R}(\cdot)$ and $\mathcal{N}(\cdot)$ respectively denoting the range and nullspace of a matrix: $\mathcal{R}(\mathfrak{B}\mathfrak{D})\subseteq\mathcal{R}(\mathfrak{B})=\mathcal{N}(\mathfrak{R})$. Only if $\mathfrak{D}$ is full row rank $\mathcal{R}(\mathfrak{B}\mathfrak{D})=\mathcal{N}(\mathfrak{R})$.
\setcounter{thm}{0}
\begin{lem}[\cite{Willems2005}, Cor.~2(iii)]
    If the input sequence $\{u_k\}_{k=i}^{i+\epsilon+q-2}$ of a discrete \ac{LTI} system without noise and controllable $(A,B)$ is persistently exciting of order $\epsilon+n$ then the matrix $\left[X_{i,1,q}^\top\;U_{i,\epsilon,q}^\top\right]^\top$ is full row rank.
\end{lem}
Since controllability of $(A,B)$ implies controllability of $(A,[B,K])$, this yields the following corollary for extension to non-deterministic systems.
\setcounter{thm}{0}
\begin{cor}
    If for a controllable non-deterministic \ac{LTI} system of the form given by \eqref{eqn:SS_innovation} the sequence of inputs and noise $\{[u_k^\top\;e_k^\top]^\top\}_{k=i}^{i+\epsilon+q-2}$
\end{cor}


% ==============================================================================================================================================================
% ==============================================================================================================================================================
% \subsection{The basic idea}
% To derive a variant of \ac{DeePC} that does not suffer from the aforementioned closed-loop identification issue lets start by considering regular \ac{DeePC}. Closed-loop identification bias can be avoided by using a step-ahead predictor~\citep{Ljung1996}. However, such a short prediction horizon length is typically not conducive to good performance in a receding horizon control setting. Hence, to obtain another output prediction, the previous regular \ac{DeePC} problem is repeated with the same past data (but a different vector $g$ to span it) to obtain trajectories of input-output data that are shifted forwards one time step. This procedure can be repeated to obtain a desired prediction horizon length $f$. The entire procedure is succinctly described by Fig.~\ref{fig:CL-DeePC} and \eqref{eq:CL_DeePC_no_IVs}, \todo{check length conditions DeePC}

% in which $i$, $i_p$, $\hat{i}$, and $\hat{i}_p$ are discrete time indices (the first three indices lie in the past, and the last index $\hat{i}_p$ resembles the first future time index), and $G$ defines a matrix with columns given by the vectors $\{g_k\}^f_{k=1}$. Furthermore, note that $\Phi_{i,1,N}$ and $\Phi_{\hat{i},1,f}$ are present on respectively the top left and top right hand side.

% Treatment of different \ac{CL-DeePC} solution strategies is deferred to Section \ref{sec:SolutionMethods}. Suffice it for now to take note of the structure of \eqref{eq:CL_DeePC_no_IVs} illustrated by Fig.~\ref{fig:CL-DeePC} and to say that if the input is sufficiently persistently exciting such that $\Phi_{i,1,N}$ is full row rank~\cite[Chapt.~9.6.1]{Verhaegen2007a} then making $\Phi_{i,1,N}$ square and invertible by selecting $N=(p+1)r+pl$ minimizes the number of optimization variables.
% 
% As with regular \ac{DeePC} the idea is to find an optimal combination of allowable future inputs and outputs that minimizes a cost function that is possibly subject to constraints. To see how \eqref{eq:CL_DeePC_no_IVs} can be used in a receding horizon optimal control setting, first consider the top three blocks of the past data matrix. If the input is sufficiently persistently exciting then this matrix, $\Phi_{i,1,N}$, is full row rank~\cite[Chapt.~9.6.1]{Verhaegen2007a}. If, furthermore, ${N=(p+1)r+pl}$, then $\Phi_{i,1,N}$ becomes square and invertible. Hence, a unique solution for $G$ can then be obtained from the top three block equations of \eqref{eq:CL_DeePC_no_IVs} (in terms of future inputs and outputs), which can then be used to obtain output predictions by using the bottom block equation. The structure of \eqref{eq:CL_DeePC_no_IVs} is visualized by Fig.~\ref{fig:CL-DeePC}. This figure demonstrates that successive future output predictions are dependent on preceding input-output data as well as their concurrent input, opening the door to the sequential construction of an output predictor. This is described in \todo{section}, and can be used in a receding horizon optimal control framework.
% 
% 
% ==============================================================================================================================================================
% ==============================================================================================================================================================
% \subsection{The data equations}\label{sec:DerivingDataEquations}
% To motivate a noise mitigation strategy based that is based on \ac{IVs} that is explained hereafter, the data equations that justify this approach are first derived here using a state-space approach.

% To this end, it is straightforward to show by iterative application of respectively \eqref{eqn:SS_innovation} and \eqref{eqn:SS_predictor} that%
% \begin{align}
%     Y_{k_p,s,q} &= \Gamma_s X_{k_p,1,q} + \mathcal{T}_s^\mathrm{u} U_{k_p,s,q} + \mathcal{H}_s E_{k_p,s,q}\label{eq:Yf1},\\
%     \begin{split}%
%     Y_{k_p,s,q} &= \widetilde{\Gamma}_s X_{k_p,1,q} + \widetilde{\mathcal{T}}_s^\mathrm{u} U_{k_p,s,q} + E_{k_p,s,q}\\
%     &\phantom{=}+(I_{sl}-\widetilde{\mathcal{H}}_s)Y_{k_p,s,q}.
%     \end{split}\label{eq:Yf2}
% \end{align}
% It is possible to rewrite the initial states in terms of preceding states and input-output data using \eqref{eqn:SS_predictor} as%
% \begin{align}\label{eq:Xip}
%     X_{k_p,1,q} = \tilde{A}^p X_{k,1,q} + \tKp{u} U_{k,p,q} + \tKp{y} Y_{k,p,q}.
%     % \begin{bmatrix}
%     %     Y_{i,p,q}\\
%     %     U_{i,p,q}
%     % \end{bmatrix}.
% \end{align}
% % in which $\tKp{}=\big[\tKp{y}\;\;\tKp{u}\big]$.
% Substitute \eqref{eq:Xip} into \eqref{eq:Yf1} and \eqref{eq:Yf2} and apply Assumption~\ref{assum:initial_contribution} to obtain two so called data equations:
% \begin{align}
%     Y_{k_p,s,q} &= L_s \Phi_{k,s,q} + \mathcal{H}_s E_{k_p,s,q}\label{eq:DataEq1}\\
%     Y_{k_p,s,q} &= \widetilde{L}_s \Phi_{k,s,q} + E_{k_p,s,q} + (I_{sl}-\widetilde{\mathcal{H}}_s) Y_{k_p,s,q},\label{eq:DataEq2}
% \end{align}
% in which $L_s$, $\widetilde{L}_s$, $\Phi_{k,s,q}$ are defined in Section \ref{sec:notation}. %Similarly to \eqref{eq:DataEq1} and \eqref{eq:DataEq2}, the future outputs are defined by
% % \todo{use noiseless?}%always refer to noiseless version or beter to refer to ideal predictor here?
% % \begin{align}
% %     Y_{\hat{i}_p,s,f} &= L_s \Phi_{\hat{i},s,f} + \mathcal{H}_s E_{\hat{i}_p,s,f},\label{eq:DataEq1.2}\\
% %     Y_{\hat{i}_p,s,f} &= \widetilde{L}_s \Phi_{\hat{i},s,f} + (I_{sl}-\widetilde{\mathcal{H}}_s) Y_{\hat{i}_p,s,f} + E_{\hat{i}_p,s,f}\label{eq:DataEq2.2}.
% % \end{align}
% Although a more generic representation was kept above for later analysis, for \ac{CL-DeePC}, $s=1$. This reduces the complexity of the above equations since ${\widetilde{L}_1=L_1=\big[ C\tKp{u} \; D \; C\tKp{y} \big]}$ and $\widetilde{\mathcal{H}}_1=\mathcal{H}_1=I_l$.
%
% ==============================================================================================================================================================
% ==============================================================================================================================================================
\subsection{Willems' Fundamental Lemma \& Noise}
Equation~\eqref{eq:DataEq1} can be reformulated with $k=i$, $q=N$ or for an ideal noiseless output prediction with $k=\hat{i}$ as respectively
\begin{alignat}{2}
    \begin{bmatrix}
        -L_s & I_{sl}
    \end{bmatrix}&
    \begin{bmatrix}
        \Phi_{i,s,N}\\
        Y_{i_p,s,N}-\mathcal{H}_s E_{i_p,s,N}
    \end{bmatrix} = \mathcal{O},\label{eq:NoisyWFL1}\\%\mathcal{H}_s E_{i_p,s,N}, 
    \begin{bmatrix}
        -L_s & I_{sl}
    \end{bmatrix}&
    \begin{bmatrix}
        \Phi_{\hat{i},s,q}\\
        \widehat{Y}^*_{\hat{i}_p,s,q}
    \end{bmatrix} = \mathcal{O}, \label{eq:NoisyWFL2}
\end{alignat}
in which the asterisk indicates that the output prediction is ideal in the sense of being asymptotically unbiased. Multiplying \eqref{eq:NoisyWFL1} by $\mathcal{Z}^\top G\in\mathbb{R}^{N\times q}$, and subtracting \eqref{eq:NoisyWFL2} obtains
\begin{align}\label{eq:NoisyWFL3}
    \mkern-14mu\begin{bmatrix}
        \shortminus L_s & I_{sl}
    \end{bmatrix}
    \mkern-9mu\left(\mkern-3mu%
    \begin{bmatrix}
        \Phi_{i,s,N}\\
        Y_{i_p,s,N}\shortminus\mathcal{H}_s E_{i_p,s,N}
    \end{bmatrix}%
    \mkern-4mu\mathcal{Z}^\top G%\mkern-2mu
    -%-%
    \mkern-5mu\begin{bmatrix}
        \Phi_{\hat{i},s,q}\\
        \widehat{Y}^*_{\hat{i}_p,s,q}
    \end{bmatrix}\mkern-3mu\right)\mkern-6mu=\mkern-3mu\mathcal{O}\mkern-5mu%\mathcal{H}_s E_{i_p,s,N}\mathcal{Z}G
\end{align}
in which $\mathcal{Z}$ represents a yet unspecified matrix and $G$ represents a matrix that is akin to the likewise defined matrix from \eqref{eq:CL_DeePC_no_IVs} that contains all of the vectors $g_k$.

If the columns of the matrix with data on the left hand side of \eqref{eq:NoisyWFL1} span the entire nullspace of $\left[\shortminus L_s\;I_{sl}\right]$ and $\mathcal{Z}$ is full rank then all solutions to \eqref{eq:NoisyWFL3} are described by equating the term inside the parenthesis to zero. For now, consider the case that $\mathcal{Z}=I_N$, $s=f$, and $q=1$ in the absence of noise to recover the regular deterministic \ac{DeePC} equation~\citep{Coulson2019}. %Then one possible solution (since the matrix $\left[\shortminus L_s\;I_{sl}\right]$ is not full column rank) to \eqref{eq:NoisyWFL3} with $s=f$ and $q=1$ is given by the regular deterministic \ac{DeePC} equation~\cite{Coulson2019}:
\begin{align}\label{eq:regular_DeePC}
    \begin{bmatrix}
        \Phi_{i,f,N}\\
        Y_{i_p,f,N}
    \end{bmatrix}g=%
    \begin{bmatrix}
        \Phi_{\hat{i},f,1}\\
        \widehat{Y}_{\hat{i}_p,f,1}
    \end{bmatrix}.
\end{align}
Willems' Fundamental Lemma makes use of Assumptions~\ref{assum:PE} and~\ref{assum:controllability} to ensure that the entire nullspace of $\left[\shortminus L_f\;I_{fl}\right]$ is spanned by the data matrix on the left hand side of \eqref{eq:regular_DeePC}~\citep{Willems2005}. Assumption~\ref{assum:unique_initial} is furthermore necessary to guarantee the existence of a unique initial state and therefore output predictor. %This clearly reflects Willems' Fundamental Lemma, which states that for a deterministic \ac{LTI} system, any sufficiently persistently exciting past input-output trajectory parameterizes all possible future input-output trajectories~\cite{Willems2005}.\todo{WFL: what about nullspace in (12)}

In the presence of (unknown) noise, the term $Y_{i_p,s,N}-\mathcal{H}_s E_{i_p,s,N}$ from \eqref{eq:NoisyWFL3} cannot be determined to obtain an ideal output predictor. Instead, linear combinations of a noise-corrupted output $Y_{i_p,s,N}$ as in \eqref{eq:regular_DeePC} are taken, resulting in an error of the obtained output predictor due to implicit sampling of $\mathcal{H}_s E_{i_p,s,N}$. Moreover, the regular \ac{DeePC} formulation provided by \eqref{eq:regular_DeePC} may become inconsistent in the presence of noise, prompting the use of, e.g., slack variables and regularization~\citep{Coulson2019}.
%
% ==============================================================================================================================================================
% ==============================================================================================================================================================
\subsection{Noise mitigation using \acl{IVs}}
Notwithstanding potential benefits of beforementioned mechanisms to cope with noise, such methods do not provide a systematic way to mitigate noise at the source. To that end this section introduces the use of an \ac{IV}: $\mathcal{Z}\neq I_N$. In this context, the \ac{IV} is defined such that it is uncorrelated with the noise $E_{i_p,s,N}$ and preserves the (full row) rank of the data matrix $\Phi_{i,s,N}$ obtained from a sufficiently persistently exciting input. These two conditions are respectively formulated as
%
\begin{align}
    &\lim_{N\rightarrow\infty} \frac{1}{N}E_{i_p,s,N}\mathcal{Z}^\top = \mathcal{O},\label{eq:uncorrelated}\\
    \text{rank}\biggl(&\lim_{N\rightarrow\infty} \frac{1}{N}\Phi_{i,s,N}\mathcal{Z}^\top\biggl) =  \text{rank}(\Phi_{i,s,N}),\label{eq:rankconservation}
\end{align}
%
which motivates choosing $\mathcal{Z}=\Phi_{i,s,N}$~\cite[Chapt. 9.6]{Verhaegen2007a}. An important assumption that is hereby introduced to satisfy \eqref{eq:uncorrelated} is that inputs are uncorrelated with noise. To fulfill this assumption Section~\ref{sec:CL_ID_issue} will motivate the choice $s=1$. Furthermore, to then still obtain a multi-step-ahead predictor, $q=f$ is chosen.

Since the noise contribution in \eqref{eq:NoisyWFL3} is then asymptotically attenuated with increasing $N$ this motivates the use of
\begin{align}\label{eq:CL_DeePC_with_IV}
    \begin{bmatrix}
   \Phi_{i,1,N}\Phi_{i,1,N}^\top\\
   \hline
   Y_{i_p,1,N}\Phi_{i,1,N}^\top
    \end{bmatrix}
G =
\begin{bmatrix}
    \Phi_{\hat{i},1,f}\\
    \hline
    \widehat{Y}_{\hat{i}_p,1,f}
\end{bmatrix},
\end{align}
for sufficiently large $N$. Note that the structure of this equation is very similar to \eqref{eq:CL_DeePC_no_IVs} as shown by Fig.~\ref{fig:CL-DeePC}. The main difference is that the matrix with past data on the left hand side loses its indicated block-anti-diagonal structure and has $(p+1)r+pl$ instead of $N$ columns.

Solving \eqref{eq:CL_DeePC_with_IV} for the output predictor using the data equation examplified by \eqref{eq:DataEq1} yields
\begin{align}\label{eq:OutputPredictor}
    \widehat{Y}_{\hat{i}_p,1,f} = L_1 \Phi_{\hat{i},1,f} + \mathcal{H}_1 E_{i_p,1,N}\Phi_{i,1,N}^\dagger\Phi_{\hat{i},1,f},
\end{align}
in which the dagger $\dagger$ denotes the right inverse: ${\Phi_{i,1,N}^\dagger=\Phi_{i,1,N}^\top\left(\Phi_{i,1,N}\Phi_{i,1,N}^\top\right)\inv}$. Similar scrutiny of \eqref{eq:NoisyWFL3} demonstrates that according to \eqref{eq:uncorrelated} the ideal output predictor is recovered from \eqref{eq:OutputPredictor} in the limit $N\rightarrow\infty$.
% In obtaining an output predictor, no systematic noise mitigation strategy is yet applied by \eqref{eq:CL_DeePC_no_IVs} as the columns of $G$ simply take linear combinations of the noise in the output demonstrated by \eqref{eq:DataEq1}. An altered formulation of \eqref{eq:CL_DeePC_no_IVs} is therefore considered that allows the use of \ac{IVs} ($\mathcal{Z}_\mathrm{IV}$) as in~\cite{vanWingerden2022}
% \begin{align}\label{eq:CL_DeePC_with_IV}
%     \begin{bmatrix}
%    \Phi_{i,1,N}\\
%    \hline
%    Y_{i_p,1,N}
%     \end{bmatrix}
% {\mathcal{Z}_\mathrm{IV}}^\top G =
% \begin{bmatrix}
%     \Phi_{\hat{i},1,f}\\
%     \hline
%     \widehat{Y}_{\hat{i}_p,1,f}
% \end{bmatrix},
% \end{align}
% in which $G$ may be different from its previous definition depending on the definition of $\mathcal{Z}_\mathrm{IV}$, which follows shortly. Note that \eqref{eq:CL_DeePC_no_IVs} is recovered with $\mathcal{Z}_\mathrm{IV}=I_N$.

% From \eqref{eq:DataEq1} and \eqref{eq:CL_DeePC_with_IV} the output predictor becomes
% \begin{align}\label{eq:OutputPred}
%     \widehat{Y}_{\hat{i}_p,1,f} = L_1 \Phi_{\hat{i},1,f} + \mathcal{H}_1 E_{i_p,1,N}{\mathcal{Z}_\mathrm{IV}}^\top G.
% \end{align}
% To obtain an output estimate that best resembles a noiseless version of the actual future outputs given by \eqref{eq:DataEq1.2} it is desirable to reduce the noise contribution on the right hand side above. In addition, from an optimization point of view, it would be favorable to choose an \acs{IV} that uniquely determines $G$ from \eqref{eq:CL_DeePC_with_IV} given $\Phi_{\hat{i},1,f}$ and a sufficiently persistently exciting input that ensures that $\Phi_{i,1,N}$ is full row rank.

% This motivates choosing $\mathcal{Z}_\mathrm{IV}=\Phi_{i,1,N}$ as an \acs{IV} since\footnote{Since ${\mathcal{Z}_\mathrm{IV}}^\top G$ is fixed by \eqref{eq:CL_DeePC_IV} scalar multiples of the chosen \ac{IV} would be equally valid, simply resulting in a different $G$.}%
% \begin{align}
%     &\lim_{N\rightarrow\infty} \frac{1}{N}E_{i_p,1,N}{\Phi_{i,1,N}}^\top = \mathcal{O},\\%\label{eq:uncorrelated}\\
%     \text{rank}\biggl(&\lim_{N\rightarrow\infty} \frac{1}{N}\Phi_{i,1,N}{\Phi_{i,1,N}}^\top\biggl) =  \text{rank}(\Phi_{i,1,N}).%\label{eq:rankconservation}
% \end{align}
% As demonstrated by \eqref{eq:uncorrelated}, the instrumental variable and the noise are uncorrelated\footnote{Note that the choice $s=1$ is essential here to avoid correlation between inputs and noise during operation with data obtained in closed-loop.}. Hence, \eqref{eq:OutputPred} asymptotically converges to an ideal, noiseless output predictor with increasing $N$. In addition, provided that the input is sufficiently persistently exciting, $\Phi_{i,1,N}$ is full row rank such that \eqref{eq:rankconservation} permits only a single, unique solution for $G$ in \eqref{eq:CL_DeePC_IV}.
\section{Closed-loop \acs{DeePC} solution methods}\label{sec:SolutionMethods}
Having derived a new \ac{CL-DeePC} framework without and with \ac{IVs}, this section describes several methods to employ the stated equations in a receding horizon optimal control framework. Note that whether applying \eqref{eq:CL_DeePC_no_IVs} or \eqref{eq:CL_DeePC_IV} makes little difference to the solution method as long as the input is sufficiently persistently exciting such that $\mathcal{Z}_{i,1,N}$ is full row rank and the equation reduces to a form given by Fig.~\ref{fig:CL-DeePC} with ${m=(p+1)r+pl}$. Selecting a larger number of columns $N$ in \eqref{eq:CL_DeePC_no_IVs} unnecessarily adds more optimization variables and is therefore not considered here.
%
% =====================================================================================================================
% =====================================================================================================================
\subsection{Direct application in a solver}
One obvious way to apply \ac{CL-DeePC} is simply to apply \eqref{eq:CL_DeePC_no_IVs} or \eqref{eq:CL_DeePC_IV} in an optimizer together with possible constraints and a typically quadratic cost function and repeat the procedure for each new time step. This is possible because as Fig.~\ref{fig:CL-DeePC} demonstrates \ac{CL-DeePC} makes use of a system of $f(p+1)(r+l)$ equations that, due to the block-Hankel structure on the right hand side, contain $fm+f(r+l)$ unknowns. Since, as described above, ${m=(p+1)r+pl}$, this leaves $fr$ degrees of freedom, which lends itself well to optimization over future inputs $U_{\hat{i}_p,1,f}$.
%
% =====================================================================================================================
% =====================================================================================================================
\subsection{Constructing a predictor that is explicit in inputs}
Imagine explicitly solving for $G$ in terms of $\mathcal{Z}_{\hat{i},1,f}$ and using this as an output predictor. Then with $\mathcal{I}=I_N$ for \eqref{eq:CL_DeePC_no_IVs} and $\mathcal{I}=\mathcal{Z}_{i,1,N}$ for \eqref{eq:CL_DeePC_IV}:
\begin{align}\label{eq:OutputPred2}
     \widehat{Y}_{\hat{i}_p,1,f} = Y_{i_p,1,N}{\mathcal{I}}^\top \left(\mathcal{Z}_{i,1,N}{\mathcal{I}}^\top\right)\inv\mathcal{Z}_{\hat{i},1,f}.
\end{align}
The relationship between the output predictor and $\mathcal{Z}_{\hat{i},1,f}$ in \eqref{eq:OutputPred2} demonstrates that predicted future outputs are dependent on concurrent and preceding inputs as well as preceding outputs. The dependence of future output predictions on past inputs is contained in part implicitly by the dependence on preceding outputs, as demonstrated by the combination of \eqref{eq:OutputPred2} and Fig.~\ref{fig:CL-DeePC}. This motivates the construction of a multi-step output predictor that contains only an explicit dependence on future inputs over which one can then optimize a (potentially constrained) cost function in a receding horizon framework.
\subsubsection{One-shot}
One way to arrive at such an explicit description of the output predictor is to reformulate the \ac{CL-DeePC} formulations from \eqref{eq:CL_DeePC_no_IVs} or \eqref{eq:CL_DeePC_IV} to separate the known and unknown variables as vectors. To this end, consider the vectorization of \eqref{eq:CL_DeePC_IV}, which yields
\begin{align}\label{eq:Vectorize1}
    \underbrace{\left(\mkern-3mu I_f \otimes \begin{bmatrix}
        \mathcal{Z}_{i,1,N}\\
        Y_{i_p,1,N}
    \end{bmatrix}{\mathcal{I}_\mathrm{IV}}^\top\mkern-3mu\right)}_{=\mathcal{D}^\mathrm{p}}\text{vec}(G)=\text{vec}\mkern-3mu\left(\mkern-3mu\begin{bmatrix}
        \mathcal{Z}_{\hat{i},1,f}\\
        \widehat{Y}_{\hat{i}_p,1,f}
    \end{bmatrix}\mkern-3mu\right),
\end{align}%
in which $\mathcal{D}^\mathrm{p}$ is defined as shown for convenience. With reference to Fig.~\ref{fig:CL-DeePC}, the right hand side of \eqref{eq:Vectorize1} can clearly be decomposed into a vector of known past input-output data $b^\mathrm{p}$, and a vector of unknown future data from which it is straightforward to factorize the unknown future inputs $\datavec{u}{\hat{i}_p,f}$ and predicted outputs $\datavec{\hat{y}}{\hat{i}_p,f}$. The resulting reformulation of \eqref{eq:Vectorize1} is
\begin{align}
    \mathcal{D}^\mathrm{p}\text{vec}(G) = b^\mathrm{p}+\mathcal{A}^\mathrm{u}\datavec{u}{\hat{i}_p,f} + \mathcal{A}^\mathrm{y}\datavec{\hat{y}}{\hat{i}_p,f},
\end{align}
in which $\mathcal{A}^\mathrm{u}$ and $\mathcal{A}^\mathrm{y}$ are sparse matrices with full column rank that result from the aforementioned factorization of respectively unknown future inputs and outputs. The unknowns $G$ and $\datavec{\hat{y}}{\hat{i}_p,f}$ can then be obtained simultaneously from
\begin{align}
    \begin{bmatrix}
        \text{vec}(G)\\
        \datavec{\hat{y}}{\hat{i}_p,f}
    \end{bmatrix} =
    \begin{bmatrix}
        \mathcal{D}^\mathrm{p} & -\mathcal{A}^\mathrm{y}
    \end{bmatrix}\inv \left(b^\mathrm{p} + \mathcal{A}^\mathrm{u}\datavec{u}{\hat{i}_p,f}\right),
\end{align}
in which the inverse is assumed to exist.\todo{Existence conditions?}
%
% =====================================================================================================================
% =====================================================================================================================
\subsubsection{Sequential}
Alternatively, vectorization of \eqref{eq:OutputPred2} and factorization into contributions by inputs and outputs leads to
\begin{align}\label{eq:Sequential1}
    \datavec{\hat{y}}{\hat{i}_p,f} &=
    \begin{bmatrix}
        \widetilde{\mathcal{L}}^\mathrm{u}_f & \widetilde{\mathcal{G}}^\mathrm{u}_f 
    \end{bmatrix}    
    \begin{bmatrix}
        \datavec{u}{\hat{i},p}\\
        \datavec{u}{\hat{i}_p,f}
    \end{bmatrix}+
    \begin{bmatrix}
        \widetilde{\mathcal{L}}^\mathrm{y}_f & \widetilde{\mathcal{G}}^\mathrm{y}_f 
    \end{bmatrix}    
    \begin{bmatrix}
        \datavec{y}{\hat{i},p}\\
        \datavec{\hat{y}}{\hat{i}_p,f}
    \end{bmatrix},
\end{align}
in which
\begin{align*}
    % -----------------------------------------------------------------------------------------------------------------
    \begin{bmatrix}
        \widetilde{\mathcal{L}}^\mathrm{u}_f & \widetilde{\mathcal{G}}^\mathrm{u}_f 
    \end{bmatrix}&= {\scriptsize
    \begin{bmatrix}
        \tilde{\beta}_1     & \cdots      & \tilde{\beta}_{p}   & \tilde{\beta}_{p+1} & \mathcal{O} & \mathcal{O} & \cdots      & \mathcal{O}\\
        \mathcal{O} & \tilde{\beta}_1     & \cdots      & \tilde{\beta}_{p}   & \tilde{\beta}_{p+1} & \mathcal{O} & \cdots      & \mathcal{O}\\
        \vdots      & \ddots      & \ddots      &             & \ddots      & \ddots      & \ddots      & \vdots     \\
        \mathcal{O} & \cdots      & \mathcal{O} & \tilde{\beta}_1     & \cdots      & \tilde{\beta}_{p}   & \tilde{\beta}_{p+1} & \mathcal{O}\\
        \mathcal{O} & \cdots      & \mathcal{O} & \mathcal{O} & \tilde{\beta}_1     & \cdots      & \tilde{\beta}_{p}   & \tilde{\beta}_{p+1}\\
    \end{bmatrix}},\\
    % -----------------------------------------------------------------------------------------------------------------
    \begin{bmatrix}
        \widetilde{\mathcal{L}}^\mathrm{y}_f & \widetilde{\mathcal{G}}^\mathrm{y}_f 
    \end{bmatrix}&= {\scriptsize
    \begin{bmatrix}
        \tilde{\theta}_1    & \cdots      & \tilde{\theta}_{p}  & \mathcal{O}  & \mathcal{O}  & \mathcal{O} & \cdots       & \mathcal{O}\\
        \mathcal{O} & \tilde{\theta}_1    & \cdots      & \tilde{\theta}_{p}   & \mathcal{O}  & \mathcal{O} & \cdots       & \mathcal{O}\\
        \vdots      & \ddots      & \ddots      &              & \ddots       & \ddots      & \ddots       & \vdots     \\
        \mathcal{O} & \cdots      & \mathcal{O} & \tilde{\theta}_1     & \cdots       & \tilde{\theta}_{p}  & \mathcal{O}  & \mathcal{O}\\
        \mathcal{O} & \cdots      & \mathcal{O} & \mathcal{O}  & \tilde{\theta}_1     & \cdots      & \tilde{\theta}_{p}   & \mathcal{O}\\
    \end{bmatrix}},
\end{align*}
with the matrix blocks $\tilde{\beta}_k\in\mathbb{R}^{l\times r}$, $\tilde{\theta}_k\in\mathbb{R}^{l\times l}$, and large matrices defined such that $\widetilde{\mathcal{L}}^\mathrm{u}_f\in\mathbb{R}^{fl\times pr}$, $\widetilde{\mathcal{G}}^\mathrm{u}_f\in\mathbb{R}^{fl\times fr}$, $\widetilde{\mathcal{L}}^\mathrm{y}_f\in\mathbb{R}^{fl\times pl}$, and $\widetilde{\mathcal{G}}^\mathrm{y}_f\in\mathbb{R}^{fl\times fl}$. The subscript of the large matrices indicates the number of block rows.

Notice that the predicted future outputs feature on both the left and right hand side of \eqref{eq:Sequential1}. Solving for these predicted outputs yields a predictor that is completely explicit in terms of its input dependency for use in a receding horizon framework:
\begin{align}\label{eq:Sequential2}
    \datavec{\hat{y}}{\hat{i}_p,f} &=
    \begin{bmatrix}
        \mathcal{L}^\mathrm{u}_f & \mathcal{L}^\mathrm{y}_f 
    \end{bmatrix}    
    \begin{bmatrix}
        \datavec{u}{\hat{i},p}\\
        \datavec{y}{\hat{i},p}
    \end{bmatrix}+
    \mathcal{G}^\mathrm{u}_f
    \datavec{u}{\hat{i}_p,f},
\end{align}
in which $\mathcal{L}^\mathrm{u}_f$, $\mathcal{G}^\mathrm{u}_f$, and $\mathcal{L}^\mathrm{y}_f$ are uniquely defined by
\begin{align}\label{eq:Sequential3}
    \left(I_{fl}-\widetilde{\mathcal{G}}^\mathrm{y}_f\right)
    \begin{bmatrix}
        \mathcal{L}^\mathrm{u}_f & \mathcal{G}^\mathrm{u}_f & \mathcal{L}^\mathrm{y}_f
    \end{bmatrix}=
    % \left(I_{fl}-\widetilde{\mathcal{G}}^\mathrm{y}_f\right)\inv
    \begin{bmatrix}
        \widetilde{\mathcal{L}}^\mathrm{u}_f & \widetilde{\mathcal{G}}^\mathrm{u}_f & \widetilde{\mathcal{L}}^\mathrm{y}_f
    \end{bmatrix}.
\end{align}
Since $I_{fl}-\widetilde{\mathcal{G}}^\mathrm{y}_f$ is invertible, \eqref{eq:Sequential3} can be solved directly for $\big[\mathcal{L}^\mathrm{u}_f \; \mathcal{G}^\mathrm{u}_f \; \mathcal{L}^\mathrm{y}_f\big]$ to construct the input-explicit predictor given by \eqref{eq:Sequential2}. However, an efficient sequential procedure is also possible that exploits the structure of $I_{fl}-\widetilde{\mathcal{G}}^\mathrm{y}_f$.

For this sequential procedure, define the $f$ block-rows of $\big[\widetilde{\mathcal{L}}^\mathrm{u}_f \; \widetilde{\mathcal{G}}^\mathrm{u}_f \; \widetilde{\mathcal{L}}^\mathrm{y}_f\big]$ and $\big[\mathcal{L}^\mathrm{u}_f \; \mathcal{G}^\mathrm{u}_f \; \mathcal{L}^\mathrm{y}_f\big]$ by respectively $\tilde{\alpha}_j$, ${\alpha_j\in\mathbb{R}^{l\times p(r+l)+fr}}$, with $j$ here representing the index of the block row: $j=0,1,\dots,f-1$. It is then straightforward to show from \eqref{eq:Sequential3} that the formulation
\begin{align}\label{eq:Sequential4}
    \alpha_j=
    \left\{\begin{array}{ll}
    \mathcal{O},     & \text{if } j<0\\
    \tilde{\alpha}_j,& \text{if } j=0\\
    \tilde{\alpha}_j + \sum\limits_{r=1}^{p}\tilde{\theta}_r\alpha_{r-p+j-1}, & \text{if } j \geq 1
    \end{array}\right.
\end{align}
 allows efficient sequential construction of $\big[\mathcal{L}^\mathrm{u}_f \; \mathcal{G}^\mathrm{u}_f \; \mathcal{L}^\mathrm{y}_f\big]$ starting from $j=0$. From the subsequent section, it will become clear that $\mathcal{G}^\mathrm{u}_f$ is a causal block-Toeplitz matrix, meaning that the matrix is fully parameterized by its leftmost block column. As such one may choose to only solve this portion of $\mathcal{G}^\mathrm{u}_f$ in \eqref{eq:Sequential3} using \eqref{eq:Sequential4} and to complete the matrix $\mathcal{G}^\mathrm{u}_f$ thereafter.
\section{Closed-loop issue with regular \ac{DeePC}}
\section{Equivalence to Closed-loop \acs{SPC}}
This section demonstrates an equivalence between the developed \ac{CL-DeePC} framework and \ac{CL-SPC} as developed in~\cite{Dong2008}. The \ac{CL-SPC} methodology is briefly explained first, based upon which the equivalence is demonstrated thereafter.

\subsection{Closed-loop \ac{SPC}}
To understand this equivalence, consider the data equations \eqref{eq:DataEq1} and \eqref{eq:DataEq2}. Just like \ac{CL-DeePC}, \ac{CL-SPC} uses $s=1$ to avoid closed-loop correlation between inputs and noise\footnote{Treatment of the detrimental effects thereof is reserved for the subsequent section.}, estimating the dynamic matrix $\widetilde{L}_1$ by least squares regression on past data:
\begin{align}\label{eq:CL-SPC-PredMarkov}
\hat{\widetilde{L}}_1 = \left[ \widehat{C\tKp{u}} \; \widehat{D} \; \widehat{C\tKp{y}} \right]=Y_{i_p,1,N}\mathcal{Z}^\top (\Phi_{i,1,N}\mathcal{Z}^\top)\inv,
\end{align}
in which $\mathcal{Z}$ is either $I_N$ or $\Phi_{i,1,N}$ depending on whether an \ac{IV} approach is used or not. Assuming  $\widetilde{A}^p=\mathcal{O}$ as before, estimates of the predictor Markov parameters contained in $\hat{\widetilde{L}}_1$ allow the construction of estimates of $\widetilde{\Gamma}_f\widetilde{K}_p^\mathrm{u}$, $\widetilde{\Gamma}_f\widetilde{K}_p^\mathrm{y}$ and $\widetilde{\mathcal{T}}_f^\mathrm{u}$ (which make up $L_f$), as well as $\tHf$. In line with \eqref{eq:DataEq2} this allows the construction of a predictor as
\begin{align}\label{eq:CL-SPC-pred1}
	\begin{split}
	\datavec{\hat{y}}{\hat{i}_p,f}&= \begin{bmatrix}\widehat{\widetilde{\Gamma}_f\tKp{u}} & \widehat{\widetilde{\mathcal{T}}_f^\mathrm{u}} \end{bmatrix} 
	\begin{bmatrix}
		\datavec{u}{\hat{i},p}\\
		\datavec{u}{\hat{i}_p,f}
	\end{bmatrix}+\\
	&\phantom{=}\mkern8mu\begin{bmatrix}
		\widehat{\widetilde{\Gamma}_f\tKp{y}} & (I_{sl}-\widehat{\widetilde{\mathcal{H}}}_f) \end{bmatrix} 
	\begin{bmatrix}
		\datavec{y}{\hat{i},p}\\
		\datavec{\hat{y}}{\hat{i}_p,f}
	\end{bmatrix}.
	\end{split}
\end{align}
However, this predictor contains the predicted output on both sides of the equation. To solve \eqref{eq:CL-SPC-pred1} for the predicted outputs first make note of the fact that~\cite{Houtzager2012}
\begin{align}\label{eq:MatrixRelations}
	\tHf\inv\begin{bmatrix}
		\widetilde{\Gamma}_f\tKp{u} & \widetilde{\mathcal{T}}_f^\mathrm{u} & \widetilde{\Gamma}_f\tKp{y}
	\end{bmatrix} =
	\begin{bmatrix}
		\Gamma_f\tKp{u} & \mathcal{T}_f^\mathrm{u} & \Gamma_f\tKp{y}
	\end{bmatrix},
\end{align}
as is also visible from the combination of \eqref{eq:DataEq1} and \eqref{eq:DataEq2} for $s=f$. Using \eqref{eq:MatrixRelations} to solve \eqref{eq:CL-SPC-pred1} (which can be done efficiently in a sequential manner as with \ac{CL-DeePC}) yields
\begin{align}\label{eq:CL-SPC-pred2}
		\datavec{\hat{y}}{\hat{i}_p,f}= \begin{bmatrix}\widehat{\Gamma_f\tKp{u}} & \widehat{\Gamma_f\tKp{y}} \end{bmatrix} 
		\begin{bmatrix}
			\datavec{u}{\hat{i},p}\\
			\datavec{y}{\hat{i},p}
		\end{bmatrix}+
		\widehat{\mathcal{T}_f^\mathrm{u}} 
		\datavec{u}{\hat{i}_p,f},
\end{align}
which is in line with \eqref{eq:DataEq1} and is used in a receding horizon optimization-based control framework.

\subsection{Equivalence between \ac{CL-DeePC} and \ac{CL-SPC}}
By comparing \eqref{eq:Sequential1} to \eqref{eq:CL-SPC-pred1} for $f=1$ it can be seen that the building blocks $\tilde{\beta}_k$ and $\tilde{\theta}_k$ in \ac{CL-DeePC} are equal to the estimated predictor Markov parameters from \eqref{eq:CL-SPC-PredMarkov} that are obtained using \ac{CL-SPC}:
\begin{subequations}\label{eq:EquivMarkov}
	\begin{align}
		\tilde{\beta}_k &= \left\{\begin{array}{ll}
			\widehat{C\tilde{A}^{p-k}\tilde{B}}, &\forall k\in\{\mathbb{Z}_{>0}|\;p \geq k \geq 1\}\\
			\widehat{D}, &\text{if } k=p+1
		\end{array}\right.,\label{eq:EquivMarkovInputs}\\
	\tilde{\theta}_k &= \widehat{C\tilde{A}^{p-k}\tilde{B}},\mkern24mu \forall k\in\{\mathbb{Z}_{>0}|\;p \geq k \geq 1\}\label{eq:EquivMarkovOutputs}.
	\end{align}
\end{subequations}
As a result it is clear that there is an equivalence between the constructed matrices in \ac{CL-SPC} and \ac{CL-DeePC} that is succinctly described by
\begin{align}\label{eq:EquivPredictorMatrices}
	\mkern-3mu\begin{bmatrix}
		\widetilde{\mathcal{L}}^\mathrm{u}_f & \widetilde{\mathcal{G}}^\mathrm{u}_f & \widetilde{\mathcal{L}}^\mathrm{y}_f & \left(I_{fl}-\widetilde{\mathcal{G}}^\mathrm{y}_f\right)
	\end{bmatrix}\mkern-3mu=\mkern-3mu\begin{bmatrix}
		\widehat{\widetilde{\Gamma}_f\tKp{u}} & \widehat{\widetilde{\mathcal{T}}_f^\mathrm{u}} & \widehat{\widetilde{\Gamma}_f\tKp{y}} & \widehat{\widetilde{\mathcal{H}}}_f
	\end{bmatrix}\mkern-3mu.
\end{align}
It follows from \eqref{eq:Sequential3}, \eqref{eq:MatrixRelations}, and \eqref{eq:EquivPredictorMatrices} that, likewise,
\begin{align}\label{eq:EquivInnovationMatrices}
	\begin{bmatrix}
		\mathcal{L}^\mathrm{u}_f & \mathcal{G}^\mathrm{u}_f & \mathcal{L}^\mathrm{y}_f
	\end{bmatrix}=\begin{bmatrix}
		\widehat{\Gamma_f\tKp{u}} & \widehat{\mathcal{T}_f^\mathrm{u}} & \widehat{\Gamma_f\tKp{y}}
	\end{bmatrix}.
\end{align}
Moreover,~\eqref{eq:EquivInnovationMatrices} demonstrates that the output predictors~\eqref{eq:Sequential2} and~\eqref{eq:CL-SPC-pred2} of respectively \ac{CL-DeePC} and \ac{CL-SPC} are equivalent. By comparison of the sequential technique from Section~\ref{sec:SequentialSolMethod} to the sequential algorithm of~\cite{Dong2008} it is furthermore possible to see that the two sequential algorithms are indeed equivalent\footnote{Note that direct-feedthrough is not considered in~\cite{Dong2008}, but that an expansion of $B=\tilde{B}+KD$ would yield an algorithm consistent with \ac{CL-DeePC} and~\eqref{eq:MatrixRelations}.}.



\bibliographystyle{elsarticle-harv}%plain       % Include this if you use bibtex 
\bibliography{autosam}           % and a bib file to produce the 
                                 % bibliography (preferred). The
                                 % correct style is generated by
                                 % Elsevier at the time of printing.

\appendix
% \section{A summary of Latin grammar}    % Each appendix must have a short title.
% \section{Some Latin vocabulary}         % Sections and subsections are supported  
                                        % in the appendices.
\end{document}