\begin{align*}
    \hline
\end{align*}
%
\begin{thm}\label{theorem:main_result_old}
    Consider the minimal discrete non-deterministic \ac{LTI} system given by~\eqref{eqn:SS_innovation} to generate input-output data in closed-loop with a strictly causal controller. Define data matrices $\Psi_{i,1,N}$ and $\overline{\Psi}_{\hat{i},1,f}$ as in \eqref{eq:Phi_def}. 
    If the input sequence $\{u_k\}_{k=i}^{i+\bar{N}-1}$ of length $\bar{N}=p+s+N-1$ %, with $N\geq(p+s+n)(r+l)+n$ and $p\geq\ell$\todo{don't forget},
    is persistently exciting of order $p+s+n$, and has sample correlations such that%
    \begin{alignat}{2}%see also https://www.cis.upenn.edu/~jean/schur-comp.pdf
    % \widehat{\Sigma}_{u,u} &> 0,\label{eq:PE_corU}\\
    &\widehat{\Sigma}_{\mathrm{ee}} - \widehat{\Sigma}_{\mathrm{ue}^\top} \widehat{\Sigma}_{\mathrm{uu}}\inv \widehat{\Sigma}_{\mathrm{ue}}\succ0,\span\span\label{eq:PE_corUE2}\\
    &&\text{with}\quad\widehat{\Sigma}_{\mathrm{ee}}&=E_{i,p+s+n,N-n}E_{i,p+s+n,N-n}^\top,\notag\\
    &&\widehat{\Sigma}_{\mathrm{ue}}&=U_{i,p+s+n,N-n}E_{i,p+s+n,N-n}^\top,\notag\\
    &&\widehat{\Sigma}_{\mathrm{uu}}&=U_{i,p+s+n,N-n}U_{i,p+s+n,N-n}^\top,\notag
    \end{alignat}
    % ------------------------------------------------------
    then \\
    $\mathrm{(i)}$ $\exists G\in\mathbb{R}^{N\times f}$ such that
    \begin{align}\tag{\ref{eq:CL_DeePC_no_IVs}}%\label{eq:Theorem1}
        \begin{bmatrix}
            \Psi_{i,1,N}\\
            Y_{i_p,1,N}
        \end{bmatrix}G =
        \begin{bmatrix}
            \overline{\Psi}_{\hat{i},1,f}\\
            \widehat{Y}_{\hat{i}_p,1,f}
        \end{bmatrix},
    \end{align}
    $\mathrm{(ii)}$ and with $\widehat{Y}_{\hat{i}_p,1,f}$ as an asymptotically unbiased predictor %with respect to both past and future noise 
    as $p\rightarrow\infty$.
\end{thm}

The proof of Theorem~\ref{theorem:main_result_old} is deferred till after the treatment of several auxiliary results.
\subsection{Auxiliary results}\label{sec:aux_results}
For the development of sufficient conditions for persistency of excitation, first consider the following result for deterministic systems.
\begin{lem}\citep[Cor.~2(iii)]{Willems2005}\label{lem:D_det_full_row_rank}
    If the input sequence $\{u_k\}_{k=i}^{i+\epsilon+q-2}$ of a controllable discrete \ac{LTI} system without noise is persistently exciting of order $\epsilon+n$ then the matrix $\left[X_{i,1,q}^\top\;U_{i,\epsilon,q}^\top\right]^\top$ is full row rank.
\end{lem}
Lemma~\ref{lem:D_det_full_row_rank} can be extended to non-deterministic systems as shown by the following lemma.
% \setcounter{thm}{0}
\begin{lem}\label{lem:D_full_row_rank}
    If for a controllable non-deterministic \ac{LTI} system of the form given by \eqref{eqn:SS_innovation} the sequence of inputs and noise $\{[u_k^\top\;e_k^\top]^\top\}_{k=i}^{i+\epsilon+q-2}$ is persistently exciting of order $\epsilon+n$ then the matrix $\left[X_{i,1,q}^\top\;U_{i,\epsilon,q}^\top\;E_{i,\epsilon,q}^\top\right]^\top$ is full row rank.
\end{lem}
\textbf{Proof:} Lemma~\ref{lem:D_full_row_rank} follows from Lemma~\ref{lem:D_det_full_row_rank} by extending the exogenous inputs to include the innovation noise, thus also requiring controllable $(A,[B\,K])$. This latter condition is satisfied if $(A,B)$ is controllable.$\hfill\qed$

Since the closed-loop identification problem arises due to correlation between inputs and noise the persistency of excitation condition in Lemma~\ref{lem:D_full_row_rank} is rewritten in terms of such correlations. For this the following lemma concerning Schur complements is used.
\begin{lem}\citep[Lem.~2.7(i)]{Verhaegen2007a}\label{lem:Schur_comp}
    Let $S\in\mathbb{R}^{(\delta+\kappa)\times(\delta+\kappa)}$ be the symmetric matrix
    \begin{align*}
        S=\begin{bmatrix}
            \mathcal{A} & \mathcal{B}\\
            \mathcal{B}^\top & \mathcal{C}
        \end{bmatrix},
    \end{align*}
    with $\mathcal{A}\in\mathbb{R}^{\delta \times \delta}$, $\mathcal{B}\in\mathbb{R}^{\delta \times \kappa}$, $\mathcal{C}\in\mathbb{R}^{\kappa \times \kappa}$. If $\mathcal{A}\succ0$ then $S\succ0$ if and only if $\mathcal{C}-\mathcal{B}^\top\mathcal{A}\inv\mathcal{B}\succ0$.
\end{lem}

This allows the following lemma to express persistency of excitation conditions using correlation matrices.
\begin{lem}\label{lem:D_full_row_rank2}
    Consider a controllable non-deterministic \ac{LTI} system of the form given by \eqref{eqn:SS_innovation} with an input sequence $\{u_k\}_{k=i}^{i+\epsilon+q-2}$ that is persistently exciting of order $\epsilon+n$ and innovation sequence $\{e_k\}_{k=i}^{i+\epsilon+q-2}$. If the sample correlation matrices between inputs and noise are such that
    \begin{align}
        & \span\span \hat{\Sigma}_{\mathrm{ee},2} - \hat{\Sigma}_{\mathrm{ue},2}^\top \hat{\Sigma}_{\mathrm{uu},2}\inv \hat{\Sigma}_{\mathrm{ue},2} \succ 0,\label{eq:PE_corUE3}\\
        & \text{with}\;\;\;&&\hat{\Sigma}_{\mathrm{ee},2}=E_{i,\epsilon+n,q-n}E_{i,\epsilon+n,q-n}^\top\notag\\
        & &&\hat{\Sigma}_{\mathrm{ue},2}=U_{i,\epsilon+n,q-n}E_{i,\epsilon+n,q-n}^\top\notag\\
        & &&\hat{\Sigma}_{\mathrm{uu},2}=U_{i,\epsilon+n,q-n}U_{i,\epsilon+n,q-n}^\top\notag
    \end{align}
    then the matrix $\left[X_{i,1,q}^\top\;U_{i,\epsilon,q}^\top\;E_{i,\epsilon,q}^\top\right]^\top$ is full row rank.
\end{lem}
\textbf{Proof:} By Lemma~\ref{lem:D_full_row_rank} the matrix $\left[X_{i,1,q}^\top\;U_{i,\epsilon,q}^\top\;E_{i,\epsilon,q}^\top\right]^\top$ is full row rank if the combined input and noise sequence is such that, by Definition~\ref{def:PE}, $q\geq(\epsilon+n)(r+l)+n$, and
\begin{align}\label{eq:PE_corUE}
    \begin{bmatrix}
        U_{i,\epsilon+n,q-n}\\
        E_{i,\epsilon+n,q-n}
    \end{bmatrix}\!\!
    \begin{bmatrix}
        U_{i,\epsilon+n,q-n}\\
        E_{i,\epsilon+n,q-n}
    \end{bmatrix}^\top\!=
    \begin{bmatrix}
        \hat{\Sigma}_{\mathrm{uu},2} & \hat{\Sigma}_{\mathrm{ue},2}\\
        \hat{\Sigma}_{\mathrm{ue},2}^\top & \hat{\Sigma}_{\mathrm{ee},2}
    \end{bmatrix}\succ 0,
\end{align}
The persistency of excitation condition of the input sequence of order $\epsilon+n$ entails by Definition~\ref{def:PE} that ${\hat{\Sigma}_{\mathrm{uu},2}\succ0}$. Then by Lemma~\ref{lem:Schur_comp}, condition \eqref{eq:PE_corUE} is met such that $\left[X_{i,1,q}^\top\;U_{i,\epsilon,q}^\top\;E_{i,\epsilon,q}^\top\right]^\top$ is full row rank if and only if \eqref{eq:PE_corUE3} is satisfied.$\hfill\qed$
\subsection{Proof of Theorem~\ref{theorem:main_result}}%\textbf{Proof:}
This section builds on the auxiliary results presented in \secref{sec:aux_results} to provide a proof of Theorem~\ref{theorem:main_result}, which follows next.

%%%%%%%%%%%%%%%%%%%%%%%% Proof of (i) %%%%%%%%%%%%%%%%%%%%%%%%%%%%%%%%%%%%%%%%
\noindent\textbf{Proof of $\mathrm{\mathbf{(i)}}$:} 
Equation \eqref{eq:DataEq1} can be rewritten with $k=i,\;q=N$ in terms of actual states, inputs, outputs, and noise or with $k=\hat{i},\;q=f$ for predictions along the lines of \eqref{eq:CL_DeePC_no_IVs} as respectively
\begin{align}
    &\begin{bmatrix}-\Gamma_s \tilde{A}^p & -L_s & I_{sl}&-\mathcal{H}_s\end{bmatrix}
    \underbrace{\begin{bmatrix}
        X_{i,1,N}\\
        \Psi_{i,s,N}\\
        Y_{i_p,s,N}\\
        E_{i_p,s,N}
    \end{bmatrix}}_{=\mathfrak{BD}_N}=\mathcal{O},\label{eq:RBD1}\\
    &\underbrace{\begin{bmatrix}-\Gamma_s \tilde{A}^p & -L_s & I_{sl}&-\mathcal{H}_s\end{bmatrix}}_{= \mathfrak{R}}
    \underbrace{\begin{bmatrix}
        \widehat{X}_{\hat{i},1,f}\\
        \overline{\Psi}_{\hat{i},s,f}\\
        \widehat{Y}_{\hat{i}_p,s,f}\\
        \widehat{E}_{\hat{i}_p,s,f}
    \end{bmatrix}}_{=\mathfrak{BD}_f}=\mathcal{O},\label{eq:RBD2}\\
    &\mathfrak{R}\in\mathbb{R}^{sl\times (n+(p+s)(r+l)+sl)},\;\mathfrak{BD}_q\in\mathbb{R}^{(n+(p+s)(r+l)+sl) \times q},\notag
\end{align}
with $\mathfrak{R}$, and $\mathfrak{BD}_q$ for $q=N$ and $q=f$ defined as indicated for brevity. The estimated states $\widehat{X}_{\hat{i},1,f}$ and future innovation noise $\widehat{E}_{\hat{i}_p,s,f}$ are needed to explain the estimates $\overline{\Psi}_{\hat{i},s,f}$ and $\widehat{Y}_{\hat{i}_p,s,f}$ in \eqref{eq:RBD2} along the lines of \eqref{eq:RBD1}. Proof that the columns of $\mathfrak{BD}_f$ indeed all lie in the nullspace of $\mathfrak{R}$ follows soon. To this end, let $\mathcal{R}(\cdot)$ and $\mathcal{N}(\cdot)$ respectively denote the range and nullspace of a matrix.

The central idea of this proof is that \eqref{eq:CL_DeePC_no_IVs} holds true if there exists a $G$ such that $\mathfrak{BD}_N G=\mathfrak{BD}_f$, for which sufficient conditions are that
\begin{enumerate}
    \item[C1.] $\mathcal{R}(\mathfrak{BD}_N)=\mathcal{N}(\mathfrak{R})$, and
    \item[C2.] $\mathcal{R}(\mathfrak{BD}_f)\;\subseteq\mathcal{N}(\mathfrak{R})$.
\end{enumerate}
To prove that these conditions are satisfied use \eqref{eq:DataEq1} to rewrite $\mathfrak{BD}_N$ and $\mathfrak{BD}_f$ as contributions of initial states and exogenous inputs (respectively denoted by $\mathfrak{D}_N$ and $\mathfrak{D}_f$) as well as a matrix that describes their effects $\mathfrak{B}$. This factorization is given by
\begin{alignat}{2}
    \begin{bmatrix}
        X_{i,1,N}\\
        \hline
        \Psi_{i,s,N}\\
        \hline
        Y_{i_p,s,N}\\
        E_{i_p,s,N}
    \end{bmatrix}&=
    \underbrace{\begin{bmatrix}
        I_n      & 0      & 0       & 0 & 0\\
        \hline
        0        & I_{pr} & 0       & 0 & 0\\
        0        & 0      & I_{sr}  & 0 & 0\\
        \Gamma_p & \mathcal{T}_p^\mathrm{u} & 0 & \mathcal{H}_p & 0\\
        \hline
        \varepsilon_1 & \varepsilon_2 & \mathcal{T}_s^\mathrm{u} & \varepsilon_3 & \mathcal{H}_s\\
        0 & 0 & 0 & 0 & I_{sl}
    \end{bmatrix}}_{=\mathfrak{B}}
    \underbrace{\begin{bmatrix}
        X_{i,1,N}\\
        U_{i,p,N}\\
        U_{i_p,s,N}\\
        E_{i,p,N}\\
        E_{i_p,s,N}
    \end{bmatrix}}_{=\mathfrak{D}_N},\notag\\%\label{eq:BD_N}\\
    \begin{bmatrix}
        \widehat{X}_{\hat{i},1,f}\\
        \overline{\Psi}_{\hat{i},s,f}\\
        \widehat{Y}_{\hat{i}_p,s,f}\\
        \widehat{E}_{\hat{i}_p,s,f}
    \end{bmatrix}&=
    \mathfrak{B}
    \underbrace{\begin{bmatrix}
        \widehat{X}_{\hat{i},1,f}\\
        U_{\hat{i},p,f}\\
        U_{\hat{i}_p,s,f}\\
        \overline{E}_{\hat{i},p,f}\\
        \widehat{E}_{\hat{i}_p,s,f}
    \end{bmatrix}}_{=\mathfrak{D}_f},\label{eq:BD_f}\\%\label{eq:BD_f}\\
    \span\mathfrak{B}\in\mathbb{R}^{(n+(p+s)(r+l)+sl)\times (n+(p+s)(r+l))},\notag\\
    \span\mathfrak{D}_N\in\mathbb{R}^{(n+(p+s)(r+l))\times N},\quad \mathfrak{D}_f\in\mathbb{R}^{(n+(p+s)(r+l))\times f}\notag
\end{alignat}
with $\mathfrak{B}$, $\mathfrak{D}_N$, and $\mathfrak{D}_f$ defined as shown and $\varepsilon_1=\Gamma_s(\tilde{A}^p+\tKp{y}\Gamma_p)$, $\varepsilon_2=\Gamma_s(\tKp{u}+\tKp{y}\mathcal{T}_p^\mathrm{u})$, and $\varepsilon_3=\Gamma_s\tKp{y}\mathcal{H}_p$. Note that $\mathfrak{RB}=\mathcal{O}$, which means that $\mathcal{R}(\mathfrak{B})\subseteq\mathcal{N}(\mathfrak{R})$. In fact, by inspection one may verify that $\mathfrak{B}$ is full column rank and that its number of columns corresponds to the nullity of $\mathfrak{R}$. Hence, $\mathcal{R}(\mathfrak{B})=\mathcal{N}(\mathfrak{R})$. This has two important implications.

Firstly, since ${\mathcal{R}(\mathfrak{BD}_f)\subseteq\mathcal{R}(\mathfrak{B})=\mathcal{N}(\mathfrak{R})}$, condition C2 is satisfied.

Secondly, by similar reasoning, it must hold that ${\mathcal{R}(\mathfrak{BD}_N)\subseteq\mathcal{R}(\mathfrak{B})=\mathcal{N}(\mathfrak{R})}$. To prove condition C1 it must therefore be shown that $\mathcal{R}(\mathfrak{BD}_N)=\mathcal{R}(\mathfrak{B})$, for which $\mathfrak{D}_N$ must be full row rank. Under the stipulated persistency of excitation of the input sequence of order $p+s+n$, condition \eqref{eq:PE_corUE2}, and presumed system controllability, Theorem~\ref{theorem:main_result} satisfies the conditions of Lemma~\ref{lem:D_full_row_rank2} for $\epsilon=p+s$ and $q=N$ such that the matrix $\mathfrak{D}_N$ is full row rank. This proves condition C1. Having already proven condition C2 this concludes the proof of $\mathrm{(i)}$. $\hfill\qed$

\noindent\textbf{Remark 1:} A necessary condition for \eqref{eq:PE_corUE2} is that ${N\geq(p+s+n)(r+l)+n}$. This follows from the above proof, in which $\mathfrak{D}_N$ is required to be full row rank. The top matrix equation of \eqref{eq:CL_DeePC_no_IVs} defines $G$, for which there are multiple solutions given the aforementioned necessary lower bound on $N$ and the dimensions of $\Psi_{i,s,N}\in\mathbb{R}^{((p+s)r+pl)\times N}$. These different possible solutions of $G$ are given by
\begin{align}%\label{eq:G_sols}
    G = \Psi_{i,s,N}^\dagger\overline{\Psi}_{\hat{i},s,f} + \Pi_{\Psi_{i,s,N}}^\bot W,
\end{align}
in which the dagger $\dagger$ denotes the right inverse ($\mathcal{Q}^\dagger=\mathcal{Q}^\top(\mathcal{Q}\mathcal{Q}^\top)\inv$ with $\mathcal{Q}$ as a real, full row rank matrix), $\Pi_{\Psi_{i,s,N}}^\bot=I_N-\Psi_{i,s,N}^\dagger\Psi_{i,s,N}$ is a projection matrix onto the orthogonal complement of the row space of $\Psi_{i,s,N}$, %see Overschee1996, pg. 19
and $W\in\mathbb{R}^{N\times f}$ is a free matrix.

\noindent\textbf{Proof of $\mathrm{(ii)}:$} To prove that \eqref{eq:CL_DeePC_no_IVs} defines $\widehat{Y}_{\hat{i}_p,s,f}$ as an asymptotically unbiased predictor as $p\rightarrow\infty$ if $s=1$, first consider the error of this prediction. By the bottom matrix equation of~\eqref{eq:CL_DeePC_no_IVs} and subsequent application of~\eqref{eq:DataEq1} to rewrite $Y_{i_p,s,N}$ and $Y_{\hat{i}_p,s,f}$ we find
\begin{align}%\label{eq:Yf_error1}
    \begin{split}
        &\!\!\!\widehat{Y}_{\hat{i}_p,s,f}-Y_{\hat{i}_p,s,f} = \Gamma_s \tilde{A}^p (\underbrace{X_{i,1,N}G}_{=\widehat{X}_{\hat{i},1,f}}-X_{\hat{i},1,f}) \\
        &+L_s(\underbrace{\Psi_{i,s,N}G}_{=\overline{\Psi}_{\hat{i},s,f}}-\Psi_{\hat{i},s,f}) +\mathcal{H}_s (\underbrace{E_{i_p,s,N}G}_{=\widehat{E}_{\hat{i}_p,s,f}}-E_{\hat{i}_p,s,f})
    \end{split}
\end{align}
The interpretations of the underbraced terms are obtained from $\mathfrak{BD}_N G=\mathfrak{BD}_f$, which was central to the preceding proof of $\mathrm{(i)}$. Applying the limit $p\rightarrow\infty$ asymptotically attenuates the top contribution by the error of the state estimates since, by the definition of $K$ in \secref{sec:sys_model}, $\tilde{A}$ has all of its eigenvalues strictly inside the unit circle. In addition, substituting $G$ from \eqref{eq:G_sols} and cancelling equal terms in $\overline{\Psi}_{\hat{i},s,f}$ and $\Psi_{\hat{i},s,f}$ results in
\begin{alignat}{2}
        &\!\!\!\!\lim_{p\rightarrow\infty} \widehat{Y}_{\hat{i}_p,s,f}-Y_{\hat{i}_p,s,f} = \Gamma_s\tKp{y}\left(\overline{Y}_{\hat{i},p,f}-Y_{\hat{i},p,f}\right)\notag\\
        &+\mathcal{H}_s\Big(E_{i_p,s,N} W-E_{\hat{i}_p,s,f}\\%\label{eq:Yf_error2}\\
        &+\underbrace{E_{i_p,s,N}\Psi_{i,s,N}^\top}(\Psi_{i,s,N}\Psi_{i,s,N}^\top)\inv(\overline{\Psi}_{\hat{i},s,f}-\Psi_{i,s,N}W)\Big)\notag.
\end{alignat}%
In the above formulation, $W$ is a free matrix that parameterizes $G$. As such, for the predictor to be unbiased, the expectation ($\mathbb{E}[\cdot]$) of this error w.r.t. the noise and conditioned on $W$ must be zero. \todo{$\mathbb{E}[\cdot]$,\\$EW$,\\$\hat{\Sigma}\hat{\Sigma}\inv$?} This expectation renders the second row zero, but not the bottom row because of the underbraced term. Due to feedback, there is a correlation between past inputs and preceding noise such that the expectation of the sample correlation
\begin{align}
E_{i_p,s,N}U_{i_p,s,N}^\top =
    \frac{1}{N}\sum\limits_{k=i_p-1}^{i_p+N-2}
    \begin{bmatrix}
        e_{k+1} u_{k+1}^\top & \cdots & e_{k+1} u_{k+s}^\top\\
        e_{k+2} u_{k+1}^\top & \cdots & e_{k+2} u_{k+s}^\top\\
        \vdots &  & \vdots\\
        e_{k+s}u_{k+1}^\top & \cdots & e_{k+s}u_{k+s}^\top
    \end{bmatrix},\notag%
\end{align}
which is contained in the underbraced term in \eqref{eq:Yf_error2}, is nonzero. Due to feedback $\mathbb{E}[e_j u_k^\top]\neq 0,\;\forall k>j$. However, if, as assumed by Theorem~\ref{theorem:main_result}, the controller has no direct feedthrough, then $\mathbb{E}[e_j u_k^\top]= 0,\;\forall k=j$. Hence, if $s=1$, the expected value of the correlation $E_{i_p,s,N}U_{i_p,s,N}^\top$ reduces to zero, rendering the expectation of the bottom row of \eqref{eq:Yf_error2} zero.

Given the structure of $\overline{\Psi}_{\hat{i},1,f}$ and $\overline{Y}_{\hat{i},p,f}$ shown in Fig.~\ref{fig:CL-DeePC}, consider \eqref{eq:Yf_error2} column by column for $s=1$. Taking the expectation leaves
\begin{align}%\label{eq:Yf_error3}
\begin{split}
    \mkern-3mu\lim_{p\rightarrow\infty} &\mathbb{E}\left[\widehat{y}_{\hat{i}_p+k}-y_{\hat{i}_p+k}\right] = \\ &\Gamma_1\tKp{y}\left(\mathbb{E}\left[\datavec{\overline{y}}{\hat{i}+k,p}-\datavec{y}{\hat{i}+k,p}\right]\right),\;\forall k\in[0,f-1].
\end{split}
\end{align}
For $k=0$, \eqref{eq:Yf_error3} is zero because none of the relevant ouput data is estimated ($\datavec{\overline{y}}{\hat{i},p}=\datavec{y}{\hat{i},p}$). For $k=1$ no bias is thereby introduced on the right hand side, rendering the expectation zero again. Repetition of this process until $k=f-1$ demonstrates that $\widehat{Y}_{\hat{i}_p,1,f}$ is indeed an asymptotically unbiased predictor in the limit $p\rightarrow\infty$. This concludes the proof of $\mathrm{(ii)}$. $\hfill\qed$

%%%%%%%%%%%%%%%%%%%%%%%%%%%%%%%%%%%%%%%%%%%%%%%%%%%%%%%%%%%%%%%%%%%%%%%%%%%%%%%%%%%%%%%%%%%%%%%%%%%%%%%%%%%%%%%%%%%%%%%%%%%%%%%%%%%%%%%
\begin{align*}
    \hline
\end{align*}
% As a result, by the Rouch\'{e}-Capelli theorem there then exists a vector $g_{k-\hat{i}+1}$ such that
% \begin{align}\label{eq:Dg}
%     \mathfrak{D} g_{k-\hat{i}+1} =
%     \begin{bmatrix}
%         x_k^\top & \datavec{u}{k,p}^\top & \datavec{u}{k_p,s}^\top & \datavec{e}{k,p}^\top & \datavec{e}{k_p,s}^\top
%     \end{bmatrix}^\top\in\mathbb{R}^{n+(p+s)(r+l)}.
% \end{align}
% With reference to~\eqref{eq:Yf1} and~\eqref{eq:DataEq1}, pre-multiplying both sides of~\eqref{eq:Dg} by $\mathfrak{B}$ from~\eqref{eq:RBD2} yields
% \begin{align}\label{eq:BDg}
%     \begin{bmatrix}
%         X_{i,1,N}\\
%         \Psi_{i,s,N}\\
%         Y_{i_p,s,N}\\
%         E_{i_p,s,N}
%     \end{bmatrix}g_{k-\hat{i}+1}=
%     \begin{bmatrix}
%         x_k\\
%         \Psi_{k,s,1}\\
%         \datavec{y}{k_p,s}\\
%         \datavec{e}{k_p,s}
%     \end{bmatrix}\in\mathcal{N}(\mathfrak{R}).
% \end{align}
% Proof of~\eqref{eq:CL_DeePC_no_IVs} is obtained by sequential application of \eqref{eq:BDg} for $k={\hat{i},\dots,\hat{i}+f-1}$ such that $G=\left[g_1\;g_2\;\cdots\;g_f\right]$, and without the unknown top and bottom matrix equations. As a result the future outputs in $\bar{\Psi}_{\hat{i},s,f}$ and $\widehat{Y}_{\hat{i}_p,s,f}$ of \eqref{eq:CL_DeePC_no_IVs} are predictions.

% To prove statement 1) consider that $G$ is determined only by the top matrix equation in~\eqref{eq:CL_DeePC_no_IVs}, which contains known past input-output data as well as future inputs that can be chosen. Under the assumed persistency of excitation conditions $\mathfrak{D}$ is full row rank such that, by inspection of the product $\mathfrak{BD}$ in~\eqref{eq:RBD2}, $\Psi_{i,s,N}$ must also be full row rank. %This matrix is invertible if $N=p(r+l)+sl$ <- not possible
% Since the condition ${N\geq(p+s+n)(r+l)+n}$ ensures that $\Psi_{i,s,N}\in\mathbb{R}^{p(r+l)+sl\times N}$ has more columns then rows, there are multiple solutions to $G$ in \eqref{eq:CL_DeePC_no_IVs}. This proves statement~1). The different solutions of $G$ are given by
% \begin{align}%\label{eq:G_sols}
%     G = \Psi_{i,s,N}^\dagger\overline{\Psi}_{\hat{i},s,f} + (I_N-\Psi_{i,s,N}^\dagger\Psi_{i,s,N})W,
% \end{align}
% in which the dagger $\dagger$ denotes the right inverse ($\mathcal{Q}^\dagger=\mathcal{Q}^\top(\mathcal{Q}\mathcal{Q}^\top)\inv$ with $\mathcal{Q}$ as a real, full row rank matrix), and $W\in\mathbb{R}^{N\times f}$ is a free matrix.

% To prove statement~2) consider the error of the output prediction, which using \eqref{eq:DataEq1} to rewrite $Y_{i_p,s,N}$ and $Y_{\hat{i},s,f}$ and by subsequent application of~\eqref{eq:CL_DeePC_no_IVs} is found as
% \begin{align}%\label{eq:Yf_error1}
%     \begin{split}
%         \widehat{Y}_{\hat{i}_p,s,f}-Y_{\hat{i}_p,s,f} = &\; L_s(\overline{\Psi}_{\hat{i},s,f}-\Psi_{\hat{i},s,f})\\
%         &+\mathcal{H}_s (E_{i_p,s,N}G-E_{\hat{i}_p,s,f})\\
%         &+ \Gamma_s \tilde{A}^p (X_{i,1,N}G-X_{\hat{i},1,f}).
%     \end{split}
% \end{align}\todo{incl. $\overline{\Psi}$}
% Note that the terms in parentheses correspond to the errors that are induced by removal of the unknown top and bottom matrix equations in~\eqref{eq:BDg}. Moreover, the bottom line is asymptotically attenuated as $p\rightarrow\infty$ because $\tilde{A}$, by its definition in \secref{sec:sys_model}, has all of its eigenvalues strictly inside the unit circle. Taking the limit $p\rightarrow\infty$ and substituting $G$ from \eqref{eq:G_sols} in \eqref{eq:Yf_error1} yields

% \begin{align}%\label{eq:Yf_error2}
%     \begin{split}
%         &\lim_{p\rightarrow\infty} \Big( \widehat{Y}_{\hat{i}_p,s,f}-Y_{\hat{i}_p,s,f} \Big)=\mathcal{H}_s\Big(E_{i_p,s,N}W-E_{\hat{i}_p,s,f} \\
%         % &-E_{i_p,s,N}\Psi_{i,s,N}^\top(\Psi_{i,s,N}\Psi_{i,s,N}^\top)\inv\\
%         &+\underbrace{E_{i_p,s,N}\Psi_{i,s,N}^\top}(\Psi_{i,s,N}\Psi_{i,s,N}^\top)\inv(\Psi_{\hat{i},s,f}-\Psi_{i,s,N}W)\Big),
%     \end{split}
% \end{align}
% in which the underbraced product is the sample correlation $[E_{i_p,s,N}U_{i,p,N}^\top\;\;E_{i_p,s,N}U_{i_p,s,N}^\top\;\;E_{i_p,s,N}Y_{i,p,N}^\top]$. By causality the expectation of the leftmost and rightmost sample correlations herein are zero. The expectation of the middle correlation is however not zero in general due to feedback in closed-loop operation. The middle sample correlation $E_{i_p,s,N}U_{i_p,s,N}^\top$ is
% \begin{align}
%     \frac{1}{N}\sum\limits_{k=i_p}^{i_p+N-1}\begin{bmatrix}
%         e_k u_k^\top & \cdots & e_k u_{k+s-1}^\top\\
%         e_{k+1} u_k^\top & \cdots & e_{k+1} u_{k+s-1}^\top\\
%         \vdots &  & \vdots\\
%         e_{k+s-1}u_k^\top & \cdots & e_{k+s-1}u_{k+s-1}^\top
%     \end{bmatrix}.\notag%
% \end{align}
% Since the employed controller is assumed to have no direct feedthrough, the correlation between noise and inputs is strictly causal: $\mathbb{E}[e_ku_j^\top]=0,\forall{j\leq k}$. With $s=1$ only such correlations are employed such that the expectation of \eqref{eq:Yf_error2} is the null matrix. This proves statement~2).

\section{Willems' Fundamental Lemma \& Noise}
\todo{Oud: reuse?}Equation~\eqref{eq:DataEq1} can be reformulated with $k=i$, $q=N$ or for an ideal noiseless output prediction with $k=\hat{i}$ as respectively
\begin{alignat}{2}
    \begin{bmatrix}
        -L_s & I_{sl}
    \end{bmatrix}&
    \begin{bmatrix}
        \Psi_{i,s,N}\\
        Y_{i_p,s,N}-\mathcal{H}_s E_{i_p,s,N}
    \end{bmatrix} = \mathcal{O},\label{eq:NoisyWFL1}\\%\mathcal{H}_s E_{i_p,s,N}, 
    \begin{bmatrix}
        -L_s & I_{sl}
    \end{bmatrix}&
    \begin{bmatrix}
        \Psi_{\hat{i},s,q}\\
        \widehat{Y}^*_{\hat{i}_p,s,q}
    \end{bmatrix} = \mathcal{O}, \label{eq:NoisyWFL2}
\end{alignat}
in which the asterisk indicates that the output prediction is ideal in the sense of being asymptotically unbiased. Multiplying \eqref{eq:NoisyWFL1} by $\mathcal{Z}^\top G\in\mathbb{R}^{N\times q}$, and subtracting \eqref{eq:NoisyWFL2} obtains
\begin{align}\label{eq:NoisyWFL3}
    \mkern-14mu\begin{bmatrix}
        \shortminus L_s & I_{sl}
    \end{bmatrix}
    \mkern-9mu\left(\mkern-3mu%
    \begin{bmatrix}
        \Psi_{i,s,N}\\
        Y_{i_p,s,N}\shortminus\mathcal{H}_s E_{i_p,s,N}
    \end{bmatrix}%
    \mkern-4mu\mathcal{Z}^\top G%\mkern-2mu
    -%-%
    \mkern-5mu\begin{bmatrix}
        \Psi_{\hat{i},s,q}\\
        \widehat{Y}^*_{\hat{i}_p,s,q}
    \end{bmatrix}\mkern-3mu\right)\mkern-6mu=\mkern-3mu\mathcal{O}\mkern-5mu%\mathcal{H}_s E_{i_p,s,N}\mathcal{Z}G
\end{align}
in which $\mathcal{Z}$ represents a yet unspecified matrix and $G$ represents a matrix that is akin to the likewise defined matrix from \eqref{eq:CL_DeePC_no_IVs} that contains all of the vectors $g_k$.

If the columns of the matrix with data on the left hand side of \eqref{eq:NoisyWFL1} span the entire nullspace of $\left[\shortminus L_s\;I_{sl}\right]$ and $\mathcal{Z}$ is full rank then all solutions to \eqref{eq:NoisyWFL3} are described by equating the term inside the parenthesis to zero. For now, consider the case that $\mathcal{Z}=I_N$, $s=f$, and $q=1$ in the absence of noise to recover the regular deterministic \ac{DeePC} equation~\citep{Coulson2019}. %Then one possible solution (since the matrix $\left[\shortminus L_s\;I_{sl}\right]$ is not full column rank) to \eqref{eq:NoisyWFL3} with $s=f$ and $q=1$ is given by the regular deterministic \ac{DeePC} equation~\cite{Coulson2019}:
\begin{align}\label{eq:regular_DeePC}
    \begin{bmatrix}
        \Psi_{i,f,N}\\
        Y_{i_p,f,N}
    \end{bmatrix}g=%
    \begin{bmatrix}
        \Psi_{\hat{i},f,1}\\
        \widehat{Y}_{\hat{i}_p,f,1}
    \end{bmatrix}.
\end{align}
Willems' Fundamental Lemma makes use of Assumptions~\ref{assum:PE} and~\ref{assum:controllability} to ensure that the entire nullspace of $\left[\shortminus L_f\;I_{fl}\right]$ is spanned by the data matrix on the left hand side of \eqref{eq:regular_DeePC}~\citep{Willems2005}. Assumption~\ref{assum:unique_initial} is furthermore necessary to guarantee the existence of a unique initial state and therefore output predictor. %This clearly reflects Willems' Fundamental Lemma, which states that for a deterministic \ac{LTI} system, any sufficiently persistently exciting past input-output trajectory parameterizes all possible future input-output trajectories~\cite{Willems2005}.\todo{WFL: what about nullspace in (12)}

In the presence of (unknown) noise, the term $Y_{i_p,s,N}-\mathcal{H}_s E_{i_p,s,N}$ from \eqref{eq:NoisyWFL3} cannot be determined to obtain an ideal output predictor. Instead, linear combinations of a noise-corrupted output $Y_{i_p,s,N}$ as in \eqref{eq:regular_DeePC} are taken, resulting in an error of the obtained output predictor due to implicit sampling of $\mathcal{H}_s E_{i_p,s,N}$. Moreover, the regular \ac{DeePC} formulation provided by \eqref{eq:regular_DeePC} may become inconsistent in the presence of noise, prompting the use of, e.g., slack variables and regularization~\citep{Coulson2019}.
%
% ==============================================================================================================================================================
% ==============================================================================================================================================================
\subsection{Noise mitigation using \acl{IVs}}
Notwithstanding potential benefits of beforementioned mechanisms to cope with noise, such methods do not provide a systematic way to mitigate noise at the source. To that end this section introduces the use of an \ac{IV}: $\mathcal{Z}\neq I_N$. In this context, the \ac{IV} is defined such that it is uncorrelated with the noise $E_{i_p,s,N}$ and preserves the (full row) rank of the data matrix $\Psi_{i,s,N}$ obtained from a sufficiently persistently exciting input. These two conditions are respectively formulated as
%
\begin{align}
    &\lim_{N\rightarrow\infty} \frac{1}{N}E_{i_p,s,N}\mathcal{Z}^\top = \mathcal{O},\label{eq:uncorrelated}\\
    \text{rank}\biggl(&\lim_{N\rightarrow\infty} \frac{1}{N}\Psi_{i,s,N}\mathcal{Z}^\top\biggl) =  \text{rank}(\Psi_{i,s,N}),\label{eq:rankconservation}
\end{align}
%
which motivates choosing $\mathcal{Z}=\Psi_{i,s,N}$~\cite[Chapt. 9.6]{Verhaegen2007a}. An important assumption that is hereby introduced to satisfy \eqref{eq:uncorrelated} is that inputs are uncorrelated with noise. To fulfill this assumption Section~\ref{sec:CL_ID_issue} will motivate the choice $s=1$. Furthermore, to then still obtain a multi-step-ahead predictor, $q=f$ is chosen.

Since the noise contribution in \eqref{eq:NoisyWFL3} is then asymptotically attenuated with increasing $N$ this motivates the use of
\begin{align}\label{eq:CL_DeePC_with_IV}
    \begin{bmatrix}
   \Psi_{i,1,N}\Psi_{i,1,N}^\top\\
   \hline
   Y_{i_p,1,N}\Psi_{i,1,N}^\top
    \end{bmatrix}
G =
\begin{bmatrix}
    \Psi_{\hat{i},1,f}\\
    \hline
    \widehat{Y}_{\hat{i}_p,1,f}
\end{bmatrix},
\end{align}
for sufficiently large $N$. Note that the structure of this equation is very similar to \eqref{eq:CL_DeePC_no_IVs} as shown by Fig.~\ref{fig:CL-DeePC}. The main difference is that the matrix with past data on the left hand side loses its indicated block-anti-diagonal structure and has $(p+1)r+pl$ instead of $N$ columns.

Solving \eqref{eq:CL_DeePC_with_IV} for the output predictor using the data equation examplified by \eqref{eq:DataEq1} yields
\begin{align}\label{eq:OutputPredictor}
    \widehat{Y}_{\hat{i}_p,1,f} = L_1 \Psi_{\hat{i},1,f} + \mathcal{H}_1 E_{i_p,1,N}\Psi_{i,1,N}^\dagger\Psi_{\hat{i},1,f},
\end{align}
in which the dagger $\dagger$ denotes the right inverse: ${\Psi_{i,1,N}^\dagger=\Psi_{i,1,N}^\top\left(\Psi_{i,1,N}\Psi_{i,1,N}^\top\right)\inv}$. Similar scrutiny of \eqref{eq:NoisyWFL3} demonstrates that according to \eqref{eq:uncorrelated} the ideal output predictor is recovered from \eqref{eq:OutputPredictor} in the limit $N\rightarrow\infty$.