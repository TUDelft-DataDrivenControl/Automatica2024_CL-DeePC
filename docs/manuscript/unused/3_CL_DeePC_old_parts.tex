% -------------------------------------------------------------- Unused --------------------------------------------------------------
\begin{lem}\label{lem:oblique_projections}\citep[Lemma~1]{Katayama1999} %VanOverschee1994
Let $a$, $b$ and $c$ be random vectors with components in the Hilbert space $\mathscr{H}$, $\mathscr{A}=\text{span}\{a\}$, $\mathscr{B}=\text{span}\{b\}$ and $\mathscr{A}\cap\mathscr{B}=\{0\}$. Furthermore, let correlation matrices be exemplified by $\Sigma_{ca}=\mathbb{E}[ca^\top]$ and their conditional counterparts by $\Sigma_{ca|b}=\mathbb{E}[(c|b^\bot)(a|b^\bot)^\top]$, with $\bot$ denoting the orthogonal complement. Then the orthogonal projection of the row space of $c$ on the joint row space of $a$ and $b$ is the sum of the oblique projections $\Pi_{ca||b}a$ ($c$ onto $\mathscr{A}$ along $\mathscr{B}$) and $\Pi_{cb||a}b$ ($c$ onto $\mathscr{B}$ along $\mathscr{A}$) as
\begin{align}
    \begin{bmatrix}
        \Sigma_{ca} & \Sigma_{cb}
    \end{bmatrix}
    \begin{bmatrix}
        \Sigma_{aa} & \Sigma_{ab}\\ \Sigma_{ba} & \Sigma_{bb}
    \end{bmatrix}^\dagger
    \begin{bmatrix}
        a \\ b
    \end{bmatrix} = \Pi_{ca||b} a + \Pi_{cb||a} b,\label{eq:oblique_project}\\
    \text{s.t. }\quad\Pi_{ca||b}\Sigma_{aa|b} = \Sigma_{ca|b},\quad \Pi_{cb||a}\Sigma_{bb|a} = \Sigma_{cb|a}.\notag
\end{align}
The conditional correlation matrices may be expressed as demonstrated by $\Sigma_{ca|b}=\Sigma_{ca}-\Sigma_{cb}\Sigma_{bb}\inv\Sigma_{ba}$ if the involved inverse exists. In addition, $\Sigma_{aa|b}$ and $\Sigma_{bb|a}$ are invertible if $\Sigma_{aa}$ and $\Sigma_{bb}$ are invertible.
\end{lem}

\begin{align}
    \begin{split}
        &=\!\Big(\!\big(C \tilde{A}^p \underbrace{X_{i,1,N}\Psi_{i,1,N}^\top}_{=\hat{\Sigma}_{x\psi}} + \underbrace{E_{i_p,1,N}\Psi_{i,1,N}^\top}_{=\hat{\Sigma}_{e\psi}}\big)\times\\
        &\phantom{==}\big(\smash{\underbrace{\Psi_{i,1,N}\Psi_{i,1,N}^\top}_{=\hat{\Sigma}_{\psi\psi}}}\big)\inv+\begin{bmatrix}C\tKp{u} & D & C\tKp{y}\end{bmatrix}\Big)\overline{\Psi}_{\hat{i},1,f},
    \end{split}
\end{align}
in which the underbraced terms define the indicated sample correlation matrices. Partitioning the matrix $\Psi_{i,1,N}=[U_{i,p+1,N}^\top\;Y_{i,p,N}^\top]^\top$ such that
\begin{alignat*}{3}
    \hat{\Sigma}_{\psi\psi}&=\begin{bmatrix}
        \hat{\Sigma}_{uu} & \hat{\Sigma}_{uy}\\
        \hat{\Sigma}_{yu} & \hat{\Sigma}_{yy}
    \end{bmatrix} &&= \begin{bmatrix} U_{i,p+1,N} \\ Y_{i,p,N} \end{bmatrix}\begin{bmatrix} U_{i,p+1,N}^\top & Y_{i,p,N}^\top \end{bmatrix},\\
    \hat{\Sigma}_{x\psi}&=\begin{bmatrix}
        \hat{\Sigma}_{xu} & \hat{\Sigma}_{xy}
    \end{bmatrix} &&= X_{i,1,N}\begin{bmatrix} U_{i,p+1,N}^\top & Y_{i,p,N}^\top \end{bmatrix},\\
    \hat{\Sigma}_{e\psi}&=\begin{bmatrix}
        \hat{\Sigma}_{eu} & \hat{\Sigma}_{ey}
    \end{bmatrix} &&= E_{i_p,1,N}\begin{bmatrix} U_{i,p+1,N}^\top & Y_{i,p,N}^\top \end{bmatrix},
\end{alignat*}
and applying Lemma~\ref{lem:oblique_projections}, \eqref{eq:Yfhat_1} is rewritten in terms of oblique sample projections as
\begin{align}%\label{eq:Yfhat_2}
    \begin{split}
    \widehat{Y}_{\hat{i}_p,1,f} = \Big(C\tilde{A}^p\hat{\Pi}_{xy||u}+\hat{\Pi}_{ey||u}
    + C\tKp{y}\Big)&\overline{Y}_{\hat{i},p,f}\\
    +\Big(C\tilde{A}^p\hat{\Pi}_{xu||y}+\hat{\Pi}_{eu||y}+
    \begin{bmatrix}C\tKp{u} & D \end{bmatrix}\!\Big)&U_{\hat{i},p+1,f},
    \end{split}
\end{align}
with the sample-based oblique projection matrices
\begin{alignat*}{3}
    \hat{\Pi}_{xy||u}&=\hat{\Sigma}_{xy|u}\hat{\Sigma}_{yy|u}\inv, \qquad &\hat{\Pi}_{ey||u}&=\hat{\Sigma}_{ey|u}\hat{\Sigma}_{yy|u}\inv,\\
    \hat{\Pi}_{xu||y}&=\hat{\Sigma}_{xu|y}\hat{\Sigma}_{uu|y}\inv,        &\hat{\Pi}_{eu||y}&=\hat{\Sigma}_{eu|y}\hat{\Sigma}_{uu|y}\inv.
\end{alignat*}
Note that the invertibility of $\hat{\Sigma}_{yy|u}$ and $\hat{\Sigma}_{uu|y}$ is ensured according to Lemma~\ref{lem:oblique_projections} by the non-singularity of $\hat{\Sigma}_{yy}$ and $\hat{\Sigma}_{uu}$, which in turn derives from the full row rank of $\Psi_{i,1,N}$.

Having derived the output predictor obtained from \eqref{eq:CL_DeePC_no_IVs}, now consider its error. Equations \eqref{eq:DataEq1} and \eqref{eq:Yfhat_2} determine this error as
%
%Consider the error of this prediction, which using \eqref{eq:DataEq1}, and considering $\Gamma_1=C$, $\mathcal{H}_1=I_l$ is written as
\begin{align}
    \begin{split}
        &\widehat{Y}_{\hat{i}_p,1,f}-Y_{\hat{i}_p,1,f} = C\tKp{y}\left(\overline{Y}_{\hat{i},p,f}-Y_{\hat{i},p,f}\right)\\
        &\;\;\;+C\tilde{A}^p\!\left(\hat{\Pi}_{xy||u}\overline{Y}_{\hat{i},p,f}+\hat{\Pi}_{xu||y} U_{\hat{i},p+1,f}-X_{\hat{i},1,f}\!\right)\\
        &\;\;\;+\hat{\Pi}_{eu||y}U_{\hat{i},p+1,f}+\hat{\Pi}_{ey||u}\overline{Y}_{\hat{i},p,f}-E_{\hat{i}_p,1,f}
    \end{split}\label{eq:Yf_error1}%\\
    % \begin{split}
    %     &\!\!\!\widehat{Y}_{\hat{i}_p,1,f}-Y_{\hat{i}_p,1,f} = C \tilde{A}^p (%\underbrace{
    %     X_{i,1,N_\mathrm{s}}G%}_{=\widehat{X}_{\hat{i},1,f}}
    %     -X_{\hat{i},1,f})-E_{\hat{i}_p,1,f}\\
    %     &\phantom{=}+ L_1(\smash{\underbrace{\Psi_{i,1,N_\mathrm{s}}G}_{\mathrlap{=\overline{\Psi}_{\hat{i},1,f}\;\because \text{ \eqref{eq:CL_DeePC_no_IVs}}}}}
    %     -\Psi_{\hat{i},1,f}) +%\underbrace{
    %     E_{i_p,1,N_\mathrm{s}}\underbrace{G}_{\mathclap{=\Psi_{i,1,N_\mathrm{s}}^{\dagger,\mathrm{r}}\overline{\Psi}_{\hat{i},1,f}\;\because\text{ \eqref{eq:G_sols}}}}%}_{=\widehat{E}_{\hat{i}_p,1,f}}
    % \end{split}\notag\\
    % \begin{split}
    %     &=C \tilde{A}^p (X_{i,1,N_\mathrm{s}}G-X_{\hat{i},1,f}) - E_{\hat{i}_p,1,f}\\
    %     &\phantom{=}+C\tKp{y}\left(\overline{Y}_{\hat{i},p,f}-Y_{\hat{i},p,f}\right)\\
    %     &\phantom{=}+\underbrace{E_{i_p,1,N_\mathrm{s}}\Psi_{i,1,N_\mathrm{s}}^\top}_{=\hat{\Sigma}_{e\psi}}(\underbrace{\Psi_{i,1,N_\mathrm{s}}\Psi_{i,1,N_\mathrm{s}}^\top}_{=\hat{\Sigma}_{\psi}})\inv \overline{\Psi}_{\hat{i},1,f},
    % \end{split}\label{eq:Yf_error1}
\end{align}
Applying the limit $p\rightarrow\infty$ asymptotically attenuates the error of the implicit initial state estimates on the second row since, by the definition of $K$ in \secref{sec:sys_model}, $\tilde{A}$ has all of its eigenvalues strictly inside the unit circle. Moreover, since $N_\mathrm{s}=(p+1)r+pl$, the limit $p\rightarrow\infty$ also ensures $N_\mathrm{s}\rightarrow\infty$. In this latter limit the sample correlation matrices approach their true correlation matrix with probability one because inputs and outputs are assumed to be quasi-stationary second-order ergodic stochastic processes. To this end, consider the sample correlation matrix $\hat{\Sigma}_{e\psi}$ that governs the oblique projection matrices $\hat{\Pi}_{eu||y}$ and $\hat{\Pi}_{ey||u}$ as indicated by \eqref{eq:oblique_project}.
\begin{align}%\label{eq:E_Phi_correlation}
    \begin{split}
        &\hat{\Sigma}_{e\psi} = \\
        &\;\frac{1}{N_\mathrm{s}}\;\sum\limits_{k=i_p}^{\mathclap{i_p+N_\mathrm{s}-1}} e_k \begin{bmatrix}u_{k-p}^\top & \cdots & u_{k-1}^\top & u_k^\top & y_{k-p}^\top & \cdots & y_{k-1}^\top \end{bmatrix}.
    \end{split}
\end{align}
Due to the feedback of a (by assumption) strictly causal controller, inputs are correlated with preceding noise (${\mathbb{E}[e_k u_j^\top]\neq0,\; \forall j>k}$), but inputs are uncorrelated with concurrent and subsequent noise (${\mathbb{E}[e_k u_j^\top]=0,\; \forall j\leq k}$). Since the innovation noise is also uncorrelated with preceding outputs (${\mathbb{E}[e_k y_j^\top]=0,\; \forall j<k}$), there is no correlation between the relevant terms in \eqref{eq:E_Phi_correlation} and the expectation of the bottom row of \eqref{eq:Yf_error2} with respect to future noise is zero in the limit $p,N_\mathrm{s}\rightarrow\infty$.


\begin{alignat}{2}
\begin{split}\label{eq:Yf_error2}
    \!\!&\lim_{p\rightarrow\infty}\widehat{Y}_{\hat{i}_p,1,f}-Y_{\hat{i}_p,1,f} = \lim_{p\rightarrow\infty}\!\Big[C\tKp{y}\left(\overline{Y}_{\hat{i},p,f}-Y_{\hat{i},p,f}\right)\\
        &%+E_{i_p,1,N_\mathrm{s}} \Pi_{\Psi_{i,1,N_\mathrm{s}}}^\bot W
        \quad+\hat{\Pi}_{eu||y}U_{\hat{i},p+1,f}+\hat{\Pi}_{ey||u}\overline{Y}_{\hat{i},p,f}-E_{\hat{i}_p,1,f}\Big].
\end{split}
\end{alignat}%
Taking the expectation (denoted by $\mathbb{E}[\cdot]$) with respect to future noise as conditioned on past data (denoted by $\mathbb{E}^\mathrm{f}[\cdot]$) of \eqref{eq:Yf_error2} removes the dependence on future noise $E_{\hat{i}_p,1,f}$. Consider the correlation matrix
% For the predictor to be unbiased, the expectation (which we will denote with $\mathbb{E}[\cdot]$) of this error w.r.t. the noise must be zero. \todo{$\mathbb{E}[\cdot]$,\\$EW$,\\$\hat{\Sigma}\hat{\Sigma}\inv$?} Consider the underbraced correlation matrix
\begin{align}%\label{eq:E_Phi_correlation}
    \begin{split}
        &\hat{\Sigma}_{e\psi} = \\
        &\;\frac{1}{N_\mathrm{s}}\;\sum\limits_{k=i_p}^{\mathclap{i_p+N_\mathrm{s}-1}} e_k \begin{bmatrix}u_{k-p}^\top & \cdots & u_{k-1}^\top & u_k^\top & y_{k-p}^\top & \cdots & y_{k-1}^\top \end{bmatrix}.
    \end{split}
\end{align}
Since $N_\mathrm{s}=(p+1)r+pl$, the limit $p\rightarrow\infty$ also ensures $N_\mathrm{s}\rightarrow\infty$. By the assumed conditions of second-order ergodicity and quasi-stationarity, the correlation matrix of \eqref{eq:E_Phi_correlation} asymptotically approaches a well-defined true correlation matrix. Due to the feedback of a (by assumption) strictly causal controller, inputs are correlated with preceding noise (${\mathbb{E}[e_k u_j^\top]\neq0,\; \forall j>k}$), but inputs are uncorrelated with concurrent and subsequent noise (${\mathbb{E}[e_k u_j^\top]=0,\; \forall j\leq k}$). Since the innovation noise is also uncorrelated with preceding outputs (${\mathbb{E}[e_k y_j^\top]=0,\; \forall j<k}$), there is no correlation between the relevant terms in \eqref{eq:E_Phi_correlation} and the expectation of the bottom row of \eqref{eq:Yf_error2} with respect to future noise is zero in the limit $p,N_\mathrm{s}\rightarrow\infty$. %Since $E_{i_p,1,N}$ and $\Psi_{i,1,N}$ are uncorrelated, $\mathbb{E}[E_{i_p,1,N} \Pi_{\Psi_{i,1,N}}^\bot W]=\mathbb{E}[E_{i_p,1,N}W]=0$ such that the expectation of the second row of \eqref{eq:Yf_error2} is also zero.

Given the structure of $\overline{\Psi}_{\hat{i},1,f}$ and $\overline{Y}_{\hat{i},p,f}$ shown in Fig.~\ref{fig:CL-DeePC}, consider \eqref{eq:Yf_error2} column by column. As discussed, taking the expectation conditioned on past data leaves
\begin{align}\label{eq:Yf_error3}
\begin{split}
    &\mkern-3mu\lim_{p\rightarrow\infty} \mathbb{E}^\mathrm{f}\left[\hat{y}_{\hat{i}_p+k}-y_{\hat{i}_p+k}\right] = \\ &\;\;\;\lim_{p\rightarrow\infty}C\tKp{y}\mathbb{E}^\mathrm{f}\left[\datavec{\overline{y}}{\hat{i}+k,p}-\datavec{y}{\hat{i}+k,p}\right],\forall k\in[0,f-1].
\end{split}
\end{align}
By \eqref{eq:Yf_error3} in the limit $p\rightarrow\infty$ the prediction $\hat{y}_{\hat{i}_p+k}$ is unbiased if the $p$ preceding output estimates are unbiased. For $k=0$, this is the case because none of the relevant preceding output data is estimated ($\datavec{\overline{y}}{\hat{i},p}=\datavec{y}{\hat{i},p}$). No bias is thereby introduced on the right hand side for $k=1$ such that $\hat{y}_{\hat{i}_p+1}$ is also unbiased. Repetition of this process until $k=f-1$ demonstrates that, in the limit $p\rightarrow\infty$, $\widehat{Y}_{\hat{i}_p,1,f}$ is indeed an asymptotically unbiased predictor with respect to future noise and conditioned on past data. This concludes the proof of $\mathrm{(ii)}$. $\hfill\qed$

\subsection{Systematic noise mitigation using \ac{IVs}}
The previous section demonstrated that the predictor that is obtained from \eqref{eq:CL_DeePC_no_IVs} is asymptotically unbiased. This section demonstrates the use of \ac{IVs} as a means to decrease the variance of this predictor.

Since the relevant innovation and input-output samples in the correlation matrix of \eqref{eq:E_Phi_correlation} are uncorrelated, the underbraced term in \eqref{eq:Yf_error2} vanishes asymptotically as $N_\mathrm{s}\rightarrow\infty$. Since $N_\mathrm{s}=(p+1)r+pl$, the rate at which the underbraced term vanishes in \eqref{eq:Yf_error2} is determined by the rate at which $p\rightarrow\infty$. For a finite $p$ it would be favorable to use a number of columns $N>N_\mathrm{s}$ such that the error induced by the correlation in \eqref{eq:E_Phi_correlation} is smaller. To this end, the following theorem makes use of an (extended) \ac{IV} $\mathcal{Z}$.
% 
% are increasing the variance of the implicitly estimated noise in \todo{left off here}$\widehat{E}_{\hat{i}_p,1,f}$. The idea is to redefine $G$ such that its columns are orthogonal to the noise. The following theorem demonstrates the use of an \ac{IV} $\mathcal{Z}$ for this purpose based on the same assumptions as Theorem~\ref{theorem:main_result}.

% \begin{thm}\label{theorem:main_result_IVs}
%     % Consider the minimal discrete non-deterministic \ac{LTI} system given by~\eqref{eqn:SS_innovation} to generate input-output data in closed-loop with a strictly causal controller. Define data matrices $\Psi_{i,1,N}$ and $\overline{\Psi}_{\hat{i},1,f}$ as in \eqref{eq:Phi_def}. %
%     % 
%     % If the joint input and noise sequences are sufficiently persistently exciting such that $\left[X_{i,1,N}^\top \; U_{i,p,N}^\top \; U_{i_p,1,N}^\top \; E_{i,p,N}^\top\right]^\top$ is full row rank, and with the choice of \ac{IV} given by $\mathcal{Z}=\Psi_{i,1,N}$ then\\
%     Consider the minimal discrete non-deterministic \ac{LTI} system given by~\eqref{eqn:SS_innovation} to generate input-output data in closed-loop by means of a causal controller without direct feedthrough, data matrices $\overline{\Psi}_{\hat{i},1,f}$ and full row rank $\Psi_{i,1,N}$ as in \eqref{eq:Phi_def}, and with an \ac{IV} given by $\mathcal{Z}=\Psi_{i,1,N}$ then\\
%     $\mathrm{(i)}$ $\exists G^\mathrm{IV}\in\mathbb{R}^{((p+s)r+pl)\times f}$ such that
%     \begin{align}\label{eq:Theorem2}
%         \begin{bmatrix}
%             \Psi_{i,1,N}\\Y_{i_p,1,N}
%         \end{bmatrix}\mathcal{Z}^\top G^\mathrm{IV} =
%         \begin{bmatrix}
%             \overline{\Psi}_{\hat{i},1,f}\\\widehat{Y}_{\hat{i}_p,1,f}^\mathrm{IV}
%         \end{bmatrix},
%     \end{align}
%     $\mathrm{(ii)}$ and with $\widehat{Y}_{\hat{i}_p,1,f}^\mathrm{IV}$ as an asymptotically unbiased predictor with respect to future noise and conditioned on past data as $p\rightarrow\infty$, $N\rightarrow\infty$.\\
%     $\mathrm{(iii)}$ that has a variance that is smaller than or equal to the variance of $\widehat{Y}_{\hat{i}_p,1,f}$.
% \end{thm}
\textbf{Proof of $\mathrm{(i)}:$} Similarly as before, $G^\mathrm{IV}$ is determined by the top matrix equations of \eqref{eq:Theorem2}, which are given by $\Psi_{i,1,N}\mathcal{Z}^\top G^\mathrm{IV}=\overline{\Psi}_{\hat{i},1,f}$. By the same reasoning as in the proof of Theorem~\ref{theorem:main_result}$\mathrm{(i)}$ the matrix $\Psi_{i,1,N}$ is full row rank. Hence, $\Psi_{i,1,N}\mathcal{Z}^\top=\Psi_{i,1,N}\Psi_{i,1,N}^\top$ is square and invertible such that $G^\mathrm{IV} = (\Psi_{i,1,N}\Psi_{i,1,N}^\top)\inv \overline{\Psi}_{\hat{i},1,f}$. $\hfill \qed$\\
\textbf{Remark 3:} Note that $G$ in Theorem~\ref{theorem:main_result} has been replaced with $\mathcal{Z}^\top G^\mathrm{IV}$ in Theorem~\ref{theorem:main_result_IVs}, which is
\begin{align*}
    \mathcal{Z}^\top G^\mathrm{IV} = \Psi_{i,1,N}^\dagger \overline{\Psi}_{\hat{i},1,f}.
\end{align*}
This is the same as $G$ as defined in \eqref{eq:G_sols}, but with $\Pi_{\Psi_{i,1,N}}^\bot W=0$. Essentially $G$ is restricted to lie in the row space of $\Psi_{i,1,N}$.\\
\textbf{Proof of $\mathrm{(ii)}$:} Based on Remark 3 the asymptotic unbiasedness of $\widehat{Y}_{\hat{i},1,f}^\mathrm{IV}$ as $p\rightarrow\infty$ follows directly from the proof of Theorem~\ref{theorem:main_result}$\mathrm{(ii)}$ with $\Pi_{\Psi_{i,1,N}}^\bot W=0$. $\hfill \qed$\\
\textbf{Proof of $\mathrm{(iii)}$:} This proof requires it to be shown that
\begin{align}\label{eq:CovarianceDecrease}
\begin{split}
     \lim_{p,N\rightarrow\infty} \mathbb{E}&\left[(\datavec{\hat{y}}{\hat{i}_p,f}-\datavec{y}{\hat{i}_p,f})(\datavec{\hat{y}}{\hat{i}_p,f}-\datavec{y}{\hat{i}_p,f})^\top\right]\\
   -\mathbb{E} &\left[(\datavec{\hat{y}}{\hat{i}_p,f}^\mathrm{IV}-\datavec{y}{\hat{i}_p,f})(\datavec{\hat{y}}{\hat{i}_p,f}^\mathrm{IV}-\datavec{y}{\hat{i}_p,f})^\top\right] \succeq 0,   
\end{split}
\end{align}
in which $\datavec{\hat{y}}{\hat{i}_p,f}=\text{vec}(\widehat{Y}_{\hat{i}_p,1,f})$, and $\datavec{\hat{y}}{\hat{i}_p,f}^\mathrm{IV}=\text{vec}(\widehat{Y}_{\hat{i}_p,1,f}^\mathrm{IV})$. Applying the limit $N\rightarrow\infty$ to \eqref{eq:Yf_error2} makes the sample correlation in \eqref{eq:E_Phi_correlation} converge to zero, thereby leaving after vectorization
\begin{align*}
\begin{split}
    \lim_{p,N\rightarrow\infty} &\datavec{\hat{y}}{\hat{i}_p,f}-\datavec{y}{\hat{i}_p,f}%\widehat{Y}_{\hat{i}_p,1,f}-Y_{\hat{i}_p,1,f}
     = \lim_{p,N\rightarrow\infty}\Big[\\
     &\big((\Pi_{\Psi_{i,1,N}}^\bot W)^\top \otimes I_l\big)\datavec{e}{i_p,N}-\datavec{e}{\hat{i}_p,f}\\
     +&\underbrace{\big(I_f \otimes C\tKp{y}\big)\text{vec}\left(\overline{Y}_{\hat{i},p,f}-Y_{\hat{i},p,f}\right)}_{=(I_{fl}-\mathcal{\widetilde{H}}_{f})(\datavec{\hat{y}}{\hat{i}_p,f}-\datavec{y}{\hat{i}_p,f})}\Big].
\end{split}
\end{align*}
The underbraced term can be replaced as indicated since $\lim_{p\rightarrow\infty}\tilde{A}^p=0$ such that
\begin{align}\label{eq:Yf_error4}
\begin{split}
    \lim_{p,N\rightarrow\infty} &\datavec{\hat{y}}{\hat{i}_p,f}-\datavec{y}{\hat{i}_p,f}%\widehat{Y}_{\hat{i}_p,1,f}-Y_{\hat{i}_p,1,f}
     = \lim_{p,N\rightarrow\infty}\Big[\\
     &\mathcal{H}_f\big((\Pi_{\Psi_{i,1,N}}^\bot W)^\top \otimes I_l\big)\datavec{e}{i_p,N}-\mathcal{H}_f\datavec{e}{\hat{i}_p,f}\Big],
\end{split}
\end{align}
where $\mathcal{H}_f=\mathcal{\widetilde{H}}_f\inv$~\citep[Lemma~1]{Houtzager2012}. Similarly then, based on Remark~3 and \eqref{eq:Yf_error4}
\begin{align}\label{eq:Yf_error5}
    \lim_{p,N\rightarrow\infty} \datavec{\hat{y}}{\hat{i}_p,f}^\mathrm{IV}-\datavec{y}{\hat{i}_p,f}%\widehat{Y}_{\hat{i}_p,1,f}-Y_{\hat{i}_p,1,f}
     = \lim_{p,N\rightarrow\infty}-\mathcal{H}_f\datavec{e}{\hat{i}_p,f}.
\end{align}
Applying the found errors from \eqref{eq:Yf_error4} and \eqref{eq:Yf_error5} to \eqref{eq:CovarianceDecrease} obtains
\begin{align}\label{eq:Cov}
\begin{split}
        &\lim_{p,N\rightarrow\infty} \mathbb{E}\left[(\datavec{\hat{y}}{\hat{i}_p,f}-\datavec{y}{\hat{i}_p,f})(\datavec{\hat{y}}{\hat{i}_p,f}-\datavec{y}{\hat{i}_p,f})^\top\right]\\
        &\qquad\mkern1mu-\mathbb{E}\left[(\datavec{\hat{y}}{\hat{i}_p,f}^\mathrm{IV}-\datavec{y}{\hat{i}_p,f})(\datavec{\hat{y}}{\hat{i}_p,f}^\mathrm{IV}-\datavec{y}{\hat{i}_p,f})^\top\right]
\end{split}\notag\\
\begin{split}
        &=\lim_{p,N\rightarrow\infty}\Big\{\mathcal{H}_f\psi\mathbb{E}[\datavec{e}{i_p,N}\datavec{e}{i_p,N}^\top]\psi^\top\mathcal{H}_f^\top\\
        &\qquad\mkern1mu-\mathcal{H}_f\psi\mathbb{E}[\datavec{e}{i_p,N}\datavec{e}{\hat{i}_p,f}^\top]-\mathbb{E}[\datavec{e}{\hat{i}_p,f}\datavec{e}{i_p,N}^\top]\psi^\top\mathcal{H}_f^\top\Big\}
\end{split}\notag\\
&=\lim_{p,N\rightarrow\infty}\Big\{\mathcal{H}_f\psi(I_N\otimes R_\mathrm{e})\psi^\top\mathcal{H}_f^\top\Big\}\notag\\
&=\lim_{p,N\rightarrow\infty}\Big\{\mathcal{H}_f\psi\big(I_N\otimes R_\mathrm{e}^{1/2}\big)\big(I_N\otimes R_\mathrm{e}^{\top/2}\big)\psi^\top\mathcal{H}_f^\top\Big\}\notag\succeq 0,
\end{align}
in which use is made of $\psi=(\Pi_{\Psi_{i,1,N}}^\bot W)^\top \otimes I_l$ for brevity, the Cholesky factorization $R_\mathrm{e}=R_\mathrm{e}^{1/2}R_\mathrm{e}^{\top/2}$ (since ${R_\mathrm{e}\succ0}$), and the fact that past and future innovations ($\datavec{e}{i_p,N}$ and $\datavec{e}{\hat{i}_p,f}$ respectively) are uncorrelated. $\hfill  \qed$
%%%%%%%%%%%%%%%%%%%%%%%%%%%%%%%%%%%%%%%%%%%%%%%%%%%%%%%%%%%%%%%%%%%%%%%%%%%%%%%%%%%%%%%%%%%%%%%%%%%%%%%%%%%%%%%%%%%%%%%%%%%%%
%%%%%%%%%%%%%%%%%%%%%%%%%%%%%%%%%%%%%%%%%%%%%%%%%%%%%%%%%%%%%%%%%%%%%%%%%%%%%%%%%%%%%%%%%%%%%%%%%%%%%%%%%%%%%%%%%%%%%%%%%%%%%